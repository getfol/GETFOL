\gfcommand{alle}{universal quantification elimination rule}
\index{alle}\index{us}

\gfsyntax{
   alle \ALT us \ARG{fact} \OPT{,} \ARG{term1} \ARG{term2} \SEQ;
}

\gfdescription{
   \renewcommand{\arraystretch}{0.5}
   \[
   \begin{array}{l}
      \\
      \forall E
   \end{array}
   \ \ 
   \begin{array}{c}
      \fraz{\forall x A}
           {A^{x}_{t}}
   \end{array}
   \]
   \renewcommand{\arraystretch}{1}

   The command uses the terms in \ARG{term1} \ARG{term2} \SEQ to instantiate
   the universally quantified variables in the order in which they appear.
   One execution of the command can instantiate more than one universally 
   quantified variable.
   If a particular term is not free for the variable to be instantiated, a 
   bound variable change is made and then the substitution is made.
   The dependencies are those of \ARG{fact}.

   In the sorted rule, let \verb+forall x. A(x)+ be the formula to be 
   instantiated, and let us suppose first that only a term {\tt t} is provided.
   If {\tt t} has he same sort as {\tt x}, say {\tt Sx}, then the resulting 
   formula is {\tt A(t)}; otherwise the resulting formula is
   {\tt Sx(t) imp A(t)}.
}

\gfrecap{
The command uses the terms in `term1' `term2' ... to instantiate
the universally quantified variables in the order in which they appear.
One execution of the command can instantiate more than one universally 
quantified variable.
If a particular term is not free for the variable to be instantiated, a 
bound variable change is made and then the substitution is made.
The dependencies are those of `fact'.
In the sorted rule, let `forall x. A(x)' be the formula to be 
instantiated, and let us suppose first that only a term `t' is provided.
If `t' has he same sort as `x', say `Sx', then the resulting 
formula is `A(t)'; otherwise the resulting formula is
`Sx(t) imp A(t)'.
}  

\gfexample+
   ***** declare predconst P 2;
   ***** declare indvar x y;
   ***** declare indconst c1 c2;
   [...]

   ***** assume forall x y. P(x y);
   1   forall x y.P(x,y)     (1)

   ***** alle 1 c1;
   2   forall y.P(c1,y)     (1)

   ***** alle 1 x c1;
   3   P(x,c1)     (1)

   ***** alle 1 c1 c2;
   4   P(c1,c2)     (1)
   
   ***** declare predconst P 2;
   ***** declare indvar x [Sx];
   ***** declare indvar y [Sy];
   ***** declare indconst c1 c2 [S];
   [...]

   ***** moregeneral Sx < S >;

   ***** assume forall x y. P(x y);
   1   forall x y.P(x,y)     (1)

   ***** alle 1 c1;
   2   forall y.P(c1,y)     (1)

   ***** alle 2 c1;
   3   Sy(c1) imp P(c1,c1)     (1)

   ***** alle 1 c1 c2;
   4   Sy(c2) imp P(c1,c2)     (1)
+

\gfnotes{
   The rule is sometimes called ``universal specialization''.
}
