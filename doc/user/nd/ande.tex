\gfcommand{ande}{and elimination rule}
\index{ande}\index{ae}

\gfsyntax{
   ande \ALT ae \ARG{fact} \OPT{,} 1 \ALT 2; \\
   ande \ALT ae \ARG{fact} \OPT{,} 1 \ALT 2 1 \ALT 2 \SEQ;
}

\gfdescription{
   \[
      \con E  \ \ 
      \fraz{A \con B}
           {A}
      \ \ 
      \fraz{A \con B}
           {B}
   \]

   The wff of \ARG{fact} must be a conjunction.
   A proof line is deduced whose wff is the left conjunct (if 1 \ALT 2 is 1)
   or the right conjunct (if 1 \ALT 2 is 2). The fact inherits \ARG{fact}'s
   dependencies.

   In the second form, the command is applied to a fact whose wff is a
   recursive conjunction of wffs, {\it e.g.} {\tt A and ((B and C) and D)}.
   The sequence ``1 \ARG 2 1 \ARG 2 \SEQ" picks up the appropriate subformula.
}

\gfrecap{
Applies and elimination rule.
The formula given as arguments must be a conjunction.
A proof line is deduced whose wff is the left conjunct (if the number is 1)
or the right conjunct (if the number is 2). The fact inherits
the dependencies of `fact'.
In the second form, the command is applied to a fact whose wff is a
recursive conjunction of wffs, eg. `A and ((B and C) and D)'.
The sequence `1 | 2 1 | 2 ...' picks up the appropriate subformula.
}

\gfexample+
   ***** declare sentconst A B C D;
   ***** assume A and ((B and C) and D);
   1  A and ((B and C) and D)  (1)

   ***** ande 1 1;
   2   A  (1)

   ***** ande 1 2 1;
   3  B and C  (1)

   ***** ande 1 2 1 2;
   4  C  (1)
+

\gfnotes{
   A proof line derived by executing the second form can always be derived by
   a sequence of executions of commands in the first form.
}
