\gfcommand{falsee}{ falsity rule in intuitionistic logic}
\index{falsee}\index{fe}

\gfsyntax{
   falsee \ALT fe \ARG{fact1} \OPT{,} \ARG{wff};\\
   falsee \ALT fe \ARG{fact1} \OPT{,} \ARG{fact2};
}

\gfdescription{
   \renewcommand{\arraystretch}{0.5}
   \[
   \begin{array}{l}
      \bot_i  
   \end{array}
   \ \ 
   \begin{array}{c}
      \fraz{\bot}
           {A}
   \end{array}
   \]
   \renewcommand{\arraystretch}{1}

   \ARG{fact1} must have wff {\tt FALSE}.
   In the first form, the wff derived is \ARG{wff} and its dependencies are 
   those of \ARG{fact1}.
   In the second form, the wff of \ARG{fact2} is derived.
}

\gfrecap{
   `fact1' must have wff `FALSE'.
   In the first form, the wff derived is `wff' and its dependencies are 
   those of `fact1'.
   In the second form, the wff of `fact2' is derived.
}

\gfexample+
   ***** declare sentconst A;
   [...]

   ***** assume FALSE;
   1  FALSE  (1)

   ***** falsee 1 A and not A;
   2  A and (not A)  (1)

   ***** reset;
   [...]

   ***** declare sentconst A;
   [...]

   ***** assume not not A not A A;
   1  not not A  (1)
   2  not A  (2)
   3  A  (3)

   ***** falsei 1 2;
   4  FALSE  (1 2)

   ***** falsee 4 3;
   5  A  (1 2);

   ***** impi 1 5;
   6  not not A imp A (2)
+

\gfnotes{
   This rule says that anything follows from a contradiction.
}
