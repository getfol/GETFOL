\gfcommand{ori}{or introduction rule}
\index{ori}\index{oi}

\gfsyntax{
   ori \ALT oi \ARG{fact} \OPT{,} \ARG{wff} \OPT{,} \OPT{lr \ALT rl};\\
   ori \ALT oi \ARG{fact} \OPT{,} \ARG{fact1} \ALT \ARG{wff1} 
       disj \ALT dj \ARG{fact2} \ALT \ARG{wff2} disj \ALT dj \SEQ
       \OPT{,} \OPT{lr \ALT rl};
}

\gfdescription{
   \renewcommand{\arraystretch}{0.5}
   \[
   \begin{array}{l}
      \dis I
   \end{array}
   \ \ 
   \begin{array}{c}
   \fraz{A}
        {A \dis B}
   \end{array}
   \ \ \ 
   \begin{array}{c}
   \fraz{A}
        {B \dis A}
   \end{array}
   \]
   \renewcommand{\arraystretch}{1}

   The command creates a new fact whose wff is the disjunction of \ARG{fact}'s
   wff and \ARG{wff}.
   The option {\tt lr \ALT rl} specifies the order of the disjuncts:
   {\bf lr} stands for ``left-right'' and it means that the left disjunct is 
   \ARG{fact}'s wff and the right one is \ARG{wff}; {\tt rl} viceversa.
   If no order is specified, then {\tt lr} is the default.
   The new proof line inherits \ARG{fact}'s dependencies.

   In the second form, the command accepts ``disjunctions of facts and wffs''
   as second argument.
   A ``disjunctions of facts and wffs'' is any parenthesized disjunctive
   expression in which all disjuncts are facts (\ARG{fact1} {\tt disj} 
   \ARG{fact2} \SEQ), wffs (\ARG{wff1} {\tt disj} \ARG{wff2} \SEQ)
   or facts and wffs (\ARG{fact1} {\tt disj} \ARG{wff2} \SEQ).
   The derived formula is the disjunction of the formulae.
   It depends on the assumptions \ARG{fact} and  all the \ARG{factI}s 
   depend on.
}

\gfrecap{
The command creates a new fact whose wff is the disjunction of `fact''s
formula and `wff'.
The option `lr \ALT rl' specifies the order of the disjuncts:
`lr' stands for ``left-right" and it means that the left disjunct is 
the formula of `fact' and the right one is `wff'; `rl' viceversa.
If no order is specified, then `lr' is the default.
The new proof line inherits dependencies of `fact'.
In the second form, the command accepts ``disjunctions of facts and wffs"
as second argument.
A ``disjunctions of facts and wffs" is any parenthesized disjunctive
expression in which all disjuncts are facts (`fact1 disj fact2 ...), 
wffs (wff1 disj wff2 ...) or facts and wffs (fact1 disj wff2 ...).
The derived formula is the disjunction of the formulae.
It depends on the assumptions `fact' and  all the `factI's 
depend on.
}

\gfexample+
   ***** declare sentconst A B;
   ...

   ***** assume A;
   1  A  (1);
   ***** ori 1 B;
   2  A or B  (1)
   ***** ori 1 B lr ;
   3  A or B (1)
   ***** ori 1 B rl ;
   4  B or A  (1)

   ***** reset;
   ***** declare sentconst A B C D E;
   ...

   ***** assume A B C;
   1  A  (1);
   2  B  (2);
   3  C  (3);

   ***** ori 1 B dj C rl;
   4  (B or C) or A   (1)

   ***** ori 1 2 dj 3;
   5  A or (B or C)   (1)

   ***** ori 1 2 dj D dj 3 dj E rl;
   6  (B or (D or (C or E))) or A   (1)
+

\gfnotes{
   A proof line derived by executing the second form can always be derived by
   a sequence of executions of the command in the first form.
}
