\section{Natural Deduction (ND)}
\label{sec-nd}
\label{sec-nd-first}

\subsection{{\GF} logic}
\label{sec-ndrules}

{\GF} uses a Natural Deduction (ND) calculus based on Prawitz's system
defined in \cite{prawitz1}.
Some resemblances exist also to the ND calculus defined by Quine in
\cite{quine3}.
For various reasons ({\it e.g.} efficiency of the implementation, elegance of
the proof theory), {\GF} also carries around the dependencies of any derived
formula.
This allows to see the {\GF} logic as a sequent calculus, where a sequent is a
pair $(\Gamma, A)$, where $A$ is a formula and $\Gamma$ a set of formulas, with
introduction and elimination in the post sequent.
We claim that the ``{\em correct}'' way to see {\GF} logic is as a ND calculus.
This is why all the commands are described in ND style, without explicitly
writing the dependencies.

Some notes about the ND rules described in figure \ref{fig-nd}:

\begin{itemize}
\item
	The notation used is the same as in \cite{prawitz1}.
\item
	The $\forall I$ and $\exists E$ rules have the following restrictions:
	$a$ must not appear free in the dependencies of $A$ (for $\forall I$)
	and $a$ must not appear free in $\exists x A$, in $B$  or in any assumption
	on which the upper occurrence of B depends other than $A^{x}_{a}$ (for
	$\exists E$).
\item
	{\GF} natural deduction rules {\em discharge all occurrences}.
\item
	In {\GF} both the negation connective {\bf not} ($\neg$) and the falsity
	sentential constant {\tt FALSE} ($\bot$) are available.
	$\neg A$ and $A \imp \bot$ are logically equivalent.
	From a wff containing an occurrence of $A \imp \bot$ it is possible to
	deduce the wff with $\neg A$ in place of $A \imp \bot$ and viceversa.
	In {\GF}, this deduction can be performed in several ways.
	The simplest is to use the decider for propositional calculus ({\tt ptaut})
	(see section \ref{sec-decproc}).
	For instance:
	%
	\begin{verbatim}
		***** declare sentconst A;
		A has been declared to be a Sentconst
		***** assume A imp FALSE;
		1   A imp FALSE     (1)
		***** ptaut not A by 1;
		2   not A     (1)
		***** assume not A;
		3   not A     (3)
		***** ptaut A imp FALSE by 3;
		4   A imp FALSE     (3)
	\end{verbatim}
\end{itemize}


The ND notation is given for the primitive ND rules, although the ND commands
implement (with different options) both basic and more complex derived inference
rules.
A complete discussion of the capabilities of the commands is given in the
description of each command.

In the syntax for the commands we will use {\em fact}, {\em fact}$_1$,
{\em fact}$_2$, $\ldots$, in place of {\em label}, {\em label}$_1$,
{\em label}$_2$, $\ldots$ (see section \ref{sec-reas}).
However, the user refers to the fact by its label.

The {\GF} command \verb+existe+ implementing the exist elimination rule 
($\exists E$) is the most radically different from the formal statement given
in Prawitz's ND calculus.
The command has some resemblance with the ``exist instantiation'' defined in
\cite{quine3}.
The {\GF} $\exists E$ command has side effects that make all the other
rules change behavior.
This is explained, in more detail, in the description of the command
\verb+existe+ in section \ref{sec-nd-first}.


\renewcommand{\arraystretch}{0.5}
\begin{figure}[htbp]
\[
	\begin{array}{|c|c||c|c|} \hline
	\multicolumn{2}{|c||}{\mbox{\bf Introduction rules}} &
	\multicolumn{2}{c|}{\mbox{\bf Elimination rules}} \\ \hline
	& & &\\
	& & &\\
	\con I  & 
	\fraz{A \ \ B}
     {A \wedge B}
	&
	\con E  & 
	\fraz{A \con B}
     {A}
	\ \ 
	\fraz{A \con B}
	     {B}
	\\
	\begin{array}{l}
	\\ \\ \\ \\ \\
	\imp I   
	\end{array}
	& 
	\begin{array}{c}
	\\ \\
	{[A]}\\
	\vdots\\
	B\\
	\hline\\
	A \imp B
	\end{array}
	&
	\begin{array}{l}
	\\ \\ \\ \\ \\
	\imp E  
	\end{array}
	& 
	\begin{array}{c}
	\\ \\ \\ \\ \\
	\fraz{A \ \ A \imp B}
	     {B}
	\end{array}
	\\
	\begin{array}{l}
	\\ \\ \\ \\ \\
	\dis I
	\end{array}
	& 
	\begin{array}{c}
	\\ \\ \\ \\ \\
	\fraz{A}
	     {A \dis B}
	\end{array}
	\ \ 
	\begin{array}{c}
	\\ \\ \\ \\ \\
	\fraz{A}
  	   {B \dis A}
	\end{array}
	&
	\begin{array}{l}
	\\ \\ \\ \\ \\
	\dis E  
	\end{array}
	& 
	\begin{array}{ccc}
	\\ \\
	&{[A]}&{[B]}\\
	&\vdots&\vdots\\
	A \dis B & C & C\\
	\hline\\
	& C &
	\end{array}
	\\
	\begin{array}{l}
	\\ \\ \\ \\ \\
	\bot_c 
	\end{array}
	&
	%\renewcommand{\arraystretch}{0.5}
	\begin{array}{c}
	\\ \\
	{[\neg A]}\\
	\vdots\\
	\bot\\
	\hline\\
	A 
	\end{array}
	&
	\begin{array}{l}
	\\ \\ \\ \\ \\
	\bot_i  
	\end{array}
	& 
	\begin{array}{c}
	\\ \\ \\ \\ \\
	\fraz{\bot}
	     {A}
	\end{array}
	\\
	\begin{array}{l}
	\\ \\ \\ \\ \\
	\imp I_{\neg}  
	\end{array}
	& 
	\begin{array}{c}
	\\ \\
	{[A]}\\
	\vdots\\
	\bot\\
	\hline\\
	\neg A
	\end{array}
	&
	\begin{array}{l}
	\\ \\ \\ \\ \\
	\imp E_{\neg}  
	\end{array}
	& 
	\begin{array}{c}
	\\ \\ \\ \\ \\
	\fraz{A \ \ \neg A}
	     {\bot}
	\end{array}
	\\
	\begin{array}{l}
	\\ \\
	\liff  I  
	\end{array}
	& 
	\begin{array}{c}
	\\ \\ 
	\fraz{A \imp B \ \ B \imp A}
	     {A \liff B}
	\end{array}
	&
	\begin{array}{l}
	\\ \\
	\liff E  
	\end{array}
	& 
	\begin{array}{c}
	\\ \\
	\fraz{A \liff B}
	     {A \imp B}
	\end{array}
	\ \
	\begin{array}{c}
	\\ \\
	\fraz{A \liff B}
	     {B \imp A}
	\end{array}
	\\
	\begin{array}{l}
	\\ \\
	\forall  I  
	\end{array}
	& 
	\begin{array}{c}
	\\ \\
	\fraz{A}
	     {\forall x A^{a}_{x}}
	\end{array}
	&
	\begin{array}{l}
	\\ \\
	\forall E
	\end{array}
	& 
	\begin{array}{c}
	\\
	\fraz{\forall x A}
	     {A^{x}_{t}}
	\end{array}
	\\
	\begin{array}{l}
	\\ \\ \\ \\ \\
	\exists  I  
	\end{array}
	& 
	\begin{array}{c}
	\\ \\ \\ \\ \\
	\fraz{A}
	     {\exists x A^{t}_{x}}
	\end{array}
	&
	\begin{array}{l}
	\\ \\ \\ \\ \\
	\exists E 
	\end{array}
	&
	\begin{array}{cc}
	\\ \\
	&{[A^{x}_{a}]}\\
	&\vdots\\
	\exists x A & B\\
	\hline\\
	B
	\end{array}
	\\ 
	&&& \\ \hline
	\end{array}
	\]
\caption{ND inference rules}
\label{fig-nd}
\end{figure}

\renewcommand{\arraystretch}{1}


\subsection{{\GF} sorted logic}

In section \ref{sec-ndrules} we described the inference rules for the unsorted
logic. 
However, the sort information present in the language has to be taken into
account when performing deduction.
The only natural deduction rules that have to be modified are the quantifier
introduction and elimination rules, where the sort of variables and terms
involved in the deduction substantially changes the applicability of the rules
and the conclusion.

The sorted ND rules are defined as follows:

$$
\begin{array}{ll}
\forall E & \left\{
\begin{array}{cl}
\fraz{\forall x A}{A^{x}_{t}} &
\mbox{if $t$ is of sort $S_x$} \\
& \\
\fraz{\forall x A}{S_x(t)\imp A^x_t} & \mbox{otherwise}
\end{array}
\right. % } to balance
\\

& \\

\forall I & \ps
\begin{array}{cl}
\fraz{A}
     {\forall x A^{a}_{x}}
& \mbox{applicable only if $x$ is of sort $S_a$}
\end{array} \\

& \\

\exists I & \left\{
\begin{array}{cl}
\fraz{A}
     {\exists x A^{t}_{x}}
& \mbox{if $t$ is of sort $S_x$} \\
& \\
\fraz{A^x_t}{S_x(t)\imp\exists x A}
& \mbox{otherwise}
\end{array} 
\right. % } to balance
\\

& \\
\exists E & \ps
\fraz{
\begin{array}{cc}
&{[A^{x}_{a}]}\\
&\vdots\\
\exists x A & B\\
\end{array}}
{B}
\pps
\mbox{applicable only if $x$ is of sort $S_a$}

\end{array}
$$

where $S_x$ and $S_a$ stand for the sort of $x$ and the sort of $a$
respectively.
The restrictions on the unsorted $\forall I$ and $\exists E$  rules apply also
to the rules above. 
