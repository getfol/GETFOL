\gfcommand{noti}{implication introduction for negation}
\index{noti}\index{ni}

\gfsyntax{
   noti \ALT ni \ARG{fact1} \OPT{,} \ARG{wff}; \\
   noti \ALT ni \ARG{fact1} \OPT{,} \ARG{fact2};
}

\gfdescription{
   \renewcommand{\arraystretch}{0.5}
   \[
   \begin{array}{l}
      \\ \\ 
      \imp I_{\neg}  
   \end{array}
   \ \ 
   \begin{array}{c}
      {[A]}\\
      \vdots\\
      \bot\\
      \hline\\
      \neg A
   \end{array}
   \]
   \renewcommand{\arraystretch}{1}

   The wff of \ARG{fact1} must be {\tt FALSE}.
   The command creates a fact whose wff is the negation of \ARG{wff}.
   It depends on all the dependencies of \ARG{fact1} less all the dependencies
   whose wff is equal to \ARG{wff} (in the first form) or \ARG{fact2}'s wff
   (in the second form).
}

\gfrecap{
   The wff of `fact1' must be `FALSE'.
   The command creates a fact whose wff is the negation of `wff'.
   It depends on all the dependencies of `fact1' less all the dependencies
   whose wff is equal to `wff' (in the first form) or the formula of 
   `fact2' (in the second form).
}
   
\gfexample+
   ***** declare sentconst A;
   [...]

   ***** assume A not A;
   1  A  (1)
   2  not A  (2)

   *****  falsei 1 2;
   3  FALSE  (1 2)

   *****  noti 3 not A;
   4  not not A  (1)

   *****  impi 1 4;
   5  A imp not not A
+

\gfnotes{
   Since $\neg A$ ai equivalent to $A \imp \bot$,
   this rule can be seen as a special case of the implication introduction
   rule ({\tt impi}), in which  the asserted line has \verb+not+ ($\neg$) as
   main symbol.
}
