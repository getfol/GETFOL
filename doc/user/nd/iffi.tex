\gfcommand{iffi}{equivalence introduction}
\index{iffi}\index{ii}

\gfsyntax{
   iffi \ALT ii \ARG{fact1} \OPT{,} \ARG{fact2};
}

\gfdescription{
   \renewcommand{\arraystretch}{0.5}
   \[
   \begin{array}{l}
      \liff  I  
   \end{array}
   \ \ 
   \begin{array}{c}
      \fraz{A \imp B \ \ B \imp A}
           {A \liff B}
   \end{array}
   \]
   \renewcommand{\arraystretch}{1}

   Both facts must be implications. 
   If \ARG{fact1}'s wff is of the form {\tt A imp B}, then \ARG{fact2}'s wff 
   must be {\tt B imp A}.
   The conclusion is {\tt A iff B} with the union of the dependencies of 
   \ARG{fact1} and \ARG{fact2}.
}

\gfrecap{
Both facts must be implications. 
If the formula of `fact1' is of the form `A imp B', then the formula of
`fact2' must be `B imp A'.
The conclusion is {\tt A iff B} with the union of the dependencies of 
`fact1' and `fact2'.
}
   
\gfexample+
   ***** declare sentconst A B;
   [...]

   ***** assume A imp B B imp A;
   1  A imp B  (1)
   2  B imp A  (2)

   ***** iffi 1 2;
   3  A iff B  (1 2);

   ***** iffi 2 1;
   4  B iff A  (1 2)

   ***** reset;
   [...]

   ***** declare sentconst A;
   [...]

   ***** assume FALSE imp A A imp FALSE;
   1  FALSE imp A  (1)
   2  A imp FALSE  (2)

   ***** iffi 1 2;
   3  FALSE iff A  (1 2);
+
