\gfcommand{existi}{existential quantification introduction rule}
\index{existi}\index{ei}

\gfsyntax{
  existi \ARG{fact}
   \OPT{\OPT{,} \ARG{term1} :} \ARG{indvar1} \OPT{occ \ARG{n11} \ARG{n12} \SEQ}
   \OPT{\OPT{,} \ARG{term2} :} \ARG{indvar2} \OPT{occ \ARG{n21} \ARG{n22} \SEQ}
   \SEQ;
}


\gfdescription{
  \renewcommand{\arraystretch}{0.5}
  \[
  \begin{array}{l}
    \exists  I  
  \end{array}
  \ \ 
  \begin{array}{c}
    \fraz{A}
    {\exists x A^{t}_{x}}
  \end{array}
  \]
  \renewcommand{\arraystretch}{1}

  The list following \ARG{fact} indicates which terms are to be 
  existentialized.
  If the optional \ARG{termI} is present, it is replaced by \ARG{indvarI}
  at each occurrence mentioned in the sequence of natural numbers
  \OPT{$n_{i1}$ $n_{i2}$ \SEQ} and then  existentialized.
  If \ARG{termI} is not present, all the occurrences of \ARG{indvarI} are
  put under the scope of the existential quantifier.
  Notice that no use can be made of an occurrence specification
  \OPT{$n_{i1}$ $n_{i2}$ \SEQ} if there is no \ARG{termI} present,
  {\GF} will return an error in this case.
  The dependencies of the conclusion are those of \ARG{fact}.

  In the sorted rule, let {\tt A(t)} be the formula to be existentialized,
  and let {\tt x} be of sort {\tt Sx}.
  Then the result of the rule is {\tt exists x.A(x)} if {\tt t} is of sort
  {\tt Sx}, {\tt Sx(t) imp exists x.A(x)} otherwise.
  This schema applies also to the case where multiple terms are to be
  existentialized.
}

\gfrecap{
The list following `fact' indicates which terms are to be existentialized.
If the optional `termI' is present, it is replaced by `indvarI' at each 
occurrence mentioned in the sequence of natural numbers [nI1 nI2 ...] and then
existentialized.
If `termI' is not present, all the occurrences of `indvarI' are put under the 
scope of the existential quantifier.
Notice that no use can be made of an occurrence specification [nI1 nI2 ...] if
there is no `termI' present, GETFOL will return an error in this case.
The dependencies of the conclusion are those of `fact'.
In the sorted rule, let `A(t)' be the formula to be existentialized, and let
`x' be of sort `Sx'.
Then the result of the rule is `exists x. A(x)' if `t' is of sort `Sx',
`Sx(t) imp exists x. A(x)' otherwise.
This schema applies also to the case where multiple terms are to be
existentialized.
}  

\gfexample+
   *****  comment ! unsorted example !
       
   ***** declare predconst P 2;
   ***** declare indvar x y x0;
   ***** declare indconst c1 c2;
   [...]

   ***** assume P(c1 c2);
   1   P(c1,c2)     (1)

   ***** existi 1 c1:x c2:y;
   2   exists y x.P(x,y)     (1)

   ***** existi 1 c1:x c1:x;
   3   exists x x.P(x,c2)     (1)

   ***** existi 1 c1:x c2:x0;
   4   exists x x0.P(x0,x)     (1)

   ***** assume P(c1 c1);
   5   P(c1,c1)     (5)

   ***** existi 5 c1:x 2;
   6   exists x.P(c1,x)     (5)
    
    
   ***** comment ! sorted example !
    
   ***** declare predconst P 2;
   ***** declare indvar x [Sx];
   ***** declare indvar y [Sy];
   ***** declare indconst c1 c2 [S];
   [...]

   ***** moregeneral Sx < S >;

   ***** assume P(c1 c2);
   1   P(c1,c2)     (1)

   ***** existi 1 c1:x c2:y;
   2   Sy(c2) imp exists y x. P(x,y)     (1)
+
