\gfcommand{nameproof}{name the current proof}
\index{nameproof}

\gfsyntax{
   nameproof \ARG{prf-name};
}

\gfdescription{
   If the current proof has no name, it  is named with \ARG{prf-name}.
}

\gfrecap{
If the current proof has no name, it is named with `prf-name'.
}

\gfexample+
   ***** show whereami;
   You are now using an unnamed context.
   You are now using an unnamed proof.
   ***** nameproof P1;
   You have named the current proof: P1
   ***** show whereami;
   You are now using an unnamed context.
   You are now using the proof: P1
   ***** declare sentconst A;
   ***** assume A;
   1   A     (1)
   ***** makeproof P2;
   You have created the empty proof: P2
   ***** switchproof P2;
   You are now using the proof: P2
   ***** declare sentconst B;
   ***** assume A or B;
   1   A or B     (1)
   ***** switchproof P1;
   You are now using the proof: P1
   ***** assume B;
   2   B     (2)
   ***** andi 1 2;
   3   A and B     (1 2)
   ***** show proof;
   1   A     (1)
   2   B     (2)
   3   A and B     (1 2)
   ***** switchproof P2;
   You are now using the proof: P2
   ***** show proof;
   1   A or B     (1)
+