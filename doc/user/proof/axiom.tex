\gfcommand{axiom}{axiom definition}
\index{axiom}

\gfsyntax{
  axiom \ARG{sym} : \ARG{wff};
}

\gfdescription{
  Creates an axiom whose {\em axlabel\/} is {\em sym\/} and whose formula is
  {\em wff}.  The command also allows you to define axiom schemas. An axiom
  schema is an axiom containing one or more {\em sentpars}, {\em
  funpars} or {\em predpars}.  When an axiom schema is used to assert a proof
  line, then it may be instantiated, that is, some or all of {\it sentpar},   
  {\it funpars} and {\it predpars} may be ``substituted'' with complex
  wffs and terms\footnote{
    A formal definition of the substitution of
    parameters can be found in \cite{kleene1}.
  }. In particular:
  %
  \begin{itemize}
   \item 
     a {\em sentpar} $S$ can be instantiated in an axiom schema $\phi$,
     with a wff $\psi$. The result is an axiom $\phi^*$ obtained simply
     by replacing every occurrence of $S$ in $\phi$ by $\psi$;
   \item 
     a {\em funpar} $f$ of arity $n$, can be instantiated in $\phi$
     with a term $t(x_1,\ldots,x_n)$. The result of the instantiation is an
     axiom $\phi^*$ in which every occurrence in $\phi$ of the form
     $f(t_1,\ldots,t_n)$ is replaced with $t(t_1,\ldots,t_n)$;
   \item 
     a {\em predpar} $P$ of arity $n$, can be instantiated in $\phi$ with a
     wff $\psi(x_1,\ldots,x_n)$. The result of the instantiation is an
     axiom $\phi^*$ in which every occurrence in $\phi$ of the form
     $P(t_1,\ldots,t_n)$ has been replaced by $\psi(t_1,\ldots,t_n)$.
  \end{itemize} 
  %
  Let us notice that:
  %
  \begin{enumerate}
    \item 
      If an occurrence of $f(t_1,\ldots,t_n)$ in $\phi$ is within the scope
      of a quantifier for a variable $x$, if $x$ is distinct form $x_1,\ldots x_n$,
      and $x$ occurs in $t(x_1,\ldots x_n)$, then {\GF} performs a renaming
      of the variable $x$ in $\phi$, with a new variable of the same sort of
      $x$. {\em predpar} instantiation is treated in the analogous way.
    \item 
      In the {\em predpars} ($S$) and {\em sentpar} ($P$) instantiations, if a 
      variable $x$ distinct from $x_1,\ldots,x_n$ is quantified in $\psi$ and occurs
      in $P(t_1,\ldots,t_n)$ or $S$, then the substitution fails. In this case an
      error message is returned.
  \end{enumerate}
  %
  The syntax to instantiate an axiom schema whose {\em axlabel} is {\em sym} is:
  %
  \begin{quote}
    {\em sym} {\em subst$_1$} {\em subst$_2$} $\ldots$
  \end{quote}
  %
  where {\em subst} is a  pair composed of a parameter and its instantiation. The
  syntax of {\em subst} is:
  %
  \begin{quote}\tt
    \ARG{sym} sentpar : \ARG{wff} \\
    \ARG{sym} funpar  : lambda \ARG{indvar$_1$} 
                        [ \ARG{indvar$_2$} \SEQ] .
                       \ARG{term} \\
    \ARG{sym} predpar : lambda \ARG{indvar$_1$} 
                        [ \ARG{indvar$_2$} \SEQ ] .
                       \ARG{wff} \\
  \end{quote}
}

\gfrecap{
Declares an axiom in the current context labeled by `sym'.
}

\gfexample#
   ***** declare sentconst A B;
   ***** declare sentpar alpha beta;
   ...
   ***** axiom axA: A;
   axA : A
   ***** impi axA axA; 
   1   A imp A     
   ***** axiom Hil1: alpha imp (beta imp alpha);
   Hil1 : alpha imp (beta imp alpha)
   ***** impe 1 Hil1 alpha: A imp A, beta:B;
   2   B imp (A imp A)     

   ***** reset;   
   ***** declare indvar x y;
   ***** declare predconst p q 1;
   ***** declare predpar P Q 1;
   ...
   ***** axiom axdistr: forall x.(P(x) and Q(x)) imp
                        (forall x.P(x) and forall x.Q(x));
   axdistr : forall x. (P(x) and Q(x)) imp (forall x. P(x) and forall x. Q(x))
   ***** assume forall x. (p(x) and q(x));
   3   forall x. (p(x) and q(x))     (3)
   ***** impe 3 axdistr P:lambda x.p(x), Q:lambda x.q(x);
   4   forall x. p(x) and forall x. q(x)     (3)
   ***** axiom renaming: forall x. P(x) imp forall y. P(y);
   renaming : forall x. P(x) imp forall y. P(y)
   ***** ande 4 2;
   5   forall x. q(x)     
   ***** impe 5  renaming P:lambda y.q(y);
   6   forall y. q(y)     (3)
   
   ***** reset;
   ***** declare indvar n m;
   ***** declare predpar P 1;
   ***** declare indvar x;
   ***** declare funconst s 1;
   ***** declare funconst + 2 [INF];
   ***** know natnums;
   ...
   ***** axiom add1:forall n. n+0 = n;
   ***** axiom add2: forall m n. n+s(m) = s(n+m);
   ***** axiom ind:  P(0) and forall n.(P(n) imp P(s(n))) imp forall n.P(n);
   ***** eval 0+0=0+0;
   1   (0 + 0) = (0 + 0)     
   ***** assume x+0=0+x;
   2   (x + 0) = (0 + x)     (2)
   ***** alle add1 x;
   3   (x + 0) = x     
   ***** subst 2 3;
   4   x = (0 + x)     (2)
   ***** eval s(x)=s(x);
   5   s(x) = s(x)     
   ***** subst 5 4 occ 2;
   6   s(x) = s(0 + x)     (2)
   ***** alle add1 s(x);
   7   (s(x) + 0) = s(x)     
   ***** subst 7 6 occ 2;
   8   (s(x) + 0) = s(0 + x)     (2)
   ***** alle add2 x 0;
   9   (0 + s(x)) = s(0 + x)     
   ***** subst 8 9 right;
   10   (s(x) + 0) = (0 + s(x))     (2)
   ***** impi 2 10;
   11   ((x + 0) = (0 + x)) imp ((s(x) + 0) = (0 + s(x)))     
   ***** alli 11 x:n;
   12   forall n. (((n + 0) = (0 + n)) imp ((s(n) + 0) = (0 + s(n))))
   ***** andi 1 12;
   13   ((0 + 0) = (0 + 0)) and forall n. (((n + 0) = (0 + n)) imp 
        ((s(n) + 0) = (0 + s(n))))     
   ***** impe 13 ind P:lambda x. x+0=0+x;
   14   forall n. ((n + 0) = (0 + n))     
   
   ***** reset;
   ***** declare sentconst A B;
   ***** declare sentpar P Q;
   ***** axiom piffqu: P iff Q;
   piffqu : P iff Q
   ***** impi TRUE piffqu P:Q Q:A;
   1   TRUE imp (A iff A)     
   ***** impi TRUE piffqu Q:A P:Q;
   2   TRUE imp (Q iff A)     
   ***** impi TRUE piffqu Q:P P:Q;   
   3   TRUE imp (Q iff Q)     
   
   ***** reset;   
   ***** declare indvar x y;
   ***** declare funconst f 1;
   ***** declare predpar P 1;
   ***** declare predconst p 1;
   ***** declare predconst q 2;
   ***** axiom Ax1: forall x.(P(x) iff P(f(x)));
   Ax1 : forall x. (P(x) iff P(f(x)))
   ***** impi TRUE Ax1 P:lambda y. q(x y);
   1   TRUE imp forall x1. (q(x,x1) iff q(x,f(x1)))     
   ***** show sym x1;
   x1 is declared to be an Indvar of sort UNIVERSAL.
   
   ***** reset;
   ***** declare sentpar alpha;
   ***** declare indvar x;
   ***** declare predconst p q 1;
   ***** declare predpar P 1;
   ***** axiom Hil5: forall x.(alpha imp P(x)) imp (alpha imp forall x.P(x));
   Hil5 : forall x. (alpha imp P(x)) imp (alpha imp forall x. P(x))
   ***** assume P(x);
   1  P(x)     (1)
   ***** impe 1 Hil5  alpha: p(x) P:lambda x. q(x);
   alpha occurs within the scope of a quantifier binding x
#
   
\gfnotes{
   The {\tt predpar} and {\tt funpar} can be assigned any {\em wff} and {\em term} 
   whose number of (lambda) variables exceeds its arity.
   If the number is lower than the arity then we have an error.
   The extra lambda variables act as universally generalized 
   parameters of the entire axiom. The other variables get converted by  
   substitution.\\
   Axioms cannot be canceled or renamed. It is impossible to 
   define two axioms with the same {\em axlabel}. Numbers cannot be 
   {\em axlabel}s.
}
