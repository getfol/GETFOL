\newcommand{\eg}{{\em e.g.~}}
\newcommand{\ie}{{\em i.e.~}}
\newcommand{\wrt}{w.r.t.~}
\newcommand{\co}[2]{\langle #1, \: #2 \rangle}


\section{Decision procedures}
\label{sec-decide}
\label{system:sec}
A detailed description of the main decision procedures of {\tt GETFOL}
is given in \cite{armando5}.

The set of procedures of the {\tt GETFOL} system is depicted in figure
\ref{system:fig}.
Each box represents either a decider ({\tt PTAUT}, {\tt PTAUTEQ},
{\tt FOLTAUT}) or a rewriting procedure ({\tt tautren}, {\tt  phexp},
{\tt  reduce}).

\begin{figure}
\begin{center}
\makebox[3.375in][l]{
  \vbox to 2.750in{
    \vfill
    \special{psfile=decide/NEWFIG.PS}
  }
  \vspace{-\baselineskip}
}
\end{center}
\caption{The system of deciders}
\label{system:fig}
\end{figure}

\subsubsection*{{\tt PTAUT} and {\tt PTAUTEQ}}
{\tt PTAUT} and {\tt PTAUTEQ}
are deciders working on a quantifier-free first order language (hereon by
abuse of language we call them propositional deciders).
{\tt PTAUT} decides the set of first order formulas provable using
only the propositional deductive machinery (moreover it returns a
falsifying assignment whenever the input formula is not a tautology).
For instance, the formula $(P(x)\con R(x,b))\imp (P(x)\dis R(x,b))$ can be
easily inferred by a single application of {\tt PTAUT}.
{\tt PTAUT} is a generalization of the Davis-Putnam-Loveland procedure
(hereon DPL) \cite{davis2,davis6} to non clausal formulas.
The core of {\tt PTAUT} is a procedure capable of partially evaluating
the input formula \wrt a partial assignment of truth-values to the atomic
subformulas. 
A step of statistical analysis (of polynomial time complexity) collects
information about the {\em polarity} of the subformulas and the existence
of {\em Top-Level Disjunctive Occurrences} (TLDO) of atomic subformulas.
A formula $\alpha$ occurs as a TLDO in $\beta$ 
if and only if $\beta$ can be rewritten into a formula either of the form
$(\alpha\dis\gamma)$ or $(\neg\alpha\dis\gamma)$ by means of rules
expressing  the usual properties of the propositional connectives such as
the associativity, commutativity and distributivity of the propositional
connectives.
The notion of positive (negative) subformula occurrence
is inductively defined as follows: $\alpha$ occurs positively in $\alpha$, 
$\alpha$ occurs negatively in $\neg\alpha$;
$\alpha$ and $\beta$ occur positively in $(\alpha\con\beta)$ and
$(\alpha\dis\beta)$;
$\alpha$ occurs negatively and $\beta$ occurs positively in $(\alpha\imp\beta)$;
finally $\alpha$ and $\beta$ occur both positively and negatively in
$(\alpha\liff\beta)$.
A subformula $\alpha$ is positive (negative) in $\beta$ if and only if
each occurrence of $\alpha$ occurs positively (negatively) in $\beta$.

The statistical analysis may suggest a partial assignment $\mu$ (\wrt which
the formula can be simplified) according to the following criteria:
\begin{itemize}
\item for each positive (negative) atomic formula $\alpha$ occurring
in $\beta$, $\mu(\alpha)=F$ ($\mu(\alpha)=T$);
\item if $\beta$ contains a positive (negative) TLDO of $\alpha$ and there
are no negative (positive) TLDO of $\alpha$, then $\mu(\alpha)=F$
($\mu(\alpha)=T$).
\end{itemize}

If $\mu$ is not completely undefined, then {\tt PTAUT} simplifies the
formula in input \wrt $\mu$ and recurs on the resulting (simplified) formula.
If the input formula contains both a positive and a negative TLDO of an
atomic formula the input formula is a tautology.
These optimizations generalize the {\em Affirmative-Negative Rule} and the
{\em Rule for the Elimination of One-Literal Clauses} of DPL.
If $\mu$ is totally undefined, then
an atomic formula is chosen, two partial assignments are created
(one assigning $T$, the other $F$ to the chosen atomic formula),
the formula is partially evaluated \wrt such assignments and finally
the procedure branches recurring on the two simplified formulas.
This last step generalizes the {\em Splitting Rule} of DPL.

{\tt PTAUTEQ} is the result of adapting {\tt PTAUT}
to take into account the properties of equality.
The main difference is that, before a formula is simplified \wrt some
assignment, the assignment is tested to check whether it is model of the
quantifier-free theory of equality.
The formula $(P(x)\con x=y)\imp (P(y)\con y=x)$ can be
easily inferred by a single application of {\tt PTAUTEQ}.

\subsubsection*{{\tt nnf} and {\tt skolemize}}
{\tt nnf} rewrites the input formula into a logically equivalent one in
{\em negative normal form}.

{\tt skolemize} computes the skolemization of the input formula.

\subsubsection*{{\tt tautren} and {\tt phexp}}
The procedures on top of the propositional deciders (namely {\tt tautren}
and {\tt phexp}) map the first-order formula in input into a quantifier-free
formula.
The mappings are such that the decision problem of the input (first-order)
formula is related to the decision problem of the returned (quantifier-free)
formula in a useful way.
In particular, {\tt tautren} atomizes equal (modulo renaming of bound
variables) quantified subformulas into newly introduced propositional
letters.
For instance the formula
\begin{equation}\label{pb29-reduced}
%\setlength{\templength}{\arraycolsep}
\setlength{\arraycolsep}{0cm}
\begin{array}{rl}
(\exists x.F(x) \con \exists x.G(x)) \imp (&( \forall x.(F(x) \imp H(x)) \con \forall x.(G(x) \imp J(x))) \liff \\
& ((\exists y.G(y) \imp \forall x.(F(x) \imp H(x))) \con\\
&\ (\exists x.F(x) \imp \forall y.(G(y) \imp J(y)))))
\end{array}
\end{equation}
is mapped into the propositional formula
\begin{equation}\label{pb29-prop}
(A \con B) \imp ((C \con D) \liff ((B \imp C) \con (A \imp D)))
\end{equation}
The relation between the decision problems of the input formula (say $\alpha$)
and of the output formula (say $\alpha'$) is that
$\der{}\alpha'$ only if $\der{}\alpha$.

A more careful reduction to the quantifier-free fragment is performed by
{\tt phexp}.
{\tt phexp} maps an existential formula $\alpha$ into a quantifier-free formula
$\alpha'$ such that $\der{}{\alpha'}$ if and only if $\der{}{\alpha}$.%
\footnote{The set of existential formualas is the class of prenex
universal-existential formulas without function symbols.}

The formula $\alpha'$ is an improved version of the Herbrand's expansion
of $\alpha$ \cite{dreben1}.
An application of {\tt phexp} to the following formula:
\begin{equation}\label{pb28}\small
    (((P(x) \con \neg Q(y)) \dis
     ((Q(a) \dis R(a)) \con (\neg Q(b) \dis \neg S(b)))) \dis
     ((F(z) \con \neg G(z)) \con S(v))) \dis
       ((\neg P(c) \dis \neg F(c)) \dis G(c))
\end{equation}
yields
\begin{equation}\label{pb28-exp}\small
\begin{array}{l}
((((P(a) \dis P(b) \dis P(c)) \con (\neg Q(a) \dis \neg Q(b) \dis
\neg Q(c)))\dis\\
((Q(a) \dis R(a)) \con (\neg Q(b) \dis \neg S(b)))) \dis \\
(((F(a) \con \neg G(a)) \dis (F(b) \con \neg G(b)) \dis
(F(c) \con \neg G(c)))\con\\
(S(a) \dis S(b) \dis S(c)))) \dis ((\neg P(c) \dis \neg F(c)) \dis G(c))\\
\end{array}
\end{equation}
In \cite{armando5} it is shown that, the size of (\ref{pb28-exp})
is 44 times smaller than the size of the standard Herbrand's expansion of
(\ref{pb28}).

\subsubsection*{{\tt reduce}}
{\tt reduce} tries a set of rewriting rules on the input
formula aiming at rewriting it into a logically equivalent formula that
can be easily turned into an existential one via skolemization.
The rewriting rules employed by {\tt reduce} are the usual rules
expressing the distributivity of quantifiers through propositional connectives
and the commutativity and associativity of propositional connectives
listed in the following table.\\

    \renewcommand{\arraystretch}{1.5}
    {\small
      $$
      \begin{array}{|c|rcl|} \hline
        (1) & Q x. \alpha[x] & \mapsto & \alpha \\ \hline
        %(2) & Q x. (\neg \alpha(x)) & \mapsto & (\neg \hat{Q} x. \alpha(x)) \\ \hline
        (2) & Q x. (\alpha \circ \beta)(x) & \mapsto & (Q x. \alpha \circ Q x. \beta) 
        \\ \hline
        (3) & Q x. (\alpha[x] + \beta(x)) & \mapsto & (\alpha[x] + Q x. \beta(x)) 
        \\ \hline \hline
        (4) & (\alpha(x) + \beta[x]) & \mapsto & (\beta[x] + \alpha(x)) \\ \hline
        (5) & ((\alpha[x] + \beta(x)) + \gamma(x)) & \mapsto & 
        (\alpha[x] + (\beta(x) + \gamma(x))) \\ \hline
        (6) & ((\alpha \circ \beta)(x) + \gamma(x)) & \mapsto & 
        ((\alpha + \gamma(x)) \circ (\beta + \gamma(x))) \\ \hline
        (7) & (\alpha(x) + (\beta[x] + \gamma(x))) & \mapsto &
        (\beta[x] + (\alpha(x) + \gamma(x))) \\ \hline
        (8) & ((\alpha(x) + (\beta \circ \gamma)(x))) & \mapsto &
        ((\alpha(x) + \beta) \circ (\alpha(x) + \gamma)) \\ \hline
      \end{array}
      $$
      }
    \renewcommand{\arraystretch}{1}
{\small
{\em Restrictions}: 
\begin{itemize}
\item In rules $\{(4)-(8)\}$ the left hand side must be a top normalizable
formula.
\item In rules $\{(7),(8)\}$ $\alpha$ must be minimal \wrt $\co{Q}{x}$.
%\item Rules $\{(4)-(8)\}$ can be applied only to subformulae (say $\alpha$)
%of a formula $Qx.\beta$ in which there is no proper
%subformula $Qy.\gamma$ of which $\alpha$ is a subformula.
\end{itemize}}

Where
$\alpha(x)$ denotes a formula in which there is at least one free occurrence
of the variable $x$.
$\alpha{[x]}$ denotes a formula in which there is no free occurrences of $x$.
$Q$ and $Q'$ stand either for $\forall$ or for $\exists$.
If $Q = \forall$, then $\circ = \con$ and $+ = \dis$.
If $Q = \exists$, then $\circ = \dis$ and $+ = \con$.
%$\con$ is said to be $\forall$-compatible and $\exists$-incompatible,
%$\dis$ is said to be $\exists$-compatible and $\forall$-incompatible.
%If $\cal S$ is a set of rewriting rules then $\mapsto_{\cal S}$ is the
%reducibility relation induced by $\cal S$ and
%$\stackrel{*}{\mapsto}_{\cal S}$ is the reflexive and transitive closure
%of $\mapsto_{\cal S}$.
The definition of {\em top normalizable formula} and of {\em minimal
formula} are given in \cite{armando5}.

For instance, a single application of {\tt reduce} turns the formula
\begin{equation}\label{pb29}
%\setlength{\templength}{\arraycolsep}
\setlength{\arraycolsep}{0cm}
\begin{array}{rl}
(\exists x.F(x) \con \exists x.G(x)) \imp (&( \forall x.(F(x) \imp H(x)) \con \forall x.(G(x) \imp J(x))) \liff \\
&(\forall x.\forall y.((F(x) \con G(y)) \imp (H(x) \con J(y)))))
\end{array}
\end{equation}
into (\ref{pb29-reduced}).
{\tt reduce} considerably enlarges the set of formulas which can be solved
by using the system of deciders.
In particular, any prenex first order formula 
$$
\forall \vec{y}_n \exists \vec{x}_n \ldots
\forall \vec{y}_i \exists \vec{x}_i \ldots
\forall \vec{y}_1 \exists \vec{x}_1 . \Phi
$$
such that each literal in $\Phi$ contains no variables in $\vec{y}_k$ and
in $\vec{x}_l$ with $k < l$, or in $\vec{x}_k$ and in $\vec{x}_l$ with
$k \neq l$ can be ``reduced" to an existential formula.
On the basis of the previous result it is a trivial consequence
that the {\em monadic calculus} can be reduced to the existential fragment
by means of {\tt reduce}.
