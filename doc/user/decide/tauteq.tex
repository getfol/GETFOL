\gfcommand{tauteq}{tautological decider with equality}
\index{tauteq}

\gfsyntax{
  tauteq \ARG{wff} \OPT{by \ARG{fact1} \ARG{fact2} \SEQ};
}

\gfdescription{
  The class of formulae decided by {\tt tauteq} is the set of formulas provable
  using:
  \begin{itemize}
  \item the introduction and elimination rules for the sentential connectives;
  \item the {\em congruence-rule};
  \item the following axioms schemata for equality:
  $$
  \begin{array}{l}
    x=x\\
    (x=y\ \imp\ y=x)\\
    ((x=y\ \con\ y=z)\ \imp\ x=z)\\
    ((x_1=y_1\ \con \ldots \con\ x_n=y_n)\ \imp\ 
    (P(x_1, \ldots \ ,x_n)\ \liff\ P(y_1, \ldots \ ,y_n)))
  \end{array}
  $$
  %
  corresponding to {\em reflexivity, symmetry, transitivity} and {\em substitution 
  into predicates}.
  \end{itemize}
}

\gfrecap{
The class of formulae decided by {\tt tauteq} is the set of formulas provable
using:
* the introduction and elimination rules for the sentential connectives;
* the ``congruence-rule'';
* the following axioms schemata for equality:
     +-------------------------------------------------------------------------+
     |   x = x                                                                 |
     |   x = y imp y = x                                                       |
     |   (x = y and y = z) imp x = z                                           |
     |   (x1 =y1 and ... and xN = yN) imp (P(x1, ..., xN) iff P(y1, ..., yN))  |
     +-------------------------------------------------------------------------+
  corresponding to ``reflexivity'', ``symmetry'', ``transitivity'' and
  ``substitution into predicates''.
}

\gfexample+
   ***** declare predconst P 1;
   ***** declare funconst f 1;
   ***** declare indvar x y;
   ***** declare indvar z;
   ***** tauteq x=x;
   1   x=x
   ***** tauteq x=y imp y=x; 
   2   x=y imp y=x
   ***** tauteq ((x=y and y=z) imp x=z);
   3   (x=y and y=z) imp x=z
   ***** tauteq (x=y imp (P(x) or not P(y)));
   4   x=y imp (P(x) or not P(y))
   ***** tauteq (f(x)=f(y) imp (P(f(x)) iff P(f(y))));
   5   f(x)=f(y) imp (P(f(x)) iff P(f(y)))
   ***** tauteq x=y imp f(x)=f(y);
   TAUTEQ couldn't prove that (x = y) imp (f(x) = f(y))
   is a tautology.
+