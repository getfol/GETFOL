\gfcommand{monad}{first order decider for monadic formulas}
\index{monad}

\gfsyntax{
  monad \ARG{wff} \OPT{by \ARG{fact1} \ARG{fact2} \SEQ};
}

\gfdescription{
It tries to establish whether the input formula is deducible from the
specified facts by using {\tt nnf}, {\tt reduce}, {\tt phexp} and finally
{\tt ptaut}.
}

\gfrecap{
It tries to establish whether the input formula is deducible from the
specified facts by using nnf, reduce, phexp and finally ptaut.
}

\gfexample+
   ***** declare predconst P 1;
   ***** declare funconst f 3;
   ***** declare indvar x y z;
   ***** declare indpar a b;
   
   *****comment | *** MONAD EXAMPLES *** |
   ***** monad forall x. exists y. (P(x) imp P(y));
   1   forall x. exists y. (P(x) imp P(y))
   ***** monad exists y. forall x. (P(x) imp P(y));
   2   exists y. forall x. (P(x) imp P(y))
   ***** monad exists y. forall x. ((P(x) imp P(y)) or P(x));
   3   exists y. forall x. ((P(x) imp P(y)) or P(x))
   
   ***** monad forall x. exists y.
    wffif P(trmif P(y) then x else y)
     then P(trmif P(y) 
             then trmif P(y) 
                   then x 
                   else trmif P(y) 
                         then x 
                         else y 
             else y) 
          or TRUE
     else P(y) or TRUE;
   4   forall x. exists y. 
       (wffif P(trmif P(y) then x else y) 
         then (P(trmif P(y) 
                 then 
                 (trmif P(y) 
                   then x 
                   else (trmif P(y) then x else y)) 
               else y) or TRUE) else (P(y) or TRUE))
   ***** monad forall x. exists y. wffif P(x) then P(y) else not P(y);
   5   forall x. exists y. (wffif P(x) then P(y) else (not P(y)))  
   ***** monad forall y. exists x. (P(f(a,b,x)) or not P(f(a,b,y)));
   6   forall y. exists x. (P(f(a,b,x)) or (not P(f(a,b,y))))
   ***** monad exists z. forall y. exists x. (P(f(z,b,x)) or not P(f(x,b,z)));
   7   exists z. forall y. exists x. (P(f(z,b,x)) or (not P(f(x,b,z)))) 
   ***** monad exists z. forall y. exists x. (P(f(z,b,x)) or not P(f(x,b,y)));
   7   exists z. forall y. exists x. (P(f(z,b,x)) or (not P(f(x,b,y)))) 
+
   
\gfnotes{
   The name ``{\tt monad}" is due to the fact that the monadic predicate
calculus is subset of the class of formulas decided by such a decider.
}   
