%............................... USER MANUAL .................................
%.............................................................................

\documentstyle[12pt]{../styfiles/GFmanual}

\title{GETFOL Manual}
\author{\bf Fausto Giunchiglia}
\date{7 March 1994}
\version{2.0}
\abstract{
	  {\GF} is an interactive reasoning system.
	  We use it as an environment for studying epistemological issues.
	  We try to look at questions like: which notions are
	  important for the development of mechanized reasoning systems?
	  What kind of conversations do we want to have with them?
	  What parts of logic should we use to represent such notions?
	  How should logic be embedded in a conversational reasoning system?
	}
\addresses{
     \begin{tabular}[c]{l}
       {\bf Fausto Giunchiglia}           \\
       Mechanized Reasoning Group		  \\
		 IRST, Povo, 38050 Trento, Italy  \\
		 e-mail: {\tt fausto@irst.it}     \\
		 phone: +39 461 314436
	\end{tabular}	
}
\published{
  \begin{tabular}{l}
	  DIST Technical Report No. 92-0010 (1994). \\
	  DIST -- University of Genoa,\\
	  Via Opera Pia 11A, 16145 Genova, Italy.\\ \\
  \end{tabular}
}

%% \newcommand{\gfbibliography}{%
%% \bibliography{/home/tarski/staff/mrg/biblio/bib/a-l,%
%% /home/tarski/staff/mrg/biblio/bib/m-z,userman}}
\newcommand{\gfbibliography}{\bibliography{}}

%% \makeindex
\begin{document}
	%  ............................. COVER ..................................
	\thispagestyle{empty}
	\maketitle

	%  ........................ TABLE OF CONTENTS ...........................
	\newpage
	\pagenumbering{roman}
	\tableofcontents

	\newpage
	\pagenumbering{arabic}
	\pagestyle{headings}

	%  ........................... INTRODUCTION ............................
	\newcommand{\eg}{{\em e.g.~}}
\newcommand{\ie}{{\em i.e.~}}
\newcommand{\wrt}{w.r.t.~}
\newcommand{\co}[2]{\langle #1, \: #2 \rangle}


\section{Decision procedures}
\label{sec-decide}
\label{system:sec}
A detailed description of the main decision procedures of {\tt GETFOL}
is given in \cite{armando5}.

The set of procedures of the {\tt GETFOL} system is depicted in figure
\ref{system:fig}.
Each box represents either a decider ({\tt PTAUT}, {\tt PTAUTEQ},
{\tt FOLTAUT}) or a rewriting procedure ({\tt tautren}, {\tt  phexp},
{\tt  reduce}).

\begin{figure}
\begin{center}
\makebox[3.375in][l]{
  \vbox to 2.750in{
    \vfill
    \special{psfile=decide/NEWFIG.PS}
  }
  \vspace{-\baselineskip}
}
\end{center}
\caption{The system of deciders}
\label{system:fig}
\end{figure}

\subsubsection*{{\tt PTAUT} and {\tt PTAUTEQ}}
{\tt PTAUT} and {\tt PTAUTEQ}
are deciders working on a quantifier-free first order language (hereon by
abuse of language we call them propositional deciders).
{\tt PTAUT} decides the set of first order formulas provable using
only the propositional deductive machinery (moreover it returns a
falsifying assignment whenever the input formula is not a tautology).
For instance, the formula $(P(x)\con R(x,b))\imp (P(x)\dis R(x,b))$ can be
easily inferred by a single application of {\tt PTAUT}.
{\tt PTAUT} is a generalization of the Davis-Putnam-Loveland procedure
(hereon DPL) \cite{davis2,davis6} to non clausal formulas.
The core of {\tt PTAUT} is a procedure capable of partially evaluating
the input formula \wrt a partial assignment of truth-values to the atomic
subformulas. 
A step of statistical analysis (of polynomial time complexity) collects
information about the {\em polarity} of the subformulas and the existence
of {\em Top-Level Disjunctive Occurrences} (TLDO) of atomic subformulas.
A formula $\alpha$ occurs as a TLDO in $\beta$ 
if and only if $\beta$ can be rewritten into a formula either of the form
$(\alpha\dis\gamma)$ or $(\neg\alpha\dis\gamma)$ by means of rules
expressing  the usual properties of the propositional connectives such as
the associativity, commutativity and distributivity of the propositional
connectives.
The notion of positive (negative) subformula occurrence
is inductively defined as follows: $\alpha$ occurs positively in $\alpha$, 
$\alpha$ occurs negatively in $\neg\alpha$;
$\alpha$ and $\beta$ occur positively in $(\alpha\con\beta)$ and
$(\alpha\dis\beta)$;
$\alpha$ occurs negatively and $\beta$ occurs positively in $(\alpha\imp\beta)$;
finally $\alpha$ and $\beta$ occur both positively and negatively in
$(\alpha\liff\beta)$.
A subformula $\alpha$ is positive (negative) in $\beta$ if and only if
each occurrence of $\alpha$ occurs positively (negatively) in $\beta$.

The statistical analysis may suggest a partial assignment $\mu$ (\wrt which
the formula can be simplified) according to the following criteria:
\begin{itemize}
\item for each positive (negative) atomic formula $\alpha$ occurring
in $\beta$, $\mu(\alpha)=F$ ($\mu(\alpha)=T$);
\item if $\beta$ contains a positive (negative) TLDO of $\alpha$ and there
are no negative (positive) TLDO of $\alpha$, then $\mu(\alpha)=F$
($\mu(\alpha)=T$).
\end{itemize}

If $\mu$ is not completely undefined, then {\tt PTAUT} simplifies the
formula in input \wrt $\mu$ and recurs on the resulting (simplified) formula.
If the input formula contains both a positive and a negative TLDO of an
atomic formula the input formula is a tautology.
These optimizations generalize the {\em Affirmative-Negative Rule} and the
{\em Rule for the Elimination of One-Literal Clauses} of DPL.
If $\mu$ is totally undefined, then
an atomic formula is chosen, two partial assignments are created
(one assigning $T$, the other $F$ to the chosen atomic formula),
the formula is partially evaluated \wrt such assignments and finally
the procedure branches recurring on the two simplified formulas.
This last step generalizes the {\em Splitting Rule} of DPL.

{\tt PTAUTEQ} is the result of adapting {\tt PTAUT}
to take into account the properties of equality.
The main difference is that, before a formula is simplified \wrt some
assignment, the assignment is tested to check whether it is model of the
quantifier-free theory of equality.
The formula $(P(x)\con x=y)\imp (P(y)\con y=x)$ can be
easily inferred by a single application of {\tt PTAUTEQ}.

\subsubsection*{{\tt nnf} and {\tt skolemize}}
{\tt nnf} rewrites the input formula into a logically equivalent one in
{\em negative normal form}.

{\tt skolemize} computes the skolemization of the input formula.

\subsubsection*{{\tt tautren} and {\tt phexp}}
The procedures on top of the propositional deciders (namely {\tt tautren}
and {\tt phexp}) map the first-order formula in input into a quantifier-free
formula.
The mappings are such that the decision problem of the input (first-order)
formula is related to the decision problem of the returned (quantifier-free)
formula in a useful way.
In particular, {\tt tautren} atomizes equal (modulo renaming of bound
variables) quantified subformulas into newly introduced propositional
letters.
For instance the formula
\begin{equation}\label{pb29-reduced}
%\setlength{\templength}{\arraycolsep}
\setlength{\arraycolsep}{0cm}
\begin{array}{rl}
(\exists x.F(x) \con \exists x.G(x)) \imp (&( \forall x.(F(x) \imp H(x)) \con \forall x.(G(x) \imp J(x))) \liff \\
& ((\exists y.G(y) \imp \forall x.(F(x) \imp H(x))) \con\\
&\ (\exists x.F(x) \imp \forall y.(G(y) \imp J(y)))))
\end{array}
\end{equation}
is mapped into the propositional formula
\begin{equation}\label{pb29-prop}
(A \con B) \imp ((C \con D) \liff ((B \imp C) \con (A \imp D)))
\end{equation}
The relation between the decision problems of the input formula (say $\alpha$)
and of the output formula (say $\alpha'$) is that
$\der{}\alpha'$ only if $\der{}\alpha$.

A more careful reduction to the quantifier-free fragment is performed by
{\tt phexp}.
{\tt phexp} maps an existential formula $\alpha$ into a quantifier-free formula
$\alpha'$ such that $\der{}{\alpha'}$ if and only if $\der{}{\alpha}$.%
\footnote{The set of existential formualas is the class of prenex
universal-existential formulas without function symbols.}

The formula $\alpha'$ is an improved version of the Herbrand's expansion
of $\alpha$ \cite{dreben1}.
An application of {\tt phexp} to the following formula:
\begin{equation}\label{pb28}\small
    (((P(x) \con \neg Q(y)) \dis
     ((Q(a) \dis R(a)) \con (\neg Q(b) \dis \neg S(b)))) \dis
     ((F(z) \con \neg G(z)) \con S(v))) \dis
       ((\neg P(c) \dis \neg F(c)) \dis G(c))
\end{equation}
yields
\begin{equation}\label{pb28-exp}\small
\begin{array}{l}
((((P(a) \dis P(b) \dis P(c)) \con (\neg Q(a) \dis \neg Q(b) \dis
\neg Q(c)))\dis\\
((Q(a) \dis R(a)) \con (\neg Q(b) \dis \neg S(b)))) \dis \\
(((F(a) \con \neg G(a)) \dis (F(b) \con \neg G(b)) \dis
(F(c) \con \neg G(c)))\con\\
(S(a) \dis S(b) \dis S(c)))) \dis ((\neg P(c) \dis \neg F(c)) \dis G(c))\\
\end{array}
\end{equation}
In \cite{armando5} it is shown that, the size of (\ref{pb28-exp})
is 44 times smaller than the size of the standard Herbrand's expansion of
(\ref{pb28}).

\subsubsection*{{\tt reduce}}
{\tt reduce} tries a set of rewriting rules on the input
formula aiming at rewriting it into a logically equivalent formula that
can be easily turned into an existential one via skolemization.
The rewriting rules employed by {\tt reduce} are the usual rules
expressing the distributivity of quantifiers through propositional connectives
and the commutativity and associativity of propositional connectives
listed in the following table.\\

    \renewcommand{\arraystretch}{1.5}
    {\small
      $$
      \begin{array}{|c|rcl|} \hline
        (1) & Q x. \alpha[x] & \mapsto & \alpha \\ \hline
        %(2) & Q x. (\neg \alpha(x)) & \mapsto & (\neg \hat{Q} x. \alpha(x)) \\ \hline
        (2) & Q x. (\alpha \circ \beta)(x) & \mapsto & (Q x. \alpha \circ Q x. \beta) 
        \\ \hline
        (3) & Q x. (\alpha[x] + \beta(x)) & \mapsto & (\alpha[x] + Q x. \beta(x)) 
        \\ \hline \hline
        (4) & (\alpha(x) + \beta[x]) & \mapsto & (\beta[x] + \alpha(x)) \\ \hline
        (5) & ((\alpha[x] + \beta(x)) + \gamma(x)) & \mapsto & 
        (\alpha[x] + (\beta(x) + \gamma(x))) \\ \hline
        (6) & ((\alpha \circ \beta)(x) + \gamma(x)) & \mapsto & 
        ((\alpha + \gamma(x)) \circ (\beta + \gamma(x))) \\ \hline
        (7) & (\alpha(x) + (\beta[x] + \gamma(x))) & \mapsto &
        (\beta[x] + (\alpha(x) + \gamma(x))) \\ \hline
        (8) & ((\alpha(x) + (\beta \circ \gamma)(x))) & \mapsto &
        ((\alpha(x) + \beta) \circ (\alpha(x) + \gamma)) \\ \hline
      \end{array}
      $$
      }
    \renewcommand{\arraystretch}{1}
{\small
{\em Restrictions}: 
\begin{itemize}
\item In rules $\{(4)-(8)\}$ the left hand side must be a top normalizable
formula.
\item In rules $\{(7),(8)\}$ $\alpha$ must be minimal \wrt $\co{Q}{x}$.
%\item Rules $\{(4)-(8)\}$ can be applied only to subformulae (say $\alpha$)
%of a formula $Qx.\beta$ in which there is no proper
%subformula $Qy.\gamma$ of which $\alpha$ is a subformula.
\end{itemize}}

Where
$\alpha(x)$ denotes a formula in which there is at least one free occurrence
of the variable $x$.
$\alpha{[x]}$ denotes a formula in which there is no free occurrences of $x$.
$Q$ and $Q'$ stand either for $\forall$ or for $\exists$.
If $Q = \forall$, then $\circ = \con$ and $+ = \dis$.
If $Q = \exists$, then $\circ = \dis$ and $+ = \con$.
%$\con$ is said to be $\forall$-compatible and $\exists$-incompatible,
%$\dis$ is said to be $\exists$-compatible and $\forall$-incompatible.
%If $\cal S$ is a set of rewriting rules then $\mapsto_{\cal S}$ is the
%reducibility relation induced by $\cal S$ and
%$\stackrel{*}{\mapsto}_{\cal S}$ is the reflexive and transitive closure
%of $\mapsto_{\cal S}$.
The definition of {\em top normalizable formula} and of {\em minimal
formula} are given in \cite{armando5}.

For instance, a single application of {\tt reduce} turns the formula
\begin{equation}\label{pb29}
%\setlength{\templength}{\arraycolsep}
\setlength{\arraycolsep}{0cm}
\begin{array}{rl}
(\exists x.F(x) \con \exists x.G(x)) \imp (&( \forall x.(F(x) \imp H(x)) \con \forall x.(G(x) \imp J(x))) \liff \\
&(\forall x.\forall y.((F(x) \con G(y)) \imp (H(x) \con J(y)))))
\end{array}
\end{equation}
into (\ref{pb29-reduced}).
{\tt reduce} considerably enlarges the set of formulas which can be solved
by using the system of deciders.
In particular, any prenex first order formula 
$$
\forall \vec{y}_n \exists \vec{x}_n \ldots
\forall \vec{y}_i \exists \vec{x}_i \ldots
\forall \vec{y}_1 \exists \vec{x}_1 . \Phi
$$
such that each literal in $\Phi$ contains no variables in $\vec{y}_k$ and
in $\vec{x}_l$ with $k < l$, or in $\vec{x}_k$ and in $\vec{x}_l$ with
$k \neq l$ can be ``reduced" to an existential formula.
On the basis of the previous result it is a trivial consequence
that the {\em monadic calculus} can be reduced to the existential fragment
by means of {\tt reduce}.


	%  .............................. MODULES ..............................
	% loading introduction to the section
\section{Parser}

This section is intended to explain:
%
\begin{itemize}
	\item
		the main functionalities of the {\GF} scanning primitives,
	\item
		what modifications must be performed in the scanner data
		structures in order to be able to change its behavior ({\em e.g.} to
		define a new escape character).
\end{itemize}

The {\GF} scanner is a {\em backupable} scanner.
This feature is necessary as the parser works in a top-down fashion and,
sometimes, needs to backtrack. 

The {\GF} scanner is able to recognize three types of tokens:
%
\begin{enumerate}
	\item {\tt IDTOKEN},
	\item {\tt NUMTOKEN} and
	\item {\tt DELTOKEN}.
\end{enumerate}

The primitives to scan such tokens are \verb+FOLSYM@+, \verb+NATNUM@+ (actually
no dedicated primitive to scan a \verb+DELTOKEN+ exists).

\verb+TK@+ scans a generic token.

The primitives \verb+SCANSTATUS-GET+ and \verb+SCANSTATUS-RESTORE+ allow
respectively to save and restore the scanner status.
They are necessary to restore the status of the scanner in case of
unsuccessful parsing.

The high level functions \verb+FOLSYM@+, \verb+NATNUM@+, \verb+TK@+ (and other
not mentioned here for brevity) use the general \verb+TOKEN-GET-NEXT+ routine
which is also able to {\em bufferize} the tokens already read (reading from
an input stream is a destructive operation so we need at a some level a
buffer mechanism if we want to perform backtracking).

The buffer is implemented by the array \verb+TOKENARRAY+ whose dimension is
given by the macro \verb+TOKENARRAY-DIMENSION+. The max number of tokens
that can be present in a command line is given by the number returned
by this macro.

\verb+TOKEN-GET-NEXT+ routine is based on the lower level primitive
\verb+TOKEN-SCAN+, which reads the next token from the input stream and returns
its type.
\verb+TOKEN-SCAN+ is able to identify the type of a token reading (via 
\verb+CH-GET-NEXT+) its first character and identifying its type (via
\verb+CHTYPE-GET+).

The characters are divided in the following types:

\begin{enumerate}
	\item
		identifiers ({\tt IDCHAR}): see file
		\verb+ascitab.fol+;
	\item
		numbers (positive integers --- {\tt NUMCHAR}):
		\verb+0 1 2 3 4 5 6 7 8 9+;
	\item
		delimiter {\tt DELCHAR}:
		\verb+( ) , . : ; [ ] { }+;
	\item
		ignored char {\tt IGNCHAR}: see file
		\verb+ascitab.fol+;
	\item
		iddelim {\tt IDDELCHAR}:
        \verb$' * + - / < > = ? @ ^ ` |$;
	\item
		escape characters {\tt ESCCHAR}:
		\verb+\+;
	\item
		special handling {\tt SPECCHAR}
\end{enumerate}

The \verb+IDDELCHAR+s are identifier characters having also the functionality
of a delimiter, that is no \verb+IDTOKEN+ can contain a character of such
type but they by themselves may be considered \verb+IDTOKEN+.
For instance the string \verb+A@B+ will be regarded by the scanner a string
formed by the three distinct \verb+IDTOKEN+ \verb+A @ B+.
The escape characters allows to coerce the type of the following
character to be \verb+IDCHAR+.
For istance the string \verb+A\@B+ will be regarded by the scanner a string
formed by the \verb+IDTOKEN A@B+.

The only think important to know at the user level it is how to modify the type
of a character so that you can, for istance, extend the set of \verb+IDDELCHAR+
with the character "\verb+_+". To do this you have only to change type
declaration for the "\verb+_+" character from \verb+IDCHAR+ to \verb+IDDELCHAR+
in the file {\tt asciitab.fol}.

Finally a low level feature: each time a command line is issued to {\GF} the low
level scanning primitives store each read character in the {\tt TUPLE
SCANBUFARRAY}, so that if a syntactic error in the parsing is detected the rest
of the line will be read and the whole line printed out to show, using an
"\verb+^+", where the syntatic error has been detected. This useful feature
imposes (as it is implemented) a limit in the length of a {\GF} command line as
the dimension of \verb+SCANBUFARRAY+ is fixed by the value returned by the macro
\verb+SCANBUFARRAY-DIMENSION+.


% loading command files
\gfcommand{backup}{backup of a {\GF} session}
\index{backup}

\gfsyntax{
  backup \ARG{file} open;\\
  backup \ARG{file} close;  
}

\gfdescription{
   {\GF} stores in \ARG{file} all the successful commands between the command
   ``{\tt backup \ARG{file} open}" and the command ``{\tt backup \ARG{file}
   close}". 
}

\gfrecap{
Stores in a file all successfull commands type in GETFOL.
}

\gfexample+
   ***** backup file open;
   I am starting to backup onto file
   
   ***** declare sentconst A;
   ***** assume A;
   1   A     (1)
   
   ***** backup file close;
   
   ***** andi 1 1;
   2   A and A     (1)
   
   ***** ^D
   
   >Bye.
   
   <host-prompt> more file
   
   declare sentconst A;
   
   assume A;
   
   <host-prompt>
+

\gfnotes{
   Only Unix file names are supported.
   The file path name can be absolute or relative to the current directory.
   Multiple open backup files can exist simultaneously.
   Files must be explicitly closed to guarantee that all the commands
   are properly stored in the backup file.
}

\gfcommand{done}{exiting a {\GF} session}
\index{done}

\gfsyntax{done;}

\gfdescription{
   Returns the control back to the {\HG} environment~\cite{giunchiglia35}.
   You can get back to {\GF} by typing {\tt (SYSBACK)} at the {\HG} prompt.
}

\gfrecap{
Returns the control back to the HGKM environment.
You can get back to GETFOL by typing (SYSBACK) at the HGKM prompt.
}

\gfexample+
   **** done;
   Returning to host
   NIL
+

\gfnotes{}

\gfcommand{fetch}{fetches {\GF} file}
\index{fetch}

\gfsyntax{
   fetch \ARG{file} \OPT{from \ARG{mark1}} \OPT{to \ARG{mark2}};
}

\gfdescription{
   Redirects the standard input to the file \ARG{file}; all {\GF} commands
   between \ARG{mark1} and \ARG{mark2} are executed.
}

\gfrecap{
Fetches commands from the file `file'.
All commands between `mark1' and `mark2' are executed.
}

\gfexample+
  ***** fetch exmarks.tst;
  ...
  ***** fetch exmarks.tst to m1;
  ...
  ***** fetch exmarks.tst from m1 to m2;
  ...
  ***** fetch exmarks.tst from m2;
+

\gfnotes{
   Only Unix file names are supported.
   The file path name can be absolute or relative to the current directory.
   Nested fetches (and marking) are allowed.
   If no marks are specified, then the whole file is fetched.
   See the command {\tt mark} in this section to set marks in a file.
}


\gfcommand{mark}{sets a mark}
\index{mark}

\gfsyntax{
   mark \ARG{sym};
}

\gfdescription{
   Sets a mark in between a sequence of {\GF} commands.
   It can be used to fetch a file from/to a certain mark (see command 
   {\tt fetch} in this section).
}

\gfrecap{
Sets a mark in a file (see fetch).
}

\gfexample+
   <host-prompt> more example
   NAMECONTEXT META;
   DECLARE SORT FACT WFF;
   DECLARE INDVAR fc [FACT];
   DECLARE FUNCONST wffof (FACT) = WFF;
   DECLARE PREDCONST THEOREM 1;
   comment | here we put the first mark m1 |
   mark m1;
   DECREP FACT;
   DECREP WFF;
   REPRESENT \{WFF\} AS WFF;
   REPRESENT \{FACT\} AS FACT;
   comment | here we put the second mark m2 |
   mark m2;
   ATTACH wffof TO [FACT = WFF] fact\-get\-wff;
   AXIOM M2: forall fc. THEOREM(wffof(fc));
   MAKECONTEXT OBJ;
   SWITCHCONTEXT OBJ;
   DECLARE SENTCONST A;
+



\gfcommand{probe}{verbose mode}
\index{probe}

\gfsyntax{
   probe;\\
   probe \ARG{activity};\\
   probe all;\\
}

\gfdescription{
	Some {\GF} {\em activities} can be executed either in verbose
	or in silent mode.
	In the former, {\GF} displays messages describing the execution of the
	command which are not displayed in the silent mode.
	Examples of activities are given below. 

	{\tt probe} lists the probed commands

	{\tt probe} \ARG{activity} sets commands in the activity to be executed
	in a verbose mode. 

	{\tt probe all} sets all activities to verbose mode.
}

\gfrecap{
Set verbose mode for certain activities.
}

\gfexample+
   ***** probe;
   Probing function set : COMMAND no
   Probing function set : IO yes
   Probing function set : DECLARE yes
   Probing function set : PROOF yes
   Probing function set : ATTACH yes
   Probing function set : SIMPLIFY yes
   Probing function set : SIMPSET yes
   Probing function set : REWRITE yes
   Probing function set : EVAL yes
   Probing function set : CONTEXT yes
   Probing function set : REFLECT yes
   
   ***** declare sentconst A B C;
   A has been declared to be a Sentconst
   B has been declared to be a Sentconst
   C has been declared to be a Sentconst
   
   ***** probe command;
   
   ***** probe;
   probe;
   Probing function set : COMMAND yes
   Probing function set : IO yes
   Probing function set : DECLARE yes
   Probing function set : PROOF yes
   Probing function set : ATTACH yes
   Probing function set : SIMPLIFY yes
   Probing function set : SIMPSET yes
   Probing function set : REWRITE yes
   Probing function set : EVAL yes
   Probing function set : CONTEXT yes
   Probing function set : REFLECT yes
   
   ***** declare sentconst D E F;
   declare sentconst D E F;
   D has been declared to be a Sentconst
   E has been declared to be a Sentconst
   F has been declared to be a Sentconst
   
   ***** unprobe all;
   unprobe all;
   
   ***** declare sentconst H G K;
+

\gfnotes{}

\gfcommand{unprobe}{silent mode}
\index{unprobe}

\gfsyntax{
   unprobe \ARG{activity};\\
   unprobe all;
}

\gfdescription{
   Some {\GF} {\em activities} can be executed either in verbose
   or in silent mode.

   {\tt unprobe} \ARG{activity} sets the silent mode for commands in
   \ARG{activity}.

   {\tt unprobe all} sets all activities to silent mode .
}

\gfrecap{
Set/unset verbose mode for certain activities.  
}

\gfexample+
   ***** declare sentconst D E F;
   declare sentconst B E F;
   D has been declared to be a Sentconst
   E has been declared to be a Sentconst
   F has been declared to be a Sentconst
   
   ***** unprobe all;
   unprobe all;

   ***** probe;
   Probing function set : COMMAND no
   Probing function set : IO no
   Probing function set : DECLARE no
   Probing function set : PROOF no
   Probing function set : ATTACH no
   Probing function set : SIMPLIFY no
   Probing function set : SIMPSET no
   Probing function set : REWRITE no
   Probing function set : EVAL no
   Probing function set : CONTEXT no
   Probing function set : REFLECT no
   
   ***** declare sentconst H G K;
+

\gfnotes{}


	% loading introduction to the section
\newpage
\section{Administration}
\label{sec-adm}

The commands described in this section manipulate the proof checker but do
not modify  the ``logical'' state of the deduction or of the computation.
Among the other things, they can be used to give alternative names to proof lines,
to load {\HG} or {\GF} files, to insert comments in {\GF} files, to show the
logical/computational status of the system.


% loading explanation of commands
\gfcommand{comment}{comments in {\GF}}
\index{comment}

\gfsyntax{
   comment \ARG{separator} \OPT{\ARG{text}} \ARG{separator}
}

\gfdescription{
   Defines a comment \ARG{text} between the two \ARG{separator}s.
   Any token can be a separator.
}

\gfrecap{
Defines a comment between two separators.
Any token can be a separator.
}

\gfexample+
   ***** comment ! this is a comment 
   enclosed between two exclamation marks !
   
   ***** comment ? this is a comment 
   enclosed between two question marks ?
   
   ***** comment com this is a comment 
   enclosed between the two words "com" com
+


\gfcommand{deflam}{defining {\HG} functions}
\index{deflam}

\gfsyntax{
   deflam \ARG{funname} \ARG{var-list} \ARG{form};
}

\gfdescription{
   Defines a {\HG} function at the {\GF} prompt.
   The function \ARG{funname} is defined as the {\HG} \ARG{form}
   with parameters the parameters in the list \ARG{var-list}.
}

\gfrecap{
Defines a HGKM function at the GETFOL prompt.
}

\gfexample+
   ***** deflam wffof (fact) (fact\-get\-wff fact);
+


\gfcommand{echo}{echoes a message to the standard output stream}
\index{echo}

\gfsyntax{
   echo \ARG{separator} \OPT{\ARG{text}} \ARG{separator}
}

\gfdescription{
   Echoes \ARG{text} between the two \ARG{separator}s to the current
   output stream.
   Any token can be a \ARG{separator}.
}

\gfrecap{
Echoes text between the two separators.
}

\gfexample+
   ***** echo ! this is an echo
   enclosed between two exclamation marks !
   this is an echo enclosed between two exclamation marks

   ***** comment ? this is an echo
   enclosed between two question marks ?
   this is an echo enclosed between two question marks

   ***** comment com this is a echo
   enclosed between the two words "com" com
   this is an echo enclosed between the two words "com"
+

\gfcommand{hgk}{{\HG} evaluation}
\index{hgk}

\gfsyntax{
   hgk \ARG{s-expr};
}

\gfdescription{
   Runs the {\HG} evaluator on \ARG{s-expr}.
}

\gfrecap{
Runs the HGKM evaluator on `s-expr'.   
}

\gfexample+
   ***** hgk (LOAD (QUOTE "file"))
+

\gfnotes{
	The command hgk tries to evaluate every element of the {\em
	s-expression} passed as argument; therefore \verb+(LOAD "file")+
	causes an error, if \verb+"file"+ has no value.
}
\gfcommand{know natnums}{allows the use of natural numbers}
\index{know natnums}

\gfsyntax{
	know natnums \OPT{\ARG{natnum1}, \SEQ, \ARG{natnumN}};
}

\gfdescription{
	It allows the use of \ARG{natnum1}, \SEQ, \ARG{natnumN}
	(all natural numbers if no natural numbers are explicitly listed).
	It declares the sort {\tt NATNUMSORT}, the representation 
	{\tt NATNUMREP} and optionally defines the extension of {\tt NATNUMSORT} 
	to be \ARG{natnum1}, \SEQ, \ARG{natnumN} (if explicitly listed).
}

\gfexample+
   ***** know natnums;
   ***** simplify (5=7);
   1   not (5 = 7)   
   ***** know natnums 1 2 3 4;
   Warning! You already know natnums.
   Now the extension of NATNUMSORT is fixed to be : (1 2 3 4)
   ***** simplify (5=7);
   simplify (5=7);
            ^
   SIMPLIFY requires a wff,fact or term here
   ***** simplify (1=3);
   2   not (1 = 3)
+

\gfcommand{load}{loading a {\HG} file}
\index{load}

\gfsyntax{
   load \ARG{file};
}

\gfdescription{
   Each form in \ARG{file} is read by the {\HG} reader and evaluated by the 
   {\HG} evaluator \cite{giunchiglia35}.
}

\gfrecap{
Each form in `file' is read and evaluated by HGKM.
}

\gfexample+
   ***** load example.hgk;
+


\gfcommand{resetprompt}{Resets the user defined prompt}
\index{resetprompt}

\gfsyntax{
   resetprompt;
}

\gfdescription{
   Comes back to the default prompt.
}
\gfrecap{
Comes back to the default prompt.
}

\gfexample+
   CARTOONIA:: resetprompt;

   ***** switchcontext Disneyland;
   You are now using context: Disneyland
   You are switching to a proof with no name.

   *****
+

\gfcommand{setprompt}{Redefines the prompt}
\index{setprompt}

\gfsyntax{
   setprompt to \ARG{s-expr};
}

\gfdescription{
   Sets {\GF}'s prompt  to the value of the s-expression
   \ARG{s-expr}  followed by  ":: ". It is particularly  useful when  you are
   working on multiple contexts as you can set the prompt to the value of
   the current context.
}

\gfrecap{
Sets GETFOL's prompt to `s-expr'.
}

\gfexample+
   ***** setprompt to (QUOTE myprompt);

   myprompt:: setprompt to (QUOTE Tweedledee\&Tweedledum);

   Tweedledee&Tweedledum:: setprompt to (CAPITALIZE (curcname\-get));

   NOTNAMED&:: namecontext Disneyland;
   You have named the current context: Disneyland

   DISNEYLAND:: makecontext Cartoonia;
   You have created the empty context: Cartoonia

   DISNEYLAND:: switchcontext Cartoonia;
   You are now using context: Cartoonia
   You are switching to a proof with no name.

   CARTOONIA:: 
+

\gfnotes{
	The command tries to evaluate the {\em s-expression} passed as argument.
	Failure of the evaluation causes a crash to the {\HG} evalautor.
}

\gfcommand{show}{shows {\GF} information}

\gfsyntax{
   show \ARG{option};
}

\gfdescription{
   Shows {\GF} information. In the example we show some of the
   options implemented in a {\GF} version.
   Notice that ``\verb+show show;+" lists the options supported by the show
   command.
}

\gfrecap{
Shows GETFOL information.
}

\gfexample+
   ***** comment | ******* SHOW SHOW ******* |
   ***** show show;
   The list of show options is the following:

   CONTEXT : WHEREAMI 

   REWRITER : SIMPSET 

   SIMPLIFIER : INT REP 

   DEFINITION : DEFINITION 

   PROOF : PREMISES FACT AXIOM 

   LANGUAGE : LGS MGS MEM SORT TYP SYM 

   SYSTEM : COM SHOW 
   
   ***** comment | ******* SHOW WHEREAMI ******* |
   
   ***** show whereami;
   You are now using an unnamed context.
   You are now using an unnamed proof.
   
   ***** comment | ******* SHOW REP and INT ******* |
   
   ***** decrep PIPPOREP;
   ***** attach A dar [PIPPOREP] a;
   ***** attach C dar [PIPPOREP] c;
   ***** attach B to  [PIPPOREP] b;
   
   ***** show rep PIPPOREP;
   The designators for the representation: PIPPOREP are:
   (c . C) (a . A)
   
   ***** show int A;
   The Indconst A is attached to 'a
   with representation PIPPOREP
   
   ***** comment | ******* SHOW SIMPSET ******* |
   
   ***** setbasicsimp simp1 at wffs
   {forall x1.F1(x1)=x1,
   forall x1 x2.F2(x1 x2)=x1,
   forall x1 x2 x3.F3(x1 x2 x3)=x1};
   
   ***** show simpset simp1;
   Wffs :
   forall x1. (F1(x1) = x1)
   forall x1 x2. (F2(x1,x2) = x1)
   forall x1 x2 x3. (F3(x1,x2,x3) = x1)
   
   ***** setbasicsimp simp2 at facts {1 2};
   
   ***** show simpset simp2;
   Proof lines :  1 2
   
   ***** comment | AX1 and AX2 are two axioms |
   ***** setbasicsimp simp3 at facts {AX1 AX2};
   ***** show simpset simp3;
   Axioms :  AX1 AX2
   
   ***** setcompsimp simpA at simp1 uni simp2 uni simp3;
   ***** show simpset simpA;
   simpA is compound by this list of basic simpsets :
   (simp1 simp2 simp3)
   
   ***** SETCOMPSIMP simpB at simpA dif simp1;
   ***** show simpset simpB;
   simpB is compound by this list of basic simpsets :(simp2 simp3)
   
   
   ***** comment | ******* SHOW PROOF, FACT and AXIOM ******* |
   
   ***** show proof;
   1   forall x1. (F1(x1) = x1)     (1)
   2   forall x1 x2. (F2(x1,x2) = x1)     (2)
   
   ***** label fact identity = 1;
   ***** show fact;
   identity   1
   
   ***** show axiom;
   AX1 : forall x1. (F1(x1) = x1)
   AX2 : forall x1 x2. (F2(x1,x2) = x1)
   
   ***** comment | ******* SHOW LSG, MGS , MEM and SORT ******* |
   
   ***** declare sort a b c d e f g h;
   moregeneral a < b c d e f g h >;
   moregeneral b < c d e f g h >;
   
   ***** show lgs a;
   No sort is strictly lessgeneral than a.
   ***** show lgs b;
   No sort is strictly lessgeneral than b.
   
   ***** show mgs a;
   The only sort strictly moregeneral than a is UNIVERSAL
   ***** show mgs f;
   The only sort strictly moregeneral than f is UNIVERSAL
   
   ***** show mem a;
   No <indsym> is declared to be of sort a.
   
   ***** declare indvar x [a];
   a is a sort
   x has been declared to be an Indvar
   ***** declare indvar y [a];
   a is a sort
   y has been declared to be an Indvar
   ***** show mem a;
   The <indsym>'s declared to be of sort a are
       y  x  
   
   ***** show sort;
   The symbols declared to be sorts are
       b  a  UNIVERSAL  
   
   ***** comment | ******* SHOW TYP, SYM ******* |
   
   ***** show typ INDVAR;
   The symbols declared to be Indvars are
       x  y  
   
   ***** show sym a;
   a is declared to be a sort.
   ***** show sym x;
   x is declared to be an Indvar of sort a.
   
   
   ***** comment | **************** SHOW PREMISES *************** |
   
   ***** declare sentconst A B C;
   ***** assume A B;
   1   A     (1)
   2   B     (2)
   ***** andi 1 2;
   3   A and B     (1 2)
   ***** ori 3 C;
   4   (A and B) or C     (1 2)
   ***** andi 3 4;
   5   (A and B) and ((A and B) or C)     (1 2)
   ***** show premises 5;
   5  (A and B) and ((A and B) or C)  (1 2)
      3  A and B  (1 2)
      4  (A and B) or C  (1 2)
   ***** show premises 5 2;
   5  (A and B) and ((A and B) or C)  (1 2)
      3  A and B  (1 2)
         1  A  (1)
         2  B  (2)
      4  (A and B) or C  (1 2)
         3  A and B  (1 2)
   *****  show premises 5 all;
   5  (A and B) and ((A and B) or C)  (1 2)
      3  A and B  (1 2)
         1  A  (1)
         2  B  (2)
      4  (A and B) or C  (1 2)
         3  A and B  (1 2)
            1  A  (1)
            2  B  (2)
   
   
   ***** comment | ******* SHOW COM ******* |
   
   ***** show com;
   The list of commands is the following:

   META : REFLECT MATTACH 

   CONTEXT : COPYLEX SWITCHCONTEXT COPYCONTEXT NAMECONTEXT MAKECONTEXT 

   DECIDER : DECIDE MONADEQ MONAD TAUTEQ TAUT PTAUT 

   EVAL : EVAL 

   SIMPLIFIER : SIMPLIFY LET HARDWARE REPRESENT ATTACH DECREP 

   REWRITER : REWRITE ASSERTSIMP SETCOMPSIMP SETBASICSIMP UNFOLD FOLD CUT
   CTC CONTRACT WK WEAKEN WFFIFI WFFIFEN WFFIFE TERMIFI TERMIFEN TERMIFE  

   Natural-Deduction : ES EXISTE EXISTI US ALLE UG ALLI IE IFFE II IFFI
   NE NOTE NI NOTI FE FALSEE FI FALSEI OE ORE OI ORI AE ANDE AI ANDI MP
   IMPE DED IMPI SUBST THEOREM ASSUME   

   DEFINITION : DEFINE 

   PROOF : LABEL CANCEL AXIOM THEORY 

   LANGUAGE : SWITCHPROOF COPYPROOF NAMEPROOF MAKEPROOF EXTENSION WFF
   AWFF TERM MOREGENERAL MOSTGENERAL SETFMAP DECLARE  

   SYSTEM : COPYLEX RESET PAGER RESETPROMPT SETPROMPT KNOW HGK SHOW ECHO
   COMMENT UNPROBE PROBE DEFLAM LOAD MARK FETCH DONE BACKUP  
+

 	% introduction to the language's section
\newpage
\section{Language}
\label{sec-lang}
\label{sec-decl}


A {\GF} language is a first-order language.
Its terms, atomic formulae, and
well-formed formulae are built from primitive symbols representing
variables, constants, functions, predicates,
connectives and quantifiers. The various syntactic categories of expressions,
corresponding to the different logical categories, are
recursively defined in terms of each other,
these definitions determining the written syntax of the
language\footnote{By written syntax we mean the syntax of the
expressions as they are typed rather than as they are represented
internally.}.

Each primitive symbol has a {\bf syntype}
which corresponds to its
logical status. Symbols of certain {\bf syntype}s
must also have other associated
information (such as their arity) which together with the {\bf syntype}
determines
how the symbol is to be used in the construction of compound expressions.

As every {\GF} language is a first-order language there are certain
logical symbols (such as connectives and quantifiers)
which are predefined. The other primitive
symbols of a language must be user defined by means of declarations 
(see the command {\bf declare}) which
specify a symbol together with its {\bf syntype}
(and any other information
required).

\subsection{{\GF} symbols}

{\GF} commands accept symbols that we will identify with
{\em sym}. The {\GF} scanner recognizes different {\em types} of
characters: {\em identifiers}, {\em numbers} (positive integers),
{\em escape characters},
{\em delimiters} and special delimiters that 
may be regarded as tokens: {\em identifier-delimiters}.
{\GF} symbols {\em sym}s are sequences of {\em identifiers}
and {\em numbers}. Identifiers are for instance:
\begin{verbatim}
                       a b c d
                       A B C D
                       ! " # $ % &
\end{verbatim}    %%% $ (fake emacs' hilite mode)
{\em delimiters} are used to separate symbols.
Some delimiters are:
\begin{verbatim}
                       ( ) , . : ; [ ] 
\end{verbatim}
{\em identifier-delimiters} may be used to separate
symbols or may identify a particular token, as for instance
\begin{verbatim}
                       < > = + - * / ? @ 
\end{verbatim}
Escape characters turn a character into an identifier.
The character $\backslash$ is {\GF}'s escape character.

Types are assigned to characters in the file {\tt asciitab.fol}
in the {\tt fol} directory.

An example:
\begin{verbatim}
***** declare sentconst a!b;
a!b has been declared to be a Sentconst
***** declare sentconst a_b;
a_b has been declared to be a Sentconst
***** declare sentconst a-b;
a has been declared to be a Sentconst
- has been declared to be a Sentconst
b has been declared to be a Sentconst
***** declare sentconst c/d;
c has been declared to be a Sentconst
/ has been declared to be a Sentconst
d has been declared to be a Sentconst

***** declare sentconst e\f;
ef has been declared to be a Sentconst
\end{verbatim}

\subsection{{\GF} special symbols}

The following strings are treated as special symbols:

\begin{itemize}
\item The logical connectives:
{\tt not, and, or, imp, iff, wffif, trmif, forall, exists};
\item The sentential constants for truth and falsity:
{\tt FALSE, TRUE}; 
\item The equality predicate symbol: {\tt =};
\item Sorts: {\tt UNIVERSAL}, {\tt NATNUMSORT};
\item Numerals: {\tt 1}, {\tt 2}, {\tt 3}, ...;
\item Representations: {\tt UNIVERSALREP}, {\tt NATNUMREP};
\item Natural numbers: {\tt 1}, {\tt 2} {\tt 3}, ...;
\item Various: {\tt TRUTHSORT}, {\tt TRUTHREP}, {\tt NOTNAMED\&}, {\tt UNDEF\&};
\end{itemize}



Using these symbols not in ``reading mode",
that is modifying their meaning,
is very dangerous. The behavior
of the system becomes unpredictable.






\subsection{Syntypes of primitive symbols}

The different {\bf syntype}s of primitive symbols are:

\gap
\begin{center}
\fbox{
\parbox{15cm}{
\begin{itemize}

\item {\indvar} --- individual variables.

\item {\indpar} --- individual parameters.

\item {\indconst} --- individual constants.

\item {\funpar} --- function parameters.

\item {\funconst} --- function constants.

\item {\predpar} --- predicate parameters.

\item {\predconst} --- predicate constants.

\item {\sentpar} --- sentential parameters.

\item {\sentconst} --- sentential constants.

\item {\sentconn} --- sentential connectives.

\item {\quant} --- quantifiers.

\end{itemize}
}}
\end{center}

\gap
Some of the above {\bf syntype}s are grouped together into hybrid syntactic
categories according to the role they play
in defining logical expressions.

\gap
\begin{center}
\fbox{
\parbox{15cm}{
{\indsym} ::= {\indvar} \I {\indpar} \I {\indconst}

{\funsym} ::= {\funpar} \I {\funconst}

{\predsym} ::= {\predpar} \I {\predconst}

{\sentsym} ::= {\sentpar} \I {\sentconst}
}}
\end{center}

\subsection{Logical expressions}

The various categories of compound logical expressions are defined
recursively in terms of each other and in terms of the categories of
primitive symbols. The most important categories, from a logical point of
view, are: {\wff} (the category of well-formed formulae), {\awff} (the
category of atomic formulae) and {\term} (the category of terms).

The following definition gives the 
grammar of expressions accepted by {\GF}. In all cases $n$
is a natural number $\geq 1$.
Superscripts of
---$^n$, ---$^P$, ---$^I$ are used with certain categories,
the categories of operator symbols, to specify other properties which the
permitted symbols must possess; the three superscripts require
respectively: that the
symbol has arity $n$, that the symbol is a unary prefix operator, that the
symbol is a binary infix operator. Operator symbols and their associated
properties are treated in full in section \ref{opsym}.

\gap
\begin{center}
\fbox{
\parbox{15cm}{
{\wff} ::= {\bf (} {\wff} {\bf )} \I {\connappl} \I {\quantwff}
\I {\wffif} \I {\awff}

{\connappl} ::= {\sentconn}$^{P}$ {\wff} \I {\wff}$_1$
{\sentconn}$^{I}$ {\wff}$_2$

{\quantwff} ::= {\quantprefix} {\wff}

{\quantprefix} ::= {\quant} {\indvar}$_1$ ... {\indvar}$_n$ {\bf .}

{\wffif} ::= {\bf wffif} {\wff}$_1$ {\bf then} {\wff}$_2$ {\bf else}
{\wff}$_3$

{\awff} ::= {\sentsym} \I {\predappl}

{\predappl} ::= {\predsym}$^{P}$ {\term} \I
                {\term}$_1$ {\predsym}$^{I}$ {\term}$_2$ \I
                {\predsym}$^n${\bf (} {\term}$_1$ [{\bf ,}] ...
                [{\bf ,}] {\term}$_n$ {\bf )}

{\term} ::= {\bf (} {\term} {\bf )} \I {\funappl} \I {\termif} \I {\indsym}

{\termif} ::= {\bf trmif} {\wff} {\bf then} {\term}$_1$ {\bf else}
{\term}$_2$

{\funappl} ::= {\funsym}$^{P}$ {\term} \I
               {\term}$_1$ {\funsym}$^{I}$ {\term}$_2$ \I
               {\funsym}$^n${\bf (} {\term}$_1$ [{\bf ,}] ...
               [{\bf ,}] {\term}$_n$ {\bf )}
}}
\end{center}

\subsection{Operator symbols}
\label{opsym}

The primitive symbols of the categories  {\funsym},
\predsym, {\sentconn} are operator symbols; each operator
symbol has an associated arity which is a natural number $\geq 1$.
\funsym$^n$, \predsym$^n$, and \sentconn$^n$ are used to denote the
subclasses of {\funsym}, {\predsym} and {\sentconn} of operators with
arity $n$.

An operator symbol with arity $1$ must be defined to be prefix. \funsym$^P$,
\predsym$^P$, and \sentconn$^P$ are used to denote the categories of prefix
operators, these are subclasses of \funsym$^1$, \predsym$^1$, and
\sentconn$^1$ respectively.

An operator symbol with arity $2$ may be defined to be infix. \funsym$^I$,
\predsym$^I$, and \sentconn$^I$  are used to denote the categories of infix
operators, these will be
subclasses of \funsym$^2$, \predsym$^2$, and \sentconn$^2$ respectively.

The use of prefix and infix operators allows the following ambiguous
expressions:


\begin{enumerate}

\item {\term}$_1$ {\funsym}$^I_1$ {\term}$_2$ {\funsym}$^I_2$ {\term}$_3$

\item {\wff}$_1$ {\sentconn}$^I_1$ {\wff}$_2$ {\sentconn}$^I_2$ {\wff}$_3$

\item {\funsym}$^P$ {\term}$_1$ {\funsym}$^I$ {\term}$_2$

\item {\sentconn}$^P$ {\wff}$_1$ {\sentconn}$^I$ {\wff}$_2$

\item {\quantprefix} {\wff}$_1$ {\sentconn}$^I$ {\wff}$_2$~\footnote{This is
a special case as a {\quantprefix} behaves like a prefix operator.}

\end{enumerate}

These ambiguities are avoided through the assignment of additional
information to certain operator symbols declared to be prefix or infix.
Each symbol in \funsym$^P$ and \sentconn$^P$ must
have an associated prefix binding priority (a natural number
$\geq 1$)\footnote{In fact each prefix and infix {\predsym} also has
associated binding priorities but this information is redundant as
ambiguities can not arise.}, this
will be denoted by prbp(---). Each symbol in \funsym$^I$ and \sentconn$^I$
must have two associated binding priorities: the left binding
priority and the right binding priority (both natural numbers $\geq 1$);
these will be denoted by lbp(---) and rbp(---) respectively.

The five ambiguous cases above are disambiguated in the following way:

\begin{enumerate}

\item If rbp({\funsym}$^I_1$) $>$ lbp({\funsym}$^I_2$)
then the expression is parsed as:

({\term}$_1$ {\funsym}$^I_1$ {\term}$_2$) {\funsym}$^I_2$ {\term}$_3$

Otherwise, if
rbp({\funsym}$^I_1$) $\leq$ lbp({\funsym}$^I_2$)
then the expression is parsed as:

{\term}$_1$ {\funsym}$^I_1$ ({\term}$_2$ {\funsym}$^I_2$ {\term}$_3$)

\item If rbp({\sentconn}$^I_1$) $>$ lbp({\sentconn}$^I_2$)
then the expression is parsed as:

({\wff}$_1$ {\sentconn}$^I_1$ {\wff}$_2$) {\sentconn}$^I_2$ {\wff}$_3$

Otherwise, if
rbp({\sentconn}$^I_1$) $\leq$ lbp({\sentconn}$^I_2$)
then the expression is parsed as:

{\wff}$_1$ {\sentconn}$^I_1$ ({\wff}$_2$ {\sentconn}$^I_2$ {\wff}$_3$)

\item If prbp({\funsym}$^P$) $>$ lbp({\funsym}$^I$)
then the expression is parsed as:

({\funsym}$^P$ {\term}$_1$) {\funsym}$^I$ {\term}$_2$

Otherwise, if
rbp({\sentconn}$^I_1$) $\leq$ lbp({\sentconn}$^I_2$)
then the expression is parsed as:

{\wff}$_1$ {\sentconn}$^I_1$ ({\wff}$_2$ {\sentconn}$^I_2$ {\wff}$_3$)

\item If prbp({\sentconn}$^P$) $>$ lbp({\sentconn}$^I$)
then the expression is parsed as:

({\sentconn}$^P$ {\wff}$_1$) {\sentconn}$^I$ {\wff}$_2$

Otherwise, if
prbp({\sentconn}$^P$) $\leq$ lbp({\sentconn}$^I$)
then the expression is parsed as:

{\sentconn}$^P$ ({\wff}$_1$ {\sentconn}$^I$ {\wff}$_2$)

\item If $1000 >$ lbp({\sentconn}$^I$)
then the expression is parsed
as \footnote{1000 is the default prefix binding
priority.}:

({\quantprefix} {\wff}$_1$) {\sentconn}$^I$ {\wff}$_2$

Otherwise, if
$1000 \leq$ lbp({\sentconn}$^I$)
then the expression is parsed as:

{\quantprefix}  ({\wff}$_1$ {\sentconn}$^I$ {\wff}$_2$)

\end{enumerate}


\subsection{The logical symbols}
\label{sec-log-symb}

The logical symbols are hardwired into the system.
The two categories {\quant} and {\sentconn}
consist entirely of logical constants and are thus fixed in advance for each
language.

\gap
\begin{center}
\fbox{
\parbox{15cm}{
{\quant} ::= {\bf forall} \I {\bf exists}


{\sentconn} ::= {\bf not} \I
                {\bf and} \I {\bf or} \I {\bf imp} \I {\bf iff}
}}
\end{center}

The logical connectives are all operator symbols
of arity 2 and declared as infix with the exception of {\bf not}
which has arity 1 and is prefix. The binding priorities of these symbols are
given in the table below.

\begin{center}
\begin{tabular}{|l|c|c|ccc|} \hline
\multicolumn{1}{|c}{Symbol} &
\multicolumn{1}{c}{Arity} &
\multicolumn{1}{c}{Type} &
\multicolumn{1}{c}{prbp} & lbp & rbp \\ \hline
{\bf not} & 1 & prefix & 1000 & n/a & n/a \\ \hline
{\bf and} & 2 & infix & n/a  & 750 & 755 \\ \hline
{\bf or} & 2 & infix & n/a  & 740 & 745 \\ \hline
{\bf imp} & 2 & infix & n/a  & 730 & 735 \\ \hline
{\bf iff} & 2 & infix & n/a  & 720 & 725 \\ \hline
\end{tabular}
\end{center}

The two truth values {\tt TRUE} and {\tt FALSE} and the equality
predicate, {\tt =}, are also logical symbols, {\tt TRUE} and
{\tt FALSE} belong to the category {\sentconst} and {\tt =} belongs to the
category {\predconst}; however, the categories {\sentconst} and
{\predconst} are not fixed
(because other primitive symbols in these categories
can be user declared).

The equality predicate, {\tt =}, has arity 2 and is declared as infix.





% user commands for language
\gfcommand{awff}{{\em awff} verifier}
\index{awff}

\gfsyntax{
   awff \ARG{awff};
}

\gfdescription{
   {\bf term}, {\bf awff} and {\bf wff} verify that their arguments
   are terms, atomic formulas and well formed formulas respectively.
   An error is signaled if an expression of the correct
   syntactic category is not provided.
}

\gfrecap{
Verifies whether the formula `awff' is atomic.
}

\gfexample+
   ***** declare indvar x y;
   ...
   ***** awff x = y;
   x = y is an <awff>.

   ***** declare predconst P 1 [pre];
   ***** declare funconst f g 1;
   ...
   ***** awff P g(f(x),y);
   P g(f(x),y) is an <awff>.

   ***** awff P x and P x;
   awff P x and P x;
                ^
   You have a legal command not ending with a semi-colon.
+
\gfcommand{declare}{language declaration}
\index{declare}
\index{declare!language}

\gfsyntax{
   declare \OPT{\ARG{sentsym} \ALT \ARG{indsym}} \ARG{sym1} \SEQ \ARG{symN}; \\
   declare \OPT{\ARG{funsym}  \ALT \ARG{predsym}} \ARG{sym1} \SEQ \ARG{symN}
   \ARG{arity};\\
   declare \OPT{\ARG{funsym} \ALT \ARG{predsym}} \ARG{sym1} \SEQ \ARG{symN}
   1 \OPT{[ pre \OPT{= \ARG{prbp}} ]};\\
   declare \OPT{\ARG{funsym} \ALT \ARG{predsym}} \ARG{sym1} \SEQ \ARG{symN}
   2 \OPT{[ inf \OPT{= \ARG{lbp} \ARG{rbp}} ] };
}

\gfdescription{
   This commands adds new symbols to the {\GF} language.
   The user can declare primitive symbols belonging to the following
   categories: {\tt sentpar}, {\tt sentconst}, {\tt indvar}, {\tt indpar},
   {\tt indconst}, {\tt funpar}, {\tt funconst}, {\tt predpar}, 
   {\tt predconst}.
   The user can specify binding priorities ($prbp$, $lbp$, $rbp$) 
   for the operator symbols being declared.
   If this is not done, then the default priorities are used
   (see section \ref{sec-log-symb}).
}

\gfrecap{
Adds new symbols to the GETFOL language.
}

\gfexample+
   ***** comment ! declarations of the first kind !

   ***** declare sentconst A B C;
   A has been declared to be a Sentconst
   B has been declared to be a Sentconst
   C has been declared to be a Sentconst

   ***** declare sentpar S1 S2;
   S1 has been declared to be a Sentpar
   S2 has been declared to be a Sentpar

   ***** declare indvar x y;
   UNIVERSAL is a sort
   x has been declared to be an Indvar
   y has been declared to be an Indvar

   ***** declare indpar a;
   UNIVERSAL is a sort
   a has been declared to be an Indpar

   ***** declare indconst alpha;
   UNIVERSAL is a sort
   alpha has been declared to be an Indconst

   ***** comment ! declarations of the second kind !
   
   ***** declare predconst R 2;
   R has been declared to be a Predconst
   
   ***** declare funconst f 1;
   f has been declared to be a Funconst
   
   ***** declare funconst g 2;
   g has been declared to be a Funconst
   
   ***** declare funconst h 4;
   h has been declared to be a Funconst
   
   ***** comment ! declarations of the third kind !
   
   ***** declare funconst f1 1 [pre = 500];
   f1 has been declared to be a Funconst
   
   ***** declare predconst P 1 [pre];
   P has been declared to be a Predconst
   
   ***** comment ! declarations of the fourth kind !
   
   ***** declare funconst g1 2 [inf = 400 405];
   g1 has been declared to be a Funconst
   
   ***** declare funconst g2 2 [inf = 605 600];
   g2 has been declared to be a Funconst
   
   ***** declare funpar g3 2 [inf];
   g3 has been declared to be a Funpar
+

\gfnotes{
   Since the {\GF} logic is sorted, the user can specify sort
   information for some categories of symbols. If this is not done, then the
   default information is used (as is the case in the examples above).
   The sorting mechanism in {\GF} is discussed in section \ref{sec-sort}.
}



\gfcommand{term}{{\em term} verifier}
\index{term}

\gfsyntax{
   term \ARG{term};
}

\gfdescription{
   {\tt term}, {\tt awff} and {\tt wff} verify that their arguments
   are terms, atomic formulas and well formed formulas respectively.
   An error is signaled if an expression of the correct
   syntactic category is not provided.
}

\gfrecap{
Verifies whether the term `term' is well formed.
}

\gfexample+
   ***** declare funconst f 1;
   ***** declare funconst g 2;
   ***** declare indvar x y;
   ...
   
   ***** term g(f(x),y);
   g(f(x),y) is a <term>.

   ***** term g(x);
   g expects 2 arguments
+
\gfcommand{wff}{{\em wff} verifier}
\index{wff}

\gfsyntax{
   wff \ARG{wff};
}

\gfdescription{
   {\tt term}, {\tt awff} and {\tt wff} verify that their arguments
   are terms, atomic formulas and well formed formulas respectively.
   An error is signaled if an expression of the correct
   syntactic category is not provided.
}

\gfrecap{
Verifies whether the formula `wff' is well formed.
}

\label{language-examples}
\gfexample+
   ***** declare predconst P 1 [pre];
   ***** declare predconst R 2;
   ***** declare indvar x y;
   ***** declare funconst g 2;
   ***** declare funconst f 1;
   ...

   ***** wff P g(f(x),y) and x = y;
   (P g(f(x),y)) and (x = y) is a <wff>.
   ***** wff exists x y . (P g(f(x),y) and x = y);
   exists x y. ((P g(f(x),y)) and (x = y)) is a <wff>.
   ***** wff forall x y. wffif P trmif x = y then f(x) else f(y)
                then R(x,y)
                else R(y,x);
   forall x y. (wffif (P (trmif (x = y) then f(x) else f(y))) then R(x,y) else
   R(y,x)) is a <wff>.

   ***** wff Z(x);
   wff Z(x);
       ^
   The scanner requires a <wff> here
+



% introduction to the sort's section
\newpage
\section{Sorts}
\label{sec-sort}

\subsection{Introduction}

The {\GF} logic is sorted, although you are not forced to make any
sort declarations.
If you never use any of the system's sort commands, you can use the 
logic as if it were unsorted. {\GF} makes sure that everything works correctly
with appropriate default sorts
(see later in this section).

Sorts in {\GF} are specially-handled unary predicates: that is, aside from the
usual features of unary predicates, they possess some additional ones.
This choice won't be deeply justified here.
The resulting logic can, however, be proved to be 
consistent and to allow more compact axiomatizations and proofs
than the unsorted one.

\subsection{Defining the sort hierarchy}
We present here the commands for the declaration of new sorts, for the
definition of the generality relations between sorts, for the definition
of a most general sort and for the declaration sort extensions.


The generality relations can be intuitively specified in this way:
stating that sort $S_1$ is {\it weakly more general} than sort $S_2$
is equivalent to
specifying an axiom of the form $\forall x(S_2(x) \imp S_1(x))$.
In this case we also say that $S_2$ is {\it weakly less general} than $S_1$.
Two sorts are said to be {\it equivalent} when they are mutually
(weakly) more general.
When $S_1$ is {\it weakly more (less) general} than $S_2$ and it is not
equivalent to it, then we say that $S_1$ is {\it strictly more (less)
general} than $S_2$.
Finally, we say that $S$ is a {\em most general sort} if it is weakly more
general than any sort. {\GF} has a default most general sort, {\tt
UNIVERSAL}, that is used when no explicit sort information is provided to
the system.

\subsection{Sorted declarations}

A sort is associated to every {\indsym} at the moment of its declaration.
In unsorted
declarations, when no sort is specified, the default most general sort
{\tt UNIVERSAL} is taken to be the sort of the symbol.

The sort information associated with a {\funsym} is a set of so-called 
{\it fmaps},
that is of $n+1$-tuples of sorts, where $n$ is the arity of the \funconst.
The fmaps for $f$ specify the sort of a term whose external {\funsym} is 
$f$, depending on the sort of its arguments. When a {\funsym} is declared
the $n+1$-tuple ({\tt UNIVERSAL} \ldots {\tt UNIVERSAL}) is always added to its
{\it fmaps}.

According to the sort information of symbols we give now the
definition of the useful notion of {\it sort of a term}:
we say that $t$ is a term of sort $S$ 
if and only if
%
\begin{itemize}
\item
  $t$ is an individual symbol, and its sort is weakly less general than $S$,
  or
\item
  $t=f(t_1,\dots,t_n)$, $f$ has an {\it fmap} of the form $(S_1,\dots,S_n,
  S_f)$, $S_f$ is weakly less general than $S$ and all $t_i$ are terms of sort
  $S_i$.
\end{itemize}


% user commands for sorts
\gfcommand{declare sort}{sort declaration}
\index{declare}
\index{declare!sort}

\gfsyntax{
  declare sort \ARG{sym1} \SEQ \ARG{symN};
}

\gfdescription{
  The declare sort command declares the \ARG{symI} to be sorts.
  The \ARG{symI} have to be either new symbols or previously declared unary
  predicates. 
}

\gfrecap{
  Declares the `symI's to be sorts.
}

\gfexample+
   ***** declare predconst S1 1;
   S1 has been declared to be a Predconst

   ***** declare sort S2;
   S2 has been declared to be a sort

   ***** declare sort S1 S2 S3;
   The unary predconst S1 has been declared to be a sort
   S2 is a sort
   S3 has been declared to be a sort

   ***** declare indvar x [S1];
   S1 is a sort
   x has been declared to be an Indvar

   ***** simplify exists x. S1(x);
   1   exists x. S1(x)     

   ***** simplify forall x. S1(x);
   2   forall x. S1(x)     
+

\gfnotes{
  Symbols with no declared sort have the default sort {\tt UNIVERSAL}.
  If \ARG{symI} is already a sort (as {\tt S2} is in the last
  example) no error is signaled. If $S$ is a sort, then $\exists x S(x)$
  and $\forall x S(x)$ are theorems where $x$ is an indvar of sort $S$.
}

\gfcommand{declare}{sorted declaration of indsyms and funsyms}
\index{declare}
\index{declare!sorted language}

\gfsyntax{
  declare indconst \ALT indpar \ALT indvar \ARG{sym1} \SEQ \ARG{symN}
  [ \ARG{sortsym} ]; \\
  declare funconst \ALT funpar \ARG{sym1} \SEQ \ARG{symN}
  ( \ARG{sortsym1} \SEQ \ARG{sortsymN} ) = \ARG{sortsym};\\
}

\gfdescription{
  The first form declares the \ARG{symI} to be indsym, and sets their
  sort to \ARG{sortsym}.
  The second form declares the \ARG{symI} to be funsyms of arity $N$, and adds
  (\ARG{sortsym1} \SEQ \ARG{sortsymN} \ARG{sortsym}) to the set of fmaps of
  the \ARG{symI}.
}

\gfrecap{
Declares symI to be an indsym (of sort `sortsym') or a funsym with fmap
(`sortsym1', ..., `sortsymN', `sortsym')
}

\gfexample+
   ***** declare predconst S4 1;
   S4 has been declared to be a Predconst
   ***** declare indvar x [S1];
   S1 is a sort
   x has been declared to be an Indvar
   ***** declare funconst f ( S4 S5 ) = S1;
   S1 is a sort
   The unary predconst S4 has been declared to be a sort
   S5 has been declared to be a sort
   f has been declared to be a Funconst
+

\gfnotes{
  In both forms of declaration \ARG{sortsym} and \ARG{sortsymI} have to be
  either new symbols or unary predicates or sorts.
  In the former cases \ARG{sortsym} and \ARG{sortsymI} are also declared to
  be sorts. 
}

\gfcommand{extension}{sort extension declaration}
\index{extension}

\gfsyntax{
  extension \ARG{sort} by \ARG{extexpr};
}

\gfdescription{
  The extension of a sort $S$ is the set of all and only the {\tt indconst}s
  of sort $S$. To say that $S$ has extension $\{c_1, \ldots ,c_n\}$,
  where $c_1, \ldots ,c_n$ are {\tt indconst}s, is the same as saying: $\forall
  x.S(x) \liff (x=c_1  \dis\ldots \dis x=c_n$), where $x$ is an {\tt indvar}
  of sort {\tt UNIVERSAL}. 
  The command declares the extension of \ARG{sort} to be \ARG{extexpr}.
  An extension expression \ARG{extexpr} is defined by the syntax below, where
  {\bf uni} stands for set union, {\bf dif} stands for set difference and {\bf
  int} stands for set intersection.
  The binding power is in the following increasing order: {\bf uni}, {\bf
  dif}, {\bf int}.
  %
  \begin{bnf}
    $extexpr$  \sep  $sort$ | {\bf \{}$indsym_1,\ldots,indsym_n${\bf \}}|\\ 
               &     $extexpr$ {\bf uni} $extexpr$      |\\
               &     $extexpr$ {\bf dif} $extexpr$      |\\  
               &     $extexpr$ {\bf int} $extexpr$ 
  \end{bnf}
  %
  A sort in \ARG{extexpr} without extension is treated as if it had an empty
  extension. 
}

\gfrecap{
  Declares the extension of a sort S (that is the set of all and only the
  indconst of sort S).
}

\gfexample+
   ***** declare sort S T R;
   S has been declared to be a sort
   T has been declared to be a sort
   R has been declared to be a sort

   ***** declare indconst a b c;
   UNIVERSAL is a sort
   a has been declared to be an Indconst
   b has been declared to be an Indconst
   c has been declared to be an Indconst

   ***** extension S by {a b c};
   Now the extension of S is fixed to be : (a b c)

   ***** extension T by S int {a c};
   Now the extension of T is fixed to be : (a c)

   ***** extension R by T uni {b};
   Now the extension of R is fixed to be : (a c b)

   ***** extension R by {a b c} dif S;
   You can't set an empty extension for R
+

\gfnotes{
  The command overwrites the previous extension of a sort without 
  warning.
  The syntactic and semantic simplifiers use the extension
  information explicitly if there is any (see the command {\tt simplify} 
  in section \ref{sec-simplify}
  and the command {\tt eval} in section \ref{sec-eval}).
}

\gfcommand{moregeneral}{new pairs in the more general relation}
\index{moregeneral}

\gfsyntax{
  moregeneral \ARG{sort1} < \ARG{sort2}, \SEQ, \ARG{sortN} >;
}

\gfdescription{
  This command makes \ARG{sort1} weakly more general than \ARG{sort2}, \SEQ,
  \ARG{sortN}.
}

\gfrecap{
Makes `sort1' weakly more general than `sort2', ..., `sortN'.
}

\gfexample+
   ***** declare sort S1 S2 S3;
   ...

   ***** moregeneral S1 < S2 S3 >;
   ***** moregeneral S2 < S1 > ;
   You realize that this makes equivalent S2 and S1
+

\gfcommand{mostgeneral}{new top for the sort hierarchy}
\index{mostgeneral}

\gfsyntax{
  mostgeneral \ARG{sym};
}

\gfdescription{
  The command declares \ARG{sym} to be a sort equivalent to the default
  most general sort ({\tt UNIVERSAL}).
}

\gfrecap{
Declares `sym' to be a sort equivalent to the most general sort.
}

\gfexample+
   ***** declare predconst P1 1;
   P1 has been declared to be a Predconst

   ***** declare sort P2 ;
   P2 has been declared to be a sort

   ***** mostgeneral P1;
   The unary predconst P1 has been declared to be a sort
   P1 is now equivalent to UNIVERSAL

   ***** mostgeneral P2;
   P2 is a sort
   P2 is now equivalent to UNIVERSAL
+

\gfnotes{
  \ARG{sym} has to be either an unused symbol or a sort or a unary predicate.
  The new symbol does not substitute for the default most general sort {\tt
  UNIVERSAL}; it is defined to be equivalent to it.
}

\gfcommand{setfmap}{adds another fmap to a funsym}
\index{setfmap}

\gfsyntax{
  setfmap \ARG{funsym} ( \ARG{sym1} \SEQ \ARG{symN} ) = \ARG{sym};
}

\gfdescription{
  The command adds (\ARG{sym1} \SEQ \ARG{symN} \ARG{sym}) to the fmaps of
  \ARG{funsym}.
}

\gfrecap{
Adds (`sym1', ..., `symN', `sym') to the fmaps of `funsym'
}

\gfexample+
   ***** declare predconst S4 1;
   S4 has been declared to be a Predconst

   ***** declare indvar x [S1];
   S1 is a sort
   x has been declared to be an Indvar

   ***** declare funconst f ( S4 S5 ) = S1;
   S1 is a sort
   The unary predconst S4 has been declared to be a sort
   S5 has been declared to be a sort
   f has been declared to be a Funconst
   
   ***** show sym f;
   f is declared to be a Funconst of arity 2.
   The FMAPs for f are
      UNIVERSAL * UNIVERSAL ==> UNIVERSAL,
      S4        * S5        ==> S1.

   ***** setfmap f ( S1 S6 ) = S5;
   S5 is a sort
   S1 is a sort
   S6 has been declared to be a sort

   ***** show sym f;
   f is declared to be a Funconst of arity 2.
   The FMAPs for f are
      S1        * S6        ==> S5,
      UNIVERSAL * UNIVERSAL ==> UNIVERSAL,
      S4        * S5        ==> S1.
+

\gfnotes{
  Note that \ARG{funsym} must have been previously declared to be a {\tt
  funsym}. 
  Its arity must be {\em N}.
}


 	% introduction to facts: the reasoning's building blocks!
\newpage
\section{Facts: reasoning building blocks}
\label{sec-reas}

Reasoning in {\GF} consists of a sequence of reasoning steps, each step being
performed on {\bf facts}.
A fact is a data structure containing a {\em hook} identifying it and some extra
information characterizing the the fact itself.
A fact can be a {\bf proof line}, an {\bf axiom}, a {\bf theory} or a {\bf
definition}.
These four categories are disjoint.
The hooks of proof lines, axioms, theories and definitions are called labels
({\em label}s), axiom labels ({\em axlabel}s), theory labels ({\em thlabel}s)
and definition labels ({\em deflabel}s) respectively.
We will simply call them {\em label}s whenever the distinction is not relevant.


\subsection{Proof lines: reasoning steps}
\label{sec-proofline}

Proof lines are the basic reasoning steps in {\GF}.
A proof line is a 4-tuple of the form: 

\begin{center}
$<$ {\em label} \ , \ {\em wff} \ , \ {\em reason} \ , \ {\em deplist} $>$
\end{center}

where
{\em label} is a ``hook" identifying the proof line itself; {\em wff} is the
well formed formula of the proof line; {\em reason} records how the proof line
has been asserted; {\em dep} records the assumptions the proof lines depends on.
For instance, the proof line 
%
\begin{center}
	{\tt < 3 , B , impe 1 2 , (1 2) >}
\end{center}
%
tells us that: the proof line {\tt 3} whose formula is ``{\tt B}'' has been
deduced by modus ponens ({\tt impe}) applied to the previously deduced proof
lines whose labels are {\tt 1} and {\tt 2} and that depends on the proof lines
{\tt 1} and {\tt 2}.
Analogously, the proof line 
%
\begin{center}
	{\tt < 4 , s(0) > 0 , alle smonotonic x 0 ,  >}
\end{center}
%
tells us that the proof line {\tt 4} whose formula is ``{\tt s(0) > 0}'' has
been deduced by instantiating the variable ``{\tt x}'' in the {\em fact} whose
{\em hook} is {\tt smonotonic} (probably an axiom asserting that
$\forall x \ (\ s(x) > x\ )$).

A {\GF} {\bf proof} is a sequence of proof lines.
The proof line labels are generated by {\GF} as increasing natural numbers
starting from $1$.
The user can ``name'' proof line labels with desired {\GF} symbols (see the
command {\tt label} in section \ref{sec-adm}) or to refer to the previous {\em
n-th} proof line in the current proof by typing \verb+^n+.


\subsection{Axioms and theories}

A {\GF} axiom is the mechanization of the notion of axiom in logic, a {\GF}
theory is the mechanization of the notion of theory, defined as a set of axioms,
in logic.
 
In {\GF} you can define the axioms of your own theory.
No check is performed on their consistency.
You are thus free to define an unsound theory.
However, the {\GF} logic is complete without need of any axioms.  
An axiom can be used to generate proof lines. 
A {\GF} axiom is a 2-tuple of the form:

\begin{center}
	$<$ {\em axlabel} \ , \ {\em wff} $>$\\
\end{center}

where {\em axlabel} is the ``hook'' and {\em wff} is the well formed formula of
the axiom. 

A theory is a set of axioms.
A theory allows you to use all the axioms of the theory as a whole. 


% user command: axioms, theories, theorems 
\gfcommand{axiom}{axiom definition}
\index{axiom}

\gfsyntax{
  axiom \ARG{sym} : \ARG{wff};
}

\gfdescription{
  Creates an axiom whose {\em axlabel\/} is {\em sym\/} and whose formula is
  {\em wff}.  The command also allows you to define axiom schemas. An axiom
  schema is an axiom containing one or more {\em sentpars}, {\em
  funpars} or {\em predpars}.  When an axiom schema is used to assert a proof
  line, then it may be instantiated, that is, some or all of {\it sentpar},   
  {\it funpars} and {\it predpars} may be ``substituted'' with complex
  wffs and terms\footnote{
    A formal definition of the substitution of
    parameters can be found in \cite{kleene1}.
  }. In particular:
  %
  \begin{itemize}
   \item 
     a {\em sentpar} $S$ can be instantiated in an axiom schema $\phi$,
     with a wff $\psi$. The result is an axiom $\phi^*$ obtained simply
     by replacing every occurrence of $S$ in $\phi$ by $\psi$;
   \item 
     a {\em funpar} $f$ of arity $n$, can be instantiated in $\phi$
     with a term $t(x_1,\ldots,x_n)$. The result of the instantiation is an
     axiom $\phi^*$ in which every occurrence in $\phi$ of the form
     $f(t_1,\ldots,t_n)$ is replaced with $t(t_1,\ldots,t_n)$;
   \item 
     a {\em predpar} $P$ of arity $n$, can be instantiated in $\phi$ with a
     wff $\psi(x_1,\ldots,x_n)$. The result of the instantiation is an
     axiom $\phi^*$ in which every occurrence in $\phi$ of the form
     $P(t_1,\ldots,t_n)$ has been replaced by $\psi(t_1,\ldots,t_n)$.
  \end{itemize} 
  %
  Let us notice that:
  %
  \begin{enumerate}
    \item 
      If an occurrence of $f(t_1,\ldots,t_n)$ in $\phi$ is within the scope
      of a quantifier for a variable $x$, if $x$ is distinct form $x_1,\ldots x_n$,
      and $x$ occurs in $t(x_1,\ldots x_n)$, then {\GF} performs a renaming
      of the variable $x$ in $\phi$, with a new variable of the same sort of
      $x$. {\em predpar} instantiation is treated in the analogous way.
    \item 
      In the {\em predpars} ($S$) and {\em sentpar} ($P$) instantiations, if a 
      variable $x$ distinct from $x_1,\ldots,x_n$ is quantified in $\psi$ and occurs
      in $P(t_1,\ldots,t_n)$ or $S$, then the substitution fails. In this case an
      error message is returned.
  \end{enumerate}
  %
  The syntax to instantiate an axiom schema whose {\em axlabel} is {\em sym} is:
  %
  \begin{quote}
    {\em sym} {\em subst$_1$} {\em subst$_2$} $\ldots$
  \end{quote}
  %
  where {\em subst} is a  pair composed of a parameter and its instantiation. The
  syntax of {\em subst} is:
  %
  \begin{quote}\tt
    \ARG{sym} sentpar : \ARG{wff} \\
    \ARG{sym} funpar  : lambda \ARG{indvar$_1$} 
                        [ \ARG{indvar$_2$} \SEQ] .
                       \ARG{term} \\
    \ARG{sym} predpar : lambda \ARG{indvar$_1$} 
                        [ \ARG{indvar$_2$} \SEQ ] .
                       \ARG{wff} \\
  \end{quote}
}

\gfrecap{
Declares an axiom in the current context labeled by `sym'.
}

\gfexample#
   ***** declare sentconst A B;
   ***** declare sentpar alpha beta;
   ...
   ***** axiom axA: A;
   axA : A
   ***** impi axA axA; 
   1   A imp A     
   ***** axiom Hil1: alpha imp (beta imp alpha);
   Hil1 : alpha imp (beta imp alpha)
   ***** impe 1 Hil1 alpha: A imp A, beta:B;
   2   B imp (A imp A)     

   ***** reset;   
   ***** declare indvar x y;
   ***** declare predconst p q 1;
   ***** declare predpar P Q 1;
   ...
   ***** axiom axdistr: forall x.(P(x) and Q(x)) imp
                        (forall x.P(x) and forall x.Q(x));
   axdistr : forall x. (P(x) and Q(x)) imp (forall x. P(x) and forall x. Q(x))
   ***** assume forall x. (p(x) and q(x));
   3   forall x. (p(x) and q(x))     (3)
   ***** impe 3 axdistr P:lambda x.p(x), Q:lambda x.q(x);
   4   forall x. p(x) and forall x. q(x)     (3)
   ***** axiom renaming: forall x. P(x) imp forall y. P(y);
   renaming : forall x. P(x) imp forall y. P(y)
   ***** ande 4 2;
   5   forall x. q(x)     
   ***** impe 5  renaming P:lambda y.q(y);
   6   forall y. q(y)     (3)
   
   ***** reset;
   ***** declare indvar n m;
   ***** declare predpar P 1;
   ***** declare indvar x;
   ***** declare funconst s 1;
   ***** declare funconst + 2 [INF];
   ***** know natnums;
   ...
   ***** axiom add1:forall n. n+0 = n;
   ***** axiom add2: forall m n. n+s(m) = s(n+m);
   ***** axiom ind:  P(0) and forall n.(P(n) imp P(s(n))) imp forall n.P(n);
   ***** eval 0+0=0+0;
   1   (0 + 0) = (0 + 0)     
   ***** assume x+0=0+x;
   2   (x + 0) = (0 + x)     (2)
   ***** alle add1 x;
   3   (x + 0) = x     
   ***** subst 2 3;
   4   x = (0 + x)     (2)
   ***** eval s(x)=s(x);
   5   s(x) = s(x)     
   ***** subst 5 4 occ 2;
   6   s(x) = s(0 + x)     (2)
   ***** alle add1 s(x);
   7   (s(x) + 0) = s(x)     
   ***** subst 7 6 occ 2;
   8   (s(x) + 0) = s(0 + x)     (2)
   ***** alle add2 x 0;
   9   (0 + s(x)) = s(0 + x)     
   ***** subst 8 9 right;
   10   (s(x) + 0) = (0 + s(x))     (2)
   ***** impi 2 10;
   11   ((x + 0) = (0 + x)) imp ((s(x) + 0) = (0 + s(x)))     
   ***** alli 11 x:n;
   12   forall n. (((n + 0) = (0 + n)) imp ((s(n) + 0) = (0 + s(n))))
   ***** andi 1 12;
   13   ((0 + 0) = (0 + 0)) and forall n. (((n + 0) = (0 + n)) imp 
        ((s(n) + 0) = (0 + s(n))))     
   ***** impe 13 ind P:lambda x. x+0=0+x;
   14   forall n. ((n + 0) = (0 + n))     
   
   ***** reset;
   ***** declare sentconst A B;
   ***** declare sentpar P Q;
   ***** axiom piffqu: P iff Q;
   piffqu : P iff Q
   ***** impi TRUE piffqu P:Q Q:A;
   1   TRUE imp (A iff A)     
   ***** impi TRUE piffqu Q:A P:Q;
   2   TRUE imp (Q iff A)     
   ***** impi TRUE piffqu Q:P P:Q;   
   3   TRUE imp (Q iff Q)     
   
   ***** reset;   
   ***** declare indvar x y;
   ***** declare funconst f 1;
   ***** declare predpar P 1;
   ***** declare predconst p 1;
   ***** declare predconst q 2;
   ***** axiom Ax1: forall x.(P(x) iff P(f(x)));
   Ax1 : forall x. (P(x) iff P(f(x)))
   ***** impi TRUE Ax1 P:lambda y. q(x y);
   1   TRUE imp forall x1. (q(x,x1) iff q(x,f(x1)))     
   ***** show sym x1;
   x1 is declared to be an Indvar of sort UNIVERSAL.
   
   ***** reset;
   ***** declare sentpar alpha;
   ***** declare indvar x;
   ***** declare predconst p q 1;
   ***** declare predpar P 1;
   ***** axiom Hil5: forall x.(alpha imp P(x)) imp (alpha imp forall x.P(x));
   Hil5 : forall x. (alpha imp P(x)) imp (alpha imp forall x. P(x))
   ***** assume P(x);
   1  P(x)     (1)
   ***** impe 1 Hil5  alpha: p(x) P:lambda x. q(x);
   alpha occurs within the scope of a quantifier binding x
#
   
\gfnotes{
   The {\tt predpar} and {\tt funpar} can be assigned any {\em wff} and {\em term} 
   whose number of (lambda) variables exceeds its arity.
   If the number is lower than the arity then we have an error.
   The extra lambda variables act as universally generalized 
   parameters of the entire axiom. The other variables get converted by  
   substitution.\\
   Axioms cannot be canceled or renamed. It is impossible to 
   define two axioms with the same {\em axlabel}. Numbers cannot be 
   {\em axlabel}s.
}

\gfcommand{theorem}{axiom creation from facts with no dependencies}
\index{theorem}

\gfsyntax{
  theorem \ARG{sym} \ARG{hook};
}

\gfdescription{
  The command creates an axiom from the fact whose hook
  is \ARG{hook}. The {\em axlabel} is \ARG{sym} and the {\em wff} is the formula
  of the fact; \ARG{hook} can identify an axiom, a theory 
  or a proof line without dependencies.
}

\gfrecap{
  Adds a fact without dependencies to the list of axioms.
}

\gfexample+
   ***** declare sentconst A;
   ...

   ***** assume A;
   1   A     (1)
   ***** theorem th1 1;
   theorem th1 1;
   A <fact> with dependencies cannot be made into a theorem
   ***** impi 1 1;
   2   A imp A     
   ***** theorem th1 2;
   th1 : A imp A
   ***** show axiom;
   th1 : A imp A
   ***** axiom ax1 : A or not A;
   ax1 : A or (not A)
   ***** theorem ExcludedMiddle ax1;
   ExcludedMiddle : A or (not A)
   ***** show axiom;
   th1 : A imp A
   ax1 : A or (not A)
   ExcludedMiddle : A or (not A)

   ***** declare predpar P 1;
   ***** declare predconst p 1;
   ***** declare indvar x;
   ***** axiom PI: P(x);
   ...
   ***** andi PI P:lambda x.p(x) PI P:lambda x.p(x);
   3   p(x) and p(x)     
   ***** theorem thPI PI;
   thPI : P(x)
   ***** andi thPI P:lambda x.p(x) thPI P:lambda x.p(x);
   4   p(x) and p(x)     
+
   
\gfnotes{
  The axiom may be an axiom schema in which case it becomes then 
  possible to instantiate it.
  In {\GF} proof lines containing {\tt predpar}s
  and/or {\tt funpar}s can {\it not} be instantiated. 
  If the user wants to prove $A \imp A$ and $B \imp B$,
  he can carry out the proof with non-instantiated axioms and get the
  schema $\alpha \imp \alpha$. $\alpha \imp \alpha$ can then be instantiated to
  $A \imp A$ and $B \imp B$. This capability is provided 
  by the command {\tt theorem}.
}





\gfcommand{theory}{theory definition}
\index{theory}

\gfsyntax{
  theory \ARG{thlabel} : \ARG{wff1} \OPT{\ARG{wff2} \SEQ};\\
  theory \ARG{thlabel} : \ARG{axlabel1} : \ARG{wff}
                         \OPT{\ARG{axlabel2} : \ARG{wff} \SEQ};
}

\gfdescription{
  Creates a theory, {\it ie.} a set of axioms that can be used as a whole
  in a deduction.
  The theory contains as many axioms as
  the number of wffs specified in \ARG{wff1} \OPT{\ARG{wff2} \SEQ}
  or the axioms specified in \ARG{axlabel1} : \ARG{wff}
  \OPT{\ARG{axlabel2} : \ARG{wff} \SEQ}.\\
  %
  In the first form, each axiom is given an axiom label (a name) by {\GF}; 
  the label is generated by taking the theory label {\em thlabel} and 
  concatenating an increasing integer starting from 1.\\
  %
  In the second form, the axiom's label is given by the user
  ({\em axlabel$_i$}).\\
  %
  A theory can be used to perform inference 
  ({\it e.g.} by applying inference rules on the theory itself). In this case,
  it is considered as
  a fact whose formula is the conjunction of the theory axioms' formulas.
  Any axiom in the theory can also be used individually by specifying its
  axiom label.\\
}

\gfrecap{
  Declares a theory (a set of axioms) in the current context; axioms in a
  theory are indexed by `axlabelN'.
}

\gfexample+
   ***** declare sentconst A B C;
   ...

   ***** theory hilbert : 
          A imp (B imp C) 
          (A imp (B imp C)) imp ((A imp B) imp (A imp C));
   hilbert
   hilbert1 : A imp (B imp C)
   hilbert2 : (A imp (B imp C)) imp ((A imp B) imp (A imp C))

   ***** mp hilbert1 hilbert2;
   1   (A imp B) imp (A imp C)

   ***** andi hilbert hilbert;
   4   (((A imp B) imp C) and 
       (((A imp B) imp C) imp ((A imp B) imp (A imp C)))) and 
       (((A imp B) imp C) and 
       (((A imp B) imp C) imp ((A imp B) imp (A imp C))))

   ***** theory tautologies: 
          IMP: A imp A 
          OR: A or not A;
   tautologies
   IMP : A imp A
   OR : A or (not A)
+


% introduction to the multiple proof's section
\newpage
\section{Multiple proofs}
\label{sec-proof}

\subsection{Introduction}

In {\GF} the user can build multiple distinct proofs (sequences of proof lines).
All the proofs within a context (see section \ref{sec-cxt}) share the language,
the  axioms, and the computational model of the context.
However, proofs (in the same context) differ by their proof lines. A proof  line
belongs to one and only  one proof.
The proof you are working in is the {\em current  proof}.
When you enter the system the current proof does not contain proof lines and has
no name (the {\em un-named} proof).
If you want to leave the proof to build another one, you have to give it a name
(by the command {\tt nameproof}).
This allows us to refer to it later on.
You can create a new proof by using {\tt makeproof}, and  switch to it by using
{\tt switchproof}. The proof you switch to becomes then the current proof.


% user commands for multiple proofs
\gfcommand{cancel}{proof lines elimination}
\index{cancel}

\gfsyntax{
   cancel \OPT{\ARG{label}};
}

\gfdescription{
   Cancels all the proof lines in the current proof starting from 
   the one with label \ARG{label}. 
   If \ARG{label} is not specified then it cancels the last asserted 
   proof line.
}

\gfrecap{
Cancels all the proof lines in the current proof starting from the one
labelled `label'.
If `label' is not specified then it cancels the last asserted proof line.
}

\gfexample+
   ***** show proof;
   1   A     (1)
   2   B     (2)
   3   C     (3)
   4   D     (4)
   ***** cancel;
   1   A     (1)
   2   B     (2)
   3   C     (3)
   ***** cancel 2;
   1   A     (1)
+
\input{proof/copyproof}
\gfcommand{label}{naming proof lines}
\index{label}

\gfsyntax{
   label fact \ARG{sym};\\
   label fact \ARG{sym} = \ARG{label};
}

\gfdescription{
   Allows us to ``name" proof lines with desired {\GF} symbols.
   We will then be able to refer to the proof line with {\em sym} rather
   than with its ``hook" ({\em label}). 
   In the first form the next asserted proof line will get the name \ARG{sym}. 
   In the second form the proof line whose label is {\em label} gets
   the alternative name \ARG{sym}.
}

\gfrecap{
Names a proof line with a GETFOL symbol.
Named proof lines can be accessed through its name.
In the first form the next asserted proof line will get the name `sym'.
In the second form the proof line whose label is `label' gets the
alternative name `sym'
}


\gfexample+
   ***** declare sentconst A;
   ..............
   ***** show proof;
   ..............
   4   A or (not A) 
   ***** label fact ExclMiddle = 4;
   ***** ori ExclMiddle A;
   5   (A or (not A)) or A  
   ***** label fact AssumContr;
   ***** assume A and not A;
   6   A and (not A)     (6)
   ***** ande AssumContr 1;
   7   A     (6)
+
\gfcommand{makeproof}{create a new empty proof}
\index{makeproof}

\gfsyntax{
   makeproof \ARG{prf-name};
}

\gfdescription{
   A new empty proof with name \ARG{prf-name} is created.
}

\gfrecap{
A new empty proof with name prf-name is created.
}

\gfexample+
   ***** show whereami;
   You are now using an unnamed context.
   You are now using an unnamed proof.
   ***** nameproof P1;
   You have named the current proof: P1
   ***** show whereami;
   You are now using an unnamed context.
   You are now using the proof: P1
   ***** declare sentconst A;
   ***** assume A;
   1   A     (1)
   ***** makeproof P2;
   You have created the empty proof: P2
   ***** switchproof P2;
   You are now using the proof: P2
   ***** declare sentconst B;
   ***** assume A or B;
   1   A or B     (1)
   ***** switchproof P1;
   You are now using the proof: P1
   ***** assume B;
   2   B     (2)
   ***** andi 1 2;
   3   A and B     (1 2)
   ***** show proof;
   1   A     (1)
   2   B     (2)
   3   A and B     (1 2)
   ***** switchproof P2;
   You are now using the proof: P2
   ***** show proof;
   1   A or B     (1)
+

\gfcommand{nameproof}{name the current proof}
\index{nameproof}

\gfsyntax{
   nameproof \ARG{prf-name};
}

\gfdescription{
   If the current proof has no name, it  is named with \ARG{prf-name}.
}

\gfrecap{
If the current proof has no name, it is named with `prf-name'.
}

\gfexample+
   ***** show whereami;
   You are now using an unnamed context.
   You are now using an unnamed proof.
   ***** nameproof P1;
   You have named the current proof: P1
   ***** show whereami;
   You are now using an unnamed context.
   You are now using the proof: P1
   ***** declare sentconst A;
   ***** assume A;
   1   A     (1)
   ***** makeproof P2;
   You have created the empty proof: P2
   ***** switchproof P2;
   You are now using the proof: P2
   ***** declare sentconst B;
   ***** assume A or B;
   1   A or B     (1)
   ***** switchproof P1;
   You are now using the proof: P1
   ***** assume B;
   2   B     (2)
   ***** andi 1 2;
   3   A and B     (1 2)
   ***** show proof;
   1   A     (1)
   2   B     (2)
   3   A and B     (1 2)
   ***** switchproof P2;
   You are now using the proof: P2
   ***** show proof;
   1   A or B     (1)
+
\gfcommand{switchproof}{switch to another proof}
\index{switchproof}

\gfsyntax{
  switchproof \ARG{prf-name};
}

\gfdescription{
  Switches from the current  proof  to the proof \ARG{prf-name} which then becomes
  the current proof.
}

\gfrecap{
Switches from the current proof to the proof prf-name which then becomes
the current proof.
}

\gfexample+
   ***** show whereami;
   You are now using an unnamed context.
   You are now using an unnamed proof.
   ***** nameproof P1;
   You have named the current proof: P1
   ***** show whereami;
   You are now using an unnamed context.
   You are now using the proof: P1
   ***** declare sentconst A;
   ***** assume A;
   1   A     (1)
   ***** makeproof P2;
   You have created the empty proof: P2
   ***** switchproof P2;
   You are now using the proof: P2
   ***** declare sentconst B;
   ***** assume A or B;
   1   A or B     (1)
   ***** switchproof P1;
   You are now using the proof: P1
   ***** assume B;
   2   B     (2)
   ***** andi 1 2;
   3   A and B     (1 2)
   ***** show proof;
   1   A     (1)
   2   B     (2)
   3   A and B     (1 2)
   ***** switchproof P2;
   You are now using the proof: P2
   ***** show proof;
   1   A or B     (1)
+

\gfnotes{
  The command fails if the current proof has no name.
}


 	% introduction to the natural deduction's section
\newpage
\section{Natural Deduction (ND)}
\label{sec-nd}
\label{sec-nd-first}

\subsection{{\GF} logic}
\label{sec-ndrules}

{\GF} uses a Natural Deduction (ND) calculus based on Prawitz's system
defined in \cite{prawitz1}.
Some resemblances exist also to the ND calculus defined by Quine in
\cite{quine3}.
For various reasons ({\it e.g.} efficiency of the implementation, elegance of
the proof theory), {\GF} also carries around the dependencies of any derived
formula.
This allows to see the {\GF} logic as a sequent calculus, where a sequent is a
pair $(\Gamma, A)$, where $A$ is a formula and $\Gamma$ a set of formulas, with
introduction and elimination in the post sequent.
We claim that the ``{\em correct}'' way to see {\GF} logic is as a ND calculus.
This is why all the commands are described in ND style, without explicitly
writing the dependencies.

Some notes about the ND rules described in figure \ref{fig-nd}:

\begin{itemize}
\item
	The notation used is the same as in \cite{prawitz1}.
\item
	The $\forall I$ and $\exists E$ rules have the following restrictions:
	$a$ must not appear free in the dependencies of $A$ (for $\forall I$)
	and $a$ must not appear free in $\exists x A$, in $B$  or in any assumption
	on which the upper occurrence of B depends other than $A^{x}_{a}$ (for
	$\exists E$).
\item
	{\GF} natural deduction rules {\em discharge all occurrences}.
\item
	In {\GF} both the negation connective {\bf not} ($\neg$) and the falsity
	sentential constant {\tt FALSE} ($\bot$) are available.
	$\neg A$ and $A \imp \bot$ are logically equivalent.
	From a wff containing an occurrence of $A \imp \bot$ it is possible to
	deduce the wff with $\neg A$ in place of $A \imp \bot$ and viceversa.
	In {\GF}, this deduction can be performed in several ways.
	The simplest is to use the decider for propositional calculus ({\tt ptaut})
	(see section \ref{sec-decproc}).
	For instance:
	%
	\begin{verbatim}
		***** declare sentconst A;
		A has been declared to be a Sentconst
		***** assume A imp FALSE;
		1   A imp FALSE     (1)
		***** ptaut not A by 1;
		2   not A     (1)
		***** assume not A;
		3   not A     (3)
		***** ptaut A imp FALSE by 3;
		4   A imp FALSE     (3)
	\end{verbatim}
\end{itemize}


The ND notation is given for the primitive ND rules, although the ND commands
implement (with different options) both basic and more complex derived inference
rules.
A complete discussion of the capabilities of the commands is given in the
description of each command.

In the syntax for the commands we will use {\em fact}, {\em fact}$_1$,
{\em fact}$_2$, $\ldots$, in place of {\em label}, {\em label}$_1$,
{\em label}$_2$, $\ldots$ (see section \ref{sec-reas}).
However, the user refers to the fact by its label.

The {\GF} command \verb+existe+ implementing the exist elimination rule 
($\exists E$) is the most radically different from the formal statement given
in Prawitz's ND calculus.
The command has some resemblance with the ``exist instantiation'' defined in
\cite{quine3}.
The {\GF} $\exists E$ command has side effects that make all the other
rules change behavior.
This is explained, in more detail, in the description of the command
\verb+existe+ in section \ref{sec-nd-first}.


\renewcommand{\arraystretch}{0.5}
\begin{figure}[htbp]
\[
	\begin{array}{|c|c||c|c|} \hline
	\multicolumn{2}{|c||}{\mbox{\bf Introduction rules}} &
	\multicolumn{2}{c|}{\mbox{\bf Elimination rules}} \\ \hline
	& & &\\
	& & &\\
	\con I  & 
	\fraz{A \ \ B}
     {A \wedge B}
	&
	\con E  & 
	\fraz{A \con B}
     {A}
	\ \ 
	\fraz{A \con B}
	     {B}
	\\
	\begin{array}{l}
	\\ \\ \\ \\ \\
	\imp I   
	\end{array}
	& 
	\begin{array}{c}
	\\ \\
	{[A]}\\
	\vdots\\
	B\\
	\hline\\
	A \imp B
	\end{array}
	&
	\begin{array}{l}
	\\ \\ \\ \\ \\
	\imp E  
	\end{array}
	& 
	\begin{array}{c}
	\\ \\ \\ \\ \\
	\fraz{A \ \ A \imp B}
	     {B}
	\end{array}
	\\
	\begin{array}{l}
	\\ \\ \\ \\ \\
	\dis I
	\end{array}
	& 
	\begin{array}{c}
	\\ \\ \\ \\ \\
	\fraz{A}
	     {A \dis B}
	\end{array}
	\ \ 
	\begin{array}{c}
	\\ \\ \\ \\ \\
	\fraz{A}
  	   {B \dis A}
	\end{array}
	&
	\begin{array}{l}
	\\ \\ \\ \\ \\
	\dis E  
	\end{array}
	& 
	\begin{array}{ccc}
	\\ \\
	&{[A]}&{[B]}\\
	&\vdots&\vdots\\
	A \dis B & C & C\\
	\hline\\
	& C &
	\end{array}
	\\
	\begin{array}{l}
	\\ \\ \\ \\ \\
	\bot_c 
	\end{array}
	&
	%\renewcommand{\arraystretch}{0.5}
	\begin{array}{c}
	\\ \\
	{[\neg A]}\\
	\vdots\\
	\bot\\
	\hline\\
	A 
	\end{array}
	&
	\begin{array}{l}
	\\ \\ \\ \\ \\
	\bot_i  
	\end{array}
	& 
	\begin{array}{c}
	\\ \\ \\ \\ \\
	\fraz{\bot}
	     {A}
	\end{array}
	\\
	\begin{array}{l}
	\\ \\ \\ \\ \\
	\imp I_{\neg}  
	\end{array}
	& 
	\begin{array}{c}
	\\ \\
	{[A]}\\
	\vdots\\
	\bot\\
	\hline\\
	\neg A
	\end{array}
	&
	\begin{array}{l}
	\\ \\ \\ \\ \\
	\imp E_{\neg}  
	\end{array}
	& 
	\begin{array}{c}
	\\ \\ \\ \\ \\
	\fraz{A \ \ \neg A}
	     {\bot}
	\end{array}
	\\
	\begin{array}{l}
	\\ \\
	\liff  I  
	\end{array}
	& 
	\begin{array}{c}
	\\ \\ 
	\fraz{A \imp B \ \ B \imp A}
	     {A \liff B}
	\end{array}
	&
	\begin{array}{l}
	\\ \\
	\liff E  
	\end{array}
	& 
	\begin{array}{c}
	\\ \\
	\fraz{A \liff B}
	     {A \imp B}
	\end{array}
	\ \
	\begin{array}{c}
	\\ \\
	\fraz{A \liff B}
	     {B \imp A}
	\end{array}
	\\
	\begin{array}{l}
	\\ \\
	\forall  I  
	\end{array}
	& 
	\begin{array}{c}
	\\ \\
	\fraz{A}
	     {\forall x A^{a}_{x}}
	\end{array}
	&
	\begin{array}{l}
	\\ \\
	\forall E
	\end{array}
	& 
	\begin{array}{c}
	\\
	\fraz{\forall x A}
	     {A^{x}_{t}}
	\end{array}
	\\
	\begin{array}{l}
	\\ \\ \\ \\ \\
	\exists  I  
	\end{array}
	& 
	\begin{array}{c}
	\\ \\ \\ \\ \\
	\fraz{A}
	     {\exists x A^{t}_{x}}
	\end{array}
	&
	\begin{array}{l}
	\\ \\ \\ \\ \\
	\exists E 
	\end{array}
	&
	\begin{array}{cc}
	\\ \\
	&{[A^{x}_{a}]}\\
	&\vdots\\
	\exists x A & B\\
	\hline\\
	B
	\end{array}
	\\ 
	&&& \\ \hline
	\end{array}
	\]
\caption{ND inference rules}
\label{fig-nd}
\end{figure}

\renewcommand{\arraystretch}{1}


\subsection{{\GF} sorted logic}

In section \ref{sec-ndrules} we described the inference rules for the unsorted
logic. 
However, the sort information present in the language has to be taken into
account when performing deduction.
The only natural deduction rules that have to be modified are the quantifier
introduction and elimination rules, where the sort of variables and terms
involved in the deduction substantially changes the applicability of the rules
and the conclusion.

The sorted ND rules are defined as follows:

$$
\begin{array}{ll}
\forall E & \left\{
\begin{array}{cl}
\fraz{\forall x A}{A^{x}_{t}} &
\mbox{if $t$ is of sort $S_x$} \\
& \\
\fraz{\forall x A}{S_x(t)\imp A^x_t} & \mbox{otherwise}
\end{array}
\right. % } to balance
\\

& \\

\forall I & \ps
\begin{array}{cl}
\fraz{A}
     {\forall x A^{a}_{x}}
& \mbox{applicable only if $x$ is of sort $S_a$}
\end{array} \\

& \\

\exists I & \left\{
\begin{array}{cl}
\fraz{A}
     {\exists x A^{t}_{x}}
& \mbox{if $t$ is of sort $S_x$} \\
& \\
\fraz{A^x_t}{S_x(t)\imp\exists x A}
& \mbox{otherwise}
\end{array} 
\right. % } to balance
\\

& \\
\exists E & \ps
\fraz{
\begin{array}{cc}
&{[A^{x}_{a}]}\\
&\vdots\\
\exists x A & B\\
\end{array}}
{B}
\pps
\mbox{applicable only if $x$ is of sort $S_a$}

\end{array}
$$

where $S_x$ and $S_a$ stand for the sort of $x$ and the sort of $a$
respectively.
The restrictions on the unsorted $\forall I$ and $\exists E$  rules apply also
to the rules above. 


% user commands for nd
\gfcommand{assume}{derive an assumption}
\index{assume}

\gfsyntax{
   assume \ARG{wff1} \OPT{\OPT{,} \ARG{wff2} \SEQ};
}

\gfdescription{
   For each \ARG{wffI}, this command builds an assumption as a new proof line
   and asserts it in the proof.
   An assumption is a proof line that depends on itself.
}

\gfrecap{
For each formula in the line, assume builds an assumption as a new proof line
and asserts it in the proof.
An assumption is a proof line that depends on itself.
}
 
\gfexample+
   ***** declare sentconst A B;
   ***** assume A and B;
   1   A and B     (1)
   
   ***** assume A and B    A imp B;
   2   A and B     (2)
   3   A imp B     (3)
+

\gfcommand{ande}{and elimination rule}
\index{ande}\index{ae}

\gfsyntax{
   ande \ALT ae \ARG{fact} \OPT{,} 1 \ALT 2; \\
   ande \ALT ae \ARG{fact} \OPT{,} 1 \ALT 2 1 \ALT 2 \SEQ;
}

\gfdescription{
   \[
      \con E  \ \ 
      \fraz{A \con B}
           {A}
      \ \ 
      \fraz{A \con B}
           {B}
   \]

   The wff of \ARG{fact} must be a conjunction.
   A proof line is deduced whose wff is the left conjunct (if 1 \ALT 2 is 1)
   or the right conjunct (if 1 \ALT 2 is 2). The fact inherits \ARG{fact}'s
   dependencies.

   In the second form, the command is applied to a fact whose wff is a
   recursive conjunction of wffs, {\it e.g.} {\tt A and ((B and C) and D)}.
   The sequence ``1 \ARG 2 1 \ARG 2 \SEQ" picks up the appropriate subformula.
}

\gfrecap{
Applies and elimination rule.
The formula given as arguments must be a conjunction.
A proof line is deduced whose wff is the left conjunct (if the number is 1)
or the right conjunct (if the number is 2). The fact inherits
the dependencies of `fact'.
In the second form, the command is applied to a fact whose wff is a
recursive conjunction of wffs, eg. `A and ((B and C) and D)'.
The sequence `1 | 2 1 | 2 ...' picks up the appropriate subformula.
}

\gfexample+
   ***** declare sentconst A B C D;
   ***** assume A and ((B and C) and D);
   1  A and ((B and C) and D)  (1)

   ***** ande 1 1;
   2   A  (1)

   ***** ande 1 2 1;
   3  B and C  (1)

   ***** ande 1 2 1 2;
   4  C  (1)
+

\gfnotes{
   A proof line derived by executing the second form can always be derived by
   a sequence of executions of commands in the first form.
}

\gfcommand{andi}{and introduction rule}
\index{andi}\index{ai}

\gfsyntax{
   andi \ALT ai \ARG{fact1} \OPT{,} \ARG{fact2}; \\
   andi \ALT ai \ARG{fact11}
                \OPT{conj \ALT cj \ARG{fact12} conj \ALT cj \ARG{fact13} \SEQ}
                \OPT{,}
                \ARG{fact21}
                \OPT{conj \ALT cj \ARG{fact22} conj \ALT cj \ARG{fact23} \SEQ};
}

\gfdescription{
   \[
   \con I \ \
   \fraz{A \ \ B}
        {A \wedge B}
   \]

   A proof line is derived whose wff is the conjunction of the wffs
   of \ARG{fact1} and \ARG{fact2} and whose dependencies are the union
   of the dependencies of \ARG{fact1} and \ARG{fact2}.

   In the second form we have ``conjunctions of facts'' rather than facts.
   A ``conjunction of facts'' is any parenthesized conjunctive expression in
   which all conjuncts are facts (\ARG{fact11} {\tt conj} \ARG{fact12} \SEQ).
   A proof line is derived whose wff is the conjunction of each $fact_{ij}$'s
   wff and whose dependencies are the union of the dependencies each
   $fact_{ij}$ depends on.
}

\gfrecap{
Applies `and' introduction to the given arguments.
In its first from a proof line is derived whose formula is the conjuction of
the formulae of `fact1' and `fact2' and whose dependencies are the union of
the dependencies of `fact1' and `fact2'.
In the second form we have ``conjunctions of facts" rather than facts.
A "conjunction of facts" is any parenthesized conjunctive expression in
which all conjuncts are facts (`fact11 conj fact12 ...').
A proof line is derived whose wff is the conjunction of each `factIJ''s
wff and whose dependencies are the union of the dependencies each
`factIJ' depends on.
}


\gfexample+
   ***** declare sentconst A B C D E;
   ***** assume A B;
   1   A     (1)
   2   B     (2)

   ***** andi 1 2;
   3   A and B     (1 2)

   ***** assume C D E;
   4   C     (4)
   5   D     (5)
   6   E     (6)

   ***** andi 1 conj 2   3;
   7   (A and B) and (A and B)     (1 2)

   ***** andi 1 conj 2   3 conj 4;
   8   (A and B) and ((A and B) and C)     (1 2 4)

   ***** andi 1 conj 2 conj 3   4;
   9   (A and (B and (A and B))) and C     (1 2 4)
+

\gfnotes{
   A proof line derived by executing the second form can always be derived by
   a sequence of executions of the command in the first form.
}

\gfcommand{falsee}{ falsity rule in intuitionistic logic}
\index{falsee}\index{fe}

\gfsyntax{
   falsee \ALT fe \ARG{fact1} \OPT{,} \ARG{wff};\\
   falsee \ALT fe \ARG{fact1} \OPT{,} \ARG{fact2};
}

\gfdescription{
   \renewcommand{\arraystretch}{0.5}
   \[
   \begin{array}{l}
      \bot_i  
   \end{array}
   \ \ 
   \begin{array}{c}
      \fraz{\bot}
           {A}
   \end{array}
   \]
   \renewcommand{\arraystretch}{1}

   \ARG{fact1} must have wff {\tt FALSE}.
   In the first form, the wff derived is \ARG{wff} and its dependencies are 
   those of \ARG{fact1}.
   In the second form, the wff of \ARG{fact2} is derived.
}

\gfrecap{
   `fact1' must have wff `FALSE'.
   In the first form, the wff derived is `wff' and its dependencies are 
   those of `fact1'.
   In the second form, the wff of `fact2' is derived.
}

\gfexample+
   ***** declare sentconst A;
   [...]

   ***** assume FALSE;
   1  FALSE  (1)

   ***** falsee 1 A and not A;
   2  A and (not A)  (1)

   ***** reset;
   [...]

   ***** declare sentconst A;
   [...]

   ***** assume not not A not A A;
   1  not not A  (1)
   2  not A  (2)
   3  A  (3)

   ***** falsei 1 2;
   4  FALSE  (1 2)

   ***** falsee 4 3;
   5  A  (1 2);

   ***** impi 1 5;
   6  not not A imp A (2)
+

\gfnotes{
   This rule says that anything follows from a contradiction.
}

\gfcommand{falsei}{implication elimination for negation or false introduction}
\index{falsei}\index{fi}

\gfsyntax{
   falsei \ALT fi \ARG{fact1} \OPT{,} \ARG{fact2}; 
}

\gfdescription{
   \renewcommand{\arraystretch}{0.5}
   \[
   \begin{array}{l}
      \imp E_{\neg}  
   \end{array}
   \ \ 
   \begin{array}{c}
      \fraz{A \ \ \neg A}
           {\bot}
   \end{array}
   \]
   \renewcommand{\arraystretch}{1}

   One fact must be the negation of the other. 
   The wff derived is {\tt FALSE} and its dependencies are the union of the
   dependencies of both facts.
}

\gfrecap{
   One fact must be the negation of the other. 
   The wff derived is `FALSE' and its dependencies are the union of the
   dependencies of both facts.
}

\gfexample+
   ***** declare sentconst A;
   [...]

   ***** assume A not A;
   1  A  (1)
   2  not A  (2)

   ***** falsei 1 2;
   3  FALSE  (1 2)

   ***** falsei 2 1;
   4  FALSE  (1 2)
+

\gfnotes{
   This rule can be seen as a special case of introduction elimination
   in which the main symbol of one of the premises must be {\tt not} ($\neg$)
   rather than {\tt imp} ($\imp$). 
}

\gfcommand{iffe}{equivalence elimination}.
\index{iffe}\index{ie}

\gfsyntax{
   iffe \ALT ie \ARG{fact} \OPT{,} 1 \ALT 2;
}

\gfdescription{
   \renewcommand{\arraystretch}{0.5}
   \[
   \begin{array}{l}
      \liff E
   \end{array}
   \ \ 
   \begin{array}{c}
      \fraz{A \liff B}
           {A \imp B}
   \end{array}
   \ \ \ 
   \begin{array}{c}
      \fraz{A \liff B}
           {B \imp A}
   \end{array}
   \]
   \renewcommand{\arraystretch}{1}

   If \ARG{fact}'s wff is of the form {\tt A iff B} then the first alternative
   gives {\tt A imp B}, the second {\tt B imp A}. The dependencies are those 
   of \ARG{fact}.
}

\gfrecap{
   If the formula of `fact' is of the form `A iff B' then the first alternative
   gives `A imp B', the second `B imp A'.
   The dependencies are those of `fact'.
}

\gfexample+
   ***** declare sentconst A;
   [...]

   ***** assume A iff not not A;
   1  A iff not not A  (1)

   ***** iffe 1 1;
   2  A imp not not A  (1)

   ***** iffe 1 2;
   2  not not A imp A  (1)
+

\gfcommand{iffi}{equivalence introduction}
\index{iffi}\index{ii}

\gfsyntax{
   iffi \ALT ii \ARG{fact1} \OPT{,} \ARG{fact2};
}

\gfdescription{
   \renewcommand{\arraystretch}{0.5}
   \[
   \begin{array}{l}
      \liff  I  
   \end{array}
   \ \ 
   \begin{array}{c}
      \fraz{A \imp B \ \ B \imp A}
           {A \liff B}
   \end{array}
   \]
   \renewcommand{\arraystretch}{1}

   Both facts must be implications. 
   If \ARG{fact1}'s wff is of the form {\tt A imp B}, then \ARG{fact2}'s wff 
   must be {\tt B imp A}.
   The conclusion is {\tt A iff B} with the union of the dependencies of 
   \ARG{fact1} and \ARG{fact2}.
}

\gfrecap{
Both facts must be implications. 
If the formula of `fact1' is of the form `A imp B', then the formula of
`fact2' must be `B imp A'.
The conclusion is {\tt A iff B} with the union of the dependencies of 
`fact1' and `fact2'.
}
   
\gfexample+
   ***** declare sentconst A B;
   [...]

   ***** assume A imp B B imp A;
   1  A imp B  (1)
   2  B imp A  (2)

   ***** iffi 1 2;
   3  A iff B  (1 2);

   ***** iffi 2 1;
   4  B iff A  (1 2)

   ***** reset;
   [...]

   ***** declare sentconst A;
   [...]

   ***** assume FALSE imp A A imp FALSE;
   1  FALSE imp A  (1)
   2  A imp FALSE  (2)

   ***** iffi 1 2;
   3  FALSE iff A  (1 2);
+

\gfcommand{impe}{implication elimination rule}
\index{impe}\index{mp}

\gfsyntax{
   impe \ALT mp \ARG{fact1} \OPT{,} \ARG{fact2};
}

\gfdescription{
   \renewcommand{\arraystretch}{0.5}
   \[
   \begin{array}{l}
      \imp E  
   \end{array}
   \ \  
   \begin{array}{c}
      \fraz{A \ \ A \imp B}
           {B}
   \end{array}
   \]
   \renewcommand{\arraystretch}{1}

   One of the two facts must be an implication and the other one
   must be its hypothesis.
   The order of the arguments is not relevant.
   The command creates a proof line whose wff is the conclusion of the
   implication and whose dependencies list is the union of the dependencies of
   \ARG{fact1} and \ARG{fact2}.
}

\gfrecap{
One of the two facts must be an implication and the other one must be its
hypothesis.
The order of the arguments is not relevant.
The command creates a proof line whose wff is the conclusion of the implication
and whose dependencies list is the union of the dependencies of `fact1' and
`fact2'.
}


\gfexample+
   ***** declare sentconst A B;
   ***** assume A A imp B;
   1  A  (1)
   2  A imp B  (2)

   ***** impe 1 2;
   3  B  (1 2)

   ***** impe 2 1;
   4  B  (1 2)
+

\gfcommand{impi}{implication introduction rule}
\index{impi}\index{ded}

\gfsyntax{
   impi \ALT ded  \ARG{fact1} \OPT{, \ALT imp} \ARG{fact};\\
   impi \ALT ded  \ARG{wff} \OPT{, \ALT imp} \ARG{fact};
}


\gfdescription{
   \renewcommand{\arraystretch}{0.5}
   \[
   \begin{array}{l}
      \\ \\ \\
      \imp I   
   \end{array}
   \ \  
   %
   \begin{array}{c}
      {[A]}\\
      \vdots\\
      B\\
      \hline\\
      A \imp B
   \end{array}
   \]
   \renewcommand{\arraystretch}{1}

   The wff derived is the implication of the wffs of \ARG{fact1} and
   \ARG{fact2}. It depends on all the dependencies of \ARG{fact2} less 
   {\em all the lines} whose wff is the same as \ARG{fact1}'s wff (in the 
   first form) or \ARG{wff} (in the second form).
}

\gfrecap{
The wff derived is the implication of the wffs of `fact1' and `fact2'.
It depends on all the dependencies of `fact2' less all the lines whose wff is
the same as `fact1''s wff (in the first form) or `wff' (in the second form).
}
   
\gfexample+
   ***** declare sentconst A;
   ***** assume A;
   1  A  (1)

   ***** impi 1 1;
   2  A imp A  

   ***** impi A 1;
   3  A imp A 
+

\gfcommand{note}{falsity rule in classical logic}.
\index{note}\index{ne}

\gfsyntax{
   note \ALT ne \ARG{fact1} \OPT{,} \ARG{wff}; \\
   note \ALT ne \ARG{fact1} \OPT{,} \ARG{fact2};
}

\gfdescription{
   \renewcommand{\arraystretch}{0.5}
   \[
   \begin{array}{l}
      \\ \\ 
      \bot_c 
   \end{array}
   \ \ 
   \begin{array}{c}
      {[\neg A]}\\
      \vdots\\
      \bot\\
      \hline\\
      A 
   \end{array}
   \]
   \renewcommand{\arraystretch}{1}

   The wff of \ARG{fact1} must be {\tt FALSE} and \ARG{wff} must be a negation
   of the form $\neg A$.
   The wff derived is $A$.
   It depends on all the dependencies of \ARG{fact1} less the dependencies of
   all those facts whose wff is equal to \ARG{wff} ($\neg A$) (in the first) 
   form or \ARG{fact1}'s wff (in the second form).
}

\gfrecap{
   The wff of `fact1' must be `FALSE' and `wff' must be a negation
   of the form `not A'.
   The wff derived is `A'.
   It depends on all the dependencies of `fact1' less the dependencies of
   all those facts whose wff is equal to `wff' (`not A') (in the first) 
   form or the formula of `fact1' (in the second form).
}

\gfexample+
   ***** declare sentconst A;
   [...]

   ***** assume not A not not A;
   1   not A     (1)
   2   not (not A)     (2)

   *****  falsei 1 2;
   3   FALSE     (1 2)

   *****  note 3 not A;
   4   A     (2)

   *****  impi 2 4;
   5   (not (not A)) imp A 
+

\gfnotes{
   This rule implements ``reductio ad absurdum''. 
   Any proof using only the ND commands but not this rule is valid
   intuitionistically. 
}

\gfcommand{noti}{implication introduction for negation}
\index{noti}\index{ni}

\gfsyntax{
   noti \ALT ni \ARG{fact1} \OPT{,} \ARG{wff}; \\
   noti \ALT ni \ARG{fact1} \OPT{,} \ARG{fact2};
}

\gfdescription{
   \renewcommand{\arraystretch}{0.5}
   \[
   \begin{array}{l}
      \\ \\ 
      \imp I_{\neg}  
   \end{array}
   \ \ 
   \begin{array}{c}
      {[A]}\\
      \vdots\\
      \bot\\
      \hline\\
      \neg A
   \end{array}
   \]
   \renewcommand{\arraystretch}{1}

   The wff of \ARG{fact1} must be {\tt FALSE}.
   The command creates a fact whose wff is the negation of \ARG{wff}.
   It depends on all the dependencies of \ARG{fact1} less all the dependencies
   whose wff is equal to \ARG{wff} (in the first form) or \ARG{fact2}'s wff
   (in the second form).
}

\gfrecap{
   The wff of `fact1' must be `FALSE'.
   The command creates a fact whose wff is the negation of `wff'.
   It depends on all the dependencies of `fact1' less all the dependencies
   whose wff is equal to `wff' (in the first form) or the formula of 
   `fact2' (in the second form).
}
   
\gfexample+
   ***** declare sentconst A;
   [...]

   ***** assume A not A;
   1  A  (1)
   2  not A  (2)

   *****  falsei 1 2;
   3  FALSE  (1 2)

   *****  noti 3 not A;
   4  not not A  (1)

   *****  impi 1 4;
   5  A imp not not A
+

\gfnotes{
   Since $\neg A$ ai equivalent to $A \imp \bot$,
   this rule can be seen as a special case of the implication introduction
   rule ({\tt impi}), in which  the asserted line has \verb+not+ ($\neg$) as
   main symbol.
}

\gfcommand{ore}{or elimination rule}.
\index{ore}\index{oe}

\gfsyntax{
   ore \ALT oe \ARG{fact1} \OPT{,} \ARG{fact2} \OPT{,} \ARG{fact3};
}

\gfdescription{
   \renewcommand{\arraystretch}{0.5}
   \[
   \begin{array}{l}
      \\ \\ 
      \dis E  
   \end{array}
   \ \ 
   \begin{array}{ccc}
      &{[A]}&{[B]}\\
      &\vdots&\vdots\\
      A \dis B & C & C\\
      \hline\\
      & C &
   \end{array}
   \]
   \renewcommand{\arraystretch}{1}

   Let \ARG{wff1}, \ARG{wff2} and \ARG{wff3} be the formulas of \ARG{fact1}, 
   \ARG{fact2}, \ARG{fact3} respectively; let {\em wff1} be a disjunction;
   let \ARG{wff2} and \ARG{wff3} be the same formula.
   Then the conclusion is \ARG{wff2} (\ARG{wff3}) with the dependencies of
   \ARG{fact1} along with those of \ARG{fact2} whose wff is not equal to the
   left disjunct of \ARG{wff1} and those of \ARG{fact3} whose wff is not equal
   to the right disjunct of \ARG{wff1}. 
}

\gfrecap{
Let `wff1', `wff2' and `wff3' be the formulas of `fact1', 
`fact2', `fact3' respectively; let {\em wff1} be a disjunction;
let `wff2' and `wff3' be the same formula.
Then the conclusion is `wff2' (`wff3') with the dependencies of
`fact1' along with those of `fact2' whose wff is not equal to the
left disjunct of `wff1' and those of `fact3' whose wff is not equal
to the right disjunct of `wff1'. 
}

\gfexample+
   ***** declare sentconst A B C;
   [...]

   ***** assume B imp A;
   1  B imp A (1)

   ***** assume C imp A;
   2  C imp A (2);

   ***** assume B;
   3  B  (3);

   ***** assume C;
   4  C  (4);

   ***** impe 3 1;
   5  A  (1 3)

   ***** impe 4 2;
   6  A  (2 4)

   ***** assume B or C;
   7  B or C  (7)

   ***** ore 7 5 6;
   8  A   (1 2 7)

   ***** impi 7 8;
   9  (B or C) imp A   (1 2)

   ***** ore 7 6 5;
   10  A   (1 2 3 4 7)
+

\gfcommand{ori}{or introduction rule}
\index{ori}\index{oi}

\gfsyntax{
   ori \ALT oi \ARG{fact} \OPT{,} \ARG{wff} \OPT{,} \OPT{lr \ALT rl};\\
   ori \ALT oi \ARG{fact} \OPT{,} \ARG{fact1} \ALT \ARG{wff1} 
       disj \ALT dj \ARG{fact2} \ALT \ARG{wff2} disj \ALT dj \SEQ
       \OPT{,} \OPT{lr \ALT rl};
}

\gfdescription{
   \renewcommand{\arraystretch}{0.5}
   \[
   \begin{array}{l}
      \dis I
   \end{array}
   \ \ 
   \begin{array}{c}
   \fraz{A}
        {A \dis B}
   \end{array}
   \ \ \ 
   \begin{array}{c}
   \fraz{A}
        {B \dis A}
   \end{array}
   \]
   \renewcommand{\arraystretch}{1}

   The command creates a new fact whose wff is the disjunction of \ARG{fact}'s
   wff and \ARG{wff}.
   The option {\tt lr \ALT rl} specifies the order of the disjuncts:
   {\bf lr} stands for ``left-right'' and it means that the left disjunct is 
   \ARG{fact}'s wff and the right one is \ARG{wff}; {\tt rl} viceversa.
   If no order is specified, then {\tt lr} is the default.
   The new proof line inherits \ARG{fact}'s dependencies.

   In the second form, the command accepts ``disjunctions of facts and wffs''
   as second argument.
   A ``disjunctions of facts and wffs'' is any parenthesized disjunctive
   expression in which all disjuncts are facts (\ARG{fact1} {\tt disj} 
   \ARG{fact2} \SEQ), wffs (\ARG{wff1} {\tt disj} \ARG{wff2} \SEQ)
   or facts and wffs (\ARG{fact1} {\tt disj} \ARG{wff2} \SEQ).
   The derived formula is the disjunction of the formulae.
   It depends on the assumptions \ARG{fact} and  all the \ARG{factI}s 
   depend on.
}

\gfrecap{
The command creates a new fact whose wff is the disjunction of `fact''s
formula and `wff'.
The option `lr \ALT rl' specifies the order of the disjuncts:
`lr' stands for ``left-right" and it means that the left disjunct is 
the formula of `fact' and the right one is `wff'; `rl' viceversa.
If no order is specified, then `lr' is the default.
The new proof line inherits dependencies of `fact'.
In the second form, the command accepts ``disjunctions of facts and wffs"
as second argument.
A ``disjunctions of facts and wffs" is any parenthesized disjunctive
expression in which all disjuncts are facts (`fact1 disj fact2 ...), 
wffs (wff1 disj wff2 ...) or facts and wffs (fact1 disj wff2 ...).
The derived formula is the disjunction of the formulae.
It depends on the assumptions `fact' and  all the `factI's 
depend on.
}

\gfexample+
   ***** declare sentconst A B;
   ...

   ***** assume A;
   1  A  (1);
   ***** ori 1 B;
   2  A or B  (1)
   ***** ori 1 B lr ;
   3  A or B (1)
   ***** ori 1 B rl ;
   4  B or A  (1)

   ***** reset;
   ***** declare sentconst A B C D E;
   ...

   ***** assume A B C;
   1  A  (1);
   2  B  (2);
   3  C  (3);

   ***** ori 1 B dj C rl;
   4  (B or C) or A   (1)

   ***** ori 1 2 dj 3;
   5  A or (B or C)   (1)

   ***** ori 1 2 dj D dj 3 dj E rl;
   6  (B or (D or (C or E))) or A   (1)
+

\gfnotes{
   A proof line derived by executing the second form can always be derived by
   a sequence of executions of the command in the first form.
}

\gfcommand{alle}{universal quantification elimination rule}
\index{alle}\index{us}

\gfsyntax{
   alle \ALT us \ARG{fact} \OPT{,} \ARG{term1} \ARG{term2} \SEQ;
}

\gfdescription{
   \renewcommand{\arraystretch}{0.5}
   \[
   \begin{array}{l}
      \\
      \forall E
   \end{array}
   \ \ 
   \begin{array}{c}
      \fraz{\forall x A}
           {A^{x}_{t}}
   \end{array}
   \]
   \renewcommand{\arraystretch}{1}

   The command uses the terms in \ARG{term1} \ARG{term2} \SEQ to instantiate
   the universally quantified variables in the order in which they appear.
   One execution of the command can instantiate more than one universally 
   quantified variable.
   If a particular term is not free for the variable to be instantiated, a 
   bound variable change is made and then the substitution is made.
   The dependencies are those of \ARG{fact}.

   In the sorted rule, let \verb+forall x. A(x)+ be the formula to be 
   instantiated, and let us suppose first that only a term {\tt t} is provided.
   If {\tt t} has he same sort as {\tt x}, say {\tt Sx}, then the resulting 
   formula is {\tt A(t)}; otherwise the resulting formula is
   {\tt Sx(t) imp A(t)}.
}

\gfrecap{
The command uses the terms in `term1' `term2' ... to instantiate
the universally quantified variables in the order in which they appear.
One execution of the command can instantiate more than one universally 
quantified variable.
If a particular term is not free for the variable to be instantiated, a 
bound variable change is made and then the substitution is made.
The dependencies are those of `fact'.
In the sorted rule, let `forall x. A(x)' be the formula to be 
instantiated, and let us suppose first that only a term `t' is provided.
If `t' has he same sort as `x', say `Sx', then the resulting 
formula is `A(t)'; otherwise the resulting formula is
`Sx(t) imp A(t)'.
}  

\gfexample+
   ***** declare predconst P 2;
   ***** declare indvar x y;
   ***** declare indconst c1 c2;
   [...]

   ***** assume forall x y. P(x y);
   1   forall x y.P(x,y)     (1)

   ***** alle 1 c1;
   2   forall y.P(c1,y)     (1)

   ***** alle 1 x c1;
   3   P(x,c1)     (1)

   ***** alle 1 c1 c2;
   4   P(c1,c2)     (1)
   
   ***** declare predconst P 2;
   ***** declare indvar x [Sx];
   ***** declare indvar y [Sy];
   ***** declare indconst c1 c2 [S];
   [...]

   ***** moregeneral Sx < S >;

   ***** assume forall x y. P(x y);
   1   forall x y.P(x,y)     (1)

   ***** alle 1 c1;
   2   forall y.P(c1,y)     (1)

   ***** alle 2 c1;
   3   Sy(c1) imp P(c1,c1)     (1)

   ***** alle 1 c1 c2;
   4   Sy(c2) imp P(c1,c2)     (1)
+

\gfnotes{
   The rule is sometimes called ``universal specialization''.
}

\gfcommand{alli}{universal quantification introduction rule}
\index{alli}\index{ug}

\gfsyntax{
   alli \ALT ug \ARG{fact} \OPT{\OPT{,} \ARG{indvar1} \ALT \ARG{indpar1} :} 
                           \ARG{indvar11}
                           \OPT{\OPT{,} \ARG{indvar2} \ALT \ARG{indpar2} :}
                           \ARG{indvar22} \SEQ;
}

\gfdescription{
   \renewcommand{\arraystretch}{0.5}
   \[
   \begin{array}{l}
      \forall  I  
   \end{array}
   \ \ 
   \begin{array}{c}
      \fraz{A}
      {\forall x A^{a}_{x}}
   \end{array}
   \]
   \renewcommand{\arraystretch}{1}

   There is the usual ND restriction on the application of this rule, namely
   the newly quantified variable must not appear free in any of the 
   dependencies of \ARG{fact}.

   Several simultaneous universal generalizations on \ARG{fact}'s wff
   can be carried out with this command.
   Each element of the substitution list may be either an individual variable
   (e.g. {\tt x}) or a pair.
   The pair elements are two individual variables ({\tt y:x}) or an individual
   parameter and an individual variable ({\tt a:x}).
   For each element in the substitution list a new universal quantifier is put
   at the front of \ARG{fact}'s wff. 
   In the case of pairs of the kind {\tt a:x},  the variable {\tt x} is
   substituted for all occurrences of the individual parameter {\tt a}
   which are not within the scope of the universal quantification of {\tt x}. 
   The individual parameter {\tt a} must not occur within the scope of a
   quantifier binding {\tt x}, otherwise an error message is returned. 

   The dependencies of the new created proof line are the same as those of
   \ARG{fact}.

   The rule with sorts must also satisfy the following restriction.
   Let us first consider a substitution list whose only substitution is
   {\tt a:x} or {\tt y:x}, where {\tt x} is a variable of sort {\tt Sx}.
   Then the sorted rule is applicable only if {\tt a} and {\tt y} are terms
   of sort weakly more general than {\tt Sx}.
   In the general case, the specified condition must hold for all the
   substitutions of the substitution list.
}

\gfrecap{
   Applies introduction of universal quantification.
   The usual natural deduction's restrictions apply.
}

\gfexample+
   ***** comment ! An example with generalization from individual parameters
                   to quantified variables !
        
   ***** declare predconst P 1;
   ***** declare indvar x;
   ***** declare indpar a;
   [...]

   ***** assume P(a);
   1   P(a)     (1)
   ***** alli 1 a:x;
   alli 1 a:x;
   Some variables appear free in the assumptions.
   ***** impi 1 1;
   2   P(a) imp P(a)
   ***** alli 2 a:x;
   3   forall x.(P(x) imp P(x))
   ***** alli 2 x;
   4   forall x.(P(a) imp P(a))
    
     
   ***** comment ! The same example when substituting more than one variable !
     
   ***** reset;
   ***** declare predconst Q 2;
   ***** declare indvar x y;
   ***** declare indpar a b;
   [...]

   ***** assume Q(a b);
   1   Q(a,b)     (1)
   ***** impi 1 1;
   2   Q(a,b) imp Q(a,b)
   ***** alli 2 a:x b:y;
   3   forall x y. (Q(x,y) imp Q(x,y))


   ***** comment ! An example with generalization from free variables
                   to quantified variables !

   ***** reset;
   ***** declare predconst P 1;
   ***** declare indvar x;
   [...]

   ***** assume P(x);
   1   P(x)     (1)
   ***** impi 1 1;
   2   P(x) imp P(x)
   ***** alli 2 x:x;
   3   forall x.(P(x) imp P(x))
   ***** alli 2 x;
   4   forall x.(P(x) imp P(x))
     

   ***** comment ! An example of non applicability of generalization from
                   individual parameters to quantified variables with sorts !
      
   ***** declare predconst P 1;
   ***** declare indvar x [Sx];
   ***** declare indpar a [Sa];
   [...]

   ***** assume P(a);
   1   P(a)     (1)

   ***** alli 1 a:x;
   alli 1 a:x;
   Some variables appear free in the assumptions.

   ***** impi 1 1;
   2   P(a) imp P(a)

   ***** alli 2 a:x;
   alli 2 a:x
   A <var> cannot be replaced by one with more general sort

   ***** moregeneral Sa < Sx >;

   ***** alli 2 a:x;
   3   forall x. (P(x) imp P(x))
+


\gfnotes{
   The rule is often called ``universal generalization".
}

\gfcommand{existe}{existential quantification elimination rule}
\index{existe}\index{es}

\gfsyntax{
   existe \ALT es \ARG{fact}  \OPT{,} \ARG{indvar1} \ALT \ARG{indpar1}
                  \OPT{,} \ARG{indvar2} \ALT \ARG{indpar2} \SEQ;
}


\gfdescription{
   \renewcommand{\arraystretch}{0.5}
   \[
   \begin{array}{l}
      \\ \\ \\ \\
      \exists E 
   \end{array}
   \ \ 
   \begin{array}{cc}
      \\ \\
      &{[A^{x}_{a}]}\\
      &\vdots\\
      \exists x A & B\\
      \hline\\
      B
   \end{array}
   \]
   \renewcommand{\arraystretch}{1}

   The implementation of this rule is the most radically different
   from the formal statement given in Prawitz's ND. This rule corresponds, in
   informal reasoning, to the following kind of argument:
   suppose we have shown that something exists with some particular
   property, e.g. $\exists y P(a,y)$. Then we say ``call this thing $b$''.
   This is like saying assume $P(a,b)$. Then we can reason about $b$.
   As soon as we have a sentence, however, that no longer mentions $b$,
   it does not depend on what we called ``$y$'',
   but only on the dependencies of the existential statement we started
   with. Thus we can discharge $P(a,b)$ from the dependencies
   and replace them with those of $\exists y P(a,y)$. {\GF} thus makes the 
   correct assumption for you, remembers it and automatically removes it at
   the first legitimate opportunity.

   The only difference with sorts is the following:
   the existentially quantified variable must be substituted by a variable
   or a parameter of weakly more general sort.

   In the example below, an existential elimination is done creating step 
   {\tt 6}.
   This fact actually has as {\tt reason}  (see
   subsection~\ref{sec-proofline}) that it was assumed.
   Fact {\tt 8} thus depends on {\tt 6}. When the existential generalization
   is done on the next fact, {\tt b} no longer appears and so fact {\tt 6}
   is removed from the dependencies of fact {\tt 9}. The user should
   convince himself that the {\GF} logic is equivalent to the ND definition
   given at the beginning of the section.
}

\gfrecap{
The implementation of this rule is the most radically different
from the formal statement given in Prawitz's ND.
This rule corresponds, in informal reasoning, to the following kind of 
argument: suppose we have shown that something exists with some particular
property, eg. `exists y P(a,y)'.
Then we say ``call this thing `b''': this is like saying assume `P(a,b)'.
Then we can reason about `b'.
As soon as we have a sentence, however, that no longer mentions `b',
it does not depend on what we called `y', but only on the dependencies of the
existential statement we started with.
Thus we can discharge `P(a,b)' from the dependencies and replace them with 
those of `exists y P(a,y)'.
GETFOL thus makes the correct assumption for you, remembers it and 
automatically removes it at the first legitimate opportunity.
The only difference with sorts is the following: the existentially quantified
variable must be substituted by a variable or a parameter of weakly more
general sort.
}

\gfexample+
   ***** comment ! Sorted example !
     
   ***** declare indvar x y;
   ***** declare indpar a b;
   ***** declare predconst P 2;
   [...]

   ***** assume forall x.exists y.P(x y) and forall x y.(P(x y) imp P(y x));
   1   forall x.exists y.P(x,y) and forall x y.(P(x,y) imp P(y,x))     (1)

   ***** ande 1 1;
   2   forall x.exists y.P(x,y)     (1)

   ***** ande 1 2;
   3   forall x y.(P(x,y) imp P(y,x))     (1)

   ***** alle 2 a;
   4   exists y.P(a,y)     (1)

   ***** alle 3 a b;
   5   P(a,b) imp P(b,a))     (1)

   ***** existe 4 b;
   6   P(a,b)     (6)

   ***** impe 6 5;
   7   P(b,a)     (1 6)

   ***** andi 6 7;
   8   P(a,b) and  P(b,a)     (1 6)

   ***** existi 8 b:y;
   9   exists y.(P(a,y) and P(y,a)     (1)

   ***** alli 9 a:x;
   10   forall x.exists y.(P(x,y) and P(x,x)     (1)

   ***** impi 1 10;
   11   forall x.exists y.P(x,y) and 
        forall x y.(P(x,y) imp P(y,x)) imp 
        forall x.exists y.(P(x,y) and P(x,y)
    
       
   ***** comment ! Sorted example !
    
   ***** reset;
   [...]

   ***** declare indvar x [Sx];
   ***** declare indpar p [S];
   ***** declare predconst A 1;
   ***** declare sentconst B;
   [...]

   ***** axiom ONE: exists x.A(x);
   ***** axiom TWO: A(p) imp B;
   ONE : exists x. A(x)
   TWO : A(p) imp B

   ***** existe ONE p;
   A <var> must be replaced by one with more general sort;

   ***** moregeneral S < Sx >;

   ***** existe ONE p;
   1   A(p)     (1)

   ***** impe TWO 1;
   2   B
+

\gfnotes{
   The rule is often called ``existential instantiation".
}
\gfcommand{existi}{existential quantification introduction rule}
\index{existi}\index{ei}

\gfsyntax{
  existi \ARG{fact}
   \OPT{\OPT{,} \ARG{term1} :} \ARG{indvar1} \OPT{occ \ARG{n11} \ARG{n12} \SEQ}
   \OPT{\OPT{,} \ARG{term2} :} \ARG{indvar2} \OPT{occ \ARG{n21} \ARG{n22} \SEQ}
   \SEQ;
}


\gfdescription{
  \renewcommand{\arraystretch}{0.5}
  \[
  \begin{array}{l}
    \exists  I  
  \end{array}
  \ \ 
  \begin{array}{c}
    \fraz{A}
    {\exists x A^{t}_{x}}
  \end{array}
  \]
  \renewcommand{\arraystretch}{1}

  The list following \ARG{fact} indicates which terms are to be 
  existentialized.
  If the optional \ARG{termI} is present, it is replaced by \ARG{indvarI}
  at each occurrence mentioned in the sequence of natural numbers
  \OPT{$n_{i1}$ $n_{i2}$ \SEQ} and then  existentialized.
  If \ARG{termI} is not present, all the occurrences of \ARG{indvarI} are
  put under the scope of the existential quantifier.
  Notice that no use can be made of an occurrence specification
  \OPT{$n_{i1}$ $n_{i2}$ \SEQ} if there is no \ARG{termI} present,
  {\GF} will return an error in this case.
  The dependencies of the conclusion are those of \ARG{fact}.

  In the sorted rule, let {\tt A(t)} be the formula to be existentialized,
  and let {\tt x} be of sort {\tt Sx}.
  Then the result of the rule is {\tt exists x.A(x)} if {\tt t} is of sort
  {\tt Sx}, {\tt Sx(t) imp exists x.A(x)} otherwise.
  This schema applies also to the case where multiple terms are to be
  existentialized.
}

\gfrecap{
The list following `fact' indicates which terms are to be existentialized.
If the optional `termI' is present, it is replaced by `indvarI' at each 
occurrence mentioned in the sequence of natural numbers [nI1 nI2 ...] and then
existentialized.
If `termI' is not present, all the occurrences of `indvarI' are put under the 
scope of the existential quantifier.
Notice that no use can be made of an occurrence specification [nI1 nI2 ...] if
there is no `termI' present, GETFOL will return an error in this case.
The dependencies of the conclusion are those of `fact'.
In the sorted rule, let `A(t)' be the formula to be existentialized, and let
`x' be of sort `Sx'.
Then the result of the rule is `exists x. A(x)' if `t' is of sort `Sx',
`Sx(t) imp exists x. A(x)' otherwise.
This schema applies also to the case where multiple terms are to be
existentialized.
}  

\gfexample+
   *****  comment ! unsorted example !
       
   ***** declare predconst P 2;
   ***** declare indvar x y x0;
   ***** declare indconst c1 c2;
   [...]

   ***** assume P(c1 c2);
   1   P(c1,c2)     (1)

   ***** existi 1 c1:x c2:y;
   2   exists y x.P(x,y)     (1)

   ***** existi 1 c1:x c1:x;
   3   exists x x.P(x,c2)     (1)

   ***** existi 1 c1:x c2:x0;
   4   exists x x0.P(x0,x)     (1)

   ***** assume P(c1 c1);
   5   P(c1,c1)     (5)

   ***** existi 5 c1:x 2;
   6   exists x.P(c1,x)     (5)
    
    
   ***** comment ! sorted example !
    
   ***** declare predconst P 2;
   ***** declare indvar x [Sx];
   ***** declare indvar y [Sy];
   ***** declare indconst c1 c2 [S];
   [...]

   ***** moregeneral Sx < S >;

   ***** assume P(c1 c2);
   1   P(c1,c2)     (1)

   ***** existi 1 c1:x c2:y;
   2   Sy(c2) imp exists y x. P(x,y)     (1)
+


% introduction to the substitution rule
\newpage
\section{Equality rules}

We describe here the rules for substitution of equality, $sub_l$ and $sub_r$.
{\GF} does not have explicit axioms or rules for reflexivity, commutativity and
transitivity.
Commutativity and transitivity, however, can be derived by $sub_l$ and $sub_r$,
and symmetry can be derived by using other {\GF} commands (for instance by
semantic simplification (see section \ref{sec-eval}), by the tautology and
monadic deciders (see the commands {\tt tauteq} and {\tt monadeq} commands in
section \ref{sec-decide}).


\[
\begin{array}{|c|c||c|c|} \hline 
	&&& \\
	sub_l &
	\fraz{A(t_1) \ \ \ \ t_1 = t_2}
	     {A(t_2)}
	&
	sub_r &
	\fraz{A(t_1) \ \ \ \ t_2 = t_1}
	     {A(t_2)}
	\\
	&&& \\ \hline
\end{array}
\]


% user command for substitution
\gfcommand{subst}{equality substitution}
\index{subst}

\gfsyntax{
  subst \ARG{fact1} \OPT{with} \ARG{fact2};\\
  subst \ARG{fact1} \OPT{with} \ARG{fact2} \OPT{right \ALT left};\\
  subst \ARG{fact1} \OPT{with} \ARG{fact2} 
                    \OPT{occ \ARG{n1} \ARG{n2} \SEQ} \OPT{right \ALT left};
}

\gfdescription{
  \[
  \begin{array}{cc}
    sub_l \ \ 
    \fraz{A(t_1) \ \ \ \ t_1 = t_2}
    {A(t_2)}
    \ \ \ 
    &
    \ \ \ 
    sub_r \ \ 
    \fraz{A(t_1) \ \ \ \ t_2 = t_1}
    {A(t_2)}
  \end{array}
  \]

  The rule substitutes a term in \ARG{fact1}'s wff with another one proved to 
  be equal to the former.
  \ARG{fact2} must be the equality {\tt t1 = t2} and \ARG{fact1} may contain 
  one or more occurrences of {\tt t1}.
  The conclusion is the result of the substitution of {\tt t1} with {\tt t2} in
  \ARG{fact1}'s wff. 
  Dependencies of derived fact are the union of those of \ARG{fact1} and
  \ARG{fact2}.
  If \ARG{fact1} does not contain {\tt t1}, then the conclusion has the same 
  wff and dependencies as \ARG{fact1}.

  The default is the substitution of {\tt t2} with {\tt t1}, corresponding to 
  the option {\bf left}. If {\bf right} is indicated, then {\tt t1} is 
  substituted with {\tt t2}.

  Individual occurrences can be substituted by specifying the optional
  \OPT{{\tt occ} \ARG{n1} \ARG{n2} \SEQ}, where \ARG{n1}, \ARG{n2}, \SEQ are
  the occurrences to be substituted.
  Without this option, all occurrences are substituted. 
}

\gfrecap{
The rule substitutes a term in the formula of `fact1' with another one
proved to be equal to the former.
`fact2' must be the equality `t1 = t2' and `fact1' may contain one or more
occurrences of `t1'.
The conclusion is the result of the substitution of `t1' with `t2' in the
formula of `fact1'. 
Dependencies of derived fact are the union of those of `fact1' and `fact2'.
If `fact1' does not contain `t1', then the conclusion has the same wff and
dependencies as `fact1'.
The default is the substitution of `t2' with `t1', corresponding to the option
`left'.
If `right' is indicated, then `t1' is substituted with `t2'.
Individual occurrences can be substituted by specifying the optional
`[occ n1 n2 ...]', where `n1', `n2', ... are the occurrences to be substituted.
Without this option, all occurrences are substituted. 
}

\gfexample+
   ***** declare predconst P Q 2;
   ***** declare funconst f 1; 
   ***** declare indvar x; declare indvar y;
   [...]

   ***** assume P(x y) imp Q(y x);
   1   P(x,y) imp Q(y,x)     (1)

   ***** assume x = f(x);
   2   x = f(x)     (2)

   ***** subst 1 2;
   3   P(f(x),y) imp Q(y,f(x))     (1 2)

   ***** subst 3 2 right;
   4   P(x,y) imp Q(y,x)     (1 2)

   ***** subst 1 2 occ 1;
   5   P(f(x),y) imp Q(y,x)     (1 2)
+




	% introduction to the section
\newpage
\section{Other rules}

\subsection{Conditional rules}
\label{sec-cond}

Conditional rules allow us to introduce and eliminate the conditional wffs
{\wffif} and conditional terms {\termif}.

\begin{bnf}
	{\wffif}  \sep {\bf wffif} {\wff}$_1$ {\bf then} {\wff}$_2$ {\bf else}
				   {\wff}$_3$ \\
	{\termif} \sep {\bf trmif} {\wff} {\bf then} {\term}$_1$ {\bf else}
				   {\term}$_2$
\end{bnf}

Note that the {\termif} construct is not first order.
However, it can easily be shown that {\termif} can be defined as a conservative
extension of first order logic using an induction on the length of deductions.

\renewcommand{\arraystretch}{0.5}
\[
\begin{array}{|c|c||c|c|} \hline 
\multicolumn{2}{|c||}{\mbox{\bf Introduction rules}} &
\multicolumn{2}{c|}{\mbox{\bf Elimination rules}} \\ \hline
\begin{array}{l}
\\ \\ \\ \\ \\
\mbox{{\em wif}} \  I  
\end{array}
&
\begin{array}{cc}
\\ \\
\begin{array}{c}
{[A]}\\
\vdots\\
B
\end{array}
\ \ \ 
\begin{array}{c}
{[\neg A]}\\
\vdots\\
C
\end{array}
\\
\hline\\
\begin{array}{c}
\mbox{{\em wffif}} \ A \ \mbox{{\em then}} \ B \ \mbox{{\em else}} \ C 
\end{array}
\end{array}
&
\begin{array}{l}
\\ \\ \\ \\ \\
\mbox{{\em wif}} \  E
\end{array}
&
\begin{array}{cc}
\\ \\
\begin{array}{c}
\\ \\ \\ \\
A
\end{array}
\ \ \ 
\begin{array}{c}
\\ \\ \\ \\
\mbox{{\em wffif}} \ A \ \mbox{{\em then}} \ B \ \mbox{{\em else}} \ C 
\end{array}
\\
\hline\\
\begin{array}{c}
B
\end{array}
\end{array}
%\fraz{\Gamma \vdash A \ \ \Delta \vdash \mbox{{\em wffif}} \ A \ \mbox{{\em then}} \ B \ 
%\mbox{{\em else}} \ C}
%     {\Gamma,\Delta \vdash B}
\\ %\hline 
& &
\begin{array}{l}
\\ \\ \\ \\ \\
\mbox{{\em wif}} \  E_{\neg}
\end{array}
&
\begin{array}{cc}
\\ \\
\begin{array}{c}
\\ \\ \\ \\
\neg A
\end{array}
\ \ \ 
\begin{array}{c}
\\ \\ \\ \\
\mbox{{\em wffif}} \ A \ \mbox{{\em then}} \ B \ \mbox{{\em else}} \ C 
\end{array}
\\
\hline\\
\begin{array}{c}
C
\end{array}
\end{array}
\\ %\hline
\begin{array}{l}
\\ \\ \\ \\ \\
\mbox{{\em tif}} \  I    
\end{array}
&
\begin{array}{cc}
\\ \\
\begin{array}{c}
{[A]}\\
\vdots\\
B(t_1)
\end{array}
\ \ \ 
\begin{array}{c}
{[\neg A]}\\
\vdots\\
B(t_2)
\end{array}
\\
\hline\\
\begin{array}{c}
B(\mbox{{\em termif}} \ A \ \mbox{{\em then}} \ t_1 \ \mbox{{\em else}} \ t_2)
\end{array}
\end{array}
&
\begin{array}{l}
\\ \\ \\ \\ \\
\mbox{{\em tif}} \  E
\end{array}
&
\begin{array}{cc}
\\ \\
\begin{array}{c}
\\ \\ \\ \\
A
\end{array}
\ \ \ 
\begin{array}{c}
\\ \\ \\ \\
B(\mbox{{\em termif}} \ A \ \mbox{{\em then}} \ t_1 \ \mbox{{\em else}} \ t_2)
\end{array}
\\
\hline\\
\begin{array}{c}
B(t_1)
\end{array}
\end{array}
\\ %\hline 
& &
\begin{array}{l}
\\ \\ \\ \\ \\
\mbox{{\em tif}} \  E_{\neg}
\end{array}
&
\begin{array}{cc}
\\ \\
\begin{array}{c}
\\ \\ \\ \\
\neg A
\end{array}
\begin{array}{c}
\\ \\ \\ \\
B(\mbox{{\em termif}} \ A \ \mbox{{\em then}} \ t_1 \ \mbox{{\em else}} \ t_2)
\end{array}
\\
\hline\\
\begin{array}{c}
B(t_2)
\end{array}
\end{array}
\\ 
& & & \\ \hline 
\end{array}
\]
\renewcommand{\arraystretch}{1}


\subsection{Structural rules}

Structural rules are useful when performing theorem proving.

% user commands for conditional rules
\gfcommand{termife}{term conditional elimination}
\index{termife}

\gfsyntax{
	termife \ARG{fact1} \ARG{fact2} \ARG{termif}; \\
	termife \ARG{fact1} \ARG{fact2} \ARG{termif} \OPT{occ \ARG{n1} \ARG{n2}
	\SEQ}; 
}

\gfdescription{
	\renewcommand{\arraystretch}{0.5}
	\[
	\mbox{{\em tif}} \  E \ \ 
	\fraz{A \ \ \ \ \ B(\mbox{{\em termif}} \ A \ \mbox{{\em then}} \ t_1 \
	\mbox{{\em else}} \ t_2)} 
	{B(t_1)}
	\]
	\renewcommand{\arraystretch}{1}

	If \ARG{termif} is {\tt iftrm A then t1 else t2}, \ARG{fact1}'s wff is
	{\tt W(iftrm A then t1 else t2)} and {\em fact}$_2$'s wff is {\tt A},  then
	the rule deduces {\tt W(t1)}.
	If {\em termif} is not a subexpression of \ARG{fact1}'s wff, then no
	substitution is performed.
	Individual occurrences can be substituted by specifying the optional 
	\ARG{n1} \ARG{n2}, \SEQ, where \ARG{n1}, \ARG{n2}, \SEQ are the occurrences
	to be substituted.
	Without this option, all occurrences are substituted. 
	The dependencies of the derived wff are the union of those of \ARG{fact1}
	and \ARG{fact2}. 
}

\gfrecap{
Term conditional elimination.
}

\gfexample+
   ***** declare sentconst A;
   ***** declare predconst P 1;
   ***** declare indpar a b;
   ***** assume P(trmif A then a else b);
   1   P(trmif A then a else b)     (1)
   ***** assume A;
   2   A     (2)
   ***** termife 1 2 trmif A then a else b;
   3   P(a)     (1 2)
+

\gfnotes{}

\gfcommand{termifen}{term conditional elimination (with negation)}
\index{termifen}

\gfsyntax{
	termifen \ARG{fact1} \ARG{fact2} \ARG{termif}; \\
	termifen \ARG{fact1} \ARG{fact2} \ARG{termif} \OPT{occ \ARG{n1} \ARG{n2}
	\SEQ}; 
}

\gfdescription{
	\renewcommand{\arraystretch}{0.5}
	\[
	\mbox{{\em tif}} \  E_{\neg} \ \ 
	\fraz{\neg A \ \ \ \ \ B(\mbox{{\em termif}} \ A \ \mbox{{\em then}} \ t_1 \
	\mbox{{\em else}} \ t_2)} 
	{B(t_2)}
	\]
	\renewcommand{\arraystretch}{1}
	
	If {\em termif} is {\tt iftrm A then t1 else t2}, {\em fact}$_1$'s wff is
	{\tt W(iftrm A then t1 else t2)} and \ARG{fact2}'s wff is {\tt not A},
	then the rule deduces {\tt W(t2)}.
	If \ARG{termif} is not a subexpression of \ARG{fact1}'s wff, then no
	substitution is performed. 
	Individual occurrences can be substituted by specifying the optional 
	\ARG{n1}, \ARG{n2}, \SEQ, where \ARG{n1}, \ARG{n2}, \SEQ are the occurrences
	to be substituted.
	Without this option, all occurrences are substituted. 
	The dependencies of the derived wff are the union of those of \ARG{fact1}
	and \ARG{fact2}.
}

\gfrecap{
Term conditional elimination (with negation).
}

\gfexample+
   ***** declare sentconst A;
   ***** declare predconst P 1;
   ***** declare indpar a b;
   ***** assume P(trmif A then a else b);
   1   P(trmif A then a else b)     (1)
   ***** assume not A;
   2   not A     (2)
   ***** termifen 1 2 trmif A then a else b;
   3   P(b)     (1 2)
   \end{verbatim}
+

\gfnotes{
	\ARG{fact2}'s wff must be negation of \ARG{termif}'s condition and not
	viceversa.
	If \ARG{termif} = {\tt trmif not A then t1 else t2} and \ARG{fact2}'s wff =
	{\tt A}, then the rule is not applicable. 
}

\gfcommand{termifi}{term conditional introduction}
\index{termifi}

\gfsyntax{
	termifi \ARG{fact1} \ARG{fact2} \ARG{wff} \ARG{term1} \ARG{term2};
}

\gfdescription{
\renewcommand{\arraystretch}{0.5}
\[
\begin{array}{l}
\\ \\ \\ 
%\\ \\
\mbox{{\em tif}} \  I    
\end{array}
\ \
\begin{array}{cc}
%\\ \\
\begin{array}{c}
{[A]}\\
\vdots\\
B(t_1)
\end{array}
\ \ \ 
\begin{array}{c}
{[\neg A]}\\
\vdots\\
B(t_2)
\end{array}
\\
\hline\\
\begin{array}{c}
B(\mbox{{\em termif}} \ A \ \mbox{{\em then}} \ t_1 \ \mbox{{\em else}} \ t_2)
\end{array}
\end{array}
\]
\renewcommand{\arraystretch}{1}

If \ARG{fact1}'s wff is {\tt B({\em term}$_1$)} and \ARG{fact2}'s wff is {\tt
B(\ARG{term2})}, then the proof line's wff is
{\tt B(iftrm wff then \ARG{term1} else \ARG{term2})}.
All assumptions in \ARG{fact1}'s dependencies whose formula is \ARG{wff} and all
assumptions in \ARG{fact2}'s dependencies whose formula is the negation of
\ARG{wff} are discharged.
}

\gfrecap{
Term conditional introduction.
}

\gfexample+
   ***** declare sentconst A;
   ***** declare predconst P 1;
   ***** declare funconst f 1;
   ***** declare indpar a b;
   ***** assume P(trmif A then a else b);
   1   P(trmif A then a else b)     (1)
   ***** assume A;
   2   A     (2)
   ***** termife 1 2 trmif A then a else b;
   3   P(a)     (1 2)
   ***** assume not A;
   4   not A     (4)
   ***** termifen 1 4 trmif A then a else b;
   5   P(b)     (1 4)
   ***** termifi 3 5 A a b;
   6   P(trmif A then a else b)     (1)
   ***** COMMENT | Dependencies are discharged |
   ***** termifi 3 5 not A a b;
   7   P(trmif (not A) then a else b)     (1 2 4)
   ***** COMMENT | Dependencies are NOT discharged |
   ***** termifi 3 5 B a b;
   8   P(trmif B then a else b)     (1 2 4)
   ***** COMMENT | TERMIFI with function symbols |
   ***** assume P(trmif A then f(a) else f(b));
   9   P(trmif A then f(a) else f(b))     (9)
   ***** termife 9 2 trmif A then f(a) else f(b);
   10   P(f(a))     (2 9)
   ***** termifen 9 4 trmif A then f(a) else f(b);
   11   P(f(b))     (4 9)
   ***** termifen 9 4 trmif A then f(a) else f(b);
   12   P(f(b))     (4 9)
   ***** termifi 10 11  A a b;
   13   P(f(trmif A then a else b))     (9)
   ***** termifi 10 11 A f(a) f(b);
   14   P(trmif A then f(a) else f(b))     (9)
+   
   
   
\gfcommand{wffife}{wff conditional elimination}
\index{wffife}

\gfsyntax{
	wffife \ARG{fact1} \ARG{fact2};
}

\gfdescription{
\renewcommand{\arraystretch}{0.5}
\[
\mbox{{\em wif}} \  E \ \
\fraz{A \ \ \ \ \ \mbox{{\em wffif}} \ A \ \mbox{{\em then}} \ B \ \mbox{{\em else}} \ C }
{B}
\]
\renewcommand{\arraystretch}{1}

If \ARG{fact1}'s wff is {\tt wffif A then B else C} and \ARG{fact2}'s wff is
{\tt A}, then the derived wff is {\tt B} and its dependencies are the union of
those of \ARG{fact1} and \ARG{fact2}.
}

\gfrecap{
If the wff of `fact1' is `wffif A then B else C' and the wff of `fact2' is
`A', then the derived wff is `B' and its dependencies are the union of those of
`fact1' and `fact2'.
}

\gfexample+
   ***** declare sentconst A B C;
   ***** assume A;
   1   A     (1)
   ***** assume wffif A then B else C;
   2   wffif A then B else C     (2)
   ***** wffife 2 1;
   3   B     (1 2)
+

\gfnotes{}

\gfcommand{wffifen}{wff conditional elimination (with negation)}
\index{wffifen}

\gfsyntax{
	wffifen \ARG{fact1} \ARG{fact2};
}

\gfdescription{
	\renewcommand{\arraystretch}{0.5}
	\[
	\mbox{{\em wif}} \  E_{\neg} \ \
	\fraz{\neg A \ \ \ \ \ \mbox{{\em wffif}} \ A \ \mbox{{\em then}} \ B \
	\mbox{{\em else}} \ C } 
	{C}
	\]
	\renewcommand{\arraystretch}{1}
	
	If {\em fact}$_1$'s wff is {\tt wffif A then B else C} and {\em fact}$_2$'s
	wff is {\tt not A}, then the derived wff is {\tt C} and its dependencies are
	the union of those of {\em fact}$_1$ and {\em fact}$_2$.
}

\gfrecap{
If the wff of `fact1' is `wffif A then B else C' and the wff of `fact' wff is
`not A', then the derived wff is `C' and its dependencies are the union of those
of `fact1' and `fact2'.
}

\gfexample+
   ***** declare sentconst A B C; 
   ...

   ***** assume not A;
   1   not A     (1)
   ***** assume wffif A then B else C;
   2   wffif A then B else C     (2)
   ***** wffifen 2 1;
   3   C     (1 2)
+

\gfnotes{}

\gfcommand{wffifi}{wff conditional introduction}
\index{wffifi}

\gfsyntax{
	wffifi \ARG{wff} \ARG{fact1} \ARG{fact2};
}

\gfdescription{
\renewcommand{\arraystretch}{0.5}
\[
\begin{array}{l}
\\ \\ \\ 
%\\ \\
\mbox{{\em wif}} \  I  
\end{array}
\ \ 
\begin{array}{cc}
%\\ \\
\begin{array}{c}
{[A]}\\
\vdots\\
B
\end{array}
\ \ \ 
\begin{array}{c}
{[\neg A]}\\
\vdots\\
C
\end{array}
\\
\hline\\
\begin{array}{c}
\mbox{{\em wffif}} \ A \ \mbox{{\em then}} \ B \ \mbox{{\em else}} \ C 
\end{array}
\end{array}
\]
\renewcommand{\arraystretch}{1}

If \ARG{wff} is {\tt A}, \ARG{fact1}'s wff is {\tt B} and \ARG{fact2}'s wff is
{\tt C}, then the conclusion is {\tt wffif A then B else C}.
The rule discharges all the dependencies of {\em fact}$_1$ whose wff is {\tt A}
and of {\em fact}$_2$ whose wff is {\tt not A}.
}

\gfrecap{
If `wff' is `A', the wff of `fact1' is `B' and the wff of `fact2' is `C', then
the conclusion is `wffif A then B else C'.
The rule discharges all the dependencies of `fact1' whose wff is `A' and of
`fact2' whose wff is `not A'.
}

\gfexample+
   ***** declare sentconst A;
   ...
   ***** assume A;
   1   A     (1)
   ***** assume not A;
   2   not A     (2)
   ***** wffifi A 1 2;
   3   wffif A then A else (not A)
   ***** wffifi A 2 1;
   4   wffif A then (not A) else A     (1 2)
   ***** wffifi not A 2 1;
   5   wffif (not A) then (not A) else A     (1)
   ***** wffifi not A 1 2;
   6   wffif (not A) then A else (not A)     (1 2)
+

\gfnotes{}


% introduction to the sort's section
\gfcommand{contract}{contraction rule}
\index{contract}

\gfsyntax{
	contract \ALT ctc  \ARG{fact} by \ARG{assumption1} \SEQ  \ARG{assumptionN}; 
}

\gfdescription{
	Every \ARG{assumptionI} must occur in the list of dependencies of \ARG{fact}.

	The proof line's dependencies are those of \ARG{fact} less all the dependencies 
	with the same wff of some \ARG{assumptionI}s.
}

\gfrecap{
	Every `assumptionI' must occur in the list of dependencies of `fact'.

	The proof line's dependencies are those of `fact' less all the dependencies 
	with the same wff of some `assumptionI's.
}

\gfexample+
   ***** declare sentconst A B C D;
   ***** assume A A A A B B C D;
   1   A     (1)
   2   A     (2)
   3   A     (3)
   4   A     (4)
   5   B     (5)
   6   B     (6)
   7   C     (7)
   8   D     (8)
   ***** wk 8 by 1 2 3 4 5 6 7;
   9   D     (1 2 3 4 5 6 7 8)
   ***** ctc 9 by 1 5;
   10   D     (1 5 7 8)
   ***** ctc 9 by 1 2 3 5 7;
   11   D     (1 2 3 5 7 8)
   ***** ctc 9 by 10;
   I can only contract facts which occur in the assumption list!
   ***** ctc 11 by 4;
   I cannot contract using a fact not occurring in the assumption list!
+

\gfnotes{}



\gfcommand{cut}{cut rule}
\index{cut}

\gfsyntax{
	cut \ARG{fact1} \ARG{fact2};\\
	cut \ARG{fact1} \ARG{fact2} \OPT{keep \ARG{assumption1} \SEQ
	\ARG{assumptionN}};
}

\gfdescription{
	This command generates a new proof line obtained from \ARG{fact2} 
	by eliminating dependencies of all facts whose wff is equal to \ARG{fact1}'s
	wff, and then adding the dependencies of \ARG{fact1}.
	In the second form, every \ARG{assumptionI} in the list of dependencies
	is kept. 
	Every \ARG{assumptionI} must occur in the list of dependencies of
	\ARG{fact2} . 
}

\gfrecap{
This command generates a new proof line obtained from `fact2'
by eliminating dependencies of all facts whose wff is equal to `fact1''s
wff, and then adding the dependencies of `fact1'.
In the second form, every `assumptionI' in the list of dependencies is kept. 
Every `assumptionI' must occur in the list of dependencies of `fact2'. 
}


\gfexample+
   ***** declare sentconst A B C;
   ***** axiom AAA : A;
   AAA : A
   ***** assume A A A A B C;
   1   A     (1)
   2   A     (2)
   3   A     (3)
   4   A     (4)
   5   B     (5)
   6   C     (6)
   ***** wk 5 by 1 2 3 4;
   7   B     (1 2 3 4 5)
   ***** wk AAA by 6;
   8   A     (6)
   ***** cut 8 7;
   9   B     (5 6)
   ***** wk 7 by 6;
   10   B     (1 2 3 4 5 6)
   ***** cut AAA 10;
   11   B     (5 6)
   ***** cut 8 7 keep 3 2;
   12   B     (2 3 5 6)
+

\gfcommand{weaken}{weakening rule}
\index{weaken}

\gfsyntax{
	weaken \ALT wk \ARG{fact} by \ARG{fact1} \SEQ \ARG{factN};
}

\gfdescription{
	Each \ARG{factI} can be an assumption or, more generally, a fact.
	In the first case, the derived fact depends on the dependencies of 
	\ARG{fact} plus those of \ARG{factI}.
	In the second case, the dependencies of the \ARG{factI} are added to the
	list of dependencies of \ARG{fact}. 
}

\gfrecap{
Each `factI' can be an assumption or, more generally, a fact.
In the first case, the derived fact depends on the dependencies of `fact' plus
those of `factI'.
In the second case, the dependencies of the `factI' are added to the list of
dependencies of `fact'. 
}

\gfexample+
   ***** declare sentconst A B C;
   ***** assume A A A B B C;
   1   A     (1)
   2   A     (2)
   3   A     (3)
   4   B     (4)
   5   B     (5)
   6   C     (6)
   ***** axiom AAAA : A;
   AAAA : A
   ***** ori 3 4;
   7   A or B     (3 4)
   ***** ori 5 6;
   8   B or C     (5 6)
   ***** wk 7 by 6 2 1 3;
   9   A or B     (1 2 3 4 6)
   ***** wk 1 by 7 8;
   10   A     (1 3 4 5 6)
   ***** wk AAAA by 3 7 2 8;
   11   A     (2 3 4 5 6)
+

\gfnotes{}

 	% introduction to the deciders
\newpage
\newcommand{\eg}{{\em e.g.~}}
\newcommand{\ie}{{\em i.e.~}}
\newcommand{\wrt}{w.r.t.~}
\newcommand{\co}[2]{\langle #1, \: #2 \rangle}


\section{Decision procedures}
\label{sec-decide}
\label{system:sec}
A detailed description of the main decision procedures of {\tt GETFOL}
is given in \cite{armando5}.

The set of procedures of the {\tt GETFOL} system is depicted in figure
\ref{system:fig}.
Each box represents either a decider ({\tt PTAUT}, {\tt PTAUTEQ},
{\tt FOLTAUT}) or a rewriting procedure ({\tt tautren}, {\tt  phexp},
{\tt  reduce}).

\begin{figure}
\begin{center}
\makebox[3.375in][l]{
  \vbox to 2.750in{
    \vfill
    \special{psfile=decide/NEWFIG.PS}
  }
  \vspace{-\baselineskip}
}
\end{center}
\caption{The system of deciders}
\label{system:fig}
\end{figure}

\subsubsection*{{\tt PTAUT} and {\tt PTAUTEQ}}
{\tt PTAUT} and {\tt PTAUTEQ}
are deciders working on a quantifier-free first order language (hereon by
abuse of language we call them propositional deciders).
{\tt PTAUT} decides the set of first order formulas provable using
only the propositional deductive machinery (moreover it returns a
falsifying assignment whenever the input formula is not a tautology).
For instance, the formula $(P(x)\con R(x,b))\imp (P(x)\dis R(x,b))$ can be
easily inferred by a single application of {\tt PTAUT}.
{\tt PTAUT} is a generalization of the Davis-Putnam-Loveland procedure
(hereon DPL) \cite{davis2,davis6} to non clausal formulas.
The core of {\tt PTAUT} is a procedure capable of partially evaluating
the input formula \wrt a partial assignment of truth-values to the atomic
subformulas. 
A step of statistical analysis (of polynomial time complexity) collects
information about the {\em polarity} of the subformulas and the existence
of {\em Top-Level Disjunctive Occurrences} (TLDO) of atomic subformulas.
A formula $\alpha$ occurs as a TLDO in $\beta$ 
if and only if $\beta$ can be rewritten into a formula either of the form
$(\alpha\dis\gamma)$ or $(\neg\alpha\dis\gamma)$ by means of rules
expressing  the usual properties of the propositional connectives such as
the associativity, commutativity and distributivity of the propositional
connectives.
The notion of positive (negative) subformula occurrence
is inductively defined as follows: $\alpha$ occurs positively in $\alpha$, 
$\alpha$ occurs negatively in $\neg\alpha$;
$\alpha$ and $\beta$ occur positively in $(\alpha\con\beta)$ and
$(\alpha\dis\beta)$;
$\alpha$ occurs negatively and $\beta$ occurs positively in $(\alpha\imp\beta)$;
finally $\alpha$ and $\beta$ occur both positively and negatively in
$(\alpha\liff\beta)$.
A subformula $\alpha$ is positive (negative) in $\beta$ if and only if
each occurrence of $\alpha$ occurs positively (negatively) in $\beta$.

The statistical analysis may suggest a partial assignment $\mu$ (\wrt which
the formula can be simplified) according to the following criteria:
\begin{itemize}
\item for each positive (negative) atomic formula $\alpha$ occurring
in $\beta$, $\mu(\alpha)=F$ ($\mu(\alpha)=T$);
\item if $\beta$ contains a positive (negative) TLDO of $\alpha$ and there
are no negative (positive) TLDO of $\alpha$, then $\mu(\alpha)=F$
($\mu(\alpha)=T$).
\end{itemize}

If $\mu$ is not completely undefined, then {\tt PTAUT} simplifies the
formula in input \wrt $\mu$ and recurs on the resulting (simplified) formula.
If the input formula contains both a positive and a negative TLDO of an
atomic formula the input formula is a tautology.
These optimizations generalize the {\em Affirmative-Negative Rule} and the
{\em Rule for the Elimination of One-Literal Clauses} of DPL.
If $\mu$ is totally undefined, then
an atomic formula is chosen, two partial assignments are created
(one assigning $T$, the other $F$ to the chosen atomic formula),
the formula is partially evaluated \wrt such assignments and finally
the procedure branches recurring on the two simplified formulas.
This last step generalizes the {\em Splitting Rule} of DPL.

{\tt PTAUTEQ} is the result of adapting {\tt PTAUT}
to take into account the properties of equality.
The main difference is that, before a formula is simplified \wrt some
assignment, the assignment is tested to check whether it is model of the
quantifier-free theory of equality.
The formula $(P(x)\con x=y)\imp (P(y)\con y=x)$ can be
easily inferred by a single application of {\tt PTAUTEQ}.

\subsubsection*{{\tt nnf} and {\tt skolemize}}
{\tt nnf} rewrites the input formula into a logically equivalent one in
{\em negative normal form}.

{\tt skolemize} computes the skolemization of the input formula.

\subsubsection*{{\tt tautren} and {\tt phexp}}
The procedures on top of the propositional deciders (namely {\tt tautren}
and {\tt phexp}) map the first-order formula in input into a quantifier-free
formula.
The mappings are such that the decision problem of the input (first-order)
formula is related to the decision problem of the returned (quantifier-free)
formula in a useful way.
In particular, {\tt tautren} atomizes equal (modulo renaming of bound
variables) quantified subformulas into newly introduced propositional
letters.
For instance the formula
\begin{equation}\label{pb29-reduced}
%\setlength{\templength}{\arraycolsep}
\setlength{\arraycolsep}{0cm}
\begin{array}{rl}
(\exists x.F(x) \con \exists x.G(x)) \imp (&( \forall x.(F(x) \imp H(x)) \con \forall x.(G(x) \imp J(x))) \liff \\
& ((\exists y.G(y) \imp \forall x.(F(x) \imp H(x))) \con\\
&\ (\exists x.F(x) \imp \forall y.(G(y) \imp J(y)))))
\end{array}
\end{equation}
is mapped into the propositional formula
\begin{equation}\label{pb29-prop}
(A \con B) \imp ((C \con D) \liff ((B \imp C) \con (A \imp D)))
\end{equation}
The relation between the decision problems of the input formula (say $\alpha$)
and of the output formula (say $\alpha'$) is that
$\der{}\alpha'$ only if $\der{}\alpha$.

A more careful reduction to the quantifier-free fragment is performed by
{\tt phexp}.
{\tt phexp} maps an existential formula $\alpha$ into a quantifier-free formula
$\alpha'$ such that $\der{}{\alpha'}$ if and only if $\der{}{\alpha}$.%
\footnote{The set of existential formualas is the class of prenex
universal-existential formulas without function symbols.}

The formula $\alpha'$ is an improved version of the Herbrand's expansion
of $\alpha$ \cite{dreben1}.
An application of {\tt phexp} to the following formula:
\begin{equation}\label{pb28}\small
    (((P(x) \con \neg Q(y)) \dis
     ((Q(a) \dis R(a)) \con (\neg Q(b) \dis \neg S(b)))) \dis
     ((F(z) \con \neg G(z)) \con S(v))) \dis
       ((\neg P(c) \dis \neg F(c)) \dis G(c))
\end{equation}
yields
\begin{equation}\label{pb28-exp}\small
\begin{array}{l}
((((P(a) \dis P(b) \dis P(c)) \con (\neg Q(a) \dis \neg Q(b) \dis
\neg Q(c)))\dis\\
((Q(a) \dis R(a)) \con (\neg Q(b) \dis \neg S(b)))) \dis \\
(((F(a) \con \neg G(a)) \dis (F(b) \con \neg G(b)) \dis
(F(c) \con \neg G(c)))\con\\
(S(a) \dis S(b) \dis S(c)))) \dis ((\neg P(c) \dis \neg F(c)) \dis G(c))\\
\end{array}
\end{equation}
In \cite{armando5} it is shown that, the size of (\ref{pb28-exp})
is 44 times smaller than the size of the standard Herbrand's expansion of
(\ref{pb28}).

\subsubsection*{{\tt reduce}}
{\tt reduce} tries a set of rewriting rules on the input
formula aiming at rewriting it into a logically equivalent formula that
can be easily turned into an existential one via skolemization.
The rewriting rules employed by {\tt reduce} are the usual rules
expressing the distributivity of quantifiers through propositional connectives
and the commutativity and associativity of propositional connectives
listed in the following table.\\

    \renewcommand{\arraystretch}{1.5}
    {\small
      $$
      \begin{array}{|c|rcl|} \hline
        (1) & Q x. \alpha[x] & \mapsto & \alpha \\ \hline
        %(2) & Q x. (\neg \alpha(x)) & \mapsto & (\neg \hat{Q} x. \alpha(x)) \\ \hline
        (2) & Q x. (\alpha \circ \beta)(x) & \mapsto & (Q x. \alpha \circ Q x. \beta) 
        \\ \hline
        (3) & Q x. (\alpha[x] + \beta(x)) & \mapsto & (\alpha[x] + Q x. \beta(x)) 
        \\ \hline \hline
        (4) & (\alpha(x) + \beta[x]) & \mapsto & (\beta[x] + \alpha(x)) \\ \hline
        (5) & ((\alpha[x] + \beta(x)) + \gamma(x)) & \mapsto & 
        (\alpha[x] + (\beta(x) + \gamma(x))) \\ \hline
        (6) & ((\alpha \circ \beta)(x) + \gamma(x)) & \mapsto & 
        ((\alpha + \gamma(x)) \circ (\beta + \gamma(x))) \\ \hline
        (7) & (\alpha(x) + (\beta[x] + \gamma(x))) & \mapsto &
        (\beta[x] + (\alpha(x) + \gamma(x))) \\ \hline
        (8) & ((\alpha(x) + (\beta \circ \gamma)(x))) & \mapsto &
        ((\alpha(x) + \beta) \circ (\alpha(x) + \gamma)) \\ \hline
      \end{array}
      $$
      }
    \renewcommand{\arraystretch}{1}
{\small
{\em Restrictions}: 
\begin{itemize}
\item In rules $\{(4)-(8)\}$ the left hand side must be a top normalizable
formula.
\item In rules $\{(7),(8)\}$ $\alpha$ must be minimal \wrt $\co{Q}{x}$.
%\item Rules $\{(4)-(8)\}$ can be applied only to subformulae (say $\alpha$)
%of a formula $Qx.\beta$ in which there is no proper
%subformula $Qy.\gamma$ of which $\alpha$ is a subformula.
\end{itemize}}

Where
$\alpha(x)$ denotes a formula in which there is at least one free occurrence
of the variable $x$.
$\alpha{[x]}$ denotes a formula in which there is no free occurrences of $x$.
$Q$ and $Q'$ stand either for $\forall$ or for $\exists$.
If $Q = \forall$, then $\circ = \con$ and $+ = \dis$.
If $Q = \exists$, then $\circ = \dis$ and $+ = \con$.
%$\con$ is said to be $\forall$-compatible and $\exists$-incompatible,
%$\dis$ is said to be $\exists$-compatible and $\forall$-incompatible.
%If $\cal S$ is a set of rewriting rules then $\mapsto_{\cal S}$ is the
%reducibility relation induced by $\cal S$ and
%$\stackrel{*}{\mapsto}_{\cal S}$ is the reflexive and transitive closure
%of $\mapsto_{\cal S}$.
The definition of {\em top normalizable formula} and of {\em minimal
formula} are given in \cite{armando5}.

For instance, a single application of {\tt reduce} turns the formula
\begin{equation}\label{pb29}
%\setlength{\templength}{\arraycolsep}
\setlength{\arraycolsep}{0cm}
\begin{array}{rl}
(\exists x.F(x) \con \exists x.G(x)) \imp (&( \forall x.(F(x) \imp H(x)) \con \forall x.(G(x) \imp J(x))) \liff \\
&(\forall x.\forall y.((F(x) \con G(y)) \imp (H(x) \con J(y)))))
\end{array}
\end{equation}
into (\ref{pb29-reduced}).
{\tt reduce} considerably enlarges the set of formulas which can be solved
by using the system of deciders.
In particular, any prenex first order formula 
$$
\forall \vec{y}_n \exists \vec{x}_n \ldots
\forall \vec{y}_i \exists \vec{x}_i \ldots
\forall \vec{y}_1 \exists \vec{x}_1 . \Phi
$$
such that each literal in $\Phi$ contains no variables in $\vec{y}_k$ and
in $\vec{x}_l$ with $k < l$, or in $\vec{x}_k$ and in $\vec{x}_l$ with
$k \neq l$ can be ``reduced" to an existential formula.
On the basis of the previous result it is a trivial consequence
that the {\em monadic calculus} can be reduced to the existential fragment
by means of {\tt reduce}.


% user commands for deciders
\gfcommand{decide}{general purpose decider}
\index{decide}

\gfsyntax{
  decide \ARG{wff}  \OPT{by \ARG{fact1} \ARG{fact2} \SEQ} using 
  \OPT{\{ \ARG{rewriter} \SEQ \}} \ARG{decider};
}

\gfdescription{
 {\em rewriter} and {\em decider} are defined by the following grammar:
 %
 \begin{bnf}
    \T{rewriter} \sep nnf | reduce | skolemize |
                      phexp | tautren \\ 
    \T{decider}  \sep ptaut | ptauteq
 \end{bnf}
 %
 Let {\it wff}$_i$ be the formula of {\it fact}$_{i}$, $i = 1, 2, ...\ .$
 The command tries to verify whether the formula 
 %
 \begin{equation}
   {\it wff_1} \con ... \con {\it wff}_n \imp {\it wff} \label{eq-dec}
 \end{equation}
 %
 is a theorem by applying the specified rewriters in the same order they
appear on the command line and finally applying the decider to the resulting
formula.
 If this succeeds, it asserts a fact whose formula is \ARG{wff}
 and whose dependencies are the union of the dependencies of 
 \ARG{fact1}, \ARG{fact2}, \SEQ.
Otherwise failure is reported.\\
 %
 Whenever the formula in input to one of the specified routines does not  
 match the syntactic  restrictions of the routine, an error  message is
 printed, together with hints on how to modify the strategy. 
}

\gfrecap{
  `rewriter' is one of the following: nnf, reduce, skolemize, phexp, tautren.
  `decider'  is either ptaut or taut.
The command tries to verify whether the formula:
   +-------------------------------------------------------+
   | (wff1 and ... and wffN) imp wff                   (1) |
   +-------------------------------------------------------+
(where `wffI' if the formula of `factI') is a theorem by applying
the rewriters `rewriter ...' (in the order in which they appear) and 
then the decider `decider' to (1).
}

\gfexample+
   ***** declare indvar x,y,z;
   ***** declare predconst P,Q 1;
   ***** decide exists x.forall y.forall z.((P(y) imp Q(z) imp
               (P(x) imp Q(x))))  using nnf,phexp,ptaut;
   
   PHEXP requires the formula to be in existential form.
   Try using SKOLEMIZE before .... 
   
   ***** decide exists x.forall y.forall z.((P(y) imp Q(z) imp
               (P(x) imp Q(x)))) using nnf,skolemize,phexp,ptaut;
   
   decide couldn't prove that exists x. forall y z. ((P(y) imp Q(z))
    imp (P(x) imp Q(x))) is a tautology using nnf,skolemize,phexp,ptaut.
   
   ***** decide exists x.forall y.forall z.((P(y) imp Q(z) imp
               (P(x) imp Q(x)))) using nnf,reduce,skolemize,phexp,ptaut;
   
   1   exists x. forall y z. ((P(y) imp Q(z)) imp (P(x) imp Q(x)))     
+

\gfcommand{monad}{first order decider for monadic formulas}
\index{monad}

\gfsyntax{
  monad \ARG{wff} \OPT{by \ARG{fact1} \ARG{fact2} \SEQ};
}

\gfdescription{
It tries to establish whether the input formula is deducible from the
specified facts by using {\tt nnf}, {\tt reduce}, {\tt phexp} and finally
{\tt ptaut}.
}

\gfrecap{
It tries to establish whether the input formula is deducible from the
specified facts by using nnf, reduce, phexp and finally ptaut.
}

\gfexample+
   ***** declare predconst P 1;
   ***** declare funconst f 3;
   ***** declare indvar x y z;
   ***** declare indpar a b;
   
   *****comment | *** MONAD EXAMPLES *** |
   ***** monad forall x. exists y. (P(x) imp P(y));
   1   forall x. exists y. (P(x) imp P(y))
   ***** monad exists y. forall x. (P(x) imp P(y));
   2   exists y. forall x. (P(x) imp P(y))
   ***** monad exists y. forall x. ((P(x) imp P(y)) or P(x));
   3   exists y. forall x. ((P(x) imp P(y)) or P(x))
   
   ***** monad forall x. exists y.
    wffif P(trmif P(y) then x else y)
     then P(trmif P(y) 
             then trmif P(y) 
                   then x 
                   else trmif P(y) 
                         then x 
                         else y 
             else y) 
          or TRUE
     else P(y) or TRUE;
   4   forall x. exists y. 
       (wffif P(trmif P(y) then x else y) 
         then (P(trmif P(y) 
                 then 
                 (trmif P(y) 
                   then x 
                   else (trmif P(y) then x else y)) 
               else y) or TRUE) else (P(y) or TRUE))
   ***** monad forall x. exists y. wffif P(x) then P(y) else not P(y);
   5   forall x. exists y. (wffif P(x) then P(y) else (not P(y)))  
   ***** monad forall y. exists x. (P(f(a,b,x)) or not P(f(a,b,y)));
   6   forall y. exists x. (P(f(a,b,x)) or (not P(f(a,b,y))))
   ***** monad exists z. forall y. exists x. (P(f(z,b,x)) or not P(f(x,b,z)));
   7   exists z. forall y. exists x. (P(f(z,b,x)) or (not P(f(x,b,z)))) 
   ***** monad exists z. forall y. exists x. (P(f(z,b,x)) or not P(f(x,b,y)));
   7   exists z. forall y. exists x. (P(f(z,b,x)) or (not P(f(x,b,y)))) 
+
   
\gfnotes{
   The name ``{\tt monad}" is due to the fact that the monadic predicate
calculus is subset of the class of formulas decided by such a decider.
}   

\gfcommand{monadeq}{first order decider for monadic formulas with equality}
\index{monadeq}

\gfsyntax{
  monadeq \ARG{wff} \OPT{by \ARG{fact1} \ARG{fact2} \SEQ};
}


\gfdescription{
It tries to establish whether the input formula is deducible from the
specified facts by using {\tt nnf}, {\tt reduce}, {\tt phexp} and finally
{\tt ptauteq}.
}

\gfrecap{
It tries to establish whether the input formula is deducible from the
specified facts by using nnf, reduce, phexp and finally ptauteq.
}

\gfexample+
   ***** declare predconst P 1;
   ***** declare funconst f 3;
   ***** declare indvar x y z;
   ***** declare indpar a b;

   *****comment | *** MONADEQ EXAMPLES *** |
   ***** monadeq forall x. exists y. (x=y);
   1   forall x. exists y. (x=y)
   ***** monadeq forall x.  forall y. (x=y imp (P(x) imp P(y)));
   2   forall x.  forall y. (x=y imp (P(x) imp P(y)))
   ***** monad (x=y imp (P(x) imp P(y))) by 2;
   3   (x=y imp (P(x) imp P(y)))  (2)
+

\gfcommand{ptaut}{tautological decider}
\index{ptaut}
\label{sec-decproc}

\gfsyntax{
  ptaut \ARG{wff} \OPT{by \ARG{fact1} \ARG{fact2} \SEQ};
}

\gfdescription{
It decides the quantifier-free formulas provable only by means of the
propositional deductive machinery.
}

\gfrecap{
It decides the quantifier-free formulas provable only by means of the
propositional deductive machinery.
}

\gfexample+
   ***** declare sentconst A B;
   ***** ptaut (A imp (B imp A));
   1   A imp (B imp A)
   ***** assume A B;
   2   A     (2)
   3   B     (3)
   ***** ptaut (A imp (B imp A)) by 2 3;
   4   A imp (B imp A)  (2 3)
   ***** ptaut (A and B) by 2 3;
   5   A and B  (2 3)
   ***** ptaut A by 3;
   PTAUT couldn't prove that A
   is a logical consequence of facts.
   
   ***** declare predconst P 1;
   ***** declare indconst c;
   ***** declare indvar x;

   ***** ptaut (A imp (P(c) imp A));
   6   A imp (P(c) imp A)     
   ***** ptaut (A imp (forall x.P(x) imp A));
   The formula passed to PTAUT is not propositional !
+   

\gfcommand{taut}{tautological first order decider}
\index{taut}

\gfsyntax{
  taut \ARG{wff} \OPT{by \ARG{fact1} \ARG{fact2} \SEQ};
}

\gfdescription{
  The class of formulae decided by {\tt taut} is the set of first order
  formulas provable using only the introduction and elimination rules for 
  the sentential connectives plus the following rule ({\em congruence-rule}):\\
  %
  \[
      \fraz{\forall x A(x)}{\forall y A(y)}
  \]
}

\gfrecap{
The class of formulae decided by `taut' is the set of first order formulas
provable using only the introduction and elimination rules for the 
sentencial connectives plus the following rule (congruence rule)::
              +------------------------------------+
              | forall x. A(x) imp forall y. A(y)  |
              +------------------------------------+
}

\gfexample+
   ***** declare sentconst A;
   ***** declare predconst P 1;
   ***** declare indconst c;

   ***** taut (A imp (P(c) imp A));
   1   A imp (P(c) imp A)

   ***** declare indvar x y [S1];
   ***** declare indvar z [S2];

   ***** taut forall x.P(x) iff forall y.P(y);
   2   forall x.P(x) iff forall y.P(y);
   ***** taut forall x.P(x) iff forall z.P(z);
   TAUT couldn't prove that forall x. P(x) iff forall z. P(z)
   is a tautology.
+

\gfcommand{tauteq}{tautological decider with equality}
\index{tauteq}

\gfsyntax{
  tauteq \ARG{wff} \OPT{by \ARG{fact1} \ARG{fact2} \SEQ};
}

\gfdescription{
  The class of formulae decided by {\tt tauteq} is the set of formulas provable
  using:
  \begin{itemize}
  \item the introduction and elimination rules for the sentential connectives;
  \item the {\em congruence-rule};
  \item the following axioms schemata for equality:
  $$
  \begin{array}{l}
    x=x\\
    (x=y\ \imp\ y=x)\\
    ((x=y\ \con\ y=z)\ \imp\ x=z)\\
    ((x_1=y_1\ \con \ldots \con\ x_n=y_n)\ \imp\ 
    (P(x_1, \ldots \ ,x_n)\ \liff\ P(y_1, \ldots \ ,y_n)))
  \end{array}
  $$
  %
  corresponding to {\em reflexivity, symmetry, transitivity} and {\em substitution 
  into predicates}.
  \end{itemize}
}

\gfrecap{
The class of formulae decided by {\tt tauteq} is the set of formulas provable
using:
* the introduction and elimination rules for the sentential connectives;
* the ``congruence-rule'';
* the following axioms schemata for equality:
     +-------------------------------------------------------------------------+
     |   x = x                                                                 |
     |   x = y imp y = x                                                       |
     |   (x = y and y = z) imp x = z                                           |
     |   (x1 =y1 and ... and xN = yN) imp (P(x1, ..., xN) iff P(y1, ..., yN))  |
     +-------------------------------------------------------------------------+
  corresponding to ``reflexivity'', ``symmetry'', ``transitivity'' and
  ``substitution into predicates''.
}

\gfexample+
   ***** declare predconst P 1;
   ***** declare funconst f 1;
   ***** declare indvar x y;
   ***** declare indvar z;
   ***** tauteq x=x;
   1   x=x
   ***** tauteq x=y imp y=x; 
   2   x=y imp y=x
   ***** tauteq ((x=y and y=z) imp x=z);
   3   (x=y and y=z) imp x=z
   ***** tauteq (x=y imp (P(x) or not P(y)));
   4   x=y imp (P(x) or not P(y))
   ***** tauteq (f(x)=f(y) imp (P(f(x)) iff P(f(y))));
   5   f(x)=f(y) imp (P(f(x)) iff P(f(y)))
   ***** tauteq x=y imp f(x)=f(y);
   TAUTEQ couldn't prove that (x = y) imp (f(x) = f(y))
   is a tautology.
+

 	% introduction to the semantic simplification's section
\newpage
\section{Semantic simplification}
\label{sec-comp}

The subsections \ref{sec-ss-intro}, \ref{sec-ss-model} and \ref{sec-ss-repr} 
of this section have been taken from \cite{rww3}.


\subsection{Introduction}
\label{sec-ss-intro}

{\GF} is intended to express a variety of methods of human reasoning.
Though the word "reasoning" usually connotes a logical deductive process of 
using facts and assertions to obtain conclusions, much of human intelligence 
relies more upon observation than upon deduction.
We look at a book. The book is seen to be "green", as an immediate observation,
not as a deduction involving, say, analysis of wavelengths of light and 
sensory receptors  in the eye. Similarly, humans cross streets without 
conscious analysis of the traffic flow, add numbers without resorting to basic
set theory, and play chess without considering each move in terms of the 
geometry of the board. 

Any system which hopes to express a variety of reasoning processes, therefore 
needs a method of doing purely computational tasks.
In {\GF}, the {\bf semantic interpretation mechanism}, which provides this 
ability, consists of two parts:
\begin{itemize}
\item {\GF}'s {\bf semantic attachment mechanism} permits the user to define a
      ``correspondence'' between the various constants (function symbols,
      predicate constants, individual constants) of the language and
      corresponding objects of the programming language {\HG}.
\item facts about the {\HG} structure can be used directly in the proof
      via the {\bf semantic simplification mechanism}, eliminating the 
      necessity of a possibly complicated deduction.
\end{itemize}
For example, obvious attachments to the function symbol $+$ and to the 
individual constants $17$, $34$, $51$ would allow to conclude $17+34=51$ in 
one step, instead of computing $34$ successors of $17$.
In order to explain this more clearly we first give an informal account of the 
technical details.

\subsection{``Intended'' and ``computational'' models}
\label{sec-ss-model}

The declarations made by a {\GF} user specify a first order language 
$L=\langle P,F,C\rangle$, where $P$ is the list of {\predconst}s, $F$ the list 
of {\funconst}s, and $C$ the list of {\indconst}s (see section \ref{sec-decl}).

A model for such a language is a structure $M=\langle D,P',F',C'\rangle$ where
$D$ is a set and $P'$,$F'$ and $C'$ are lists of predicates over $D$, functions 
over $D$, and individuals of $D$ such that the arities of the symbols in $P$ and 
$F$ match the arities of the predicates and functions at the correspondent 
positions in $P'$ and $F'$.
The idea here is that the language $L$ is used for making statements about 
structures such as $M$ (what we call {\bf ``intended'' or ``standard'' model}). 
In particular, when the user writes down a theory in {\GF}, he generally has 
in mind some particular model for his language, and the axioms of his theory 
are intended to express the properties of this particular model.

The fact that {\GF} is really a {\HG} program running in a LISP 
environment, inspires the following idea: some parts of a model for a {\GF} 
language can often be expressed computationally in the sense that the elements 
of $D$ can be represented by s-expressions, and the predicates and functions 
on $D$ can be represented by {\HG} functions and predicates.
It should then be possible to use the computational representation to aid 
{\GF} deductions concerning the model.
For example, suppose the theory we are interested in, is first order number 
theory, and the model that we have in mind is the set of natural numbers 
together with the operations of successor, addition and multiplication.
The numerals have natural representations as {\HG} numbers, and the 
functions in question have {\tt PLUS1}, {\tt PLUS}, {\tt TIMES} as their {\HG}
counterparts.
As mentioned above it should then be possible to use the computational 
representation to provide swift deductions of such statements as $25+37=52$.

The semantic attachment mechanism in {\GF} allows the user to set up these 
computational representations of his subject matter, and the semantic
interpretation mechanism allows to use these representations to aid deduction 
in {\GF}.

With the above overview in mind, let us proceed to the details.

Given a language $L=\langle P,F,C\rangle$ and a model 
$M=\langle D,P',F',C'\rangle$, we define an interpretation function $I$.
For each {\term} $t$ of $L$ in which no free variable occurs, $I(t)$ is the 
individual in $D$ which $t$ denotes.
In particular we define the interpretation of an {\indconst} $c$ to be the 
individual $c'$ in $D$, and where $f$ is a {\funconst}, and the interpretation 
of {\term} $t_1,\ldots,t_n$ are defined, we inductively define the 
interpretation of the {\term} 
$f(t_1,\ldots,t_n)$ to be $f'(I(t_1),\ldots,I(t_n))$.
We may extend the interpretation function to formulas (again without free 
variables) over $L$ by defining $I(w)$ to be the object {\tt TRUE} exactly 
when the formula $w$ is true of the model (for a technical definition 
see \cite{kleene2}).

When $f'$ is the function in a model corresponding to the {\funconst} $f$ in 
$L$, we will also say that $f'$ is the interpretation of $f$, and similarly 
for predconsts.

Now we define a {\bf computational model} to be an object
$K=\langle D',P'',F'',C''\rangle$, where it is understood that $D'$ is a set
of s-expressions, and $P''$, $F''$ and $C''$ are lists of {\HG} predicates,
functions and s-expressions respectively, with the appropriate restrictions on 
arities.

From the extensional point of view, a computational model is for a language
just like a set-theoretic model for a language, except that we do not require
that the functions and predicates concerned be total; that is functions and
predicates may be undefined (non-terminating) for some elements
of $D'$.

We define an {\bf attachment map} $att$ from terms and formulas of $L$ into
$K$ in a manner exactly analogous to the definition of $I$ given above.

We have one last map to worry about, the map {\bf $rep$} which gives, for each
object in the domain $D'$ of the computational model $K$, the object it
represents in the domain $D$ of the model $M$.

Now we may define precisely the meaning of attachments made in the {\GF}
system: the attachment of an {\indconst} $c$ to an s-expression $c''$
signifies that $c$ and $c''$ represent the same object in the model, that is
to say, $I(c)=rep(c'')$.
Similarly, the attachment of a {\funconst} $f$ to a {\HG} function $f''$
signifies that the result of applying $f''$ to an s-expression $c''$ which
represents an individual $c'$ in the model, is a s-expression which represents
the individual $f'(c')$ in the model.
The analogous statements hold for attachments to {\predconst}s.

The above conditions are equivalent to the statement that the 
diagram in figure \ref{fig-ss} commutes.

\begin{figure}[htb]
\begin{center}
\setlength{\unitlength}{0.0125in}%
\begin{picture}(288,260)(92,540)
\thicklines
\put(340,580){\oval(80,80)}
\put(160,580){\oval(80,80)}
\put(160,760){\oval(80,80)}
\put(200,580){\vector( 1, 0){100}}
\put(194,726){\vector( 1,-1){112}}
\put(160,720){\vector( 0,-1){100}}
\put(340,559){\makebox(0,0)[b]{\raisebox{0pt}[0pt][0pt]{\twlrm model}}}
\put(340,577){\makebox(0,0)[b]{\raisebox{0pt}[0pt][0pt]{\twlrm of intended}}}
\put(340,595){\makebox(0,0)[b]{\raisebox{0pt}[0pt][0pt]{\twlrm Domain }}}
\put(160,562){\makebox(0,0)[b]{\raisebox{0pt}[0pt][0pt]{\twlrm sexpr}}}
\put(160,580){\makebox(0,0)[b]{\raisebox{0pt}[0pt][0pt]{\twlrm {\HG}}}}
\put(160,752){\makebox(0,0)[b]{\raisebox{0pt}[0pt][0pt]{\twlrm Terms}}}
\put(160,770){\makebox(0,0)[b]{\raisebox{0pt}[0pt][0pt]{\twlrm {\GF}}}}
\put(250,560){\makebox(0,0)[b]{\raisebox{0pt}[0pt][0pt]{\twlrm Representation}}}
\put(280,660){\makebox(0,0)[b]{\raisebox{0pt}[0pt][0pt]{\twlrm I}}}
\put(120,660){\makebox(0,0)[b]{\raisebox{0pt}[0pt][0pt]{\twlrm attachment}}}
\end{picture}
\end{center}
\label{fig-ss}
\caption{intended model - computational model mappings}
\end{figure}


\subsection{Multiple representation functions}
\label{sec-ss-repr} 

The semantic attachment mechanism allows several representation of the model 
by {\HG} s-expressions to be in force at the same time.
We will seek to motivate this aspect of the semantic attachment mechanism 
by means of an example: consider a theory of chess with includes a general 
theory of lists as a subtheory (this subtheory would be applied in arguments 
about lists of pieces, lists of game positions and so on).
The intended model of such a theory includes at least two kinds of objects: 
chess positions and lists.
Lists and positions form disjoint domains in the model, though it may be 
possible to build lists of chess position.
If we are going to build a computational representation of this model, we will
need to represent positions and lists by s-expressions in such a way that no 
s-expression represents both a list and a position.
The natural representation of a chess position as an s-expression is as a 
list of eight lists, each of which is a list of eight piece names (one of 
which is "empty" or some such), and the natural representation of lists as 
s-expressions is the direct representation as {\HG} lists.
This representation scheme cannot be used, since it will not be possible to 
decide whether a given list of eight lists of eight piece names represents a 
chess board or a list of list of pieces. 
That is to say, the map $rep$ will not be well defined. 
It is of course not hard to solve this problem by the use of some slightly 
fancier coding, but a general solution to the problem of disambiguating 
computational representations is available.
Suppose that the intended model of a {\GF} theory $T$ includes the disjoint 
domains $D_1,\ldots,D_n$, and suppose further that we have a different coding 
function for each of these domains.
That is we have $n$ different {\bf representation functions} $rep_i$ which map 
the domain of s-expressions into domain of the model, with the property that 
the range of $rep_i$ is a subset of $D_i$.
Then it is possible that a single s-expression codes two different objects 
$d_i$, $d_j$ in the model, but as long as we know what coding function $rep_i$
to apply, there is no ambiguity. 


Then the definition of the $att$ map may be extended to take account of the 
possibility of multiple representations in the following way: the domain of 
the $att$ map will still consist of the set of {\GF} terms and formulas, but
its range will now lie in the set of pairs of the form $\langle$ representation
function, s-expression $\rangle$.

The soundness condition for the $att$ map is now that, when 
$att(t)=\langle rep, c'' \rangle$, we have $rep(c'')=I(t)$.
In order to specify this new more complicated $att$ map, the user of the {\GF}
system must give representation information concerning his attachments.

Specifically, each representation function must be given a name and when the 
attachment to an {\indconst} is given, the name of the associate 
representation function must be given as well.
Similarly, when the attachment $f''$ to a {\funconst} $f$ is specified, the 
(names of the) representations of its arguments and of the value it returns 
must be given, and when the attachment to a {\predconst} is specified, the 
representations of its arguments must also be specified.

The significance of specifying that the representations of the arguments and 
value of the attachment $f''$ to a {\funconst} $f$ are 
$R_1,\ldots,R_n$ and $R_{n+1}$ respectively, is that 
$R_{n+1}(f''(c''_1,\ldots,c''_n))=f'(R_1(c''_1),\ldots,R_n(c''_n))$ 
where $f'$ is the interpretation of $f$, whenever $c''_1$,..,$c''_n$ are 
s-expressions in the domains of $R_1$,..,$R_n$.
The same holds for attachments to {\predconst}, mutatis mutandis.
Given the attachments with representation information for individual symbols,
the map $att$ on the domain of terms and formulas is defined inductively in 
the obvious way: if $f$ is attached to $f''$ and the declared representation 
of the arguments of $f''$ are $R_1,\ldots,R_n$ and terms $t_1$,..,$t_n$ have 
attachments with representations $R_1$,..,$R_n$ then 
$att(f(t_1,\ldots,t_n))=f''(att(t_1),\ldots,att(t_n))$.
Under this definition the diagram above commutes for each individual 
representation function.

Note that if the representation of the attachment of any term $t_i$ does not 
match that of its place in the argument list, then 
$f''(att(t_1),\ldots,att(t_n))$ cannot be expected to represent the 
interpretation of $f(t_1,\ldots,t_n)$.
The reason for this is that the correctness of a computation which purports to
represent a mathematical function depends on the representation of the 
arguments of the function as data objects.
For example, no one would expect a floating point multiplication algorithm to 
behave correctly if its arguments were encoded as integers rather than 
floating point numbers.

Finally, note that the attachment map, as well as the s-expressions which 
represent functions, may be partial.
The user is never required to provide an attachment for any {\GF} symbol, nor is
any attachment to a {\funconst} or {\predconst} required to be complete.

The semantic simplification mechanism will use whatever information is 
available and if there will be insufficient information, it will return this 
fact to the user. 




% user commands for semantic simplification
\gfcommand{attach}{semantic attachment}
\index{attach}

\gfsyntax{
  attach \ARG{indconst}  to \ALT dar [ rep ] \ARG{sexpr};\\
  attach \ARG{sentconst} to T \ALT NIL \ALT UNDEF;\\
  attach \ARG{funconst} \ALT \ARG{predconst} to \ARG{atom};\\
  attach \ARG{funconst}  to [ \ARG{rep1}, \SEQ, \ARG{repN} = \ARG{repM} ]
  \ARG{atom};\\
  attach \ARG{predconst} to [ \ARG{rep1}, \SEQ, \ARG{repN} ] \ARG{atom};
}

\gfdescription{
  Defines the attachment for the {\GF} constants.
  \ARG{repI} can be a representation function or an asterisk; if it is an
  asterisk or no representation is specified, then the default representation
  function {\tt UNIVERSALREP} is taken.
  \ARG{indconst}s can be attached either ``one way'' (using {\tt to}) or ``two ways''
  (using {\tt dar}).
  The ``two ways'' attachment tells the semantic interpretation mechanism 
  that whenever \ARG{sexpr} is computed as the {\HG} representation of 
  a term $t$, then the attached {\GF} \ARG{indconst} should be returned as the 
  simplified version of $t$.
  That is, not only \ARG{sexpr} is the {\HG} representation of \ARG{indconst}, but 
  \ARG{indconst} is the preferred {\GF} name of (the intended model value denoted 
  by) the {\HG} object \ARG{sexpr}.
  \ARG{sentconst}s can be attached to the three possible truth values
  corresponding to true, false and undefined\cite{kleene1}.
  This is done by attaching a sentconst to {\tt T}, {\tt NIL}, {\tt UNDEF}
  respectively.
  \ARG{funconst}s and \ARG{predconst}s can be attached to a {\HG} \ARG{atom}
  \cite{giunchiglia35}.
  \ARG{atom} will be used as the identifier of a {\HG} function, whose number of 
  arguments is supposed to match the arity of the {\GF} symbol.
}

\gfrecap{
Defines the attachment for the GETFOL constants.
`repI' can be a representation function or an asterisk; if it is an
asterisk or no representation is specified, then the default representation
function `UNIVERSALREP' is taken.
`indconst's can be attached either ``one way'' (using `to') or ``two ways''
(using `dar').
The ``two ways'' attachment tells the semantic interpretation mechanism 
that whenever `sexpr' is computed as the HGKM representation of 
a term `t', then the attached GETFOL `indconst' should be returned as the 
simplified version of `t'.
That is, not only `sexpr' is the HGKM representation of `indconst', but 
`indconst' is the preferred GETFOL name of (the intended model value denoted 
by) the HGKM object `sexpr'.
`sentconst's can be attached to the three possible truth values
corresponding to true, false and undefined.
This is done by attaching a sentconst to `T', `NIL', `UNDEF'
respectively.
`funconst's and `predconst's can be attached to a HGKM `atom'
`atom' will be used as the identifier of a HGKM function, whose number of 
arguments is supposed to match the arity of the GETFOL symbol.
}

\gfexample+
   ***** declare indconst a b;
   ***** declare sentconst s;
   ***** declare funconst f 1;
   ***** declare predconst p 1;
   ***** decrep rep;

   ***** attach a to a;
   a attached to 'a
   ***** attach a dar a;
   a attached to 'a
   a is the preferred name of a
   ***** attach a dar [rep]a;
   a attached to 'a
   ***** attach a dar [rep]b;
   a is already an preferred name in this representation
   ***** attach b dar [rep]a;
   a has already a preferred name in this representation

   ***** COMMENT | deflam defines an HGKM function |;
   ***** deflam f(x) x;
   ***** attach f to f;
   f attached to f
   ***** deflam p(x) (IF (EQUAL x (QUOTE a))TRUE FALSE);
   ***** attach p to p;
   p attached to p
   attach f to [repin=repout]f;
   f attached to f
   ***** deflam p1(x) (IF (EQUAL x (QUOTE b))TRUE FALSE);
   ***** attach p to [repin]p1;
   p attached to p1
   ***** attach p to [repin]p1;
   p has already an attachment with these representation informations
   ***** attach s to T;
   s attached to 'T
   ***** attach s to NIL;
   s has already an attachment
+

\gfcommand{decrep}{representation declaration}
\index{decrep}

\gfsyntax{
  decrep \ARG{replabel1} \OPT{\SEQ \ARG{replabelN}};
}

\gfdescription{
  It declares \ARG{replabelI} to be representation functions.\\
  The only builtin representation functions are {\tt NATNUMREP}, {\tt TRUTHREP}
  and {\tt UNIVERSALREP}, the representation functions for natural numbers, for 
  truth values and for default representation respectively.
  Numerals have a builtin attachment to {\HG} numbers in the representation 
  function {\tt NUMERALREP}.
}

\gfrecap{
It declares `replabelI' to be representation functions.
The only builtin representation functions are `NATNUMREP', `TRUTHREP'
and `UNIVERSALREP', the representation functions for natural numbers, for 
truth values and for default representation respectively.
Numerals have a builtin attachment to HGKM numbers in the representation 
function `NUMERALREP'.
}

\gfexample+
   ***** decrep rep1 rep2;
+

\gfnotes{
  Since the intended model itself appears nowhere in the {{\GF}} system, there
  is no need for the user to give any detailed information about the nature of 
  the representation maps which he has in mind.
  {\tt NATNUMREP} is known by {\GF} only after typing {\tt know natnums}.
}

\gfcommand{hardware}{semantic attachment to values of a s-expression}
\index{hardware}

\gfsyntax{
  hardware \ARG{indconst} to \ALT dar \ARG{sexpr};\\
  hardware \ARG{indconst} to \ALT dar [ \ARG{rep} ] \ARG{sexpr};
}

\gfdescription{
  This command is similar to the attach command for \ARG{indconst}.
  The difference is that, if \ARG{sexpr} changes value over time, then so does
  the value of the attachment.
  It is a "dynamic attachment" in the sense that it is attached to the values the
  \ARG{sexpr} assumes over time.
}

\gfrecap{
This command is similar to the attach command for `indconst'.
The difference is that, if `sexpr' changes value over time, then so does
the value of the attachment.
It is a "dynamic attachment" in the sense that it is attached to the values the
`sexpr' assumes over time.
}

\gfexample+
   ***** declare indconst clock t0 t1;
   ***** attach t0 to 0;
   t0 attached to '0
   ***** attach t1 to 1;
   t1 attached to '1
   ***** hardware clock to time;
   clock attached to time
   ***** done;
   >(SETQ time 0)
   0
   >(GETFOL)
   Hi!  Glad your back.  What would you like to talk about now?
   ***** simplify clock = t0;
   1   clock = t0     
   ***** done;
   >(SETQ time 1)
   1
   >(GETFOL)
   Hi!  Glad your back.  What would you like to talk about now?
   ***** simplify clock = t1;
   2   clock = t1     
+

\gfnotes{
  This command gives the possibility of changing the intended model. 
}

\gfcommand{represent}{default representation for sorts}
\index{represent}
\label{sec-rep-sort}

\gfsyntax{
  represent \{ \ARG{sort1}, \SEQ, \ARG{sortN} \} as \ARG{rep} \ALT \ARG{*};
}

\gfdescription{
  Sets the default representation for \ARG{sort1}, \SEQ, \ARG{sortN} to be
  \ARG{rep} ({\tt UNIVERSALREP} if \ARG{*} is specified).
  The default representation is used by the reflect command (see section
  \ref{sec-refl}).
}

\gfrecap{
Sets the default representation for `sort1', ..., `sortN' to be `rep'
(`UNIVERSALREP' if `*' is specified).
The default representation is used by the `reflect' command.
}

\gfexample+
   ***** declare sort s t;
   ***** decrep rep;
   ***** represent {s t} as rep;
   ***** represent {s t} as rep;
   s has already a default representation
+

\gfcommand{simplify}{semantic simplification}
\index{simplify}
\label{sec-simplify}

\gfsyntax{
  simplify \ARG{wff} \ALT \ARG{fact} \ALT \ARG{term};
}

\gfdescription{
  If \ARG{term} is provided as argument, three steps are performed:\\
  %
  \begin{itemize}
  \item 
    the interpretation of \ARG{term} in the computational model is computed;
  \item a preferred name for the interpretation is found;
  \item the equality of \ARG{term} with the preferred name is asserted as the next
    line of the proof.
    No action is taken if \ARG{term} has no interpretation in the computational
    model (has an undefined interpretation) or a preferred name for its
    interpretation does not exist.
  \end{itemize}
  %
  If {\em wff} is provided as an argument, two steps are performed:
  %
  \begin{itemize}
  \item the interpretation of {\em wff} in the computational model is computed 
    (in this case the interpretation will be {\tt TRUE}, {\tt FALSE} or an 
    undefined truth value)
  \item if the interpretation of {\em wff} is {\tt TRUE}, {\em wff} 
    is asserted as the next line of the proof, if it is {\tt FALSE} 
    the negation of {\em wff} is asserted.
  \end{itemize}
  %
  When \ARG{fact} is provided as argument, the simplify command works on the wff
  of \ARG{fact}.
}

\gfrecap{
  If `term' is provided as argument, three steps are performed:
      * the interpretation of `term' in the computational model is computed;
      * a preferred name for the interpretation is found;
      * the equality of $term$ with the preferred name is asserted as the next
        line of the proof. No action is taken if `term' has no interpretation
        in the computational model (has an undefined interpretation) or a
        preferred name for its interpretation does not exist.
  If `wff' is provided as an argument, two steps are performed:
      * the interpretation of `wff' in the computational model is computed 
        (in this case the interpretation will be `TRUE', `FALSE' or an 
        undefined truth value)
      * if the interpretation of `wff' is `TRUE', `wff' is asserted as the next
        line of the proof, if it is `FALSE' the negation of `wff' is asserted.
  When `fact' is provided as argument, the simplify command works on the wff
  of `fact'.
}

\gfexample+
   ***** declare indconst a b c;
   ***** decrep REP;
   ***** attach a dar [REP]a;
   a attached to 'a
   a is the preferred name of a
   ***** attach b dar [REP]b;
   b attached to 'b
   b is the preferred name of b
   ***** attach c to [REP]c;
   c attached to 'c
   ***** declare funconst F 1;
   F has been declared to be a Funconst
   ***** DEFLAM F(x) (IF (EQ x (QUOTE a)) (QUOTE b) 
                         (IF (EQ x (QUOTE b)) (QUOTE c)
                          (QUOTE UNDEF&)));
   ***** attach F to [REP=REP]F;
   F attached to F
   ***** simplify F(a);
   1   F(a) = b    
   ***** simplify F(b);
   F(b) : No simplification is possible.
   ***** simplify F(c);
   F(c) : No simplification is possible.
   ***** declare predconst P 1;
   P has been declared to be a Predconst
   ***** DEFLAM P(x) (IF (EQ x (QUOTE a)) TRUE 
                         (IF (EQ x (QUOTE b)) FALSE 
                          (QUOTE UNDEF&)));
   ***** attach P to [REP]P;
   P attached to P
   ***** simplify P(a);
   2   P(a)  
   ***** simplify P(b);
   3   not P(b)    
   ***** simplify P(c);
   P(c) : No simplification is possible.
   ***** extension UNIVERSAL by {a b c};
   Now the extension of UNIVERSAL is fixed to be : (a b c)
   ***** declare indvar x;
   UNIVERSAL is a sort
   x has been declared to be an Indvar
   ***** simplify exists x.F(x)=b;
   4   exists x. (F(x) = b)
   ***** simplify forall x.P(x);
   5   not forall x. P(x) 
+

\gfnotes{
  In the case of sorts with extensions (see the command {\tt extension} in section 
  \ref{sec-sort}) quantification is considered as {\bf bounded quantification}.
  In other words, let $P$ be a predicate and $x$ an indvar of sort $S$, where
  $S$ has extension $\{s_1,\ldots,s_n\}$.
  Then the following equivalences hold:
  $$ 
  \forall x P(x)\liff (P(s_1) \con \ldots \con P(s_n))
  $$
  $$
  \exists x P(x)\liff (P(s_1) \dis \ldots \dis P(s_n))
  $$
  %
  The command explicitly unfolds universal/existential statements into 
  their propositional equivalents. 
}



% introduction to the syntactic simplification's section
\newpage
\section{Syntactic simplification}
\label{sec-rew}

The subsections \ref{sec-rew-intro} and \ref{sec-rew-simpset} 
of this section have been taken from \cite{rww3}.


\subsection{Introduction}
\label{sec-rew-intro}

The basic idea  of syntactic simplification is repeated substitution of 
selected equalities and equivalences into a given expression.
More precisely, let $E$ be a set of universally quantified equations and 
equivalences ("rewrite rules"), so members of $E$ look like:
\begin{itemize}
\item $\forall\ {\vec x}.(t_1=t_2)$
\item $\forall\ {\vec y}.(F_1\ \liff \ F_2)$
\end{itemize}
where ${\vec x}$ and ${\vec y}$ are the {\indvar} sequences $x_1$...$x_n$
and $y_1$..,$y_m$, $t_1$ and $t_2$ are {\term}s, and $F_1$, $F_2$ are {\wff}s.

A match, or an immediate simplification, of a {{\GF}} expression $exp$ consists
 of replacing an occurrence of $t_1[x \leftarrow u]$($F_1[y \leftarrow v]$) 
in $exp$ by $t_2[x \leftarrow u]$($F_2[y \leftarrow v]$), where $u$($v$) is a 
sequence of terms and where $\leftarrow$ indicates substitution.

There are two problems to solve:
\begin{enumerate}
\item There may be more than one equation (or equivalence) whose left half 
      matches a given expression, so one has to establish a precedence 
      hierarchy for matching.

\item The order used by the algorithm to consider the subexpressions of a 
      given expression.
\end{enumerate}

{{\GF}}'s solution to the first problem is the following ordering expression:
each simplification expression (i.e., left half of a rewrite rule) is 
regarded as a linear string of atoms.
Each atom is either:

\begin{itemize}
\item a {\bf constant} (which is not bound by the universal quantifier in the
                       prefix);

\item an {\bf old variable} (which is bound by the universal quantifier in 
                            the prefix and which has occurred before in the 
                            linear string);

\item a {\bf new variable} (which is bound by the universal quantifiers in the
                           prefix and which has not occurred before in the 
                           linear string);
\end{itemize}

If we think of concatenating different atoms to a given initial string, then 
the atoms have this precedence ordering:
\begin{center}
constants $<$ old variables $<$ new variables
\end{center}
and expressions are ordered lexicographically in accordance with this 
ordering on atoms.

Let's consider, for example, the precedence relations among the simplification
 expressions :
$f(a,b,b)$, $f(a,b,c)$, $f(a,a,x)$, $f(a,x,x)$, $f(a,x,y)$, $f(x,x,x)$,
$f(x,x,y)$, 
where $f$, $a$, $b$, $c$ are constants and $x$, $y$ are variables.

The last four expressions are linearly ordered:
$$
f(a,x,x)<f(a,x,y)<f(x,x,x)f<(x,x,y)
$$
and each of the first three expressions is less than $f(a,x,x)$ and 
incomparable to the other two of the first three expressions:
$$
f(a,b,b)<f(a,x,x)
$$
$$
f(a,b,c)<f(a,x,x)
$$
$$
f(a,a,x)<f(a,x,x)
$$
Together with transitivity, these inequalities completely define the 
precedence relation.

As far as regard the second problem, {{\GF}}'s syntactic simplification code 
basically considers subexpressions of $exp$ in the usual left-to-right order.
The exceptions occur after a subexpression $exp'$ has been matched (and 
substituted for).
The algorithm then begins again at the subexpression one level above $exp'$.

The syntactic simplification algorithm has the usual problems of rewrite rules.
A typical difficulty is the infinitely recurring substitutions: 
for example if one uses $\forall\ x.x+y=y+x$ as simplification equation, 
the algorithm will attempt to make this substitution without end.

\subsection{Simplification sets}
\label{sec-rew-simpset}

Syntactic simplification in {\GF} is performed by using 
{\bf syntactic simplification sets} (called {\bf simpsets} from now on).
Simpsets contain a label ({\em simplabel}) to identify the
rewrite rules used to rewrite expressions.
{\GF} has built-in simpsets, but the user can define his own ones: he can specify a 
set of formulae or facts as rewrite rules in a {\bf basic simspset}
or he can compose already defined simpsets in {\bf compound simpsets}.

The {{\GF}} builtin simpsets (see figure \ref{fig-simpset}) are
{\bf \tt LPROPTREE}, {\bf \tt LQUANTREE}, {\bf \tt LARGIFTREE} and 
{\tt LOGICTREE}.
{\tt LPROPTREE} contains a set of rewrite rules 
corresponding to basic logical equivalences (e.g. $P\con\neg P\liff \bot$). 
{\tt LQUANTREE} contains a set of rewrite rules 
corresponding to logic equivalences for quantified formulas.
{\tt LARGIFTREE} contains a set of rewrite rules corresponding to logic
equalities and equivalences for conditional terms and formulas.
{\tt LOGICTREE} is the union of all the previous builtin simpsets.

\newpage

\begin{figure}[htbp]
\begin{center}
\fbox{
\parbox{16cm}{
$$
\begin{array}{ll}
\neg \neg P \liff P    & \ \ \                         \\
\neg {\tt TRUE} \liff \bot   & \ \ \  \neg \bot \liff {\tt TRUE}    \\
P \con \bot \liff \bot & \ \ \  \bot \con P\liff \bot   \\
P \con {\tt TRUE}  \liff  P  & \ \ \  {\tt TRUE}  \con P\liff P     \\
\neg P\con P\liff\bot  & \ \ \  P \con \neg P\liff \bot \\
P\con P\liff P         & \ \ \                         \\
P     \dis \bot \liff P  & \ \ \  \bot \dis P \liff P       \\
P \dis {\tt TRUE} \liff {\tt TRUE}   & \ \ \  {\tt TRUE}  \dis P \liff {\tt TRUE}   \\
\neg P \dis P\liff {\tt TRUE}  & \ \ \  P \dis \neg P \liff {\tt TRUE}  \\
P     \dis P     \liff P & \ \ \                         \\
P \imp \bot\liff\neg P & \ \ \  \bot \imp P\liff {\tt TRUE}   \\
P\imp {\tt TRUE}\liff {\tt TRUE}   & \ \ \  {\tt TRUE}  \imp P     \liff P\\
\neg P \imp P \liff P  & \ \ \  P \imp \neg P\liff\neg p\\
P \imp P\liff {\tt TRUE}     & \ \ \                         \\
P\liff \bot\liff\neg P  & \ \ \  \bot \liff P\liff\neg P  \\
P\liff {\tt TRUE}\liff P      & \ \ \  {\tt TRUE}  \liff P\liff P     \\
\neg P \liff P\liff\bot & \ \ \  P\liff\neg P\liff\neg\bot\\
P\liff P\liff {\tt TRUE}      & \ \ \                         \\
\end{array}
$$
}}
\fbox{
\parbox{16cm}{
$$
\begin{array}{ll}
\forall x.{\tt TRUE}  \liff {\tt TRUE} & \ \ \  \forall x.\bot \liff \bot    \\
\exists x.{\tt TRUE}  \liff {\tt TRUE} & \ \ \  \exists x.\bot \liff \bot    \\
\end{array}
$$
}}
\vspace{0.3cm} 
\fbox{
\parbox{16cm}{
$$
\begin{array}{ll}
\forall x y.{\em trmif}\ \bot{\em then}\ x\ {\em else}\ y = y    & \ \ \ 
\forall x y.{\em trmif}\ {\tt TRUE}\ {\em then}\ x\ {\em else}\ y\ = x \\
\forall x.  {\em trmif}\ P\ {\em then}\ x\ {\em else}\ x = x & \ \ \  \\
{\em wffif}\ \bot\ {\em then}\ P1\ {\em else}\ P2\ \liff P2 & \ \ \ 
{\em wffif}\ {\tt TRUE}\ {\em then}\ P1\ {\em else}\ P2\ \liff\ P1 \\
{\em wffif}\ P\ {\em then}\ P1\ {\em else}\ P1\ \liff\ P1 & \ \ \  \\
\end{array}
\\
$$
}}
\caption{The rewrite rules of {\tt LPROPTREE}, {\tt LQUANTREE}
\label{fig-simpset}
and {\tt LARGIFTREE}.}
\end{center}
\end{figure}


% user commands for syntactic simplification
\gfcommand{assertsimp}{simpset command}
\index{assertsimp}

\gfsyntax{
  assertsimp \ARG{simplabel}; 
}

\gfdescription{
  It generates a proof step for each formula contained in the rewrite rules
  of \ARG{simplabel}, which cannot be a builtin simpset.
}

\gfrecap{
It generates a proof step for each formula contained in the rewrite rules
of `simplabel', which cannot be a builtin simpset.
}

\gfexample+
   ***** declare indconst a b;
   ***** declare predconst q r 1;
   ***** declare indvar x;
   ***** setbasicsimp s1 at wffs {q(a), forall x.(q(x) iff r(x))};
   ***** assume a=b;
   1   a = b     (1)
   ***** assume q(b);
   2   q(b)     (2)
   ***** setbasicsimp s2 at facts {1,2};
   ***** setbasicsimp s2 at facts {1};
   s2 is already the label of a simpset
   ***** setcompsimp s4 at s1 uni s2;
   ***** assertsimp s1;
   3   q(a)     
   4   forall x. (q(x) iff r(x))     
   ***** assertsimp s2;
   s2 does not contain any wff to assert.
   ***** assertsimp s4;
   5   q(a)
   6   forall x. (q(x) iff r(x))
   ***** assertsimp LOGICTREE;
   LOGICTREE is the label of a builtin simpset, you can't assert it.
+
\gfcommand{rewrite}{syntactic simplifier command}
\index{rewrite}

\gfsyntax{
  rewrite \ARG{wff} \ALT \ARG{fact} \ALT \ARG{term} \OPT{by \ARG{simpexpr}};
}

\gfdescription{
  It rewrites the given expression by using the union of the rewrite rules 
  indicated by the \ARG{simpexpr}.
  If \ARG{simpexpr} is not specified, {\tt LOGICTREE} is used.
  If \ARG{term} is provided as an argument, two steps are performed:
  %
  \begin{itemize}
  \item \ARG{term} is rewritten by using the set of rewrite rules 
    indicated by \ARG{simpexpr}.
  \item the equality of \ARG{term} with its rewritten form is asserted as the next 
    line of the proof. The dependencies depend on the simpsets actually used
    during the syntactic simplification.
  \end{itemize}
  %
  If \ARG{wff} is provided as argument, also two steps are performed:
  %
  \begin{itemize}
  \item \ARG{wff} is rewritten by the set of rewrite rules of the
    \ARG{simpexpr} result.
  \item if \ARG{wff} is rewritten to {\tt TRUE}, \ARG{wff} is asserted as the
    next line in the proof, if \ARG{wff} is rewritten to {\tt FALSE},
    $\neg$ \ARG{wff} is asserted, otherwise the equivalence of \ARG{wff} with 
    its rewritten form is asserted.
    The dependencies depend on the simpsets used during the syntactic
    simplification.
  \end{itemize}
  %
  When \ARG{fact} is provided as an argument, the rewrite command works on the wff
  of the \ARG{fact}.
}

\gfrecap{
It rewrites the given expression by using the union of the rewrite rules 
indicated by the `simpexpr'.
If `simpexpr' is not specified, LOGICTREE is used.
}

\gfexample+
   ***** declare indconst A,B;
   ***** declare indvar X,Y;
   ***** declare funconst F 2;
   ***** declare funconst G 1;
   ***** declare sentconst P;
   ***** assume forall X . F(X,A) = A;
   1   forall X. (F(X,A) = A)     (1)
   ***** assume forall X . F(X,X) = G(X);
   2   forall X. (F(X,X) = G(X))     (2)
   ***** assume forall X Y . F(X,Y) =Y;
   3   forall X Y. (F(X,Y) = Y)     (3)
   ***** axiom F1:forall X . F(X,A) = A;
   F1 : forall X. (F(X,A) = A)
   ***** axiom F2:forall X . F(X,X) = G(X);
   F2 : forall X. (F(X,X) = G(X))
   ***** axiom F3:forall X Y. F(X,Y) = Y;
   F3 : forall X Y. (F(X,Y) = Y)
   ***** setbasicsimp S1 at facts {1};
   ***** setbasicsimp S2 at facts {2};
   ***** setbasicsimp S3 at facts {3};
   ***** setbasicsimp S4 at facts {F1};
   ***** setbasicsimp S5 at facts {F2};
   ***** setbasicsimp S6 at facts {F3};
   ***** setbasicsimp SIMPEQ at wffs {forall X.(X=X iff TRUE)};
   ***** setcompsimp S7 at S1 uni S2 uni S3;
   ***** rewrite F(A,A) by S6;
   4   F(A,A) = A     
   ***** rewrite F(A,A) by S5;
   5   F(A,A) = G(A)     
   ***** rewrite F(A,A) by S4;
   6   F(A,A) = A     
   ***** rewrite F(A,A) by S1;
   7   F(A,A) = A     (1)
   ***** rewrite F(A,A) by S2;
   8   F(A,A) = G(A)     (2)
   ***** rewrite F(A,A) by S3;
   9   F(A,A) = A     (3)
   ***** rewrite F(A,A) by S7;
   10   F(A,A) = A     (1)
   ***** rewrite F(B,B) by S1 uni S3;
   11   F(B,B) = B     (3)
   ***** rewrite F(B,B) by S1;
   F(B,B): No simplification is possible
   ***** rewrite not TRUE by S1;
   not TRUE: No simplification is possible
   ***** rewrite not TRUE;
   12   not (not TRUE)     
   ***** rewrite TRUE imp (P imp X=X);
   13   (TRUE imp (P imp (X = X))) iff (P imp (X = X)) 
   ***** rewrite TRUE imp (P imp X=X) by SIMPEQ uni LOGICTREE;
   14   TRUE imp (P imp (X = X))
   ***** rewrite F(A,A) by S7;
   15   F(A,A) = A     (1)
   ***** rewrite F(A,A)=A by S7;
   16   F(A,A) = A  iff (A = A)   (1)
   ***** rewrite F(A,A)=A by S7 uni SIMPEQ;
   17   F(A,A) = A     (1)
   ***** rewrite F(A,A)=G(A) by S7;
   18   (F(A,A) = G(A)) iff (A = G(A))     (1)
   ***** rewrite F(B,B) by S7;
   19   F(B,B) = G(B)     (2)
   ***** rewrite F(B,B)=G(B) by S7 uni SIMPEQ;
   20   F(B,B) = G(B)     (2)
   ***** rewrite F(B,B)=G(B) and F(A,A)=A by S7 uni SIMPEQ uni LOGICTREE;
   21   (F(B,B) = G(B)) and (F(A,A) = A)     (1 2)
   ***** rewrite F(A,A) by S7 dif S1 ;
   22   F(A,A) = G(A)     (2)
   ***** rewrite F(A,A) by S7 dif (S1 uni S2);
   23   F(A,A) = A     (3)
   ***** rewrite F(A,A)=A by S3 dif (S1 uni S2) uni SIMPEQ;
   24   F(A,A) = A     (3)
+
   


% introduction to the syntactic/semantic simplification's section
\newpage
\section{Syntactic and semantic simplification}

Some of the commands perform both syntactic and semantic simplifications.


% user commands for the syntactic/semantic simplification's section
\gfcommand{eval}{mixed simplifier command}
\index{eval}
\label{sec-eval}

\gfsyntax{
  eval \ARG{wff} \ALT \ARG{fact} \ALT \ARG{term} \OPT{by \ARG{simpexpr}};
}

\gfdescription{
  This command evaluates the expression (\ARG{wff}, the wff of \ARG{fact}
  or \ARG{term} respectively) by combining the semantic evaluation of the
  expression in the simulation structure and the syntactical rewriting
  performed by using the union of the rewrite rules indicated by 
  \ARG{simpexpr}.
}

\gfrecap{
  This command evaluates the expression (`wff', the wff of `fact'
  or `term' respectively) by combining the semantic evaluation of the
  expression in the simulation structure and the syntactical rewriting
  performed by using the union of the rewrite rules indicated by 
  `simpexpr'.
}

\gfexample+
   ***** declare indconst a b c;
   ***** decrep REP;
   ***** attach a dar [REP]a;
   a attached to 'a
   a is the preferred name of a
   ***** attach b dar [REP]b;
   b attached to 'b
   b is the preferred name of b
   ***** attach c dar [REP]c;
   c attached to 'c
   c is the preferred name of c
   ***** declare funconst G 2;
   ***** declare indvar x y;
   ***** setbasicsimp S at wffs {forall x y.G(x y)=x};
   ***** declare predconst P 1;
   ***** DEFLAM P(x) (IF (EQ x (QUOTE a)) TRUE 
                         (IF (EQ x (QUOTE b)) FALSE 
                          (QUOTE UNDEF&)));
   ***** attach P to [REP]P;
   P attached to P
   ***** eval P(G(a,G(b,c))) by S;
   1   P(G(a,G(b,c)))     
   ***** eval P(G(b,c)) and P(c) by S;
   2   not (P(G(b,c)) and P(c))
   ***** eval P(G(c,a)) by S;
   3   P(G(c,a)) iff P(c)   
   ***** eval forall x.P(G(x x)) by S;
   4   forall x. P(G(x,x)) iff forall x. P(x)     
   ***** extension UNIVERSAL by {a b c};
   Now the extension of UNIVERSAL is fixed to be : (a b c)
   ***** eval forall x.P(G(x x)) by S;
   5   not forall x. P(G(x,x)) 
+

\gfnotes{
  In the case of sorts with extensions (see the command {\tt extension} in section 
  \ref{sec-sort}) quantification is considered as  {\bf bounded quantification}.
  In other words, let $P$ be a predicate and $x$ an indvar of sort $S$, where
  $S$ has extension $\{s_1,\ldots,s_n\}$.
  Then the following equivalences hold:
  $$ 
  \forall x(P(x)\liff P(s_1) \con \ldots P(s_n))
  $$
  $$
  \exists x(P(x)\liff P(s_1) \dis \ldots P(s_n))
  $$
  %
  The command explicitly unfolds universal/existential statements into 
  their propositional equivalents. The expansion is performed 
  syntactically, that is the formula $\forall x P(x)$ [$\exists x P(x)$]
  is rewritten as $P(s_1) \con \ldots \con P(s_n)$
  [$P(s_1) \dis \ldots \dis P(s_n)$].
  The expansion mechanism embedded in {\tt simplify} is not used by {\tt eval}.
}


\gfcommand{let}{evaluation plus attachment}
\index{let}

\gfsyntax{
  let \ARG{\indconst} to \ALT dar [ \ARG{rep} ] \ARG{term};
}

\gfdescription{
  This command evaluates the \ARG{term} using the mixed evaluation mechanism
  of the {\tt eval} command.
  If the evaluation returns a {\HG} representation for \ARG{term}, then it is
  attached to  \ARG{indconst} with representation function \ARG{rep}. Then the 
  equality of \ARG{indconst} with \ARG{term} is asserted as the next line in the
  proof, otherwise an error message is given. If \ARG{rep} is not specified the
  representation is the default representation. If this does not happen {\GF}
  outputs an error message.
}

\gfrecap{
This command evaluates the `term' using the mixed evaluation mechanism
of the {\tt eval} command.
If the evaluation returns a HGKM representation for `term', then it is
attached to  `indconst' with representation function `rep'. Then the 
equality of `indconst' with `term' is asserted as the next line in the
proof, otherwise an error message is given. If `rep' is not specified the
representation is the default representation. If this does not happen GETFOL
outputs an error message.
}

\gfexample+
   ***** declare indconst a b c;
   ***** attach b to b;
   b attached to 'b
   ***** attach c to c;
   c attached to 'c
   ***** declare funconst h 2;
   ***** DEFLAM h(x y) (QUOTE d);
   ***** attach h to h;
   h attached to h
   ***** let a dar h(b c);
   a attached to 'd
   a is the preferred name of d
   1   a = h(b,c)     
+



 	% loading introduction to the section
\newpage
\section{Multiple contexts}
\label{sec-cxt}

\subsection{Introduction}

\begin{quote}\em
  ... When reasoning, people seem to be able to switch focus of their
  attention and make always some sort of local reasoning ...
  \cite{giunchiglia2}
\end{quote}

Structuring the knowledge into {\em distinguished partial descriptions of the
world}, has been hinted as a cognitively plausible hypotheses. 
Distinct partial descriptions can be represented by {\GF} {\bf contexts}.
A {\GF} context contains its own language defined by a set of declarations, 
its own axioms and definitions, and its own computational model.
Reasoning can be performed within a context. 
You can type any command defined so far within any context in {\GF}.
Multiple proofs can be performed within a context.
When you work in {\GF} you are always in one context.
The context in which you are working in is called the {\em current context}.
When you enter the system the current context is empty and without name.
If you want to leave the context to work in another one, you have to give the
context a name to refer to it later (by the command {\tt namecontext}).
You can create a new context by using {\tt makecontext}, and  switch to it
by using {\tt switchcontext}.
The context you switch to then becomes the current context.


% loading explanation of commands
\gfcommand{copycontext}{Multiple contexts manipulation}
\index{copycontext}

\gfsyntax{
  copycontext \ARG{ctx-name};
}

\gfdescription{
  A new context with name \ARG{ctx-name} is created, and the current context
  is copied in it.
}

\gfrecap{
  A new context with name `ctx-name' is created, and the current context
  is copied in it.
}

\gfexample+
   ***** show whereami;
   You are now using an unnamed context.
   You are now using an unnamed proof.
   ***** namecontext C1;
   You have named the current context: C1
   ***** show whereami;
   You are now using context: C1
   You are now using an unnamed proof.
   ***** declare sentconst A;
   ***** makecontext C2;
   You have created the empty context: C2
   ***** switchcontext C2;
   You are now using context: C2
   ***** declare indvar A;
+
\gfcommand{copylex}{Language declaration through contexts}
\index{copylex}

\gfsyntax{
	copylex	\ARG{ctx-name};
}

\gfdescription{
	This command copies in the current context all the symbols and sorts
	declared in the context {\em ctx-name}.
	The command has no effects in the case there is at least a symbol in the
	current context that has the same name as a symbol in the context
	{\em ctx-name}.
}

\gfrecap{
This command copies in the current context all the symbols and sorts
declared in the context `ctx-name'.
The command has no effects in the case there is at least a symbol in the
current context that has the same name as a symbol in the context `ctx-name'.
}


\gfexample+
   ***** declare indconst a b;
   ***** declare sentconst A B;
   ***** declare sort S1 S2;
   ***** namecontext C1;
   You have named the current context: C1
   ***** makecontext C2;
   You have created the empty context: C2
   ***** switchcontext C2;
   You are now using context: C2
   You are switching to a proof with no name.
   ***** probe declare;
   ***** copylex C1;
   S1 has been declared to be a sort
   S2 has been declared to be a sort
   A has been declared to be a Sentconst
   B has been declared to be a Sentconst
   a has been declared to be an Indconst
   b has been declared to be an Indconst
   ***** copylex C1;
   COPYLEX cannot be done: A has already been declared
   ***** copylex C2;
   You cannot copy the lex of the current context
+

\gfcommand{makecontext}{Multiple contexts' manipulation}
\index{makecontext}

\gfsyntax{
  makecontext \ARG{ctx-name};
}

\gfdescription{
  A new empty context  with name  \ARG{ctx-name} is created.
}

\gfrecap{
  A new empty context  with name  `ctx-name' is created.
}

\gfexample+
   ***** show whereami;
   You are now using an unnamed context.
   You are now using an unnamed proof.
   ***** namecontext C1;
   You have named the current context: C1
   ***** show whereami;
   You are now using context: C1
   You are now using an unnamed proof.
   ***** declare sentconst A;
   ***** makecontext C2;
   You have created the empty context: C2
   ***** switchcontext C2;
   You are now using context: C2
   ***** declare indvar A;
+

\gfcommand{namecontext}{Multiple contexts' manipulation}
\index{namecontext}

\gfsyntax{
  namecontext \ARG{ctx-name};
}

\gfdescription{
  If the current context has no  name,  it is named with \ARG{ctx-name}.
}

\gfrecap{
  If the current context has no  name,  it is named with `ctx-name'.
}

\gfexample+
   ***** show whereami;
   You are now using an unnamed context.
   You are now using an unnamed proof.
   ***** namecontext C1;
   You have named the current context: C1
   ***** show whereami;
   You are now using context: C1
   You are now using an unnamed proof.
   ***** declare sentconst A;
   ***** makecontext C2;
   You have created the empty context: C2
   ***** switchcontext C2;
   You are now using context: C2
   ***** declare indvar A;
+
\gfcommand{reset}{{\GF} reset}
\index{reset}

\gfsyntax{
  reset;
}

\gfdescription{
  Resets the whole {\GF} system.
}

\gfrecap{
  Resets the whole GETFOL system.
}

\gfexample+
   ***** namecontext c; nameproof p;
   You have named the current context: c
   You have named the current proof: p
   
   ***** show whereami;
   You are now using context: c
   You are now using the proof: p
   
   ***** declare sentconst A;
   A has been declared to be a Sentconst
   ***** assume A;
   1   A     (1)
   ***** show proof;
   1   A     (1)
   ***** makecontext c1;
   You have created the empty context: c1
   ***** switchcontext c1;
   You are now using context: c1
   You are switching to a proof with no name.
   
   ***** reset;
   Resetting the whole GETFOL-system
   
   ***** show whereami;
   You are now using an unnamed context.
   You are now using an unnamed proof.
   
   ***** show proof;
+

\gfcommand{switchcontext}{Multiple contexts' manipulation}
\index{switchcontext}

\gfsyntax{
  switchcontext \ARG{ctx-name};
}

\gfdescription{ 
  Switches from the current context to the context \ARG{ctx-name} which
  becomes the current context.
}

\gfrecap{ 
  Switches from the current context to the context `ctx-name' which
  becomes the current context.
}

\gfexample+
   ***** show whereami;
   You are now using an unnamed context.
   You are now using an unnamed proof.
   ***** namecontext C1;
   You have named the current context: C1
   ***** show whereami;
   You are now using context: C1
   You are now using an unnamed proof.
   ***** declare sentconst A;
   ***** makecontext C2;
   You have created the empty context: C2
   ***** switchcontext C2;
   You are now using context: C2
   ***** declare indvar A;
+

 	% loading introduction to the section
\newpage
\section{Metareasoning}
\label{sec-meta}

\subsection{Introduction}

A special context is {\meta}.
{\GF} recognizes {\meta} as a metatheory of all the other contexts.
The context {\meta} can be used to perform {\em metareasoning}, that is to
describe other contexts and to reason about them. 
Metareasoning in {\meta} is performed by employing the following novel
features:
%
\begin{itemize}
  \item
    the metatheory is, in general, distinct from the object theories it
    describes; 
  \item
    the link between the metalanguage and the object language is not performed
    by encoding, but rather by naming \cite{giunchiglia3}.
    Naming is implemented by using the commands which implement reasoning in
    the computational model of a context (see section \ref{sec-comp}).
    These features are available to the user by the commands \C{attach},
    \C{simplify}, \C{eval} etc.  
  \item
    Metareasoning and object reasoning can be mixed via the reflection rule 
    \cite{giunchiglia3}:

    \begin{equation}
      R_{down}
      \fraz{\der{M} Theorem(``w'')}{\der{O} w}
      \label{refl}
    \end{equation}

    where $M$ and $O$ stand for {\meta} and object theory respectively. 
    This rule is implemented in the {\GF} command \C{reflect}.
    The command knows that some form of metareasoning must be performed in
    {\meta} to deduce the metastatement $Theorem(``w'')$.
    The command can use the reflection rule (\ref{refl}) to assert a new proof
    line in the object level context (the context in which object level 
    reasoning is performed and where the command \C{reflect} is typed in).
  \item
    Any context can be the object level context, {\meta} itself.
    The amalgamation of the object and meta level is a particular case of {\GF}
    metareasoning. 
\end{itemize}

In {\meta}, the user is free to declare any language, any set of axioms and to 
define any computational model. This amounts to say that {\meta} is the 
``metatheory'' of a theory represented in a context as far as the user defines
the appropriate attachments and axioms.
A special unary predicate symbol which can be declared in {\meta} is
{\tt THEOREM}: this is the predicate recognized as meaning theoremhood by the
the reflect rule (\ref{refl}) in the command {\tt reflect}.



% loading explanation of commands
\gfcommand{mattach}{semantic meta attachment}
\index{mattach}

\gfsyntax{
   mattach \ARG{indconst} to \ALT dar \OPT{[rep]}
   \ARG{cname}:\ARG{pname}:\ARG{sort}:\ARG{object};
}

\gfdescription{
   Defines an attachment for a constant of the context {\em meta}.

   \ARG{rep}, if present, can be a representation function or \verb+*+.
   If it is \verb+*+ or no representation is specified, then the default representation function 
   {\tt UNIVERSALREP} is taken.
   \ARG{indconst} is a symbol declared to be an INDCONST in {\em meta}; \ARG{cname} is the name
   of the context to which \ARG{object} belongs; \ARG{pname} is the name of the proof in which
   \ARG{object} is present; \ARG{sort} is a sort of the meta-context associated to one of the
   syntactic categories reported with the {\tt reflect} command; \ARG{object} is an object of
   type \ARG{sort}.

   This command implements the mechanism of ``naming'' symbols or objects belonging to the 
   context \ARG{cname}, {\em ie.} the creation of names denoting objects of \ARG{cname}.
   \ARG{indconst} can be attached ``one way'' (using {\tt to}) or ``two ways'' (using {\tt dar}).
   The ``one way'' attachment tells the semantic interpretation mechanism that \ARG{indconst} is
   the name in {\meta} of the {\GF} object in the context \ARG{cname} corresponding to
   \ARG{object}.
   The two ways attachment tells the semantic interpretation mechanism that whenever the 
   (data structure representing) the \ARG{object} is computed as the representation of a 
   term {\em t}, then \ARG{indconst} should be returned as the simplified version of {\em t}.
}

\gfrecap{
This command implements the mechanism of ``naming'' symbols or objects belonging to the 
context `cname', ie. the creation of names denoting objects of `cname'.
`indconst' can be attached ``one way'' (using `to') or ``two ways'' (using `dar').
The ``one way'' attachment tells the semantic interpretation mechanism that `indconst' is
the name in meta of the GETFOL object in the context `cname' corresponding to
`object'.
The two ways attachment tells the semantic interpretation mechanism that whenever the 
(data structure representing) the `object' is computed as the representation of a 
term `t', then `indconst' should be returned as the simplified version of `t'.
}

\gfexample+
   ***** namecontext META;
   ***** nameproof P1;

   ***** declare indconst sc [SENTCONST];
   ***** declare indconst ic [INDCONST];
   ***** declare indconst vl [FACT];
   ***** declare indconst f1 [FACT];

   ***** DECREP  SENTCONST INDCONST FACT;

   ***** represent { SENTCONST } as SENTCONST;
   ***** represent { INDCONST } as INDCONST;
   ***** represent { FACT } as FACT;

   ***** makecontext C;
   ***** switchcontext C;
   ***** declare indconst c;
   ***** declare sentconst A;
   ***** nameproof P1;
   You have named the current proof: P1

   ***** assume c=c;
   1   c = c     (1)

   ***** makeproof P2;
   You have created the empty proof: P2

   ***** switchproof P2;
   You are now using the proof: P2

   ***** assume A imp A;
   1   A imp A     (1)

   ***** label fact ax = 1;

   ***** switchcontext META;

   ***** MATTACH sc TO  C::SENTCONST:A;
   ctext-get: I changed context to: C
   ctext-get: I changed context to: META
   sc attached to 'A

   ***** MATTACH ic DAR C:P2:INDCONST:c;
   ctext-get: I changed context to: C
   ctext-get: I changed context to: META
   ic attached to 'c
   ic is the preferred name of c

   ***** MATTACH vl DAR [SENTCONST] C:P1:FACT:1;
   ctext-get: I changed context to: C
   proof-get: I changed proof to: P1
   proof-get: I changed proof to: P2
   ctext-get: I changed context to: META
   vl attached to '(1 (= c c) (1) ASSUME (%WFF% = c c))
   vl is the preferred name of (1 (= c c) (1) ASSUME (%WFF% = c c))

   ***** MATTACH f1 TO  C:P2:FACT:1;
   ctext-get: I changed context to: C
   ctext-get: I changed context to: META
   f1 attached to '(1 (imp A A) (1) ASSUME (%WFF% imp A A))

   ***** MATTACH f1 DAR C:P2:FACT:ax;
   ctext-get: I changed context to: C
   ctext-get: I changed context to: META
   f1 attached to '(1 (imp A A) (1) ASSUME (%WFF% imp A A))
   f1 is the preferred name of (1 (imp A A) (1) ASSUME (%WFF% imp A A))

+

\gfnotes{}


\gfcommand{reflect}{reflection}
\index{reflect}
\label{sec-refl}

\gfsyntax{
  reflect \ARG{M-fact} \ARG{arg1} \ARG{arg2} \SEQ \ARG{argN};
}

\gfdescription{
  In the   following description we  call ``object context''  the 
  context where the {\tt reflect} command is executed.

  \ARG{M-fact} is any fact of  the context {\meta} whose wff is of the
  form $\forall x_1 x_2\ldots x_n A(x_1, x_2,\ldots,x_n)$, $(n \geq 0)$,
  where the sorts of the variables $x_1, x_2,\ldots,x_n$ 
  correspond to some {\GF} syntactic category ({\em term, wff, fact} ... ).
  For any syntactic category corresponding to a sort in {\meta},
  {\GF} provides the necessary parsing routine.
  This parsing routine is necessary to run the reflect command.
  The relation between sorts in {\meta} and the associated parsed syntactic
  category is the following:

  \begin{figure}
    \begin{tabular}{|l|l|}
   \hline
   {\bf sort}&  {\bf syntactic category} \hspace{7cm} \\ \hline \hline
   SENTCONST &  a {\em sentconst}; \\
   QUANT     &  a {\em quant} (quantifier: {\tt forall} or {\tt exists}); \\
   SORT      &  a symbol declared as a sort; \\
   DECSYM    &  any declared symbol: {\em sym};\\
   FACT      &  a {\em fact};\\
   WFF       &  a {\em wff}; \\
   WFFIF     &  a {\em wffif};\\
   QUANTWFF  &  a {\em quantwff} (of the form  {\tt forall ... }  or 
                {\tt exists  ... });\\
   AWFF      &  an atomic wff (a wff of the form {\tt P( ... ))};\\
   TERM      &  a {\em term}; \\
   TERMIF    &  a {\em termif};\\
   INDSYM    &  a symbol declared as an {\em indconst} or {\em indvar} or
                {\em indpar}; \\
   INDVAR    &  a symbol declared as an {\em indvar};\\
   INDPAR    &  a symbol declared as an {\em indpar};\\
   INDCONST  &  a symbol declared as an {\em indconst};\\
   SENTSYM   &  a symbol declared as a {\em sentconst} or a {\em sentpar}; \\
   SENTPAR   &  a symbol declared as a {\em sentpar}; \\
   SENTCONST &  a symbol declared as a {\em sentconst}; \\
   APPLSYM   &  a symbol declared as a {\em funconst, funpar, predconst, 
                predpar} \\
             &  or a boolean connective; \\ 
   PREDSYM   &  a symbol declared as a {\em predconst} or {\em predpar}; \\
   PREDPAR   &  a symbol declared as a {\em predpar}; \\
   PREDCONST &  a symbol declared as a {\em predconst}; \\
   FUNSYM    &  a symbol declared as a {\em funconst} or {\em funpar}; \\
   FUNPAR    &  a symbol declared as a {\em funpar}; \\
   FUNCONST  &  a symbol declared as a {\em funconst}; \\ \hline 
   \end{tabular}
   \caption{
	Sorts for {\GF} syntactic categories.
   }
   \end{figure}

   If, for example,  the sort of the variable $x_1$  is SENTCONST,  we
   expect $x_1$ to denote any object that is declared  as  a {\em sentconst}
   in the object  context .  Therefore \ARG{arg1}, \ARG{arg2}, \SEQ, \ARG{argN}
   are objects in the object context, such that all \ARG{arg$_i$} are of
   the syntactic category corresponding to the sort of the variables $x_i$
   in \ARG{M-fact}.

   In the following we give an explanation of the major steps performed during
   the execution of the command {\tt reflect} \cite{giunchiglia3}~\footnote{
   This description is the generalization of the example reported below
   and copied from \cite{giunchiglia3}.}:

   \begin{enumerate}
   \item 
     In the object context, when \C{REFLECT} is executed, {\GF}, after parsing 
     \C{REFLECT}, knows that the next argument is the label of a fact in 
     {\meta}. 
     Thus, {\GF} switches context automatically and goes to {\meta}.
   \item
     In {\meta}, the first argument of the command (\ARG{M-fact}) is parsed 
     and the formula of the fact whose label is \ARG{M-fact} is returned:
     $\forall x_1 x_2\ldots x_n A(x_1, x_2,\ldots,x_n)$.
     The variables $x_1, x_2,\ldots,x_n$ must be instantiated to constants
     in {\meta} which will be the names of the objects in the object context
     \ARG{arg1}, \ARG{arg2}, \SEQ, \ARG{argN}.
     The syntactic type of these objects must correspond to the sort of the 
     variables in {\meta}.
     For instance, if the sort of $x_1$ is WFF, then it must be instantiated
     to a constant denoting a {\it wff} in the object context. At this point
     {\GF} knows that \ARG{arg1} must be a {\it wff} in the object context,
     and so {\em arg}$_1$ can be parsed.
     This step provides {\GF} with the information needed to parse \ARG{arg1},
     \SEQ, \ARG{argN} in the object context.
   \item 
     {\GF} switches to the object context and parses  \ARG{arg1}, \ARG{arg2}, 
     \SEQ, \ARG{argN}, using the parsing functions of the syntactic categories
     corresponding to the variables $x_1, x_2,\ldots,x_n$ respectively.
   \item 
     {\GF} switches to {\meta} and automatically declares $n$ constants in 
     {\meta}. Let them be $c_1, c_2, ... c_n$, with the sorts of $x_1, x_2, 
     ... x_n$ respectively.
     Any constant in $c_1, ... c_n$ is automatically ``attached'' to the objects
     returned by parsing \ARG{arg1}, \ARG{arg2}, \SEQ, \ARG{argN} respectively.
     The representation associated to these constants is the representation
     declared for their sorts (see the command {\bf represent}, section
     \ref{sec-rep-sort}).
   \item 
     Still in {\meta}, an universal elimination is performed on 
     $\forall x_1 x_2\ldots x_n A(x_1, x_2,\ldots,x_n)$ obtaining
     $A(c_1, \ldots ,c_n)$.
   \item 
     Still in {\meta}, the formula $A(c_1,...,c_n)$ is automatically evaluated
     by the command {\tt eval} (see section \ref{sec-eval}).
     In this step {\GF}, by using the command {\bf eval}, performs metalevel
     reasoning by computation in the model.
     This metalevel reasoning is used to compute a formula of the form 
     $Theorem(``w'')$.
     If the metalevel reasoning does not lead to $Theorem(``w'')$, then an error
     message is returned. Otherwise the evaluation of the ground term ``$w$''
     gives $w$, the formula of a fact to be asserted in the object context by
     $R_{down}$
   \item
     At this point the reflection rule $R_{down}$ can be applied.
     Thus {\GF} forgets the constants $c_1 ... c_n$ declared in {\meta} and the
     attachments, switches back to the object context and asserts a new fact
     whose formula is $w$ and whose dependencies are the union of the 
     dependencies of the facts in {\em arg$_1$ ... arg$_n$} if there are any.
   \end{enumerate}

	Let us define two context {\tt OBJ} and {\tt META} as follows:

	\gfsourcefile{example.tst}{
	  NAMECONTEXT META; \\
	  DECLARE SORT TERM WFF;\\
	  DECLARE PREDCONST THEOREM 1;\\
	  DECLARE FUNCONST mkequal (INDVAR, INDVAR) = WFF;\\
	  DECLARE INDVAR x [TERM];\\
	  AXIOM M1: forall x. THEOREM(mkequal(x,x));\\
	  DECREP TERM;\\
	  DECREP WFF;\\
	  REPRESENT \{TERM\} AS TERM;\\
	  REPRESENT \{WFF\} AS WFF;\\
	  ATTACH mkequal TO [TERM,TERM=WFF] mkequ;\\
	  MAKECONTEXT OBJ;\\
	  SWITCHCONTEXT OBJ;\\
	  DECLARE INDCONST c;\\
	  DECLARE INDVAR   x;\\
	  DECLARE FUNCONST f 2;
	}

	Let's now type the following lines in {\GF}:

	\begin{quote}\tt
	   ***** FETCH example.tst;
	   ...

	   ***** REFLECT M1 c;
	   1  c = c;
	   ***** REFLECT M1 f(x,f(c,c));
	   2  f(x,f(c,c)) = f(x,f(c,c))
	\end{quote}

	Let us describe  step by step what happened during  the execution of
	the two  previous command \C{REFLECT}.

	\begin{enumerate}
	\item
    In {\tt OBJ} the word \C{REFLECT}  is parsed.
    The next argument must be  the name of a fact  in {\meta}.  Thus {\GF}
    automatically switches context and goes to {\meta}.
	\item
    In  {\meta}, {\tt M1} is parsed  and the axiom
    {\tt forall  x.THEOREM(mkequ(x,x))} is returned.  The variable {\tt x}
    in {\tt M1} must be instantiated  to a constant  in {\meta} which will
    be the name of a symbol of the language  of {\tt OBJ}.   Since the
    sort of  $x$ is {\em TERM}, then  the symbol of  the language  of {\tt
    OBJ}  must have  syntactic  type {\em term}   (it must be  a term).
	\item
    {\GF} switches to the context {\tt OBJ}.
    In  {\tt OBJ} the   term  {\tt  c} [{\tt f(x,f(c,c))} in  the second case]
    is  parsed.  
	\item
    Since no more arguments are needed, {\GF} switches back to {\meta}.
    In {\meta} a new constant, say {\tt C1} of sort {\tt TERM} is created and
    added to the language of {\meta}.
    {\tt   C1} is {\em attached} to the term {\tt c}  [{\tt f(x,f(c,c))}] 
    of the language of {\tt OBJ}.  
    The representation associated to {\tt C1} is {\tt TERM} which is associated
    to the sort {\tt TERM} by the command {\tt REPRESENT} in example.tst.
    In this way,  {\tt C1} is defined  as the name in  {\meta} 
    of {\tt c} [{\tt f(x,f(c,c))}].   
	\item
    Still in {\meta}, an   universal elimination is performed on {\tt  M1}
    obtaining\\
    {\tt THEOREM(mkequ(C1,C1))}.
	\item
    Still in {\meta}, {\tt  THEOREM(mkequ(C1,C1))} is evaluated in {\meta}'s 
    model.
    {\tt THEOREM} has no interpretation, {\tt mkequ(C1,C1)} evaluates to 
    {\tt c = c} [{\tt f(x,f(c,c)) = f(x,f(c,c))}], namely {\tt mkequ(C1,C1)}
    turns out to be the name of {\tt c = c} [{\tt f(x,f(c,c)) = f(x,f(c,c))}].
    So the result of this step is something that could be written as 
    {\tt THEOREM(``c =  c'')}.
    [{\tt THEOREM(``f(x,f(c,c)) = f(x,f(c,c))'')}],  where {\tt ``c = c''}
    [{\tt ``f(x,f(c,c)) = f(x,f(c,c))''}]  should be read as  the  name  of
    {\tt c = c} [{\tt (f(f(c,c),c),f(x,c)) = f(x,f(c,c))}]
	\item
    At  this point the  reflection rule can be applied.
    {\GF} forgets everything in {\meta}, (in this case {\tt C1} from the
    language and {\tt c} [{\tt f(x,f(c,c))}] from the domain of the
   	interpretation).
    {\GF} switches back to the context {\tt OBJ}, and assert a new fact
    with wff  {\tt  c = c} [{\tt  f(x,f(c,c))}] and with the empty deplist.
    \end{enumerate}

    The following example shows how {\GF} computes the deplist of a fact derived 
    by a {\tt reflect} command whose arguments contain a fact.
}

\gfrecap{
	Reflection
}

\gfexample+
   <host prompt> cat example.tst
   NAMECONTEXT META;\\
   DECLARE SORT FACT WFF;\\
   DECLARE INDVAR fc [FACT];\\
   DECLARE FUNCONST wffof (FACT) = WFF;\\
   DECLARE PREDCONST THEOREM 1;\\
   DECREP FACT;\\
   DECREP WFF; \\
   REPRESENT \{WFF\} AS WFF; \\
   REPRESENT \{FACT\} AS FACT;\\
   ATTACH wffof TO [FACT = WFF] fact\-get\-wff;\\
   AXIOM M2: forall fc. THEOREM(wffof(fc));\\
   MAKECONTEXT OBJ;\\
   SWITCHCONTEXT OBJ;\\
   DECLARE SENTCONST A;

   <host prompt> GETFOL
   ...

   ***** FETCH example.tst;
   ***** ASSUME A;
   1   A     (1)
   ***** REFLECT M2 1;
   2   A     (1)
+



 
	% ........................... QUICK REFERENCE ..........................
	\newpage
	\appendix
	\section{Open problems}
\label{app-op}

If a {\tt reflect} command is used when a context {\tt META} does not
exist the error produced is that the relevant fact can not be found.
The error message should say that {\tt META} does not exist.

\gap
The {\tt subst} command has problems distinguishing
proof line labels from numbers in equalities.

\gap
The {\tt label} command does not check for overlaps between
labels for proof lines, axioms and theories.



	\newpage
	\newcommand{\syndes}[2]{\item[\parbox{\textwidth}{#1}]\hfill #2}
\newcommand{\module}[1]{\subsection{#1}}

\section{Syntax of commands}

\module{admin}

\begin{description}
\syndes{
   comment \ARG{separator} \OPT{\ARG{text}} \ARG{separator}
}
{}
%3rd-begin%3rd-end

\syndes{
   deflam \ARG{funname} \ARG{var-list} \ARG{form};
}
{}
%3rd-begin%3rd-end

\syndes{
   deflam \ARG{funname} \ARG{var-list} \ARG{form};
}
{}
%3rd-begin%3rd-end

\syndes{
   echo \ARG{separator} \OPT{\ARG{text}} \ARG{separator}
}
{}
%3rd-begin%3rd-end

\syndes{
   hgk \ARG{s-expr};
}
{}
%3rd-begin%3rd-end

\syndes{
   hgk \ARG{s-expr};
}
{}
%3rd-begin%3rd-end

\syndes{
   know natnums \OPT{\ARG{natnum1}, \SEQ \ARG{natnumN}};
}
{}
%3rd-begin%3rd-end

\syndes{
   load \ARG{file};
}
{}
%3rd-begin%3rd-end

\syndes{
   resetprompt;
}
{}
%3rd-begin%3rd-end

\syndes{
   setprompt to \ARG{s-expr};
}
{}
%3rd-begin%3rd-end

\syndes{
   setprompt to \ARG{s-expr};
}
{}
%3rd-begin%3rd-end

\syndes{
   show \ARG{option};
}
{}
%3rd-begin%3rd-end

\end{description}

\module{context}

\begin{description}
\syndes{
  copycontext \ARG{ctx-name};
}
{}
%3rd-begin%3rd-end

\syndes{
	copylex	\ARG{ctx-name};
}
{}
%3rd-begin%3rd-end

\syndes{
  makecontext \ARG{ctx-name};
}
{}
%3rd-begin%3rd-end

\syndes{
  namecontext \ARG{ctx-name};
}
{}
%3rd-begin%3rd-end

\syndes{
  reset;
}
{}
%3rd-begin%3rd-end

\syndes{
  switchcontext \ARG{ctx-name};
}
{}
%3rd-begin%3rd-end

\end{description}

\module{decide}

\begin{description}
\syndes{
  decide \ARG{wff}  \OPT{by \ARG{fact1} \ARG{fact2} \SEQ} using 
  \OPT{\{ \ARG{rewriter} \SEQ \}} \ARG{decider};
}
{}
%3rd-begin%3rd-end

\syndes{
  monad \ARG{wff} \OPT{by \ARG{fact1} \ARG{fact2} \SEQ};
}
{}
%3rd-begin%3rd-end

\syndes{
  monadeq \ARG{wff} \OPT{by \ARG{fact1} \ARG{fact2} \SEQ};
}
{}
%3rd-begin%3rd-end

\syndes{
  ptaut \ARG{wff} \OPT{by \ARG{fact1} \ARG{fact2} \SEQ};
}
{}
%3rd-begin%3rd-end

\syndes{
  taut \ARG{wff} \OPT{by \ARG{fact1} \ARG{fact2} \SEQ};
}
{}
%3rd-begin%3rd-end

\syndes{
  tauteq \ARG{wff} \OPT{by \ARG{fact1} \ARG{fact2} \SEQ};
}
{}
%3rd-begin%3rd-end

\end{description}

\module{eval}

\begin{description}
\syndes{
  assertsimp \ARG{simplabel}; 
}
{}
%3rd-begin%3rd-end

\syndes{
  attach \ARG{indconst}  to \ALT dar [ rep ] \ARG{sexpr};\\
  attach \ARG{sentconst} to T \ALT NIL \ALT UNDEF;\\
  attach \ARG{funconst} \ALT \ARG{predconst} to \ARG{atom};\\
  attach \ARG{funconst}  to [ \ARG{rep1}, \SEQ, \ARG{repN} = \ARG{repM} ]
  \ARG{atom};\\
  attach \ARG{predconst} to [ \ARG{rep1}, \SEQ, \ARG{repN} ] \ARG{atom};
}
{}
%3rd-begin%3rd-end

\syndes{
  decrep \ARG{replabel1} \OPT{\SEQ \ARG{replabelN}};
}
{}
%3rd-begin%3rd-end

\syndes{
  eval \ARG{wff} \ALT \ARG{fact} \ALT \ARG{term} \OPT{by \ARG{simpexpr}};
}
{}
%3rd-begin%3rd-end

\syndes{
  hardware \ARG{indconst} to \ALT dar \ARG{sexpr};\\
  hardware \ARG{indconst} to \ALT dar [ \ARG{rep} ] \ARG{sexpr};
}
{}
%3rd-begin%3rd-end

\syndes{
  let \ARG{\indconst} to \ALT dar [ \ARG{rep} ] \ARG{term};
}
{}
%3rd-begin%3rd-end

\syndes{
  represent \{ \ARG{sort1}, \SEQ, \ARG{sortN} \} as \ARG{rep} \ALT \ARG{*};
}
{}
%3rd-begin%3rd-end

\syndes{
  rewrite \ARG{wff} \ALT \ARG{fact} \ALT \ARG{term} \OPT{by \ARG{simpexpr}};
}
{}
%3rd-begin%3rd-end

\syndes{
  setbasicsimp \ARG{simplabel} at wffs \{ \ARG{wff1} \SEQ \ARG{wffN} \};\\
  setbasicsimp \ARG{simplabel} at facts \{ \ARG{fact1} \SEQ \ARG{factN} \};
}
{}
%3rd-begin%3rd-end

\syndes{
  simplify \ARG{wff} \ALT \ARG{fact} \ALT \ARG{term};
}
{}
%3rd-begin%3rd-end

\end{description}

\module{language}

\begin{description}
\syndes{
   awff \ARG{awff};
}
{}
%3rd-begin%3rd-end

\syndes{
   declare \OPT{\ARG{sentsym} \ALT \ARG{indsym}} \ARG{sym1} \SEQ \ARG{symN}; \\
   declare \OPT{\ARG{funsym}  \ALT \ARG{predsym}} \ARG{sym1} \SEQ \ARG{symN}
   \ARG{arity};\\
   declare \OPT{\ARG{funsym} \ALT \ARG{predsym}} \ARG{sym1} \SEQ \ARG{symN}
   1 \OPT{[ pre \OPT{= \ARG{prbp}} ]};\\
   declare \OPT{\ARG{funsym} \ALT \ARG{predsym}} \ARG{sym1} \SEQ \ARG{symN}
   2 \OPT{[ inf \OPT{= \ARG{lbp} \ARG{rbp}} ] };
}
{}
%3rd-begin%3rd-end

\syndes{
  declare sort \ARG{sym1} \SEQ \ARG{symN};
}
{}
%3rd-begin%3rd-end

\syndes{
  declare indconst \ALT indpar \ALT indvar \ARG{sym1} \SEQ \ARG{symN}
  [ \ARG{sortsym} ]; \\
  declare funconst \ALT funpar \ARG{sym1} \SEQ \ARG{symN}
  ( \ARG{sortsym1} \SEQ \ARG{sortsymN} ) = \ARG{sortsym};\\
}
{}
%3rd-begin%3rd-end

\syndes{
  extension \ARG{sort} by \ARG{extexpr};
}
{}
%3rd-begin%3rd-end

\syndes{
  moregeneral \ARG{sort1} < \ARG{sort2}, \SEQ, \ARG{sortN} >;
}
{}
%3rd-begin%3rd-end

\syndes{
  mostgeneral \ARG{sym};
}
{}
%3rd-begin%3rd-end

\syndes{
  setfmap \ARG{funsym} ( \ARG{sym1} \SEQ \ARG{symN} ) = \ARG{sym};
}
{}
%3rd-begin%3rd-end

\syndes{
   term \ARG{term};
}
{}
%3rd-begin%3rd-end

\syndes{
   wff \ARG{wff};
}
{}
%3rd-begin%3rd-end

\end{description}

\module{meta}

\begin{description}
\syndes{
   mattach \ARG{indconst} to \ALT dar \OPT{[rep]}
   \ARG{cname}:\ARG{pname}:\ARG{sort}:\ARG{object};
}
{}
%3rd-begin%3rd-end

\syndes{
  reflect \ARG{M-fact} \ARG{arg1} \ARG{arg2} \SEQ \ARG{argN};
}
{}
%3rd-begin%3rd-end

\end{description}

\module{nd}

\begin{description}
\syndes{
   alle \ALT us \ARG{fact} \OPT{,} \ARG{term1} \ARG{term2} \SEQ;
}
{}
%3rd-begin%3rd-end

\syndes{
   alli \ALT ug \ARG{fact} \OPT{\OPT{,} \ARG{indvar1} \ALT \ARG{indpar1} :} 
                           \ARG{indvar11}
                           \OPT{\OPT{,} \ARG{indvar2} \ALT \ARG{indpar2} :}
                           \ARG{indvar22} \SEQ;
}
{}
%3rd-begin%3rd-end

\syndes{
   ande \ALT ae \ARG{fact} \OPT{,} 1 \ALT 2; \\
   ande \ALT ae \ARG{fact} \OPT{,} 1 \ALT 2 1 \ALT 2 \SEQ;
}
{}
%3rd-begin%3rd-end

\syndes{
   andi \ALT ai \ARG{fact1} \OPT{,} \ARG{fact2}; \\
   andi \ALT ai \ARG{fact11}
                \OPT{conj \ALT cj \ARG{fact12} conj \ALT cj \ARG{fact13} \SEQ}
                \OPT{,}
                \ARG{fact21}
                \OPT{conj \ALT cj \ARG{fact22} conj \ALT cj \ARG{fact23} \SEQ};
}
{}
%3rd-begin%3rd-end

\syndes{
   assume \ARG{wff1} \OPT{\OPT{,} \ARG{wff2} \SEQ};
}
{}
%3rd-begin%3rd-end

\syndes{
   existe \ALT es \ARG{fact}  \OPT{,} \ARG{indvar1} \ALT \ARG{indpar1}
                  \OPT{,} \ARG{indvar2} \ALT \ARG{indpar2} \SEQ;
}
{}
%3rd-begin%3rd-end

\syndes{
  existi \ARG{fact}
   \OPT{\OPT{,} \ARG{term1} :} \ARG{indvar1} \OPT{occ \ARG{n11} \ARG{n12} \SEQ}
   \OPT{\OPT{,} \ARG{term2} :} \ARG{indvar2} \OPT{occ \ARG{n21} \ARG{n22} \SEQ}
   \SEQ;
}
{}
%3rd-begin%3rd-end

\syndes{
   falsee \ALT fe \ARG{fact1} \OPT{,} \ARG{wff};\\
   falsee \ALT fe \ARG{fact1} \OPT{,} \ARG{fact2};
}
{}
%3rd-begin%3rd-end

\syndes{
   falsei \ALT fi \ARG{fact1} \OPT{,} \ARG{fact2}; 
}
{}
%3rd-begin%3rd-end

\syndes{
   iffe \ALT ie \ARG{fact} \OPT{,} 1 \ALT 2;
}
{}
%3rd-begin%3rd-end

\syndes{
   iffi \ALT ii \ARG{fact1} \OPT{,} \ARG{fact2};
}
{}
%3rd-begin%3rd-end

\syndes{
   impe \ALT mp \ARG{fact1} \OPT{,} \ARG{fact2};
}
{}
%3rd-begin%3rd-end

\syndes{
   impi \ALT ded  \ARG{fact1} \OPT{, \ALT imp} \ARG{fact};\\
   impi \ALT ded  \ARG{wff} \OPT{, \ALT imp} \ARG{fact};
}
{}
%3rd-begin%3rd-end

\syndes{
   note \ALT ne \ARG{fact1} \OPT{,} \ARG{wff}; \\
   note \ALT ne \ARG{fact1} \OPT{,} \ARG{fact2};
}
{}
%3rd-begin%3rd-end

\syndes{
   noti \ALT ni \ARG{fact1} \OPT{,} \ARG{wff}; \\
   noti \ALT ni \ARG{fact1} \OPT{,} \ARG{fact2};
}
{}
%3rd-begin%3rd-end

\syndes{
   ore \ALT oe \ARG{fact1} \OPT{,} \ARG{fact2} \OPT{,} \ARG{fact2};
}
{}
%3rd-begin%3rd-end

\syndes{
   ori \ALT oi \ARG{fact} \OPT{,} \ARG{wff} \OPT{,} \OPT{lr \ALT rl};\\
   ori \ALT oi \ARG{fact} \OPT{,} \ARG{fact1} \ALT \ARG{wff1} 
       disj \ALT dj \ARG{fact2} \ALT \ARG{wff2} disj \ALT dj \SEQ
       \OPT{,} \OPT{lr \ALT rl};
}
{}
%3rd-begin%3rd-end

\syndes{
  subst \ARG{fact1} \OPT{with} \ARG{fact2};\\
  subst \ARG{fact1} \OPT{with} \ARG{fact2} \OPT{right \ALT left};\\
  subst \ARG{fact1} \OPT{with} \ARG{fact2} 
                    \OPT{occ \ARG{n1} \ARG{n2} \SEQ} \OPT{right \ALT left};
}
{}
%3rd-begin%3rd-end

\end{description}

\module{parser}

\begin{description}
\syndes{
  backup \ARG{file} open;\\
  backup \ARG{file} close;  
}
{}
%3rd-begin%3rd-end

\syndes{done;}
{}
%3rd-begin%3rd-end

\syndes{
   fetch \ARG{file} \OPT{from \ARG{mark1}} \OPT{to \ARG{mark2}};
}
{}
%3rd-begin%3rd-end

\syndes{
   mark \ARG{sym};
}
{}
%3rd-begin%3rd-end

\syndes{
   probe;\\
   probe \ARG{activity};\\
   probe all;\\
}
{}
%3rd-begin%3rd-end

\syndes{
   unprobe \ARG{activity};\\
   unprobe all;
}
{}
%3rd-begin%3rd-end

\end{description}

\module{proof}

\begin{description}
\syndes{
  axiom \ARG{sym} : \ARG{wff};
}
{}
%3rd-begin%3rd-end

\syndes{
   cancel \OPT{\ARG{label}};
}
{}
%3rd-begin%3rd-end

\syndes{
  copyproof \ARG{prf-name};
}
{}
%3rd-begin%3rd-end

\syndes{
   label fact \ARG{sym};\\
   label fact \ARG{sym} = \ARG{label};
}
{}
%3rd-begin%3rd-end

\syndes{
   makeproof \ARG{prf-name};
}
{}
%3rd-begin%3rd-end

\syndes{
   nameproof \ARG{prf-name};
}
{}
%3rd-begin%3rd-end

\syndes{
  switchproof \ARG{prf-name};
}
{}
%3rd-begin%3rd-end

\syndes{
  theorem \ARG{sym} \ARG{hook};
}
{}
%3rd-begin%3rd-end

\syndes{
  theory \ARG{thlabel} : \ARG{wff1} \OPT{\ARG{wff2} \SEQ};\\
  theory \ARG{thlabel} : \ARG{axlabel1} : \ARG{wff}
                         \OPT{\ARG{axlabel2} : \ARG{wff} \SEQ};
}
{}
%3rd-begin%3rd-end

\end{description}

\module{rules}

\begin{description}
\syndes{
	contract \ALT ctc  \ARG{fact} by \ARG{assumption1} \SEQ  \ARG{assumptionN}; 
}
{}
%3rd-begin%3rd-end

\syndes{
	cut \ARG{fact1} \ARG{fact2};\\
	cut \ARG{fact1} \ARG{fact2} \OPT{keep \ARG{assumption1} \SEQ
	\ARG{assumptionN}};
}
{}
%3rd-begin%3rd-end

\syndes{
	termife \ARG{fact1} \ARG{fact2} \ARG{termif}; \\
	termife \ARG{fact1} \ARG{fact2} \ARG{termif} \OPT{occ \ARG{n1} \ARG{n2}
	\SEQ}; 
}
{}
%3rd-begin%3rd-end

\syndes{
	termifen \ARG{fact1} \ARG{fact2} \ARG{termif}; \\
	termifen \ARG{fact1} \ARG{fact2} \ARG{termif} \OPT{occ \ARG{n1} \ARG{n2}
	\SEQ}; 
}
{}
%3rd-begin%3rd-end

\syndes{
	termifi \ARG{fact1} \ARG{fact2} \ARG{wff} \ARG{term1} \ARG{term2};
}
{}
%3rd-begin%3rd-end

\syndes{
	weaken \ALT wk \ARG{fact} by \ARG{fact1} \SEQ \ARG{factN};
}
{}
%3rd-begin%3rd-end

\syndes{
	wffife \ARG{fact1} \ARG{fact2};
}
{}
%3rd-begin%3rd-end

\syndes{
	wffifen \ARG{fact1} \ARG{fact2};
}
{}
%3rd-begin%3rd-end

\syndes{
	wffifi \ARG{wff} \ARG{fact1} \ARG{fact2};
}
{}
%3rd-begin%3rd-end

\end{description}



	% ............................. BIBLIOGRAPHY ...........................
	\newpage
	\bibliographystyle{alpha}
    \gfbibliography
	
	% ................................ INDEX ...............................
	\newpage
	%............................... USER MANUAL .................................
%.............................................................................

\documentstyle[12pt]{../styfiles/GFmanual}

\title{GETFOL Manual}
\author{\bf Fausto Giunchiglia}
\date{7 March 1994}
\version{2.0}
\abstract{
	  {\GF} is an interactive reasoning system.
	  We use it as an environment for studying epistemological issues.
	  We try to look at questions like: which notions are
	  important for the development of mechanized reasoning systems?
	  What kind of conversations do we want to have with them?
	  What parts of logic should we use to represent such notions?
	  How should logic be embedded in a conversational reasoning system?
	}
\addresses{
     \begin{tabular}[c]{l}
       {\bf Fausto Giunchiglia}           \\
       Mechanized Reasoning Group		  \\
		 IRST, Povo, 38050 Trento, Italy  \\
		 e-mail: {\tt fausto@irst.it}     \\
		 phone: +39 461 314436
	\end{tabular}	
}
\published{
  \begin{tabular}{l}
	  DIST Technical Report No. 92-0010 (1994). \\
	  DIST -- University of Genoa,\\
	  Via Opera Pia 11A, 16145 Genova, Italy.\\ \\
  \end{tabular}
}

%% \newcommand{\gfbibliography}{%
%% \bibliography{/home/tarski/staff/mrg/biblio/bib/a-l,%
%% /home/tarski/staff/mrg/biblio/bib/m-z,userman}}
\newcommand{\gfbibliography}{\bibliography{}}

%% \makeindex
\begin{document}
	%  ............................. COVER ..................................
	\thispagestyle{empty}
	\maketitle

	%  ........................ TABLE OF CONTENTS ...........................
	\newpage
	\pagenumbering{roman}
	\tableofcontents

	\newpage
	\pagenumbering{arabic}
	\pagestyle{headings}

	%  ........................... INTRODUCTION ............................
	\newcommand{\eg}{{\em e.g.~}}
\newcommand{\ie}{{\em i.e.~}}
\newcommand{\wrt}{w.r.t.~}
\newcommand{\co}[2]{\langle #1, \: #2 \rangle}


\section{Decision procedures}
\label{sec-decide}
\label{system:sec}
A detailed description of the main decision procedures of {\tt GETFOL}
is given in \cite{armando5}.

The set of procedures of the {\tt GETFOL} system is depicted in figure
\ref{system:fig}.
Each box represents either a decider ({\tt PTAUT}, {\tt PTAUTEQ},
{\tt FOLTAUT}) or a rewriting procedure ({\tt tautren}, {\tt  phexp},
{\tt  reduce}).

\begin{figure}
\begin{center}
\makebox[3.375in][l]{
  \vbox to 2.750in{
    \vfill
    \special{psfile=decide/NEWFIG.PS}
  }
  \vspace{-\baselineskip}
}
\end{center}
\caption{The system of deciders}
\label{system:fig}
\end{figure}

\subsubsection*{{\tt PTAUT} and {\tt PTAUTEQ}}
{\tt PTAUT} and {\tt PTAUTEQ}
are deciders working on a quantifier-free first order language (hereon by
abuse of language we call them propositional deciders).
{\tt PTAUT} decides the set of first order formulas provable using
only the propositional deductive machinery (moreover it returns a
falsifying assignment whenever the input formula is not a tautology).
For instance, the formula $(P(x)\con R(x,b))\imp (P(x)\dis R(x,b))$ can be
easily inferred by a single application of {\tt PTAUT}.
{\tt PTAUT} is a generalization of the Davis-Putnam-Loveland procedure
(hereon DPL) \cite{davis2,davis6} to non clausal formulas.
The core of {\tt PTAUT} is a procedure capable of partially evaluating
the input formula \wrt a partial assignment of truth-values to the atomic
subformulas. 
A step of statistical analysis (of polynomial time complexity) collects
information about the {\em polarity} of the subformulas and the existence
of {\em Top-Level Disjunctive Occurrences} (TLDO) of atomic subformulas.
A formula $\alpha$ occurs as a TLDO in $\beta$ 
if and only if $\beta$ can be rewritten into a formula either of the form
$(\alpha\dis\gamma)$ or $(\neg\alpha\dis\gamma)$ by means of rules
expressing  the usual properties of the propositional connectives such as
the associativity, commutativity and distributivity of the propositional
connectives.
The notion of positive (negative) subformula occurrence
is inductively defined as follows: $\alpha$ occurs positively in $\alpha$, 
$\alpha$ occurs negatively in $\neg\alpha$;
$\alpha$ and $\beta$ occur positively in $(\alpha\con\beta)$ and
$(\alpha\dis\beta)$;
$\alpha$ occurs negatively and $\beta$ occurs positively in $(\alpha\imp\beta)$;
finally $\alpha$ and $\beta$ occur both positively and negatively in
$(\alpha\liff\beta)$.
A subformula $\alpha$ is positive (negative) in $\beta$ if and only if
each occurrence of $\alpha$ occurs positively (negatively) in $\beta$.

The statistical analysis may suggest a partial assignment $\mu$ (\wrt which
the formula can be simplified) according to the following criteria:
\begin{itemize}
\item for each positive (negative) atomic formula $\alpha$ occurring
in $\beta$, $\mu(\alpha)=F$ ($\mu(\alpha)=T$);
\item if $\beta$ contains a positive (negative) TLDO of $\alpha$ and there
are no negative (positive) TLDO of $\alpha$, then $\mu(\alpha)=F$
($\mu(\alpha)=T$).
\end{itemize}

If $\mu$ is not completely undefined, then {\tt PTAUT} simplifies the
formula in input \wrt $\mu$ and recurs on the resulting (simplified) formula.
If the input formula contains both a positive and a negative TLDO of an
atomic formula the input formula is a tautology.
These optimizations generalize the {\em Affirmative-Negative Rule} and the
{\em Rule for the Elimination of One-Literal Clauses} of DPL.
If $\mu$ is totally undefined, then
an atomic formula is chosen, two partial assignments are created
(one assigning $T$, the other $F$ to the chosen atomic formula),
the formula is partially evaluated \wrt such assignments and finally
the procedure branches recurring on the two simplified formulas.
This last step generalizes the {\em Splitting Rule} of DPL.

{\tt PTAUTEQ} is the result of adapting {\tt PTAUT}
to take into account the properties of equality.
The main difference is that, before a formula is simplified \wrt some
assignment, the assignment is tested to check whether it is model of the
quantifier-free theory of equality.
The formula $(P(x)\con x=y)\imp (P(y)\con y=x)$ can be
easily inferred by a single application of {\tt PTAUTEQ}.

\subsubsection*{{\tt nnf} and {\tt skolemize}}
{\tt nnf} rewrites the input formula into a logically equivalent one in
{\em negative normal form}.

{\tt skolemize} computes the skolemization of the input formula.

\subsubsection*{{\tt tautren} and {\tt phexp}}
The procedures on top of the propositional deciders (namely {\tt tautren}
and {\tt phexp}) map the first-order formula in input into a quantifier-free
formula.
The mappings are such that the decision problem of the input (first-order)
formula is related to the decision problem of the returned (quantifier-free)
formula in a useful way.
In particular, {\tt tautren} atomizes equal (modulo renaming of bound
variables) quantified subformulas into newly introduced propositional
letters.
For instance the formula
\begin{equation}\label{pb29-reduced}
%\setlength{\templength}{\arraycolsep}
\setlength{\arraycolsep}{0cm}
\begin{array}{rl}
(\exists x.F(x) \con \exists x.G(x)) \imp (&( \forall x.(F(x) \imp H(x)) \con \forall x.(G(x) \imp J(x))) \liff \\
& ((\exists y.G(y) \imp \forall x.(F(x) \imp H(x))) \con\\
&\ (\exists x.F(x) \imp \forall y.(G(y) \imp J(y)))))
\end{array}
\end{equation}
is mapped into the propositional formula
\begin{equation}\label{pb29-prop}
(A \con B) \imp ((C \con D) \liff ((B \imp C) \con (A \imp D)))
\end{equation}
The relation between the decision problems of the input formula (say $\alpha$)
and of the output formula (say $\alpha'$) is that
$\der{}\alpha'$ only if $\der{}\alpha$.

A more careful reduction to the quantifier-free fragment is performed by
{\tt phexp}.
{\tt phexp} maps an existential formula $\alpha$ into a quantifier-free formula
$\alpha'$ such that $\der{}{\alpha'}$ if and only if $\der{}{\alpha}$.%
\footnote{The set of existential formualas is the class of prenex
universal-existential formulas without function symbols.}

The formula $\alpha'$ is an improved version of the Herbrand's expansion
of $\alpha$ \cite{dreben1}.
An application of {\tt phexp} to the following formula:
\begin{equation}\label{pb28}\small
    (((P(x) \con \neg Q(y)) \dis
     ((Q(a) \dis R(a)) \con (\neg Q(b) \dis \neg S(b)))) \dis
     ((F(z) \con \neg G(z)) \con S(v))) \dis
       ((\neg P(c) \dis \neg F(c)) \dis G(c))
\end{equation}
yields
\begin{equation}\label{pb28-exp}\small
\begin{array}{l}
((((P(a) \dis P(b) \dis P(c)) \con (\neg Q(a) \dis \neg Q(b) \dis
\neg Q(c)))\dis\\
((Q(a) \dis R(a)) \con (\neg Q(b) \dis \neg S(b)))) \dis \\
(((F(a) \con \neg G(a)) \dis (F(b) \con \neg G(b)) \dis
(F(c) \con \neg G(c)))\con\\
(S(a) \dis S(b) \dis S(c)))) \dis ((\neg P(c) \dis \neg F(c)) \dis G(c))\\
\end{array}
\end{equation}
In \cite{armando5} it is shown that, the size of (\ref{pb28-exp})
is 44 times smaller than the size of the standard Herbrand's expansion of
(\ref{pb28}).

\subsubsection*{{\tt reduce}}
{\tt reduce} tries a set of rewriting rules on the input
formula aiming at rewriting it into a logically equivalent formula that
can be easily turned into an existential one via skolemization.
The rewriting rules employed by {\tt reduce} are the usual rules
expressing the distributivity of quantifiers through propositional connectives
and the commutativity and associativity of propositional connectives
listed in the following table.\\

    \renewcommand{\arraystretch}{1.5}
    {\small
      $$
      \begin{array}{|c|rcl|} \hline
        (1) & Q x. \alpha[x] & \mapsto & \alpha \\ \hline
        %(2) & Q x. (\neg \alpha(x)) & \mapsto & (\neg \hat{Q} x. \alpha(x)) \\ \hline
        (2) & Q x. (\alpha \circ \beta)(x) & \mapsto & (Q x. \alpha \circ Q x. \beta) 
        \\ \hline
        (3) & Q x. (\alpha[x] + \beta(x)) & \mapsto & (\alpha[x] + Q x. \beta(x)) 
        \\ \hline \hline
        (4) & (\alpha(x) + \beta[x]) & \mapsto & (\beta[x] + \alpha(x)) \\ \hline
        (5) & ((\alpha[x] + \beta(x)) + \gamma(x)) & \mapsto & 
        (\alpha[x] + (\beta(x) + \gamma(x))) \\ \hline
        (6) & ((\alpha \circ \beta)(x) + \gamma(x)) & \mapsto & 
        ((\alpha + \gamma(x)) \circ (\beta + \gamma(x))) \\ \hline
        (7) & (\alpha(x) + (\beta[x] + \gamma(x))) & \mapsto &
        (\beta[x] + (\alpha(x) + \gamma(x))) \\ \hline
        (8) & ((\alpha(x) + (\beta \circ \gamma)(x))) & \mapsto &
        ((\alpha(x) + \beta) \circ (\alpha(x) + \gamma)) \\ \hline
      \end{array}
      $$
      }
    \renewcommand{\arraystretch}{1}
{\small
{\em Restrictions}: 
\begin{itemize}
\item In rules $\{(4)-(8)\}$ the left hand side must be a top normalizable
formula.
\item In rules $\{(7),(8)\}$ $\alpha$ must be minimal \wrt $\co{Q}{x}$.
%\item Rules $\{(4)-(8)\}$ can be applied only to subformulae (say $\alpha$)
%of a formula $Qx.\beta$ in which there is no proper
%subformula $Qy.\gamma$ of which $\alpha$ is a subformula.
\end{itemize}}

Where
$\alpha(x)$ denotes a formula in which there is at least one free occurrence
of the variable $x$.
$\alpha{[x]}$ denotes a formula in which there is no free occurrences of $x$.
$Q$ and $Q'$ stand either for $\forall$ or for $\exists$.
If $Q = \forall$, then $\circ = \con$ and $+ = \dis$.
If $Q = \exists$, then $\circ = \dis$ and $+ = \con$.
%$\con$ is said to be $\forall$-compatible and $\exists$-incompatible,
%$\dis$ is said to be $\exists$-compatible and $\forall$-incompatible.
%If $\cal S$ is a set of rewriting rules then $\mapsto_{\cal S}$ is the
%reducibility relation induced by $\cal S$ and
%$\stackrel{*}{\mapsto}_{\cal S}$ is the reflexive and transitive closure
%of $\mapsto_{\cal S}$.
The definition of {\em top normalizable formula} and of {\em minimal
formula} are given in \cite{armando5}.

For instance, a single application of {\tt reduce} turns the formula
\begin{equation}\label{pb29}
%\setlength{\templength}{\arraycolsep}
\setlength{\arraycolsep}{0cm}
\begin{array}{rl}
(\exists x.F(x) \con \exists x.G(x)) \imp (&( \forall x.(F(x) \imp H(x)) \con \forall x.(G(x) \imp J(x))) \liff \\
&(\forall x.\forall y.((F(x) \con G(y)) \imp (H(x) \con J(y)))))
\end{array}
\end{equation}
into (\ref{pb29-reduced}).
{\tt reduce} considerably enlarges the set of formulas which can be solved
by using the system of deciders.
In particular, any prenex first order formula 
$$
\forall \vec{y}_n \exists \vec{x}_n \ldots
\forall \vec{y}_i \exists \vec{x}_i \ldots
\forall \vec{y}_1 \exists \vec{x}_1 . \Phi
$$
such that each literal in $\Phi$ contains no variables in $\vec{y}_k$ and
in $\vec{x}_l$ with $k < l$, or in $\vec{x}_k$ and in $\vec{x}_l$ with
$k \neq l$ can be ``reduced" to an existential formula.
On the basis of the previous result it is a trivial consequence
that the {\em monadic calculus} can be reduced to the existential fragment
by means of {\tt reduce}.


	%  .............................. MODULES ..............................
	% loading introduction to the section
\section{Parser}

This section is intended to explain:
%
\begin{itemize}
	\item
		the main functionalities of the {\GF} scanning primitives,
	\item
		what modifications must be performed in the scanner data
		structures in order to be able to change its behavior ({\em e.g.} to
		define a new escape character).
\end{itemize}

The {\GF} scanner is a {\em backupable} scanner.
This feature is necessary as the parser works in a top-down fashion and,
sometimes, needs to backtrack. 

The {\GF} scanner is able to recognize three types of tokens:
%
\begin{enumerate}
	\item {\tt IDTOKEN},
	\item {\tt NUMTOKEN} and
	\item {\tt DELTOKEN}.
\end{enumerate}

The primitives to scan such tokens are \verb+FOLSYM@+, \verb+NATNUM@+ (actually
no dedicated primitive to scan a \verb+DELTOKEN+ exists).

\verb+TK@+ scans a generic token.

The primitives \verb+SCANSTATUS-GET+ and \verb+SCANSTATUS-RESTORE+ allow
respectively to save and restore the scanner status.
They are necessary to restore the status of the scanner in case of
unsuccessful parsing.

The high level functions \verb+FOLSYM@+, \verb+NATNUM@+, \verb+TK@+ (and other
not mentioned here for brevity) use the general \verb+TOKEN-GET-NEXT+ routine
which is also able to {\em bufferize} the tokens already read (reading from
an input stream is a destructive operation so we need at a some level a
buffer mechanism if we want to perform backtracking).

The buffer is implemented by the array \verb+TOKENARRAY+ whose dimension is
given by the macro \verb+TOKENARRAY-DIMENSION+. The max number of tokens
that can be present in a command line is given by the number returned
by this macro.

\verb+TOKEN-GET-NEXT+ routine is based on the lower level primitive
\verb+TOKEN-SCAN+, which reads the next token from the input stream and returns
its type.
\verb+TOKEN-SCAN+ is able to identify the type of a token reading (via 
\verb+CH-GET-NEXT+) its first character and identifying its type (via
\verb+CHTYPE-GET+).

The characters are divided in the following types:

\begin{enumerate}
	\item
		identifiers ({\tt IDCHAR}): see file
		\verb+ascitab.fol+;
	\item
		numbers (positive integers --- {\tt NUMCHAR}):
		\verb+0 1 2 3 4 5 6 7 8 9+;
	\item
		delimiter {\tt DELCHAR}:
		\verb+( ) , . : ; [ ] { }+;
	\item
		ignored char {\tt IGNCHAR}: see file
		\verb+ascitab.fol+;
	\item
		iddelim {\tt IDDELCHAR}:
        \verb$' * + - / < > = ? @ ^ ` |$;
	\item
		escape characters {\tt ESCCHAR}:
		\verb+\+;
	\item
		special handling {\tt SPECCHAR}
\end{enumerate}

The \verb+IDDELCHAR+s are identifier characters having also the functionality
of a delimiter, that is no \verb+IDTOKEN+ can contain a character of such
type but they by themselves may be considered \verb+IDTOKEN+.
For instance the string \verb+A@B+ will be regarded by the scanner a string
formed by the three distinct \verb+IDTOKEN+ \verb+A @ B+.
The escape characters allows to coerce the type of the following
character to be \verb+IDCHAR+.
For istance the string \verb+A\@B+ will be regarded by the scanner a string
formed by the \verb+IDTOKEN A@B+.

The only think important to know at the user level it is how to modify the type
of a character so that you can, for istance, extend the set of \verb+IDDELCHAR+
with the character "\verb+_+". To do this you have only to change type
declaration for the "\verb+_+" character from \verb+IDCHAR+ to \verb+IDDELCHAR+
in the file {\tt asciitab.fol}.

Finally a low level feature: each time a command line is issued to {\GF} the low
level scanning primitives store each read character in the {\tt TUPLE
SCANBUFARRAY}, so that if a syntactic error in the parsing is detected the rest
of the line will be read and the whole line printed out to show, using an
"\verb+^+", where the syntatic error has been detected. This useful feature
imposes (as it is implemented) a limit in the length of a {\GF} command line as
the dimension of \verb+SCANBUFARRAY+ is fixed by the value returned by the macro
\verb+SCANBUFARRAY-DIMENSION+.


% loading command files
\gfcommand{backup}{backup of a {\GF} session}
\index{backup}

\gfsyntax{
  backup \ARG{file} open;\\
  backup \ARG{file} close;  
}

\gfdescription{
   {\GF} stores in \ARG{file} all the successful commands between the command
   ``{\tt backup \ARG{file} open}" and the command ``{\tt backup \ARG{file}
   close}". 
}

\gfrecap{
Stores in a file all successfull commands type in GETFOL.
}

\gfexample+
   ***** backup file open;
   I am starting to backup onto file
   
   ***** declare sentconst A;
   ***** assume A;
   1   A     (1)
   
   ***** backup file close;
   
   ***** andi 1 1;
   2   A and A     (1)
   
   ***** ^D
   
   >Bye.
   
   <host-prompt> more file
   
   declare sentconst A;
   
   assume A;
   
   <host-prompt>
+

\gfnotes{
   Only Unix file names are supported.
   The file path name can be absolute or relative to the current directory.
   Multiple open backup files can exist simultaneously.
   Files must be explicitly closed to guarantee that all the commands
   are properly stored in the backup file.
}

\gfcommand{done}{exiting a {\GF} session}
\index{done}

\gfsyntax{done;}

\gfdescription{
   Returns the control back to the {\HG} environment~\cite{giunchiglia35}.
   You can get back to {\GF} by typing {\tt (SYSBACK)} at the {\HG} prompt.
}

\gfrecap{
Returns the control back to the HGKM environment.
You can get back to GETFOL by typing (SYSBACK) at the HGKM prompt.
}

\gfexample+
   **** done;
   Returning to host
   NIL
+

\gfnotes{}

\gfcommand{fetch}{fetches {\GF} file}
\index{fetch}

\gfsyntax{
   fetch \ARG{file} \OPT{from \ARG{mark1}} \OPT{to \ARG{mark2}};
}

\gfdescription{
   Redirects the standard input to the file \ARG{file}; all {\GF} commands
   between \ARG{mark1} and \ARG{mark2} are executed.
}

\gfrecap{
Fetches commands from the file `file'.
All commands between `mark1' and `mark2' are executed.
}

\gfexample+
  ***** fetch exmarks.tst;
  ...
  ***** fetch exmarks.tst to m1;
  ...
  ***** fetch exmarks.tst from m1 to m2;
  ...
  ***** fetch exmarks.tst from m2;
+

\gfnotes{
   Only Unix file names are supported.
   The file path name can be absolute or relative to the current directory.
   Nested fetches (and marking) are allowed.
   If no marks are specified, then the whole file is fetched.
   See the command {\tt mark} in this section to set marks in a file.
}


\gfcommand{mark}{sets a mark}
\index{mark}

\gfsyntax{
   mark \ARG{sym};
}

\gfdescription{
   Sets a mark in between a sequence of {\GF} commands.
   It can be used to fetch a file from/to a certain mark (see command 
   {\tt fetch} in this section).
}

\gfrecap{
Sets a mark in a file (see fetch).
}

\gfexample+
   <host-prompt> more example
   NAMECONTEXT META;
   DECLARE SORT FACT WFF;
   DECLARE INDVAR fc [FACT];
   DECLARE FUNCONST wffof (FACT) = WFF;
   DECLARE PREDCONST THEOREM 1;
   comment | here we put the first mark m1 |
   mark m1;
   DECREP FACT;
   DECREP WFF;
   REPRESENT \{WFF\} AS WFF;
   REPRESENT \{FACT\} AS FACT;
   comment | here we put the second mark m2 |
   mark m2;
   ATTACH wffof TO [FACT = WFF] fact\-get\-wff;
   AXIOM M2: forall fc. THEOREM(wffof(fc));
   MAKECONTEXT OBJ;
   SWITCHCONTEXT OBJ;
   DECLARE SENTCONST A;
+



\gfcommand{probe}{verbose mode}
\index{probe}

\gfsyntax{
   probe;\\
   probe \ARG{activity};\\
   probe all;\\
}

\gfdescription{
	Some {\GF} {\em activities} can be executed either in verbose
	or in silent mode.
	In the former, {\GF} displays messages describing the execution of the
	command which are not displayed in the silent mode.
	Examples of activities are given below. 

	{\tt probe} lists the probed commands

	{\tt probe} \ARG{activity} sets commands in the activity to be executed
	in a verbose mode. 

	{\tt probe all} sets all activities to verbose mode.
}

\gfrecap{
Set verbose mode for certain activities.
}

\gfexample+
   ***** probe;
   Probing function set : COMMAND no
   Probing function set : IO yes
   Probing function set : DECLARE yes
   Probing function set : PROOF yes
   Probing function set : ATTACH yes
   Probing function set : SIMPLIFY yes
   Probing function set : SIMPSET yes
   Probing function set : REWRITE yes
   Probing function set : EVAL yes
   Probing function set : CONTEXT yes
   Probing function set : REFLECT yes
   
   ***** declare sentconst A B C;
   A has been declared to be a Sentconst
   B has been declared to be a Sentconst
   C has been declared to be a Sentconst
   
   ***** probe command;
   
   ***** probe;
   probe;
   Probing function set : COMMAND yes
   Probing function set : IO yes
   Probing function set : DECLARE yes
   Probing function set : PROOF yes
   Probing function set : ATTACH yes
   Probing function set : SIMPLIFY yes
   Probing function set : SIMPSET yes
   Probing function set : REWRITE yes
   Probing function set : EVAL yes
   Probing function set : CONTEXT yes
   Probing function set : REFLECT yes
   
   ***** declare sentconst D E F;
   declare sentconst D E F;
   D has been declared to be a Sentconst
   E has been declared to be a Sentconst
   F has been declared to be a Sentconst
   
   ***** unprobe all;
   unprobe all;
   
   ***** declare sentconst H G K;
+

\gfnotes{}

\gfcommand{unprobe}{silent mode}
\index{unprobe}

\gfsyntax{
   unprobe \ARG{activity};\\
   unprobe all;
}

\gfdescription{
   Some {\GF} {\em activities} can be executed either in verbose
   or in silent mode.

   {\tt unprobe} \ARG{activity} sets the silent mode for commands in
   \ARG{activity}.

   {\tt unprobe all} sets all activities to silent mode .
}

\gfrecap{
Set/unset verbose mode for certain activities.  
}

\gfexample+
   ***** declare sentconst D E F;
   declare sentconst B E F;
   D has been declared to be a Sentconst
   E has been declared to be a Sentconst
   F has been declared to be a Sentconst
   
   ***** unprobe all;
   unprobe all;

   ***** probe;
   Probing function set : COMMAND no
   Probing function set : IO no
   Probing function set : DECLARE no
   Probing function set : PROOF no
   Probing function set : ATTACH no
   Probing function set : SIMPLIFY no
   Probing function set : SIMPSET no
   Probing function set : REWRITE no
   Probing function set : EVAL no
   Probing function set : CONTEXT no
   Probing function set : REFLECT no
   
   ***** declare sentconst H G K;
+

\gfnotes{}


	% loading introduction to the section
\newpage
\section{Administration}
\label{sec-adm}

The commands described in this section manipulate the proof checker but do
not modify  the ``logical'' state of the deduction or of the computation.
Among the other things, they can be used to give alternative names to proof lines,
to load {\HG} or {\GF} files, to insert comments in {\GF} files, to show the
logical/computational status of the system.


% loading explanation of commands
\gfcommand{comment}{comments in {\GF}}
\index{comment}

\gfsyntax{
   comment \ARG{separator} \OPT{\ARG{text}} \ARG{separator}
}

\gfdescription{
   Defines a comment \ARG{text} between the two \ARG{separator}s.
   Any token can be a separator.
}

\gfrecap{
Defines a comment between two separators.
Any token can be a separator.
}

\gfexample+
   ***** comment ! this is a comment 
   enclosed between two exclamation marks !
   
   ***** comment ? this is a comment 
   enclosed between two question marks ?
   
   ***** comment com this is a comment 
   enclosed between the two words "com" com
+


\gfcommand{deflam}{defining {\HG} functions}
\index{deflam}

\gfsyntax{
   deflam \ARG{funname} \ARG{var-list} \ARG{form};
}

\gfdescription{
   Defines a {\HG} function at the {\GF} prompt.
   The function \ARG{funname} is defined as the {\HG} \ARG{form}
   with parameters the parameters in the list \ARG{var-list}.
}

\gfrecap{
Defines a HGKM function at the GETFOL prompt.
}

\gfexample+
   ***** deflam wffof (fact) (fact\-get\-wff fact);
+


\gfcommand{echo}{echoes a message to the standard output stream}
\index{echo}

\gfsyntax{
   echo \ARG{separator} \OPT{\ARG{text}} \ARG{separator}
}

\gfdescription{
   Echoes \ARG{text} between the two \ARG{separator}s to the current
   output stream.
   Any token can be a \ARG{separator}.
}

\gfrecap{
Echoes text between the two separators.
}

\gfexample+
   ***** echo ! this is an echo
   enclosed between two exclamation marks !
   this is an echo enclosed between two exclamation marks

   ***** comment ? this is an echo
   enclosed between two question marks ?
   this is an echo enclosed between two question marks

   ***** comment com this is a echo
   enclosed between the two words "com" com
   this is an echo enclosed between the two words "com"
+

\gfcommand{hgk}{{\HG} evaluation}
\index{hgk}

\gfsyntax{
   hgk \ARG{s-expr};
}

\gfdescription{
   Runs the {\HG} evaluator on \ARG{s-expr}.
}

\gfrecap{
Runs the HGKM evaluator on `s-expr'.   
}

\gfexample+
   ***** hgk (LOAD (QUOTE "file"))
+

\gfnotes{
	The command hgk tries to evaluate every element of the {\em
	s-expression} passed as argument; therefore \verb+(LOAD "file")+
	causes an error, if \verb+"file"+ has no value.
}
\gfcommand{know natnums}{allows the use of natural numbers}
\index{know natnums}

\gfsyntax{
	know natnums \OPT{\ARG{natnum1}, \SEQ, \ARG{natnumN}};
}

\gfdescription{
	It allows the use of \ARG{natnum1}, \SEQ, \ARG{natnumN}
	(all natural numbers if no natural numbers are explicitly listed).
	It declares the sort {\tt NATNUMSORT}, the representation 
	{\tt NATNUMREP} and optionally defines the extension of {\tt NATNUMSORT} 
	to be \ARG{natnum1}, \SEQ, \ARG{natnumN} (if explicitly listed).
}

\gfexample+
   ***** know natnums;
   ***** simplify (5=7);
   1   not (5 = 7)   
   ***** know natnums 1 2 3 4;
   Warning! You already know natnums.
   Now the extension of NATNUMSORT is fixed to be : (1 2 3 4)
   ***** simplify (5=7);
   simplify (5=7);
            ^
   SIMPLIFY requires a wff,fact or term here
   ***** simplify (1=3);
   2   not (1 = 3)
+

\gfcommand{load}{loading a {\HG} file}
\index{load}

\gfsyntax{
   load \ARG{file};
}

\gfdescription{
   Each form in \ARG{file} is read by the {\HG} reader and evaluated by the 
   {\HG} evaluator \cite{giunchiglia35}.
}

\gfrecap{
Each form in `file' is read and evaluated by HGKM.
}

\gfexample+
   ***** load example.hgk;
+


\gfcommand{resetprompt}{Resets the user defined prompt}
\index{resetprompt}

\gfsyntax{
   resetprompt;
}

\gfdescription{
   Comes back to the default prompt.
}
\gfrecap{
Comes back to the default prompt.
}

\gfexample+
   CARTOONIA:: resetprompt;

   ***** switchcontext Disneyland;
   You are now using context: Disneyland
   You are switching to a proof with no name.

   *****
+

\gfcommand{setprompt}{Redefines the prompt}
\index{setprompt}

\gfsyntax{
   setprompt to \ARG{s-expr};
}

\gfdescription{
   Sets {\GF}'s prompt  to the value of the s-expression
   \ARG{s-expr}  followed by  ":: ". It is particularly  useful when  you are
   working on multiple contexts as you can set the prompt to the value of
   the current context.
}

\gfrecap{
Sets GETFOL's prompt to `s-expr'.
}

\gfexample+
   ***** setprompt to (QUOTE myprompt);

   myprompt:: setprompt to (QUOTE Tweedledee\&Tweedledum);

   Tweedledee&Tweedledum:: setprompt to (CAPITALIZE (curcname\-get));

   NOTNAMED&:: namecontext Disneyland;
   You have named the current context: Disneyland

   DISNEYLAND:: makecontext Cartoonia;
   You have created the empty context: Cartoonia

   DISNEYLAND:: switchcontext Cartoonia;
   You are now using context: Cartoonia
   You are switching to a proof with no name.

   CARTOONIA:: 
+

\gfnotes{
	The command tries to evaluate the {\em s-expression} passed as argument.
	Failure of the evaluation causes a crash to the {\HG} evalautor.
}

\gfcommand{show}{shows {\GF} information}

\gfsyntax{
   show \ARG{option};
}

\gfdescription{
   Shows {\GF} information. In the example we show some of the
   options implemented in a {\GF} version.
   Notice that ``\verb+show show;+" lists the options supported by the show
   command.
}

\gfrecap{
Shows GETFOL information.
}

\gfexample+
   ***** comment | ******* SHOW SHOW ******* |
   ***** show show;
   The list of show options is the following:

   CONTEXT : WHEREAMI 

   REWRITER : SIMPSET 

   SIMPLIFIER : INT REP 

   DEFINITION : DEFINITION 

   PROOF : PREMISES FACT AXIOM 

   LANGUAGE : LGS MGS MEM SORT TYP SYM 

   SYSTEM : COM SHOW 
   
   ***** comment | ******* SHOW WHEREAMI ******* |
   
   ***** show whereami;
   You are now using an unnamed context.
   You are now using an unnamed proof.
   
   ***** comment | ******* SHOW REP and INT ******* |
   
   ***** decrep PIPPOREP;
   ***** attach A dar [PIPPOREP] a;
   ***** attach C dar [PIPPOREP] c;
   ***** attach B to  [PIPPOREP] b;
   
   ***** show rep PIPPOREP;
   The designators for the representation: PIPPOREP are:
   (c . C) (a . A)
   
   ***** show int A;
   The Indconst A is attached to 'a
   with representation PIPPOREP
   
   ***** comment | ******* SHOW SIMPSET ******* |
   
   ***** setbasicsimp simp1 at wffs
   {forall x1.F1(x1)=x1,
   forall x1 x2.F2(x1 x2)=x1,
   forall x1 x2 x3.F3(x1 x2 x3)=x1};
   
   ***** show simpset simp1;
   Wffs :
   forall x1. (F1(x1) = x1)
   forall x1 x2. (F2(x1,x2) = x1)
   forall x1 x2 x3. (F3(x1,x2,x3) = x1)
   
   ***** setbasicsimp simp2 at facts {1 2};
   
   ***** show simpset simp2;
   Proof lines :  1 2
   
   ***** comment | AX1 and AX2 are two axioms |
   ***** setbasicsimp simp3 at facts {AX1 AX2};
   ***** show simpset simp3;
   Axioms :  AX1 AX2
   
   ***** setcompsimp simpA at simp1 uni simp2 uni simp3;
   ***** show simpset simpA;
   simpA is compound by this list of basic simpsets :
   (simp1 simp2 simp3)
   
   ***** SETCOMPSIMP simpB at simpA dif simp1;
   ***** show simpset simpB;
   simpB is compound by this list of basic simpsets :(simp2 simp3)
   
   
   ***** comment | ******* SHOW PROOF, FACT and AXIOM ******* |
   
   ***** show proof;
   1   forall x1. (F1(x1) = x1)     (1)
   2   forall x1 x2. (F2(x1,x2) = x1)     (2)
   
   ***** label fact identity = 1;
   ***** show fact;
   identity   1
   
   ***** show axiom;
   AX1 : forall x1. (F1(x1) = x1)
   AX2 : forall x1 x2. (F2(x1,x2) = x1)
   
   ***** comment | ******* SHOW LSG, MGS , MEM and SORT ******* |
   
   ***** declare sort a b c d e f g h;
   moregeneral a < b c d e f g h >;
   moregeneral b < c d e f g h >;
   
   ***** show lgs a;
   No sort is strictly lessgeneral than a.
   ***** show lgs b;
   No sort is strictly lessgeneral than b.
   
   ***** show mgs a;
   The only sort strictly moregeneral than a is UNIVERSAL
   ***** show mgs f;
   The only sort strictly moregeneral than f is UNIVERSAL
   
   ***** show mem a;
   No <indsym> is declared to be of sort a.
   
   ***** declare indvar x [a];
   a is a sort
   x has been declared to be an Indvar
   ***** declare indvar y [a];
   a is a sort
   y has been declared to be an Indvar
   ***** show mem a;
   The <indsym>'s declared to be of sort a are
       y  x  
   
   ***** show sort;
   The symbols declared to be sorts are
       b  a  UNIVERSAL  
   
   ***** comment | ******* SHOW TYP, SYM ******* |
   
   ***** show typ INDVAR;
   The symbols declared to be Indvars are
       x  y  
   
   ***** show sym a;
   a is declared to be a sort.
   ***** show sym x;
   x is declared to be an Indvar of sort a.
   
   
   ***** comment | **************** SHOW PREMISES *************** |
   
   ***** declare sentconst A B C;
   ***** assume A B;
   1   A     (1)
   2   B     (2)
   ***** andi 1 2;
   3   A and B     (1 2)
   ***** ori 3 C;
   4   (A and B) or C     (1 2)
   ***** andi 3 4;
   5   (A and B) and ((A and B) or C)     (1 2)
   ***** show premises 5;
   5  (A and B) and ((A and B) or C)  (1 2)
      3  A and B  (1 2)
      4  (A and B) or C  (1 2)
   ***** show premises 5 2;
   5  (A and B) and ((A and B) or C)  (1 2)
      3  A and B  (1 2)
         1  A  (1)
         2  B  (2)
      4  (A and B) or C  (1 2)
         3  A and B  (1 2)
   *****  show premises 5 all;
   5  (A and B) and ((A and B) or C)  (1 2)
      3  A and B  (1 2)
         1  A  (1)
         2  B  (2)
      4  (A and B) or C  (1 2)
         3  A and B  (1 2)
            1  A  (1)
            2  B  (2)
   
   
   ***** comment | ******* SHOW COM ******* |
   
   ***** show com;
   The list of commands is the following:

   META : REFLECT MATTACH 

   CONTEXT : COPYLEX SWITCHCONTEXT COPYCONTEXT NAMECONTEXT MAKECONTEXT 

   DECIDER : DECIDE MONADEQ MONAD TAUTEQ TAUT PTAUT 

   EVAL : EVAL 

   SIMPLIFIER : SIMPLIFY LET HARDWARE REPRESENT ATTACH DECREP 

   REWRITER : REWRITE ASSERTSIMP SETCOMPSIMP SETBASICSIMP UNFOLD FOLD CUT
   CTC CONTRACT WK WEAKEN WFFIFI WFFIFEN WFFIFE TERMIFI TERMIFEN TERMIFE  

   Natural-Deduction : ES EXISTE EXISTI US ALLE UG ALLI IE IFFE II IFFI
   NE NOTE NI NOTI FE FALSEE FI FALSEI OE ORE OI ORI AE ANDE AI ANDI MP
   IMPE DED IMPI SUBST THEOREM ASSUME   

   DEFINITION : DEFINE 

   PROOF : LABEL CANCEL AXIOM THEORY 

   LANGUAGE : SWITCHPROOF COPYPROOF NAMEPROOF MAKEPROOF EXTENSION WFF
   AWFF TERM MOREGENERAL MOSTGENERAL SETFMAP DECLARE  

   SYSTEM : COPYLEX RESET PAGER RESETPROMPT SETPROMPT KNOW HGK SHOW ECHO
   COMMENT UNPROBE PROBE DEFLAM LOAD MARK FETCH DONE BACKUP  
+

 	% introduction to the language's section
\newpage
\section{Language}
\label{sec-lang}
\label{sec-decl}


A {\GF} language is a first-order language.
Its terms, atomic formulae, and
well-formed formulae are built from primitive symbols representing
variables, constants, functions, predicates,
connectives and quantifiers. The various syntactic categories of expressions,
corresponding to the different logical categories, are
recursively defined in terms of each other,
these definitions determining the written syntax of the
language\footnote{By written syntax we mean the syntax of the
expressions as they are typed rather than as they are represented
internally.}.

Each primitive symbol has a {\bf syntype}
which corresponds to its
logical status. Symbols of certain {\bf syntype}s
must also have other associated
information (such as their arity) which together with the {\bf syntype}
determines
how the symbol is to be used in the construction of compound expressions.

As every {\GF} language is a first-order language there are certain
logical symbols (such as connectives and quantifiers)
which are predefined. The other primitive
symbols of a language must be user defined by means of declarations 
(see the command {\bf declare}) which
specify a symbol together with its {\bf syntype}
(and any other information
required).

\subsection{{\GF} symbols}

{\GF} commands accept symbols that we will identify with
{\em sym}. The {\GF} scanner recognizes different {\em types} of
characters: {\em identifiers}, {\em numbers} (positive integers),
{\em escape characters},
{\em delimiters} and special delimiters that 
may be regarded as tokens: {\em identifier-delimiters}.
{\GF} symbols {\em sym}s are sequences of {\em identifiers}
and {\em numbers}. Identifiers are for instance:
\begin{verbatim}
                       a b c d
                       A B C D
                       ! " # $ % &
\end{verbatim}    %%% $ (fake emacs' hilite mode)
{\em delimiters} are used to separate symbols.
Some delimiters are:
\begin{verbatim}
                       ( ) , . : ; [ ] 
\end{verbatim}
{\em identifier-delimiters} may be used to separate
symbols or may identify a particular token, as for instance
\begin{verbatim}
                       < > = + - * / ? @ 
\end{verbatim}
Escape characters turn a character into an identifier.
The character $\backslash$ is {\GF}'s escape character.

Types are assigned to characters in the file {\tt asciitab.fol}
in the {\tt fol} directory.

An example:
\begin{verbatim}
***** declare sentconst a!b;
a!b has been declared to be a Sentconst
***** declare sentconst a_b;
a_b has been declared to be a Sentconst
***** declare sentconst a-b;
a has been declared to be a Sentconst
- has been declared to be a Sentconst
b has been declared to be a Sentconst
***** declare sentconst c/d;
c has been declared to be a Sentconst
/ has been declared to be a Sentconst
d has been declared to be a Sentconst

***** declare sentconst e\f;
ef has been declared to be a Sentconst
\end{verbatim}

\subsection{{\GF} special symbols}

The following strings are treated as special symbols:

\begin{itemize}
\item The logical connectives:
{\tt not, and, or, imp, iff, wffif, trmif, forall, exists};
\item The sentential constants for truth and falsity:
{\tt FALSE, TRUE}; 
\item The equality predicate symbol: {\tt =};
\item Sorts: {\tt UNIVERSAL}, {\tt NATNUMSORT};
\item Numerals: {\tt 1}, {\tt 2}, {\tt 3}, ...;
\item Representations: {\tt UNIVERSALREP}, {\tt NATNUMREP};
\item Natural numbers: {\tt 1}, {\tt 2} {\tt 3}, ...;
\item Various: {\tt TRUTHSORT}, {\tt TRUTHREP}, {\tt NOTNAMED\&}, {\tt UNDEF\&};
\end{itemize}



Using these symbols not in ``reading mode",
that is modifying their meaning,
is very dangerous. The behavior
of the system becomes unpredictable.






\subsection{Syntypes of primitive symbols}

The different {\bf syntype}s of primitive symbols are:

\gap
\begin{center}
\fbox{
\parbox{15cm}{
\begin{itemize}

\item {\indvar} --- individual variables.

\item {\indpar} --- individual parameters.

\item {\indconst} --- individual constants.

\item {\funpar} --- function parameters.

\item {\funconst} --- function constants.

\item {\predpar} --- predicate parameters.

\item {\predconst} --- predicate constants.

\item {\sentpar} --- sentential parameters.

\item {\sentconst} --- sentential constants.

\item {\sentconn} --- sentential connectives.

\item {\quant} --- quantifiers.

\end{itemize}
}}
\end{center}

\gap
Some of the above {\bf syntype}s are grouped together into hybrid syntactic
categories according to the role they play
in defining logical expressions.

\gap
\begin{center}
\fbox{
\parbox{15cm}{
{\indsym} ::= {\indvar} \I {\indpar} \I {\indconst}

{\funsym} ::= {\funpar} \I {\funconst}

{\predsym} ::= {\predpar} \I {\predconst}

{\sentsym} ::= {\sentpar} \I {\sentconst}
}}
\end{center}

\subsection{Logical expressions}

The various categories of compound logical expressions are defined
recursively in terms of each other and in terms of the categories of
primitive symbols. The most important categories, from a logical point of
view, are: {\wff} (the category of well-formed formulae), {\awff} (the
category of atomic formulae) and {\term} (the category of terms).

The following definition gives the 
grammar of expressions accepted by {\GF}. In all cases $n$
is a natural number $\geq 1$.
Superscripts of
---$^n$, ---$^P$, ---$^I$ are used with certain categories,
the categories of operator symbols, to specify other properties which the
permitted symbols must possess; the three superscripts require
respectively: that the
symbol has arity $n$, that the symbol is a unary prefix operator, that the
symbol is a binary infix operator. Operator symbols and their associated
properties are treated in full in section \ref{opsym}.

\gap
\begin{center}
\fbox{
\parbox{15cm}{
{\wff} ::= {\bf (} {\wff} {\bf )} \I {\connappl} \I {\quantwff}
\I {\wffif} \I {\awff}

{\connappl} ::= {\sentconn}$^{P}$ {\wff} \I {\wff}$_1$
{\sentconn}$^{I}$ {\wff}$_2$

{\quantwff} ::= {\quantprefix} {\wff}

{\quantprefix} ::= {\quant} {\indvar}$_1$ ... {\indvar}$_n$ {\bf .}

{\wffif} ::= {\bf wffif} {\wff}$_1$ {\bf then} {\wff}$_2$ {\bf else}
{\wff}$_3$

{\awff} ::= {\sentsym} \I {\predappl}

{\predappl} ::= {\predsym}$^{P}$ {\term} \I
                {\term}$_1$ {\predsym}$^{I}$ {\term}$_2$ \I
                {\predsym}$^n${\bf (} {\term}$_1$ [{\bf ,}] ...
                [{\bf ,}] {\term}$_n$ {\bf )}

{\term} ::= {\bf (} {\term} {\bf )} \I {\funappl} \I {\termif} \I {\indsym}

{\termif} ::= {\bf trmif} {\wff} {\bf then} {\term}$_1$ {\bf else}
{\term}$_2$

{\funappl} ::= {\funsym}$^{P}$ {\term} \I
               {\term}$_1$ {\funsym}$^{I}$ {\term}$_2$ \I
               {\funsym}$^n${\bf (} {\term}$_1$ [{\bf ,}] ...
               [{\bf ,}] {\term}$_n$ {\bf )}
}}
\end{center}

\subsection{Operator symbols}
\label{opsym}

The primitive symbols of the categories  {\funsym},
\predsym, {\sentconn} are operator symbols; each operator
symbol has an associated arity which is a natural number $\geq 1$.
\funsym$^n$, \predsym$^n$, and \sentconn$^n$ are used to denote the
subclasses of {\funsym}, {\predsym} and {\sentconn} of operators with
arity $n$.

An operator symbol with arity $1$ must be defined to be prefix. \funsym$^P$,
\predsym$^P$, and \sentconn$^P$ are used to denote the categories of prefix
operators, these are subclasses of \funsym$^1$, \predsym$^1$, and
\sentconn$^1$ respectively.

An operator symbol with arity $2$ may be defined to be infix. \funsym$^I$,
\predsym$^I$, and \sentconn$^I$  are used to denote the categories of infix
operators, these will be
subclasses of \funsym$^2$, \predsym$^2$, and \sentconn$^2$ respectively.

The use of prefix and infix operators allows the following ambiguous
expressions:


\begin{enumerate}

\item {\term}$_1$ {\funsym}$^I_1$ {\term}$_2$ {\funsym}$^I_2$ {\term}$_3$

\item {\wff}$_1$ {\sentconn}$^I_1$ {\wff}$_2$ {\sentconn}$^I_2$ {\wff}$_3$

\item {\funsym}$^P$ {\term}$_1$ {\funsym}$^I$ {\term}$_2$

\item {\sentconn}$^P$ {\wff}$_1$ {\sentconn}$^I$ {\wff}$_2$

\item {\quantprefix} {\wff}$_1$ {\sentconn}$^I$ {\wff}$_2$~\footnote{This is
a special case as a {\quantprefix} behaves like a prefix operator.}

\end{enumerate}

These ambiguities are avoided through the assignment of additional
information to certain operator symbols declared to be prefix or infix.
Each symbol in \funsym$^P$ and \sentconn$^P$ must
have an associated prefix binding priority (a natural number
$\geq 1$)\footnote{In fact each prefix and infix {\predsym} also has
associated binding priorities but this information is redundant as
ambiguities can not arise.}, this
will be denoted by prbp(---). Each symbol in \funsym$^I$ and \sentconn$^I$
must have two associated binding priorities: the left binding
priority and the right binding priority (both natural numbers $\geq 1$);
these will be denoted by lbp(---) and rbp(---) respectively.

The five ambiguous cases above are disambiguated in the following way:

\begin{enumerate}

\item If rbp({\funsym}$^I_1$) $>$ lbp({\funsym}$^I_2$)
then the expression is parsed as:

({\term}$_1$ {\funsym}$^I_1$ {\term}$_2$) {\funsym}$^I_2$ {\term}$_3$

Otherwise, if
rbp({\funsym}$^I_1$) $\leq$ lbp({\funsym}$^I_2$)
then the expression is parsed as:

{\term}$_1$ {\funsym}$^I_1$ ({\term}$_2$ {\funsym}$^I_2$ {\term}$_3$)

\item If rbp({\sentconn}$^I_1$) $>$ lbp({\sentconn}$^I_2$)
then the expression is parsed as:

({\wff}$_1$ {\sentconn}$^I_1$ {\wff}$_2$) {\sentconn}$^I_2$ {\wff}$_3$

Otherwise, if
rbp({\sentconn}$^I_1$) $\leq$ lbp({\sentconn}$^I_2$)
then the expression is parsed as:

{\wff}$_1$ {\sentconn}$^I_1$ ({\wff}$_2$ {\sentconn}$^I_2$ {\wff}$_3$)

\item If prbp({\funsym}$^P$) $>$ lbp({\funsym}$^I$)
then the expression is parsed as:

({\funsym}$^P$ {\term}$_1$) {\funsym}$^I$ {\term}$_2$

Otherwise, if
rbp({\sentconn}$^I_1$) $\leq$ lbp({\sentconn}$^I_2$)
then the expression is parsed as:

{\wff}$_1$ {\sentconn}$^I_1$ ({\wff}$_2$ {\sentconn}$^I_2$ {\wff}$_3$)

\item If prbp({\sentconn}$^P$) $>$ lbp({\sentconn}$^I$)
then the expression is parsed as:

({\sentconn}$^P$ {\wff}$_1$) {\sentconn}$^I$ {\wff}$_2$

Otherwise, if
prbp({\sentconn}$^P$) $\leq$ lbp({\sentconn}$^I$)
then the expression is parsed as:

{\sentconn}$^P$ ({\wff}$_1$ {\sentconn}$^I$ {\wff}$_2$)

\item If $1000 >$ lbp({\sentconn}$^I$)
then the expression is parsed
as \footnote{1000 is the default prefix binding
priority.}:

({\quantprefix} {\wff}$_1$) {\sentconn}$^I$ {\wff}$_2$

Otherwise, if
$1000 \leq$ lbp({\sentconn}$^I$)
then the expression is parsed as:

{\quantprefix}  ({\wff}$_1$ {\sentconn}$^I$ {\wff}$_2$)

\end{enumerate}


\subsection{The logical symbols}
\label{sec-log-symb}

The logical symbols are hardwired into the system.
The two categories {\quant} and {\sentconn}
consist entirely of logical constants and are thus fixed in advance for each
language.

\gap
\begin{center}
\fbox{
\parbox{15cm}{
{\quant} ::= {\bf forall} \I {\bf exists}


{\sentconn} ::= {\bf not} \I
                {\bf and} \I {\bf or} \I {\bf imp} \I {\bf iff}
}}
\end{center}

The logical connectives are all operator symbols
of arity 2 and declared as infix with the exception of {\bf not}
which has arity 1 and is prefix. The binding priorities of these symbols are
given in the table below.

\begin{center}
\begin{tabular}{|l|c|c|ccc|} \hline
\multicolumn{1}{|c}{Symbol} &
\multicolumn{1}{c}{Arity} &
\multicolumn{1}{c}{Type} &
\multicolumn{1}{c}{prbp} & lbp & rbp \\ \hline
{\bf not} & 1 & prefix & 1000 & n/a & n/a \\ \hline
{\bf and} & 2 & infix & n/a  & 750 & 755 \\ \hline
{\bf or} & 2 & infix & n/a  & 740 & 745 \\ \hline
{\bf imp} & 2 & infix & n/a  & 730 & 735 \\ \hline
{\bf iff} & 2 & infix & n/a  & 720 & 725 \\ \hline
\end{tabular}
\end{center}

The two truth values {\tt TRUE} and {\tt FALSE} and the equality
predicate, {\tt =}, are also logical symbols, {\tt TRUE} and
{\tt FALSE} belong to the category {\sentconst} and {\tt =} belongs to the
category {\predconst}; however, the categories {\sentconst} and
{\predconst} are not fixed
(because other primitive symbols in these categories
can be user declared).

The equality predicate, {\tt =}, has arity 2 and is declared as infix.





% user commands for language
\gfcommand{awff}{{\em awff} verifier}
\index{awff}

\gfsyntax{
   awff \ARG{awff};
}

\gfdescription{
   {\bf term}, {\bf awff} and {\bf wff} verify that their arguments
   are terms, atomic formulas and well formed formulas respectively.
   An error is signaled if an expression of the correct
   syntactic category is not provided.
}

\gfrecap{
Verifies whether the formula `awff' is atomic.
}

\gfexample+
   ***** declare indvar x y;
   ...
   ***** awff x = y;
   x = y is an <awff>.

   ***** declare predconst P 1 [pre];
   ***** declare funconst f g 1;
   ...
   ***** awff P g(f(x),y);
   P g(f(x),y) is an <awff>.

   ***** awff P x and P x;
   awff P x and P x;
                ^
   You have a legal command not ending with a semi-colon.
+
\gfcommand{declare}{language declaration}
\index{declare}
\index{declare!language}

\gfsyntax{
   declare \OPT{\ARG{sentsym} \ALT \ARG{indsym}} \ARG{sym1} \SEQ \ARG{symN}; \\
   declare \OPT{\ARG{funsym}  \ALT \ARG{predsym}} \ARG{sym1} \SEQ \ARG{symN}
   \ARG{arity};\\
   declare \OPT{\ARG{funsym} \ALT \ARG{predsym}} \ARG{sym1} \SEQ \ARG{symN}
   1 \OPT{[ pre \OPT{= \ARG{prbp}} ]};\\
   declare \OPT{\ARG{funsym} \ALT \ARG{predsym}} \ARG{sym1} \SEQ \ARG{symN}
   2 \OPT{[ inf \OPT{= \ARG{lbp} \ARG{rbp}} ] };
}

\gfdescription{
   This commands adds new symbols to the {\GF} language.
   The user can declare primitive symbols belonging to the following
   categories: {\tt sentpar}, {\tt sentconst}, {\tt indvar}, {\tt indpar},
   {\tt indconst}, {\tt funpar}, {\tt funconst}, {\tt predpar}, 
   {\tt predconst}.
   The user can specify binding priorities ($prbp$, $lbp$, $rbp$) 
   for the operator symbols being declared.
   If this is not done, then the default priorities are used
   (see section \ref{sec-log-symb}).
}

\gfrecap{
Adds new symbols to the GETFOL language.
}

\gfexample+
   ***** comment ! declarations of the first kind !

   ***** declare sentconst A B C;
   A has been declared to be a Sentconst
   B has been declared to be a Sentconst
   C has been declared to be a Sentconst

   ***** declare sentpar S1 S2;
   S1 has been declared to be a Sentpar
   S2 has been declared to be a Sentpar

   ***** declare indvar x y;
   UNIVERSAL is a sort
   x has been declared to be an Indvar
   y has been declared to be an Indvar

   ***** declare indpar a;
   UNIVERSAL is a sort
   a has been declared to be an Indpar

   ***** declare indconst alpha;
   UNIVERSAL is a sort
   alpha has been declared to be an Indconst

   ***** comment ! declarations of the second kind !
   
   ***** declare predconst R 2;
   R has been declared to be a Predconst
   
   ***** declare funconst f 1;
   f has been declared to be a Funconst
   
   ***** declare funconst g 2;
   g has been declared to be a Funconst
   
   ***** declare funconst h 4;
   h has been declared to be a Funconst
   
   ***** comment ! declarations of the third kind !
   
   ***** declare funconst f1 1 [pre = 500];
   f1 has been declared to be a Funconst
   
   ***** declare predconst P 1 [pre];
   P has been declared to be a Predconst
   
   ***** comment ! declarations of the fourth kind !
   
   ***** declare funconst g1 2 [inf = 400 405];
   g1 has been declared to be a Funconst
   
   ***** declare funconst g2 2 [inf = 605 600];
   g2 has been declared to be a Funconst
   
   ***** declare funpar g3 2 [inf];
   g3 has been declared to be a Funpar
+

\gfnotes{
   Since the {\GF} logic is sorted, the user can specify sort
   information for some categories of symbols. If this is not done, then the
   default information is used (as is the case in the examples above).
   The sorting mechanism in {\GF} is discussed in section \ref{sec-sort}.
}



\gfcommand{term}{{\em term} verifier}
\index{term}

\gfsyntax{
   term \ARG{term};
}

\gfdescription{
   {\tt term}, {\tt awff} and {\tt wff} verify that their arguments
   are terms, atomic formulas and well formed formulas respectively.
   An error is signaled if an expression of the correct
   syntactic category is not provided.
}

\gfrecap{
Verifies whether the term `term' is well formed.
}

\gfexample+
   ***** declare funconst f 1;
   ***** declare funconst g 2;
   ***** declare indvar x y;
   ...
   
   ***** term g(f(x),y);
   g(f(x),y) is a <term>.

   ***** term g(x);
   g expects 2 arguments
+
\gfcommand{wff}{{\em wff} verifier}
\index{wff}

\gfsyntax{
   wff \ARG{wff};
}

\gfdescription{
   {\tt term}, {\tt awff} and {\tt wff} verify that their arguments
   are terms, atomic formulas and well formed formulas respectively.
   An error is signaled if an expression of the correct
   syntactic category is not provided.
}

\gfrecap{
Verifies whether the formula `wff' is well formed.
}

\label{language-examples}
\gfexample+
   ***** declare predconst P 1 [pre];
   ***** declare predconst R 2;
   ***** declare indvar x y;
   ***** declare funconst g 2;
   ***** declare funconst f 1;
   ...

   ***** wff P g(f(x),y) and x = y;
   (P g(f(x),y)) and (x = y) is a <wff>.
   ***** wff exists x y . (P g(f(x),y) and x = y);
   exists x y. ((P g(f(x),y)) and (x = y)) is a <wff>.
   ***** wff forall x y. wffif P trmif x = y then f(x) else f(y)
                then R(x,y)
                else R(y,x);
   forall x y. (wffif (P (trmif (x = y) then f(x) else f(y))) then R(x,y) else
   R(y,x)) is a <wff>.

   ***** wff Z(x);
   wff Z(x);
       ^
   The scanner requires a <wff> here
+



% introduction to the sort's section
\newpage
\section{Sorts}
\label{sec-sort}

\subsection{Introduction}

The {\GF} logic is sorted, although you are not forced to make any
sort declarations.
If you never use any of the system's sort commands, you can use the 
logic as if it were unsorted. {\GF} makes sure that everything works correctly
with appropriate default sorts
(see later in this section).

Sorts in {\GF} are specially-handled unary predicates: that is, aside from the
usual features of unary predicates, they possess some additional ones.
This choice won't be deeply justified here.
The resulting logic can, however, be proved to be 
consistent and to allow more compact axiomatizations and proofs
than the unsorted one.

\subsection{Defining the sort hierarchy}
We present here the commands for the declaration of new sorts, for the
definition of the generality relations between sorts, for the definition
of a most general sort and for the declaration sort extensions.


The generality relations can be intuitively specified in this way:
stating that sort $S_1$ is {\it weakly more general} than sort $S_2$
is equivalent to
specifying an axiom of the form $\forall x(S_2(x) \imp S_1(x))$.
In this case we also say that $S_2$ is {\it weakly less general} than $S_1$.
Two sorts are said to be {\it equivalent} when they are mutually
(weakly) more general.
When $S_1$ is {\it weakly more (less) general} than $S_2$ and it is not
equivalent to it, then we say that $S_1$ is {\it strictly more (less)
general} than $S_2$.
Finally, we say that $S$ is a {\em most general sort} if it is weakly more
general than any sort. {\GF} has a default most general sort, {\tt
UNIVERSAL}, that is used when no explicit sort information is provided to
the system.

\subsection{Sorted declarations}

A sort is associated to every {\indsym} at the moment of its declaration.
In unsorted
declarations, when no sort is specified, the default most general sort
{\tt UNIVERSAL} is taken to be the sort of the symbol.

The sort information associated with a {\funsym} is a set of so-called 
{\it fmaps},
that is of $n+1$-tuples of sorts, where $n$ is the arity of the \funconst.
The fmaps for $f$ specify the sort of a term whose external {\funsym} is 
$f$, depending on the sort of its arguments. When a {\funsym} is declared
the $n+1$-tuple ({\tt UNIVERSAL} \ldots {\tt UNIVERSAL}) is always added to its
{\it fmaps}.

According to the sort information of symbols we give now the
definition of the useful notion of {\it sort of a term}:
we say that $t$ is a term of sort $S$ 
if and only if
%
\begin{itemize}
\item
  $t$ is an individual symbol, and its sort is weakly less general than $S$,
  or
\item
  $t=f(t_1,\dots,t_n)$, $f$ has an {\it fmap} of the form $(S_1,\dots,S_n,
  S_f)$, $S_f$ is weakly less general than $S$ and all $t_i$ are terms of sort
  $S_i$.
\end{itemize}


% user commands for sorts
\gfcommand{declare sort}{sort declaration}
\index{declare}
\index{declare!sort}

\gfsyntax{
  declare sort \ARG{sym1} \SEQ \ARG{symN};
}

\gfdescription{
  The declare sort command declares the \ARG{symI} to be sorts.
  The \ARG{symI} have to be either new symbols or previously declared unary
  predicates. 
}

\gfrecap{
  Declares the `symI's to be sorts.
}

\gfexample+
   ***** declare predconst S1 1;
   S1 has been declared to be a Predconst

   ***** declare sort S2;
   S2 has been declared to be a sort

   ***** declare sort S1 S2 S3;
   The unary predconst S1 has been declared to be a sort
   S2 is a sort
   S3 has been declared to be a sort

   ***** declare indvar x [S1];
   S1 is a sort
   x has been declared to be an Indvar

   ***** simplify exists x. S1(x);
   1   exists x. S1(x)     

   ***** simplify forall x. S1(x);
   2   forall x. S1(x)     
+

\gfnotes{
  Symbols with no declared sort have the default sort {\tt UNIVERSAL}.
  If \ARG{symI} is already a sort (as {\tt S2} is in the last
  example) no error is signaled. If $S$ is a sort, then $\exists x S(x)$
  and $\forall x S(x)$ are theorems where $x$ is an indvar of sort $S$.
}

\gfcommand{declare}{sorted declaration of indsyms and funsyms}
\index{declare}
\index{declare!sorted language}

\gfsyntax{
  declare indconst \ALT indpar \ALT indvar \ARG{sym1} \SEQ \ARG{symN}
  [ \ARG{sortsym} ]; \\
  declare funconst \ALT funpar \ARG{sym1} \SEQ \ARG{symN}
  ( \ARG{sortsym1} \SEQ \ARG{sortsymN} ) = \ARG{sortsym};\\
}

\gfdescription{
  The first form declares the \ARG{symI} to be indsym, and sets their
  sort to \ARG{sortsym}.
  The second form declares the \ARG{symI} to be funsyms of arity $N$, and adds
  (\ARG{sortsym1} \SEQ \ARG{sortsymN} \ARG{sortsym}) to the set of fmaps of
  the \ARG{symI}.
}

\gfrecap{
Declares symI to be an indsym (of sort `sortsym') or a funsym with fmap
(`sortsym1', ..., `sortsymN', `sortsym')
}

\gfexample+
   ***** declare predconst S4 1;
   S4 has been declared to be a Predconst
   ***** declare indvar x [S1];
   S1 is a sort
   x has been declared to be an Indvar
   ***** declare funconst f ( S4 S5 ) = S1;
   S1 is a sort
   The unary predconst S4 has been declared to be a sort
   S5 has been declared to be a sort
   f has been declared to be a Funconst
+

\gfnotes{
  In both forms of declaration \ARG{sortsym} and \ARG{sortsymI} have to be
  either new symbols or unary predicates or sorts.
  In the former cases \ARG{sortsym} and \ARG{sortsymI} are also declared to
  be sorts. 
}

\gfcommand{extension}{sort extension declaration}
\index{extension}

\gfsyntax{
  extension \ARG{sort} by \ARG{extexpr};
}

\gfdescription{
  The extension of a sort $S$ is the set of all and only the {\tt indconst}s
  of sort $S$. To say that $S$ has extension $\{c_1, \ldots ,c_n\}$,
  where $c_1, \ldots ,c_n$ are {\tt indconst}s, is the same as saying: $\forall
  x.S(x) \liff (x=c_1  \dis\ldots \dis x=c_n$), where $x$ is an {\tt indvar}
  of sort {\tt UNIVERSAL}. 
  The command declares the extension of \ARG{sort} to be \ARG{extexpr}.
  An extension expression \ARG{extexpr} is defined by the syntax below, where
  {\bf uni} stands for set union, {\bf dif} stands for set difference and {\bf
  int} stands for set intersection.
  The binding power is in the following increasing order: {\bf uni}, {\bf
  dif}, {\bf int}.
  %
  \begin{bnf}
    $extexpr$  \sep  $sort$ | {\bf \{}$indsym_1,\ldots,indsym_n${\bf \}}|\\ 
               &     $extexpr$ {\bf uni} $extexpr$      |\\
               &     $extexpr$ {\bf dif} $extexpr$      |\\  
               &     $extexpr$ {\bf int} $extexpr$ 
  \end{bnf}
  %
  A sort in \ARG{extexpr} without extension is treated as if it had an empty
  extension. 
}

\gfrecap{
  Declares the extension of a sort S (that is the set of all and only the
  indconst of sort S).
}

\gfexample+
   ***** declare sort S T R;
   S has been declared to be a sort
   T has been declared to be a sort
   R has been declared to be a sort

   ***** declare indconst a b c;
   UNIVERSAL is a sort
   a has been declared to be an Indconst
   b has been declared to be an Indconst
   c has been declared to be an Indconst

   ***** extension S by {a b c};
   Now the extension of S is fixed to be : (a b c)

   ***** extension T by S int {a c};
   Now the extension of T is fixed to be : (a c)

   ***** extension R by T uni {b};
   Now the extension of R is fixed to be : (a c b)

   ***** extension R by {a b c} dif S;
   You can't set an empty extension for R
+

\gfnotes{
  The command overwrites the previous extension of a sort without 
  warning.
  The syntactic and semantic simplifiers use the extension
  information explicitly if there is any (see the command {\tt simplify} 
  in section \ref{sec-simplify}
  and the command {\tt eval} in section \ref{sec-eval}).
}

\gfcommand{moregeneral}{new pairs in the more general relation}
\index{moregeneral}

\gfsyntax{
  moregeneral \ARG{sort1} < \ARG{sort2}, \SEQ, \ARG{sortN} >;
}

\gfdescription{
  This command makes \ARG{sort1} weakly more general than \ARG{sort2}, \SEQ,
  \ARG{sortN}.
}

\gfrecap{
Makes `sort1' weakly more general than `sort2', ..., `sortN'.
}

\gfexample+
   ***** declare sort S1 S2 S3;
   ...

   ***** moregeneral S1 < S2 S3 >;
   ***** moregeneral S2 < S1 > ;
   You realize that this makes equivalent S2 and S1
+

\gfcommand{mostgeneral}{new top for the sort hierarchy}
\index{mostgeneral}

\gfsyntax{
  mostgeneral \ARG{sym};
}

\gfdescription{
  The command declares \ARG{sym} to be a sort equivalent to the default
  most general sort ({\tt UNIVERSAL}).
}

\gfrecap{
Declares `sym' to be a sort equivalent to the most general sort.
}

\gfexample+
   ***** declare predconst P1 1;
   P1 has been declared to be a Predconst

   ***** declare sort P2 ;
   P2 has been declared to be a sort

   ***** mostgeneral P1;
   The unary predconst P1 has been declared to be a sort
   P1 is now equivalent to UNIVERSAL

   ***** mostgeneral P2;
   P2 is a sort
   P2 is now equivalent to UNIVERSAL
+

\gfnotes{
  \ARG{sym} has to be either an unused symbol or a sort or a unary predicate.
  The new symbol does not substitute for the default most general sort {\tt
  UNIVERSAL}; it is defined to be equivalent to it.
}

\gfcommand{setfmap}{adds another fmap to a funsym}
\index{setfmap}

\gfsyntax{
  setfmap \ARG{funsym} ( \ARG{sym1} \SEQ \ARG{symN} ) = \ARG{sym};
}

\gfdescription{
  The command adds (\ARG{sym1} \SEQ \ARG{symN} \ARG{sym}) to the fmaps of
  \ARG{funsym}.
}

\gfrecap{
Adds (`sym1', ..., `symN', `sym') to the fmaps of `funsym'
}

\gfexample+
   ***** declare predconst S4 1;
   S4 has been declared to be a Predconst

   ***** declare indvar x [S1];
   S1 is a sort
   x has been declared to be an Indvar

   ***** declare funconst f ( S4 S5 ) = S1;
   S1 is a sort
   The unary predconst S4 has been declared to be a sort
   S5 has been declared to be a sort
   f has been declared to be a Funconst
   
   ***** show sym f;
   f is declared to be a Funconst of arity 2.
   The FMAPs for f are
      UNIVERSAL * UNIVERSAL ==> UNIVERSAL,
      S4        * S5        ==> S1.

   ***** setfmap f ( S1 S6 ) = S5;
   S5 is a sort
   S1 is a sort
   S6 has been declared to be a sort

   ***** show sym f;
   f is declared to be a Funconst of arity 2.
   The FMAPs for f are
      S1        * S6        ==> S5,
      UNIVERSAL * UNIVERSAL ==> UNIVERSAL,
      S4        * S5        ==> S1.
+

\gfnotes{
  Note that \ARG{funsym} must have been previously declared to be a {\tt
  funsym}. 
  Its arity must be {\em N}.
}


 	% introduction to facts: the reasoning's building blocks!
\newpage
\section{Facts: reasoning building blocks}
\label{sec-reas}

Reasoning in {\GF} consists of a sequence of reasoning steps, each step being
performed on {\bf facts}.
A fact is a data structure containing a {\em hook} identifying it and some extra
information characterizing the the fact itself.
A fact can be a {\bf proof line}, an {\bf axiom}, a {\bf theory} or a {\bf
definition}.
These four categories are disjoint.
The hooks of proof lines, axioms, theories and definitions are called labels
({\em label}s), axiom labels ({\em axlabel}s), theory labels ({\em thlabel}s)
and definition labels ({\em deflabel}s) respectively.
We will simply call them {\em label}s whenever the distinction is not relevant.


\subsection{Proof lines: reasoning steps}
\label{sec-proofline}

Proof lines are the basic reasoning steps in {\GF}.
A proof line is a 4-tuple of the form: 

\begin{center}
$<$ {\em label} \ , \ {\em wff} \ , \ {\em reason} \ , \ {\em deplist} $>$
\end{center}

where
{\em label} is a ``hook" identifying the proof line itself; {\em wff} is the
well formed formula of the proof line; {\em reason} records how the proof line
has been asserted; {\em dep} records the assumptions the proof lines depends on.
For instance, the proof line 
%
\begin{center}
	{\tt < 3 , B , impe 1 2 , (1 2) >}
\end{center}
%
tells us that: the proof line {\tt 3} whose formula is ``{\tt B}'' has been
deduced by modus ponens ({\tt impe}) applied to the previously deduced proof
lines whose labels are {\tt 1} and {\tt 2} and that depends on the proof lines
{\tt 1} and {\tt 2}.
Analogously, the proof line 
%
\begin{center}
	{\tt < 4 , s(0) > 0 , alle smonotonic x 0 ,  >}
\end{center}
%
tells us that the proof line {\tt 4} whose formula is ``{\tt s(0) > 0}'' has
been deduced by instantiating the variable ``{\tt x}'' in the {\em fact} whose
{\em hook} is {\tt smonotonic} (probably an axiom asserting that
$\forall x \ (\ s(x) > x\ )$).

A {\GF} {\bf proof} is a sequence of proof lines.
The proof line labels are generated by {\GF} as increasing natural numbers
starting from $1$.
The user can ``name'' proof line labels with desired {\GF} symbols (see the
command {\tt label} in section \ref{sec-adm}) or to refer to the previous {\em
n-th} proof line in the current proof by typing \verb+^n+.


\subsection{Axioms and theories}

A {\GF} axiom is the mechanization of the notion of axiom in logic, a {\GF}
theory is the mechanization of the notion of theory, defined as a set of axioms,
in logic.
 
In {\GF} you can define the axioms of your own theory.
No check is performed on their consistency.
You are thus free to define an unsound theory.
However, the {\GF} logic is complete without need of any axioms.  
An axiom can be used to generate proof lines. 
A {\GF} axiom is a 2-tuple of the form:

\begin{center}
	$<$ {\em axlabel} \ , \ {\em wff} $>$\\
\end{center}

where {\em axlabel} is the ``hook'' and {\em wff} is the well formed formula of
the axiom. 

A theory is a set of axioms.
A theory allows you to use all the axioms of the theory as a whole. 


% user command: axioms, theories, theorems 
\gfcommand{axiom}{axiom definition}
\index{axiom}

\gfsyntax{
  axiom \ARG{sym} : \ARG{wff};
}

\gfdescription{
  Creates an axiom whose {\em axlabel\/} is {\em sym\/} and whose formula is
  {\em wff}.  The command also allows you to define axiom schemas. An axiom
  schema is an axiom containing one or more {\em sentpars}, {\em
  funpars} or {\em predpars}.  When an axiom schema is used to assert a proof
  line, then it may be instantiated, that is, some or all of {\it sentpar},   
  {\it funpars} and {\it predpars} may be ``substituted'' with complex
  wffs and terms\footnote{
    A formal definition of the substitution of
    parameters can be found in \cite{kleene1}.
  }. In particular:
  %
  \begin{itemize}
   \item 
     a {\em sentpar} $S$ can be instantiated in an axiom schema $\phi$,
     with a wff $\psi$. The result is an axiom $\phi^*$ obtained simply
     by replacing every occurrence of $S$ in $\phi$ by $\psi$;
   \item 
     a {\em funpar} $f$ of arity $n$, can be instantiated in $\phi$
     with a term $t(x_1,\ldots,x_n)$. The result of the instantiation is an
     axiom $\phi^*$ in which every occurrence in $\phi$ of the form
     $f(t_1,\ldots,t_n)$ is replaced with $t(t_1,\ldots,t_n)$;
   \item 
     a {\em predpar} $P$ of arity $n$, can be instantiated in $\phi$ with a
     wff $\psi(x_1,\ldots,x_n)$. The result of the instantiation is an
     axiom $\phi^*$ in which every occurrence in $\phi$ of the form
     $P(t_1,\ldots,t_n)$ has been replaced by $\psi(t_1,\ldots,t_n)$.
  \end{itemize} 
  %
  Let us notice that:
  %
  \begin{enumerate}
    \item 
      If an occurrence of $f(t_1,\ldots,t_n)$ in $\phi$ is within the scope
      of a quantifier for a variable $x$, if $x$ is distinct form $x_1,\ldots x_n$,
      and $x$ occurs in $t(x_1,\ldots x_n)$, then {\GF} performs a renaming
      of the variable $x$ in $\phi$, with a new variable of the same sort of
      $x$. {\em predpar} instantiation is treated in the analogous way.
    \item 
      In the {\em predpars} ($S$) and {\em sentpar} ($P$) instantiations, if a 
      variable $x$ distinct from $x_1,\ldots,x_n$ is quantified in $\psi$ and occurs
      in $P(t_1,\ldots,t_n)$ or $S$, then the substitution fails. In this case an
      error message is returned.
  \end{enumerate}
  %
  The syntax to instantiate an axiom schema whose {\em axlabel} is {\em sym} is:
  %
  \begin{quote}
    {\em sym} {\em subst$_1$} {\em subst$_2$} $\ldots$
  \end{quote}
  %
  where {\em subst} is a  pair composed of a parameter and its instantiation. The
  syntax of {\em subst} is:
  %
  \begin{quote}\tt
    \ARG{sym} sentpar : \ARG{wff} \\
    \ARG{sym} funpar  : lambda \ARG{indvar$_1$} 
                        [ \ARG{indvar$_2$} \SEQ] .
                       \ARG{term} \\
    \ARG{sym} predpar : lambda \ARG{indvar$_1$} 
                        [ \ARG{indvar$_2$} \SEQ ] .
                       \ARG{wff} \\
  \end{quote}
}

\gfrecap{
Declares an axiom in the current context labeled by `sym'.
}

\gfexample#
   ***** declare sentconst A B;
   ***** declare sentpar alpha beta;
   ...
   ***** axiom axA: A;
   axA : A
   ***** impi axA axA; 
   1   A imp A     
   ***** axiom Hil1: alpha imp (beta imp alpha);
   Hil1 : alpha imp (beta imp alpha)
   ***** impe 1 Hil1 alpha: A imp A, beta:B;
   2   B imp (A imp A)     

   ***** reset;   
   ***** declare indvar x y;
   ***** declare predconst p q 1;
   ***** declare predpar P Q 1;
   ...
   ***** axiom axdistr: forall x.(P(x) and Q(x)) imp
                        (forall x.P(x) and forall x.Q(x));
   axdistr : forall x. (P(x) and Q(x)) imp (forall x. P(x) and forall x. Q(x))
   ***** assume forall x. (p(x) and q(x));
   3   forall x. (p(x) and q(x))     (3)
   ***** impe 3 axdistr P:lambda x.p(x), Q:lambda x.q(x);
   4   forall x. p(x) and forall x. q(x)     (3)
   ***** axiom renaming: forall x. P(x) imp forall y. P(y);
   renaming : forall x. P(x) imp forall y. P(y)
   ***** ande 4 2;
   5   forall x. q(x)     
   ***** impe 5  renaming P:lambda y.q(y);
   6   forall y. q(y)     (3)
   
   ***** reset;
   ***** declare indvar n m;
   ***** declare predpar P 1;
   ***** declare indvar x;
   ***** declare funconst s 1;
   ***** declare funconst + 2 [INF];
   ***** know natnums;
   ...
   ***** axiom add1:forall n. n+0 = n;
   ***** axiom add2: forall m n. n+s(m) = s(n+m);
   ***** axiom ind:  P(0) and forall n.(P(n) imp P(s(n))) imp forall n.P(n);
   ***** eval 0+0=0+0;
   1   (0 + 0) = (0 + 0)     
   ***** assume x+0=0+x;
   2   (x + 0) = (0 + x)     (2)
   ***** alle add1 x;
   3   (x + 0) = x     
   ***** subst 2 3;
   4   x = (0 + x)     (2)
   ***** eval s(x)=s(x);
   5   s(x) = s(x)     
   ***** subst 5 4 occ 2;
   6   s(x) = s(0 + x)     (2)
   ***** alle add1 s(x);
   7   (s(x) + 0) = s(x)     
   ***** subst 7 6 occ 2;
   8   (s(x) + 0) = s(0 + x)     (2)
   ***** alle add2 x 0;
   9   (0 + s(x)) = s(0 + x)     
   ***** subst 8 9 right;
   10   (s(x) + 0) = (0 + s(x))     (2)
   ***** impi 2 10;
   11   ((x + 0) = (0 + x)) imp ((s(x) + 0) = (0 + s(x)))     
   ***** alli 11 x:n;
   12   forall n. (((n + 0) = (0 + n)) imp ((s(n) + 0) = (0 + s(n))))
   ***** andi 1 12;
   13   ((0 + 0) = (0 + 0)) and forall n. (((n + 0) = (0 + n)) imp 
        ((s(n) + 0) = (0 + s(n))))     
   ***** impe 13 ind P:lambda x. x+0=0+x;
   14   forall n. ((n + 0) = (0 + n))     
   
   ***** reset;
   ***** declare sentconst A B;
   ***** declare sentpar P Q;
   ***** axiom piffqu: P iff Q;
   piffqu : P iff Q
   ***** impi TRUE piffqu P:Q Q:A;
   1   TRUE imp (A iff A)     
   ***** impi TRUE piffqu Q:A P:Q;
   2   TRUE imp (Q iff A)     
   ***** impi TRUE piffqu Q:P P:Q;   
   3   TRUE imp (Q iff Q)     
   
   ***** reset;   
   ***** declare indvar x y;
   ***** declare funconst f 1;
   ***** declare predpar P 1;
   ***** declare predconst p 1;
   ***** declare predconst q 2;
   ***** axiom Ax1: forall x.(P(x) iff P(f(x)));
   Ax1 : forall x. (P(x) iff P(f(x)))
   ***** impi TRUE Ax1 P:lambda y. q(x y);
   1   TRUE imp forall x1. (q(x,x1) iff q(x,f(x1)))     
   ***** show sym x1;
   x1 is declared to be an Indvar of sort UNIVERSAL.
   
   ***** reset;
   ***** declare sentpar alpha;
   ***** declare indvar x;
   ***** declare predconst p q 1;
   ***** declare predpar P 1;
   ***** axiom Hil5: forall x.(alpha imp P(x)) imp (alpha imp forall x.P(x));
   Hil5 : forall x. (alpha imp P(x)) imp (alpha imp forall x. P(x))
   ***** assume P(x);
   1  P(x)     (1)
   ***** impe 1 Hil5  alpha: p(x) P:lambda x. q(x);
   alpha occurs within the scope of a quantifier binding x
#
   
\gfnotes{
   The {\tt predpar} and {\tt funpar} can be assigned any {\em wff} and {\em term} 
   whose number of (lambda) variables exceeds its arity.
   If the number is lower than the arity then we have an error.
   The extra lambda variables act as universally generalized 
   parameters of the entire axiom. The other variables get converted by  
   substitution.\\
   Axioms cannot be canceled or renamed. It is impossible to 
   define two axioms with the same {\em axlabel}. Numbers cannot be 
   {\em axlabel}s.
}

\gfcommand{theorem}{axiom creation from facts with no dependencies}
\index{theorem}

\gfsyntax{
  theorem \ARG{sym} \ARG{hook};
}

\gfdescription{
  The command creates an axiom from the fact whose hook
  is \ARG{hook}. The {\em axlabel} is \ARG{sym} and the {\em wff} is the formula
  of the fact; \ARG{hook} can identify an axiom, a theory 
  or a proof line without dependencies.
}

\gfrecap{
  Adds a fact without dependencies to the list of axioms.
}

\gfexample+
   ***** declare sentconst A;
   ...

   ***** assume A;
   1   A     (1)
   ***** theorem th1 1;
   theorem th1 1;
   A <fact> with dependencies cannot be made into a theorem
   ***** impi 1 1;
   2   A imp A     
   ***** theorem th1 2;
   th1 : A imp A
   ***** show axiom;
   th1 : A imp A
   ***** axiom ax1 : A or not A;
   ax1 : A or (not A)
   ***** theorem ExcludedMiddle ax1;
   ExcludedMiddle : A or (not A)
   ***** show axiom;
   th1 : A imp A
   ax1 : A or (not A)
   ExcludedMiddle : A or (not A)

   ***** declare predpar P 1;
   ***** declare predconst p 1;
   ***** declare indvar x;
   ***** axiom PI: P(x);
   ...
   ***** andi PI P:lambda x.p(x) PI P:lambda x.p(x);
   3   p(x) and p(x)     
   ***** theorem thPI PI;
   thPI : P(x)
   ***** andi thPI P:lambda x.p(x) thPI P:lambda x.p(x);
   4   p(x) and p(x)     
+
   
\gfnotes{
  The axiom may be an axiom schema in which case it becomes then 
  possible to instantiate it.
  In {\GF} proof lines containing {\tt predpar}s
  and/or {\tt funpar}s can {\it not} be instantiated. 
  If the user wants to prove $A \imp A$ and $B \imp B$,
  he can carry out the proof with non-instantiated axioms and get the
  schema $\alpha \imp \alpha$. $\alpha \imp \alpha$ can then be instantiated to
  $A \imp A$ and $B \imp B$. This capability is provided 
  by the command {\tt theorem}.
}





\gfcommand{theory}{theory definition}
\index{theory}

\gfsyntax{
  theory \ARG{thlabel} : \ARG{wff1} \OPT{\ARG{wff2} \SEQ};\\
  theory \ARG{thlabel} : \ARG{axlabel1} : \ARG{wff}
                         \OPT{\ARG{axlabel2} : \ARG{wff} \SEQ};
}

\gfdescription{
  Creates a theory, {\it ie.} a set of axioms that can be used as a whole
  in a deduction.
  The theory contains as many axioms as
  the number of wffs specified in \ARG{wff1} \OPT{\ARG{wff2} \SEQ}
  or the axioms specified in \ARG{axlabel1} : \ARG{wff}
  \OPT{\ARG{axlabel2} : \ARG{wff} \SEQ}.\\
  %
  In the first form, each axiom is given an axiom label (a name) by {\GF}; 
  the label is generated by taking the theory label {\em thlabel} and 
  concatenating an increasing integer starting from 1.\\
  %
  In the second form, the axiom's label is given by the user
  ({\em axlabel$_i$}).\\
  %
  A theory can be used to perform inference 
  ({\it e.g.} by applying inference rules on the theory itself). In this case,
  it is considered as
  a fact whose formula is the conjunction of the theory axioms' formulas.
  Any axiom in the theory can also be used individually by specifying its
  axiom label.\\
}

\gfrecap{
  Declares a theory (a set of axioms) in the current context; axioms in a
  theory are indexed by `axlabelN'.
}

\gfexample+
   ***** declare sentconst A B C;
   ...

   ***** theory hilbert : 
          A imp (B imp C) 
          (A imp (B imp C)) imp ((A imp B) imp (A imp C));
   hilbert
   hilbert1 : A imp (B imp C)
   hilbert2 : (A imp (B imp C)) imp ((A imp B) imp (A imp C))

   ***** mp hilbert1 hilbert2;
   1   (A imp B) imp (A imp C)

   ***** andi hilbert hilbert;
   4   (((A imp B) imp C) and 
       (((A imp B) imp C) imp ((A imp B) imp (A imp C)))) and 
       (((A imp B) imp C) and 
       (((A imp B) imp C) imp ((A imp B) imp (A imp C))))

   ***** theory tautologies: 
          IMP: A imp A 
          OR: A or not A;
   tautologies
   IMP : A imp A
   OR : A or (not A)
+


% introduction to the multiple proof's section
\newpage
\section{Multiple proofs}
\label{sec-proof}

\subsection{Introduction}

In {\GF} the user can build multiple distinct proofs (sequences of proof lines).
All the proofs within a context (see section \ref{sec-cxt}) share the language,
the  axioms, and the computational model of the context.
However, proofs (in the same context) differ by their proof lines. A proof  line
belongs to one and only  one proof.
The proof you are working in is the {\em current  proof}.
When you enter the system the current proof does not contain proof lines and has
no name (the {\em un-named} proof).
If you want to leave the proof to build another one, you have to give it a name
(by the command {\tt nameproof}).
This allows us to refer to it later on.
You can create a new proof by using {\tt makeproof}, and  switch to it by using
{\tt switchproof}. The proof you switch to becomes then the current proof.


% user commands for multiple proofs
\gfcommand{cancel}{proof lines elimination}
\index{cancel}

\gfsyntax{
   cancel \OPT{\ARG{label}};
}

\gfdescription{
   Cancels all the proof lines in the current proof starting from 
   the one with label \ARG{label}. 
   If \ARG{label} is not specified then it cancels the last asserted 
   proof line.
}

\gfrecap{
Cancels all the proof lines in the current proof starting from the one
labelled `label'.
If `label' is not specified then it cancels the last asserted proof line.
}

\gfexample+
   ***** show proof;
   1   A     (1)
   2   B     (2)
   3   C     (3)
   4   D     (4)
   ***** cancel;
   1   A     (1)
   2   B     (2)
   3   C     (3)
   ***** cancel 2;
   1   A     (1)
+
\input{proof/copyproof}
\gfcommand{label}{naming proof lines}
\index{label}

\gfsyntax{
   label fact \ARG{sym};\\
   label fact \ARG{sym} = \ARG{label};
}

\gfdescription{
   Allows us to ``name" proof lines with desired {\GF} symbols.
   We will then be able to refer to the proof line with {\em sym} rather
   than with its ``hook" ({\em label}). 
   In the first form the next asserted proof line will get the name \ARG{sym}. 
   In the second form the proof line whose label is {\em label} gets
   the alternative name \ARG{sym}.
}

\gfrecap{
Names a proof line with a GETFOL symbol.
Named proof lines can be accessed through its name.
In the first form the next asserted proof line will get the name `sym'.
In the second form the proof line whose label is `label' gets the
alternative name `sym'
}


\gfexample+
   ***** declare sentconst A;
   ..............
   ***** show proof;
   ..............
   4   A or (not A) 
   ***** label fact ExclMiddle = 4;
   ***** ori ExclMiddle A;
   5   (A or (not A)) or A  
   ***** label fact AssumContr;
   ***** assume A and not A;
   6   A and (not A)     (6)
   ***** ande AssumContr 1;
   7   A     (6)
+
\gfcommand{makeproof}{create a new empty proof}
\index{makeproof}

\gfsyntax{
   makeproof \ARG{prf-name};
}

\gfdescription{
   A new empty proof with name \ARG{prf-name} is created.
}

\gfrecap{
A new empty proof with name prf-name is created.
}

\gfexample+
   ***** show whereami;
   You are now using an unnamed context.
   You are now using an unnamed proof.
   ***** nameproof P1;
   You have named the current proof: P1
   ***** show whereami;
   You are now using an unnamed context.
   You are now using the proof: P1
   ***** declare sentconst A;
   ***** assume A;
   1   A     (1)
   ***** makeproof P2;
   You have created the empty proof: P2
   ***** switchproof P2;
   You are now using the proof: P2
   ***** declare sentconst B;
   ***** assume A or B;
   1   A or B     (1)
   ***** switchproof P1;
   You are now using the proof: P1
   ***** assume B;
   2   B     (2)
   ***** andi 1 2;
   3   A and B     (1 2)
   ***** show proof;
   1   A     (1)
   2   B     (2)
   3   A and B     (1 2)
   ***** switchproof P2;
   You are now using the proof: P2
   ***** show proof;
   1   A or B     (1)
+

\gfcommand{nameproof}{name the current proof}
\index{nameproof}

\gfsyntax{
   nameproof \ARG{prf-name};
}

\gfdescription{
   If the current proof has no name, it  is named with \ARG{prf-name}.
}

\gfrecap{
If the current proof has no name, it is named with `prf-name'.
}

\gfexample+
   ***** show whereami;
   You are now using an unnamed context.
   You are now using an unnamed proof.
   ***** nameproof P1;
   You have named the current proof: P1
   ***** show whereami;
   You are now using an unnamed context.
   You are now using the proof: P1
   ***** declare sentconst A;
   ***** assume A;
   1   A     (1)
   ***** makeproof P2;
   You have created the empty proof: P2
   ***** switchproof P2;
   You are now using the proof: P2
   ***** declare sentconst B;
   ***** assume A or B;
   1   A or B     (1)
   ***** switchproof P1;
   You are now using the proof: P1
   ***** assume B;
   2   B     (2)
   ***** andi 1 2;
   3   A and B     (1 2)
   ***** show proof;
   1   A     (1)
   2   B     (2)
   3   A and B     (1 2)
   ***** switchproof P2;
   You are now using the proof: P2
   ***** show proof;
   1   A or B     (1)
+
\gfcommand{switchproof}{switch to another proof}
\index{switchproof}

\gfsyntax{
  switchproof \ARG{prf-name};
}

\gfdescription{
  Switches from the current  proof  to the proof \ARG{prf-name} which then becomes
  the current proof.
}

\gfrecap{
Switches from the current proof to the proof prf-name which then becomes
the current proof.
}

\gfexample+
   ***** show whereami;
   You are now using an unnamed context.
   You are now using an unnamed proof.
   ***** nameproof P1;
   You have named the current proof: P1
   ***** show whereami;
   You are now using an unnamed context.
   You are now using the proof: P1
   ***** declare sentconst A;
   ***** assume A;
   1   A     (1)
   ***** makeproof P2;
   You have created the empty proof: P2
   ***** switchproof P2;
   You are now using the proof: P2
   ***** declare sentconst B;
   ***** assume A or B;
   1   A or B     (1)
   ***** switchproof P1;
   You are now using the proof: P1
   ***** assume B;
   2   B     (2)
   ***** andi 1 2;
   3   A and B     (1 2)
   ***** show proof;
   1   A     (1)
   2   B     (2)
   3   A and B     (1 2)
   ***** switchproof P2;
   You are now using the proof: P2
   ***** show proof;
   1   A or B     (1)
+

\gfnotes{
  The command fails if the current proof has no name.
}


 	% introduction to the natural deduction's section
\newpage
\section{Natural Deduction (ND)}
\label{sec-nd}
\label{sec-nd-first}

\subsection{{\GF} logic}
\label{sec-ndrules}

{\GF} uses a Natural Deduction (ND) calculus based on Prawitz's system
defined in \cite{prawitz1}.
Some resemblances exist also to the ND calculus defined by Quine in
\cite{quine3}.
For various reasons ({\it e.g.} efficiency of the implementation, elegance of
the proof theory), {\GF} also carries around the dependencies of any derived
formula.
This allows to see the {\GF} logic as a sequent calculus, where a sequent is a
pair $(\Gamma, A)$, where $A$ is a formula and $\Gamma$ a set of formulas, with
introduction and elimination in the post sequent.
We claim that the ``{\em correct}'' way to see {\GF} logic is as a ND calculus.
This is why all the commands are described in ND style, without explicitly
writing the dependencies.

Some notes about the ND rules described in figure \ref{fig-nd}:

\begin{itemize}
\item
	The notation used is the same as in \cite{prawitz1}.
\item
	The $\forall I$ and $\exists E$ rules have the following restrictions:
	$a$ must not appear free in the dependencies of $A$ (for $\forall I$)
	and $a$ must not appear free in $\exists x A$, in $B$  or in any assumption
	on which the upper occurrence of B depends other than $A^{x}_{a}$ (for
	$\exists E$).
\item
	{\GF} natural deduction rules {\em discharge all occurrences}.
\item
	In {\GF} both the negation connective {\bf not} ($\neg$) and the falsity
	sentential constant {\tt FALSE} ($\bot$) are available.
	$\neg A$ and $A \imp \bot$ are logically equivalent.
	From a wff containing an occurrence of $A \imp \bot$ it is possible to
	deduce the wff with $\neg A$ in place of $A \imp \bot$ and viceversa.
	In {\GF}, this deduction can be performed in several ways.
	The simplest is to use the decider for propositional calculus ({\tt ptaut})
	(see section \ref{sec-decproc}).
	For instance:
	%
	\begin{verbatim}
		***** declare sentconst A;
		A has been declared to be a Sentconst
		***** assume A imp FALSE;
		1   A imp FALSE     (1)
		***** ptaut not A by 1;
		2   not A     (1)
		***** assume not A;
		3   not A     (3)
		***** ptaut A imp FALSE by 3;
		4   A imp FALSE     (3)
	\end{verbatim}
\end{itemize}


The ND notation is given for the primitive ND rules, although the ND commands
implement (with different options) both basic and more complex derived inference
rules.
A complete discussion of the capabilities of the commands is given in the
description of each command.

In the syntax for the commands we will use {\em fact}, {\em fact}$_1$,
{\em fact}$_2$, $\ldots$, in place of {\em label}, {\em label}$_1$,
{\em label}$_2$, $\ldots$ (see section \ref{sec-reas}).
However, the user refers to the fact by its label.

The {\GF} command \verb+existe+ implementing the exist elimination rule 
($\exists E$) is the most radically different from the formal statement given
in Prawitz's ND calculus.
The command has some resemblance with the ``exist instantiation'' defined in
\cite{quine3}.
The {\GF} $\exists E$ command has side effects that make all the other
rules change behavior.
This is explained, in more detail, in the description of the command
\verb+existe+ in section \ref{sec-nd-first}.


\renewcommand{\arraystretch}{0.5}
\begin{figure}[htbp]
\[
	\begin{array}{|c|c||c|c|} \hline
	\multicolumn{2}{|c||}{\mbox{\bf Introduction rules}} &
	\multicolumn{2}{c|}{\mbox{\bf Elimination rules}} \\ \hline
	& & &\\
	& & &\\
	\con I  & 
	\fraz{A \ \ B}
     {A \wedge B}
	&
	\con E  & 
	\fraz{A \con B}
     {A}
	\ \ 
	\fraz{A \con B}
	     {B}
	\\
	\begin{array}{l}
	\\ \\ \\ \\ \\
	\imp I   
	\end{array}
	& 
	\begin{array}{c}
	\\ \\
	{[A]}\\
	\vdots\\
	B\\
	\hline\\
	A \imp B
	\end{array}
	&
	\begin{array}{l}
	\\ \\ \\ \\ \\
	\imp E  
	\end{array}
	& 
	\begin{array}{c}
	\\ \\ \\ \\ \\
	\fraz{A \ \ A \imp B}
	     {B}
	\end{array}
	\\
	\begin{array}{l}
	\\ \\ \\ \\ \\
	\dis I
	\end{array}
	& 
	\begin{array}{c}
	\\ \\ \\ \\ \\
	\fraz{A}
	     {A \dis B}
	\end{array}
	\ \ 
	\begin{array}{c}
	\\ \\ \\ \\ \\
	\fraz{A}
  	   {B \dis A}
	\end{array}
	&
	\begin{array}{l}
	\\ \\ \\ \\ \\
	\dis E  
	\end{array}
	& 
	\begin{array}{ccc}
	\\ \\
	&{[A]}&{[B]}\\
	&\vdots&\vdots\\
	A \dis B & C & C\\
	\hline\\
	& C &
	\end{array}
	\\
	\begin{array}{l}
	\\ \\ \\ \\ \\
	\bot_c 
	\end{array}
	&
	%\renewcommand{\arraystretch}{0.5}
	\begin{array}{c}
	\\ \\
	{[\neg A]}\\
	\vdots\\
	\bot\\
	\hline\\
	A 
	\end{array}
	&
	\begin{array}{l}
	\\ \\ \\ \\ \\
	\bot_i  
	\end{array}
	& 
	\begin{array}{c}
	\\ \\ \\ \\ \\
	\fraz{\bot}
	     {A}
	\end{array}
	\\
	\begin{array}{l}
	\\ \\ \\ \\ \\
	\imp I_{\neg}  
	\end{array}
	& 
	\begin{array}{c}
	\\ \\
	{[A]}\\
	\vdots\\
	\bot\\
	\hline\\
	\neg A
	\end{array}
	&
	\begin{array}{l}
	\\ \\ \\ \\ \\
	\imp E_{\neg}  
	\end{array}
	& 
	\begin{array}{c}
	\\ \\ \\ \\ \\
	\fraz{A \ \ \neg A}
	     {\bot}
	\end{array}
	\\
	\begin{array}{l}
	\\ \\
	\liff  I  
	\end{array}
	& 
	\begin{array}{c}
	\\ \\ 
	\fraz{A \imp B \ \ B \imp A}
	     {A \liff B}
	\end{array}
	&
	\begin{array}{l}
	\\ \\
	\liff E  
	\end{array}
	& 
	\begin{array}{c}
	\\ \\
	\fraz{A \liff B}
	     {A \imp B}
	\end{array}
	\ \
	\begin{array}{c}
	\\ \\
	\fraz{A \liff B}
	     {B \imp A}
	\end{array}
	\\
	\begin{array}{l}
	\\ \\
	\forall  I  
	\end{array}
	& 
	\begin{array}{c}
	\\ \\
	\fraz{A}
	     {\forall x A^{a}_{x}}
	\end{array}
	&
	\begin{array}{l}
	\\ \\
	\forall E
	\end{array}
	& 
	\begin{array}{c}
	\\
	\fraz{\forall x A}
	     {A^{x}_{t}}
	\end{array}
	\\
	\begin{array}{l}
	\\ \\ \\ \\ \\
	\exists  I  
	\end{array}
	& 
	\begin{array}{c}
	\\ \\ \\ \\ \\
	\fraz{A}
	     {\exists x A^{t}_{x}}
	\end{array}
	&
	\begin{array}{l}
	\\ \\ \\ \\ \\
	\exists E 
	\end{array}
	&
	\begin{array}{cc}
	\\ \\
	&{[A^{x}_{a}]}\\
	&\vdots\\
	\exists x A & B\\
	\hline\\
	B
	\end{array}
	\\ 
	&&& \\ \hline
	\end{array}
	\]
\caption{ND inference rules}
\label{fig-nd}
\end{figure}

\renewcommand{\arraystretch}{1}


\subsection{{\GF} sorted logic}

In section \ref{sec-ndrules} we described the inference rules for the unsorted
logic. 
However, the sort information present in the language has to be taken into
account when performing deduction.
The only natural deduction rules that have to be modified are the quantifier
introduction and elimination rules, where the sort of variables and terms
involved in the deduction substantially changes the applicability of the rules
and the conclusion.

The sorted ND rules are defined as follows:

$$
\begin{array}{ll}
\forall E & \left\{
\begin{array}{cl}
\fraz{\forall x A}{A^{x}_{t}} &
\mbox{if $t$ is of sort $S_x$} \\
& \\
\fraz{\forall x A}{S_x(t)\imp A^x_t} & \mbox{otherwise}
\end{array}
\right. % } to balance
\\

& \\

\forall I & \ps
\begin{array}{cl}
\fraz{A}
     {\forall x A^{a}_{x}}
& \mbox{applicable only if $x$ is of sort $S_a$}
\end{array} \\

& \\

\exists I & \left\{
\begin{array}{cl}
\fraz{A}
     {\exists x A^{t}_{x}}
& \mbox{if $t$ is of sort $S_x$} \\
& \\
\fraz{A^x_t}{S_x(t)\imp\exists x A}
& \mbox{otherwise}
\end{array} 
\right. % } to balance
\\

& \\
\exists E & \ps
\fraz{
\begin{array}{cc}
&{[A^{x}_{a}]}\\
&\vdots\\
\exists x A & B\\
\end{array}}
{B}
\pps
\mbox{applicable only if $x$ is of sort $S_a$}

\end{array}
$$

where $S_x$ and $S_a$ stand for the sort of $x$ and the sort of $a$
respectively.
The restrictions on the unsorted $\forall I$ and $\exists E$  rules apply also
to the rules above. 


% user commands for nd
\gfcommand{assume}{derive an assumption}
\index{assume}

\gfsyntax{
   assume \ARG{wff1} \OPT{\OPT{,} \ARG{wff2} \SEQ};
}

\gfdescription{
   For each \ARG{wffI}, this command builds an assumption as a new proof line
   and asserts it in the proof.
   An assumption is a proof line that depends on itself.
}

\gfrecap{
For each formula in the line, assume builds an assumption as a new proof line
and asserts it in the proof.
An assumption is a proof line that depends on itself.
}
 
\gfexample+
   ***** declare sentconst A B;
   ***** assume A and B;
   1   A and B     (1)
   
   ***** assume A and B    A imp B;
   2   A and B     (2)
   3   A imp B     (3)
+

\gfcommand{ande}{and elimination rule}
\index{ande}\index{ae}

\gfsyntax{
   ande \ALT ae \ARG{fact} \OPT{,} 1 \ALT 2; \\
   ande \ALT ae \ARG{fact} \OPT{,} 1 \ALT 2 1 \ALT 2 \SEQ;
}

\gfdescription{
   \[
      \con E  \ \ 
      \fraz{A \con B}
           {A}
      \ \ 
      \fraz{A \con B}
           {B}
   \]

   The wff of \ARG{fact} must be a conjunction.
   A proof line is deduced whose wff is the left conjunct (if 1 \ALT 2 is 1)
   or the right conjunct (if 1 \ALT 2 is 2). The fact inherits \ARG{fact}'s
   dependencies.

   In the second form, the command is applied to a fact whose wff is a
   recursive conjunction of wffs, {\it e.g.} {\tt A and ((B and C) and D)}.
   The sequence ``1 \ARG 2 1 \ARG 2 \SEQ" picks up the appropriate subformula.
}

\gfrecap{
Applies and elimination rule.
The formula given as arguments must be a conjunction.
A proof line is deduced whose wff is the left conjunct (if the number is 1)
or the right conjunct (if the number is 2). The fact inherits
the dependencies of `fact'.
In the second form, the command is applied to a fact whose wff is a
recursive conjunction of wffs, eg. `A and ((B and C) and D)'.
The sequence `1 | 2 1 | 2 ...' picks up the appropriate subformula.
}

\gfexample+
   ***** declare sentconst A B C D;
   ***** assume A and ((B and C) and D);
   1  A and ((B and C) and D)  (1)

   ***** ande 1 1;
   2   A  (1)

   ***** ande 1 2 1;
   3  B and C  (1)

   ***** ande 1 2 1 2;
   4  C  (1)
+

\gfnotes{
   A proof line derived by executing the second form can always be derived by
   a sequence of executions of commands in the first form.
}

\gfcommand{andi}{and introduction rule}
\index{andi}\index{ai}

\gfsyntax{
   andi \ALT ai \ARG{fact1} \OPT{,} \ARG{fact2}; \\
   andi \ALT ai \ARG{fact11}
                \OPT{conj \ALT cj \ARG{fact12} conj \ALT cj \ARG{fact13} \SEQ}
                \OPT{,}
                \ARG{fact21}
                \OPT{conj \ALT cj \ARG{fact22} conj \ALT cj \ARG{fact23} \SEQ};
}

\gfdescription{
   \[
   \con I \ \
   \fraz{A \ \ B}
        {A \wedge B}
   \]

   A proof line is derived whose wff is the conjunction of the wffs
   of \ARG{fact1} and \ARG{fact2} and whose dependencies are the union
   of the dependencies of \ARG{fact1} and \ARG{fact2}.

   In the second form we have ``conjunctions of facts'' rather than facts.
   A ``conjunction of facts'' is any parenthesized conjunctive expression in
   which all conjuncts are facts (\ARG{fact11} {\tt conj} \ARG{fact12} \SEQ).
   A proof line is derived whose wff is the conjunction of each $fact_{ij}$'s
   wff and whose dependencies are the union of the dependencies each
   $fact_{ij}$ depends on.
}

\gfrecap{
Applies `and' introduction to the given arguments.
In its first from a proof line is derived whose formula is the conjuction of
the formulae of `fact1' and `fact2' and whose dependencies are the union of
the dependencies of `fact1' and `fact2'.
In the second form we have ``conjunctions of facts" rather than facts.
A "conjunction of facts" is any parenthesized conjunctive expression in
which all conjuncts are facts (`fact11 conj fact12 ...').
A proof line is derived whose wff is the conjunction of each `factIJ''s
wff and whose dependencies are the union of the dependencies each
`factIJ' depends on.
}


\gfexample+
   ***** declare sentconst A B C D E;
   ***** assume A B;
   1   A     (1)
   2   B     (2)

   ***** andi 1 2;
   3   A and B     (1 2)

   ***** assume C D E;
   4   C     (4)
   5   D     (5)
   6   E     (6)

   ***** andi 1 conj 2   3;
   7   (A and B) and (A and B)     (1 2)

   ***** andi 1 conj 2   3 conj 4;
   8   (A and B) and ((A and B) and C)     (1 2 4)

   ***** andi 1 conj 2 conj 3   4;
   9   (A and (B and (A and B))) and C     (1 2 4)
+

\gfnotes{
   A proof line derived by executing the second form can always be derived by
   a sequence of executions of the command in the first form.
}

\gfcommand{falsee}{ falsity rule in intuitionistic logic}
\index{falsee}\index{fe}

\gfsyntax{
   falsee \ALT fe \ARG{fact1} \OPT{,} \ARG{wff};\\
   falsee \ALT fe \ARG{fact1} \OPT{,} \ARG{fact2};
}

\gfdescription{
   \renewcommand{\arraystretch}{0.5}
   \[
   \begin{array}{l}
      \bot_i  
   \end{array}
   \ \ 
   \begin{array}{c}
      \fraz{\bot}
           {A}
   \end{array}
   \]
   \renewcommand{\arraystretch}{1}

   \ARG{fact1} must have wff {\tt FALSE}.
   In the first form, the wff derived is \ARG{wff} and its dependencies are 
   those of \ARG{fact1}.
   In the second form, the wff of \ARG{fact2} is derived.
}

\gfrecap{
   `fact1' must have wff `FALSE'.
   In the first form, the wff derived is `wff' and its dependencies are 
   those of `fact1'.
   In the second form, the wff of `fact2' is derived.
}

\gfexample+
   ***** declare sentconst A;
   [...]

   ***** assume FALSE;
   1  FALSE  (1)

   ***** falsee 1 A and not A;
   2  A and (not A)  (1)

   ***** reset;
   [...]

   ***** declare sentconst A;
   [...]

   ***** assume not not A not A A;
   1  not not A  (1)
   2  not A  (2)
   3  A  (3)

   ***** falsei 1 2;
   4  FALSE  (1 2)

   ***** falsee 4 3;
   5  A  (1 2);

   ***** impi 1 5;
   6  not not A imp A (2)
+

\gfnotes{
   This rule says that anything follows from a contradiction.
}

\gfcommand{falsei}{implication elimination for negation or false introduction}
\index{falsei}\index{fi}

\gfsyntax{
   falsei \ALT fi \ARG{fact1} \OPT{,} \ARG{fact2}; 
}

\gfdescription{
   \renewcommand{\arraystretch}{0.5}
   \[
   \begin{array}{l}
      \imp E_{\neg}  
   \end{array}
   \ \ 
   \begin{array}{c}
      \fraz{A \ \ \neg A}
           {\bot}
   \end{array}
   \]
   \renewcommand{\arraystretch}{1}

   One fact must be the negation of the other. 
   The wff derived is {\tt FALSE} and its dependencies are the union of the
   dependencies of both facts.
}

\gfrecap{
   One fact must be the negation of the other. 
   The wff derived is `FALSE' and its dependencies are the union of the
   dependencies of both facts.
}

\gfexample+
   ***** declare sentconst A;
   [...]

   ***** assume A not A;
   1  A  (1)
   2  not A  (2)

   ***** falsei 1 2;
   3  FALSE  (1 2)

   ***** falsei 2 1;
   4  FALSE  (1 2)
+

\gfnotes{
   This rule can be seen as a special case of introduction elimination
   in which the main symbol of one of the premises must be {\tt not} ($\neg$)
   rather than {\tt imp} ($\imp$). 
}

\gfcommand{iffe}{equivalence elimination}.
\index{iffe}\index{ie}

\gfsyntax{
   iffe \ALT ie \ARG{fact} \OPT{,} 1 \ALT 2;
}

\gfdescription{
   \renewcommand{\arraystretch}{0.5}
   \[
   \begin{array}{l}
      \liff E
   \end{array}
   \ \ 
   \begin{array}{c}
      \fraz{A \liff B}
           {A \imp B}
   \end{array}
   \ \ \ 
   \begin{array}{c}
      \fraz{A \liff B}
           {B \imp A}
   \end{array}
   \]
   \renewcommand{\arraystretch}{1}

   If \ARG{fact}'s wff is of the form {\tt A iff B} then the first alternative
   gives {\tt A imp B}, the second {\tt B imp A}. The dependencies are those 
   of \ARG{fact}.
}

\gfrecap{
   If the formula of `fact' is of the form `A iff B' then the first alternative
   gives `A imp B', the second `B imp A'.
   The dependencies are those of `fact'.
}

\gfexample+
   ***** declare sentconst A;
   [...]

   ***** assume A iff not not A;
   1  A iff not not A  (1)

   ***** iffe 1 1;
   2  A imp not not A  (1)

   ***** iffe 1 2;
   2  not not A imp A  (1)
+

\gfcommand{iffi}{equivalence introduction}
\index{iffi}\index{ii}

\gfsyntax{
   iffi \ALT ii \ARG{fact1} \OPT{,} \ARG{fact2};
}

\gfdescription{
   \renewcommand{\arraystretch}{0.5}
   \[
   \begin{array}{l}
      \liff  I  
   \end{array}
   \ \ 
   \begin{array}{c}
      \fraz{A \imp B \ \ B \imp A}
           {A \liff B}
   \end{array}
   \]
   \renewcommand{\arraystretch}{1}

   Both facts must be implications. 
   If \ARG{fact1}'s wff is of the form {\tt A imp B}, then \ARG{fact2}'s wff 
   must be {\tt B imp A}.
   The conclusion is {\tt A iff B} with the union of the dependencies of 
   \ARG{fact1} and \ARG{fact2}.
}

\gfrecap{
Both facts must be implications. 
If the formula of `fact1' is of the form `A imp B', then the formula of
`fact2' must be `B imp A'.
The conclusion is {\tt A iff B} with the union of the dependencies of 
`fact1' and `fact2'.
}
   
\gfexample+
   ***** declare sentconst A B;
   [...]

   ***** assume A imp B B imp A;
   1  A imp B  (1)
   2  B imp A  (2)

   ***** iffi 1 2;
   3  A iff B  (1 2);

   ***** iffi 2 1;
   4  B iff A  (1 2)

   ***** reset;
   [...]

   ***** declare sentconst A;
   [...]

   ***** assume FALSE imp A A imp FALSE;
   1  FALSE imp A  (1)
   2  A imp FALSE  (2)

   ***** iffi 1 2;
   3  FALSE iff A  (1 2);
+

\gfcommand{impe}{implication elimination rule}
\index{impe}\index{mp}

\gfsyntax{
   impe \ALT mp \ARG{fact1} \OPT{,} \ARG{fact2};
}

\gfdescription{
   \renewcommand{\arraystretch}{0.5}
   \[
   \begin{array}{l}
      \imp E  
   \end{array}
   \ \  
   \begin{array}{c}
      \fraz{A \ \ A \imp B}
           {B}
   \end{array}
   \]
   \renewcommand{\arraystretch}{1}

   One of the two facts must be an implication and the other one
   must be its hypothesis.
   The order of the arguments is not relevant.
   The command creates a proof line whose wff is the conclusion of the
   implication and whose dependencies list is the union of the dependencies of
   \ARG{fact1} and \ARG{fact2}.
}

\gfrecap{
One of the two facts must be an implication and the other one must be its
hypothesis.
The order of the arguments is not relevant.
The command creates a proof line whose wff is the conclusion of the implication
and whose dependencies list is the union of the dependencies of `fact1' and
`fact2'.
}


\gfexample+
   ***** declare sentconst A B;
   ***** assume A A imp B;
   1  A  (1)
   2  A imp B  (2)

   ***** impe 1 2;
   3  B  (1 2)

   ***** impe 2 1;
   4  B  (1 2)
+

\gfcommand{impi}{implication introduction rule}
\index{impi}\index{ded}

\gfsyntax{
   impi \ALT ded  \ARG{fact1} \OPT{, \ALT imp} \ARG{fact};\\
   impi \ALT ded  \ARG{wff} \OPT{, \ALT imp} \ARG{fact};
}


\gfdescription{
   \renewcommand{\arraystretch}{0.5}
   \[
   \begin{array}{l}
      \\ \\ \\
      \imp I   
   \end{array}
   \ \  
   %
   \begin{array}{c}
      {[A]}\\
      \vdots\\
      B\\
      \hline\\
      A \imp B
   \end{array}
   \]
   \renewcommand{\arraystretch}{1}

   The wff derived is the implication of the wffs of \ARG{fact1} and
   \ARG{fact2}. It depends on all the dependencies of \ARG{fact2} less 
   {\em all the lines} whose wff is the same as \ARG{fact1}'s wff (in the 
   first form) or \ARG{wff} (in the second form).
}

\gfrecap{
The wff derived is the implication of the wffs of `fact1' and `fact2'.
It depends on all the dependencies of `fact2' less all the lines whose wff is
the same as `fact1''s wff (in the first form) or `wff' (in the second form).
}
   
\gfexample+
   ***** declare sentconst A;
   ***** assume A;
   1  A  (1)

   ***** impi 1 1;
   2  A imp A  

   ***** impi A 1;
   3  A imp A 
+

\gfcommand{note}{falsity rule in classical logic}.
\index{note}\index{ne}

\gfsyntax{
   note \ALT ne \ARG{fact1} \OPT{,} \ARG{wff}; \\
   note \ALT ne \ARG{fact1} \OPT{,} \ARG{fact2};
}

\gfdescription{
   \renewcommand{\arraystretch}{0.5}
   \[
   \begin{array}{l}
      \\ \\ 
      \bot_c 
   \end{array}
   \ \ 
   \begin{array}{c}
      {[\neg A]}\\
      \vdots\\
      \bot\\
      \hline\\
      A 
   \end{array}
   \]
   \renewcommand{\arraystretch}{1}

   The wff of \ARG{fact1} must be {\tt FALSE} and \ARG{wff} must be a negation
   of the form $\neg A$.
   The wff derived is $A$.
   It depends on all the dependencies of \ARG{fact1} less the dependencies of
   all those facts whose wff is equal to \ARG{wff} ($\neg A$) (in the first) 
   form or \ARG{fact1}'s wff (in the second form).
}

\gfrecap{
   The wff of `fact1' must be `FALSE' and `wff' must be a negation
   of the form `not A'.
   The wff derived is `A'.
   It depends on all the dependencies of `fact1' less the dependencies of
   all those facts whose wff is equal to `wff' (`not A') (in the first) 
   form or the formula of `fact1' (in the second form).
}

\gfexample+
   ***** declare sentconst A;
   [...]

   ***** assume not A not not A;
   1   not A     (1)
   2   not (not A)     (2)

   *****  falsei 1 2;
   3   FALSE     (1 2)

   *****  note 3 not A;
   4   A     (2)

   *****  impi 2 4;
   5   (not (not A)) imp A 
+

\gfnotes{
   This rule implements ``reductio ad absurdum''. 
   Any proof using only the ND commands but not this rule is valid
   intuitionistically. 
}

\gfcommand{noti}{implication introduction for negation}
\index{noti}\index{ni}

\gfsyntax{
   noti \ALT ni \ARG{fact1} \OPT{,} \ARG{wff}; \\
   noti \ALT ni \ARG{fact1} \OPT{,} \ARG{fact2};
}

\gfdescription{
   \renewcommand{\arraystretch}{0.5}
   \[
   \begin{array}{l}
      \\ \\ 
      \imp I_{\neg}  
   \end{array}
   \ \ 
   \begin{array}{c}
      {[A]}\\
      \vdots\\
      \bot\\
      \hline\\
      \neg A
   \end{array}
   \]
   \renewcommand{\arraystretch}{1}

   The wff of \ARG{fact1} must be {\tt FALSE}.
   The command creates a fact whose wff is the negation of \ARG{wff}.
   It depends on all the dependencies of \ARG{fact1} less all the dependencies
   whose wff is equal to \ARG{wff} (in the first form) or \ARG{fact2}'s wff
   (in the second form).
}

\gfrecap{
   The wff of `fact1' must be `FALSE'.
   The command creates a fact whose wff is the negation of `wff'.
   It depends on all the dependencies of `fact1' less all the dependencies
   whose wff is equal to `wff' (in the first form) or the formula of 
   `fact2' (in the second form).
}
   
\gfexample+
   ***** declare sentconst A;
   [...]

   ***** assume A not A;
   1  A  (1)
   2  not A  (2)

   *****  falsei 1 2;
   3  FALSE  (1 2)

   *****  noti 3 not A;
   4  not not A  (1)

   *****  impi 1 4;
   5  A imp not not A
+

\gfnotes{
   Since $\neg A$ ai equivalent to $A \imp \bot$,
   this rule can be seen as a special case of the implication introduction
   rule ({\tt impi}), in which  the asserted line has \verb+not+ ($\neg$) as
   main symbol.
}

\gfcommand{ore}{or elimination rule}.
\index{ore}\index{oe}

\gfsyntax{
   ore \ALT oe \ARG{fact1} \OPT{,} \ARG{fact2} \OPT{,} \ARG{fact3};
}

\gfdescription{
   \renewcommand{\arraystretch}{0.5}
   \[
   \begin{array}{l}
      \\ \\ 
      \dis E  
   \end{array}
   \ \ 
   \begin{array}{ccc}
      &{[A]}&{[B]}\\
      &\vdots&\vdots\\
      A \dis B & C & C\\
      \hline\\
      & C &
   \end{array}
   \]
   \renewcommand{\arraystretch}{1}

   Let \ARG{wff1}, \ARG{wff2} and \ARG{wff3} be the formulas of \ARG{fact1}, 
   \ARG{fact2}, \ARG{fact3} respectively; let {\em wff1} be a disjunction;
   let \ARG{wff2} and \ARG{wff3} be the same formula.
   Then the conclusion is \ARG{wff2} (\ARG{wff3}) with the dependencies of
   \ARG{fact1} along with those of \ARG{fact2} whose wff is not equal to the
   left disjunct of \ARG{wff1} and those of \ARG{fact3} whose wff is not equal
   to the right disjunct of \ARG{wff1}. 
}

\gfrecap{
Let `wff1', `wff2' and `wff3' be the formulas of `fact1', 
`fact2', `fact3' respectively; let {\em wff1} be a disjunction;
let `wff2' and `wff3' be the same formula.
Then the conclusion is `wff2' (`wff3') with the dependencies of
`fact1' along with those of `fact2' whose wff is not equal to the
left disjunct of `wff1' and those of `fact3' whose wff is not equal
to the right disjunct of `wff1'. 
}

\gfexample+
   ***** declare sentconst A B C;
   [...]

   ***** assume B imp A;
   1  B imp A (1)

   ***** assume C imp A;
   2  C imp A (2);

   ***** assume B;
   3  B  (3);

   ***** assume C;
   4  C  (4);

   ***** impe 3 1;
   5  A  (1 3)

   ***** impe 4 2;
   6  A  (2 4)

   ***** assume B or C;
   7  B or C  (7)

   ***** ore 7 5 6;
   8  A   (1 2 7)

   ***** impi 7 8;
   9  (B or C) imp A   (1 2)

   ***** ore 7 6 5;
   10  A   (1 2 3 4 7)
+

\gfcommand{ori}{or introduction rule}
\index{ori}\index{oi}

\gfsyntax{
   ori \ALT oi \ARG{fact} \OPT{,} \ARG{wff} \OPT{,} \OPT{lr \ALT rl};\\
   ori \ALT oi \ARG{fact} \OPT{,} \ARG{fact1} \ALT \ARG{wff1} 
       disj \ALT dj \ARG{fact2} \ALT \ARG{wff2} disj \ALT dj \SEQ
       \OPT{,} \OPT{lr \ALT rl};
}

\gfdescription{
   \renewcommand{\arraystretch}{0.5}
   \[
   \begin{array}{l}
      \dis I
   \end{array}
   \ \ 
   \begin{array}{c}
   \fraz{A}
        {A \dis B}
   \end{array}
   \ \ \ 
   \begin{array}{c}
   \fraz{A}
        {B \dis A}
   \end{array}
   \]
   \renewcommand{\arraystretch}{1}

   The command creates a new fact whose wff is the disjunction of \ARG{fact}'s
   wff and \ARG{wff}.
   The option {\tt lr \ALT rl} specifies the order of the disjuncts:
   {\bf lr} stands for ``left-right'' and it means that the left disjunct is 
   \ARG{fact}'s wff and the right one is \ARG{wff}; {\tt rl} viceversa.
   If no order is specified, then {\tt lr} is the default.
   The new proof line inherits \ARG{fact}'s dependencies.

   In the second form, the command accepts ``disjunctions of facts and wffs''
   as second argument.
   A ``disjunctions of facts and wffs'' is any parenthesized disjunctive
   expression in which all disjuncts are facts (\ARG{fact1} {\tt disj} 
   \ARG{fact2} \SEQ), wffs (\ARG{wff1} {\tt disj} \ARG{wff2} \SEQ)
   or facts and wffs (\ARG{fact1} {\tt disj} \ARG{wff2} \SEQ).
   The derived formula is the disjunction of the formulae.
   It depends on the assumptions \ARG{fact} and  all the \ARG{factI}s 
   depend on.
}

\gfrecap{
The command creates a new fact whose wff is the disjunction of `fact''s
formula and `wff'.
The option `lr \ALT rl' specifies the order of the disjuncts:
`lr' stands for ``left-right" and it means that the left disjunct is 
the formula of `fact' and the right one is `wff'; `rl' viceversa.
If no order is specified, then `lr' is the default.
The new proof line inherits dependencies of `fact'.
In the second form, the command accepts ``disjunctions of facts and wffs"
as second argument.
A ``disjunctions of facts and wffs" is any parenthesized disjunctive
expression in which all disjuncts are facts (`fact1 disj fact2 ...), 
wffs (wff1 disj wff2 ...) or facts and wffs (fact1 disj wff2 ...).
The derived formula is the disjunction of the formulae.
It depends on the assumptions `fact' and  all the `factI's 
depend on.
}

\gfexample+
   ***** declare sentconst A B;
   ...

   ***** assume A;
   1  A  (1);
   ***** ori 1 B;
   2  A or B  (1)
   ***** ori 1 B lr ;
   3  A or B (1)
   ***** ori 1 B rl ;
   4  B or A  (1)

   ***** reset;
   ***** declare sentconst A B C D E;
   ...

   ***** assume A B C;
   1  A  (1);
   2  B  (2);
   3  C  (3);

   ***** ori 1 B dj C rl;
   4  (B or C) or A   (1)

   ***** ori 1 2 dj 3;
   5  A or (B or C)   (1)

   ***** ori 1 2 dj D dj 3 dj E rl;
   6  (B or (D or (C or E))) or A   (1)
+

\gfnotes{
   A proof line derived by executing the second form can always be derived by
   a sequence of executions of the command in the first form.
}

\gfcommand{alle}{universal quantification elimination rule}
\index{alle}\index{us}

\gfsyntax{
   alle \ALT us \ARG{fact} \OPT{,} \ARG{term1} \ARG{term2} \SEQ;
}

\gfdescription{
   \renewcommand{\arraystretch}{0.5}
   \[
   \begin{array}{l}
      \\
      \forall E
   \end{array}
   \ \ 
   \begin{array}{c}
      \fraz{\forall x A}
           {A^{x}_{t}}
   \end{array}
   \]
   \renewcommand{\arraystretch}{1}

   The command uses the terms in \ARG{term1} \ARG{term2} \SEQ to instantiate
   the universally quantified variables in the order in which they appear.
   One execution of the command can instantiate more than one universally 
   quantified variable.
   If a particular term is not free for the variable to be instantiated, a 
   bound variable change is made and then the substitution is made.
   The dependencies are those of \ARG{fact}.

   In the sorted rule, let \verb+forall x. A(x)+ be the formula to be 
   instantiated, and let us suppose first that only a term {\tt t} is provided.
   If {\tt t} has he same sort as {\tt x}, say {\tt Sx}, then the resulting 
   formula is {\tt A(t)}; otherwise the resulting formula is
   {\tt Sx(t) imp A(t)}.
}

\gfrecap{
The command uses the terms in `term1' `term2' ... to instantiate
the universally quantified variables in the order in which they appear.
One execution of the command can instantiate more than one universally 
quantified variable.
If a particular term is not free for the variable to be instantiated, a 
bound variable change is made and then the substitution is made.
The dependencies are those of `fact'.
In the sorted rule, let `forall x. A(x)' be the formula to be 
instantiated, and let us suppose first that only a term `t' is provided.
If `t' has he same sort as `x', say `Sx', then the resulting 
formula is `A(t)'; otherwise the resulting formula is
`Sx(t) imp A(t)'.
}  

\gfexample+
   ***** declare predconst P 2;
   ***** declare indvar x y;
   ***** declare indconst c1 c2;
   [...]

   ***** assume forall x y. P(x y);
   1   forall x y.P(x,y)     (1)

   ***** alle 1 c1;
   2   forall y.P(c1,y)     (1)

   ***** alle 1 x c1;
   3   P(x,c1)     (1)

   ***** alle 1 c1 c2;
   4   P(c1,c2)     (1)
   
   ***** declare predconst P 2;
   ***** declare indvar x [Sx];
   ***** declare indvar y [Sy];
   ***** declare indconst c1 c2 [S];
   [...]

   ***** moregeneral Sx < S >;

   ***** assume forall x y. P(x y);
   1   forall x y.P(x,y)     (1)

   ***** alle 1 c1;
   2   forall y.P(c1,y)     (1)

   ***** alle 2 c1;
   3   Sy(c1) imp P(c1,c1)     (1)

   ***** alle 1 c1 c2;
   4   Sy(c2) imp P(c1,c2)     (1)
+

\gfnotes{
   The rule is sometimes called ``universal specialization''.
}

\gfcommand{alli}{universal quantification introduction rule}
\index{alli}\index{ug}

\gfsyntax{
   alli \ALT ug \ARG{fact} \OPT{\OPT{,} \ARG{indvar1} \ALT \ARG{indpar1} :} 
                           \ARG{indvar11}
                           \OPT{\OPT{,} \ARG{indvar2} \ALT \ARG{indpar2} :}
                           \ARG{indvar22} \SEQ;
}

\gfdescription{
   \renewcommand{\arraystretch}{0.5}
   \[
   \begin{array}{l}
      \forall  I  
   \end{array}
   \ \ 
   \begin{array}{c}
      \fraz{A}
      {\forall x A^{a}_{x}}
   \end{array}
   \]
   \renewcommand{\arraystretch}{1}

   There is the usual ND restriction on the application of this rule, namely
   the newly quantified variable must not appear free in any of the 
   dependencies of \ARG{fact}.

   Several simultaneous universal generalizations on \ARG{fact}'s wff
   can be carried out with this command.
   Each element of the substitution list may be either an individual variable
   (e.g. {\tt x}) or a pair.
   The pair elements are two individual variables ({\tt y:x}) or an individual
   parameter and an individual variable ({\tt a:x}).
   For each element in the substitution list a new universal quantifier is put
   at the front of \ARG{fact}'s wff. 
   In the case of pairs of the kind {\tt a:x},  the variable {\tt x} is
   substituted for all occurrences of the individual parameter {\tt a}
   which are not within the scope of the universal quantification of {\tt x}. 
   The individual parameter {\tt a} must not occur within the scope of a
   quantifier binding {\tt x}, otherwise an error message is returned. 

   The dependencies of the new created proof line are the same as those of
   \ARG{fact}.

   The rule with sorts must also satisfy the following restriction.
   Let us first consider a substitution list whose only substitution is
   {\tt a:x} or {\tt y:x}, where {\tt x} is a variable of sort {\tt Sx}.
   Then the sorted rule is applicable only if {\tt a} and {\tt y} are terms
   of sort weakly more general than {\tt Sx}.
   In the general case, the specified condition must hold for all the
   substitutions of the substitution list.
}

\gfrecap{
   Applies introduction of universal quantification.
   The usual natural deduction's restrictions apply.
}

\gfexample+
   ***** comment ! An example with generalization from individual parameters
                   to quantified variables !
        
   ***** declare predconst P 1;
   ***** declare indvar x;
   ***** declare indpar a;
   [...]

   ***** assume P(a);
   1   P(a)     (1)
   ***** alli 1 a:x;
   alli 1 a:x;
   Some variables appear free in the assumptions.
   ***** impi 1 1;
   2   P(a) imp P(a)
   ***** alli 2 a:x;
   3   forall x.(P(x) imp P(x))
   ***** alli 2 x;
   4   forall x.(P(a) imp P(a))
    
     
   ***** comment ! The same example when substituting more than one variable !
     
   ***** reset;
   ***** declare predconst Q 2;
   ***** declare indvar x y;
   ***** declare indpar a b;
   [...]

   ***** assume Q(a b);
   1   Q(a,b)     (1)
   ***** impi 1 1;
   2   Q(a,b) imp Q(a,b)
   ***** alli 2 a:x b:y;
   3   forall x y. (Q(x,y) imp Q(x,y))


   ***** comment ! An example with generalization from free variables
                   to quantified variables !

   ***** reset;
   ***** declare predconst P 1;
   ***** declare indvar x;
   [...]

   ***** assume P(x);
   1   P(x)     (1)
   ***** impi 1 1;
   2   P(x) imp P(x)
   ***** alli 2 x:x;
   3   forall x.(P(x) imp P(x))
   ***** alli 2 x;
   4   forall x.(P(x) imp P(x))
     

   ***** comment ! An example of non applicability of generalization from
                   individual parameters to quantified variables with sorts !
      
   ***** declare predconst P 1;
   ***** declare indvar x [Sx];
   ***** declare indpar a [Sa];
   [...]

   ***** assume P(a);
   1   P(a)     (1)

   ***** alli 1 a:x;
   alli 1 a:x;
   Some variables appear free in the assumptions.

   ***** impi 1 1;
   2   P(a) imp P(a)

   ***** alli 2 a:x;
   alli 2 a:x
   A <var> cannot be replaced by one with more general sort

   ***** moregeneral Sa < Sx >;

   ***** alli 2 a:x;
   3   forall x. (P(x) imp P(x))
+


\gfnotes{
   The rule is often called ``universal generalization".
}

\gfcommand{existe}{existential quantification elimination rule}
\index{existe}\index{es}

\gfsyntax{
   existe \ALT es \ARG{fact}  \OPT{,} \ARG{indvar1} \ALT \ARG{indpar1}
                  \OPT{,} \ARG{indvar2} \ALT \ARG{indpar2} \SEQ;
}


\gfdescription{
   \renewcommand{\arraystretch}{0.5}
   \[
   \begin{array}{l}
      \\ \\ \\ \\
      \exists E 
   \end{array}
   \ \ 
   \begin{array}{cc}
      \\ \\
      &{[A^{x}_{a}]}\\
      &\vdots\\
      \exists x A & B\\
      \hline\\
      B
   \end{array}
   \]
   \renewcommand{\arraystretch}{1}

   The implementation of this rule is the most radically different
   from the formal statement given in Prawitz's ND. This rule corresponds, in
   informal reasoning, to the following kind of argument:
   suppose we have shown that something exists with some particular
   property, e.g. $\exists y P(a,y)$. Then we say ``call this thing $b$''.
   This is like saying assume $P(a,b)$. Then we can reason about $b$.
   As soon as we have a sentence, however, that no longer mentions $b$,
   it does not depend on what we called ``$y$'',
   but only on the dependencies of the existential statement we started
   with. Thus we can discharge $P(a,b)$ from the dependencies
   and replace them with those of $\exists y P(a,y)$. {\GF} thus makes the 
   correct assumption for you, remembers it and automatically removes it at
   the first legitimate opportunity.

   The only difference with sorts is the following:
   the existentially quantified variable must be substituted by a variable
   or a parameter of weakly more general sort.

   In the example below, an existential elimination is done creating step 
   {\tt 6}.
   This fact actually has as {\tt reason}  (see
   subsection~\ref{sec-proofline}) that it was assumed.
   Fact {\tt 8} thus depends on {\tt 6}. When the existential generalization
   is done on the next fact, {\tt b} no longer appears and so fact {\tt 6}
   is removed from the dependencies of fact {\tt 9}. The user should
   convince himself that the {\GF} logic is equivalent to the ND definition
   given at the beginning of the section.
}

\gfrecap{
The implementation of this rule is the most radically different
from the formal statement given in Prawitz's ND.
This rule corresponds, in informal reasoning, to the following kind of 
argument: suppose we have shown that something exists with some particular
property, eg. `exists y P(a,y)'.
Then we say ``call this thing `b''': this is like saying assume `P(a,b)'.
Then we can reason about `b'.
As soon as we have a sentence, however, that no longer mentions `b',
it does not depend on what we called `y', but only on the dependencies of the
existential statement we started with.
Thus we can discharge `P(a,b)' from the dependencies and replace them with 
those of `exists y P(a,y)'.
GETFOL thus makes the correct assumption for you, remembers it and 
automatically removes it at the first legitimate opportunity.
The only difference with sorts is the following: the existentially quantified
variable must be substituted by a variable or a parameter of weakly more
general sort.
}

\gfexample+
   ***** comment ! Sorted example !
     
   ***** declare indvar x y;
   ***** declare indpar a b;
   ***** declare predconst P 2;
   [...]

   ***** assume forall x.exists y.P(x y) and forall x y.(P(x y) imp P(y x));
   1   forall x.exists y.P(x,y) and forall x y.(P(x,y) imp P(y,x))     (1)

   ***** ande 1 1;
   2   forall x.exists y.P(x,y)     (1)

   ***** ande 1 2;
   3   forall x y.(P(x,y) imp P(y,x))     (1)

   ***** alle 2 a;
   4   exists y.P(a,y)     (1)

   ***** alle 3 a b;
   5   P(a,b) imp P(b,a))     (1)

   ***** existe 4 b;
   6   P(a,b)     (6)

   ***** impe 6 5;
   7   P(b,a)     (1 6)

   ***** andi 6 7;
   8   P(a,b) and  P(b,a)     (1 6)

   ***** existi 8 b:y;
   9   exists y.(P(a,y) and P(y,a)     (1)

   ***** alli 9 a:x;
   10   forall x.exists y.(P(x,y) and P(x,x)     (1)

   ***** impi 1 10;
   11   forall x.exists y.P(x,y) and 
        forall x y.(P(x,y) imp P(y,x)) imp 
        forall x.exists y.(P(x,y) and P(x,y)
    
       
   ***** comment ! Sorted example !
    
   ***** reset;
   [...]

   ***** declare indvar x [Sx];
   ***** declare indpar p [S];
   ***** declare predconst A 1;
   ***** declare sentconst B;
   [...]

   ***** axiom ONE: exists x.A(x);
   ***** axiom TWO: A(p) imp B;
   ONE : exists x. A(x)
   TWO : A(p) imp B

   ***** existe ONE p;
   A <var> must be replaced by one with more general sort;

   ***** moregeneral S < Sx >;

   ***** existe ONE p;
   1   A(p)     (1)

   ***** impe TWO 1;
   2   B
+

\gfnotes{
   The rule is often called ``existential instantiation".
}
\gfcommand{existi}{existential quantification introduction rule}
\index{existi}\index{ei}

\gfsyntax{
  existi \ARG{fact}
   \OPT{\OPT{,} \ARG{term1} :} \ARG{indvar1} \OPT{occ \ARG{n11} \ARG{n12} \SEQ}
   \OPT{\OPT{,} \ARG{term2} :} \ARG{indvar2} \OPT{occ \ARG{n21} \ARG{n22} \SEQ}
   \SEQ;
}


\gfdescription{
  \renewcommand{\arraystretch}{0.5}
  \[
  \begin{array}{l}
    \exists  I  
  \end{array}
  \ \ 
  \begin{array}{c}
    \fraz{A}
    {\exists x A^{t}_{x}}
  \end{array}
  \]
  \renewcommand{\arraystretch}{1}

  The list following \ARG{fact} indicates which terms are to be 
  existentialized.
  If the optional \ARG{termI} is present, it is replaced by \ARG{indvarI}
  at each occurrence mentioned in the sequence of natural numbers
  \OPT{$n_{i1}$ $n_{i2}$ \SEQ} and then  existentialized.
  If \ARG{termI} is not present, all the occurrences of \ARG{indvarI} are
  put under the scope of the existential quantifier.
  Notice that no use can be made of an occurrence specification
  \OPT{$n_{i1}$ $n_{i2}$ \SEQ} if there is no \ARG{termI} present,
  {\GF} will return an error in this case.
  The dependencies of the conclusion are those of \ARG{fact}.

  In the sorted rule, let {\tt A(t)} be the formula to be existentialized,
  and let {\tt x} be of sort {\tt Sx}.
  Then the result of the rule is {\tt exists x.A(x)} if {\tt t} is of sort
  {\tt Sx}, {\tt Sx(t) imp exists x.A(x)} otherwise.
  This schema applies also to the case where multiple terms are to be
  existentialized.
}

\gfrecap{
The list following `fact' indicates which terms are to be existentialized.
If the optional `termI' is present, it is replaced by `indvarI' at each 
occurrence mentioned in the sequence of natural numbers [nI1 nI2 ...] and then
existentialized.
If `termI' is not present, all the occurrences of `indvarI' are put under the 
scope of the existential quantifier.
Notice that no use can be made of an occurrence specification [nI1 nI2 ...] if
there is no `termI' present, GETFOL will return an error in this case.
The dependencies of the conclusion are those of `fact'.
In the sorted rule, let `A(t)' be the formula to be existentialized, and let
`x' be of sort `Sx'.
Then the result of the rule is `exists x. A(x)' if `t' is of sort `Sx',
`Sx(t) imp exists x. A(x)' otherwise.
This schema applies also to the case where multiple terms are to be
existentialized.
}  

\gfexample+
   *****  comment ! unsorted example !
       
   ***** declare predconst P 2;
   ***** declare indvar x y x0;
   ***** declare indconst c1 c2;
   [...]

   ***** assume P(c1 c2);
   1   P(c1,c2)     (1)

   ***** existi 1 c1:x c2:y;
   2   exists y x.P(x,y)     (1)

   ***** existi 1 c1:x c1:x;
   3   exists x x.P(x,c2)     (1)

   ***** existi 1 c1:x c2:x0;
   4   exists x x0.P(x0,x)     (1)

   ***** assume P(c1 c1);
   5   P(c1,c1)     (5)

   ***** existi 5 c1:x 2;
   6   exists x.P(c1,x)     (5)
    
    
   ***** comment ! sorted example !
    
   ***** declare predconst P 2;
   ***** declare indvar x [Sx];
   ***** declare indvar y [Sy];
   ***** declare indconst c1 c2 [S];
   [...]

   ***** moregeneral Sx < S >;

   ***** assume P(c1 c2);
   1   P(c1,c2)     (1)

   ***** existi 1 c1:x c2:y;
   2   Sy(c2) imp exists y x. P(x,y)     (1)
+


% introduction to the substitution rule
\newpage
\section{Equality rules}

We describe here the rules for substitution of equality, $sub_l$ and $sub_r$.
{\GF} does not have explicit axioms or rules for reflexivity, commutativity and
transitivity.
Commutativity and transitivity, however, can be derived by $sub_l$ and $sub_r$,
and symmetry can be derived by using other {\GF} commands (for instance by
semantic simplification (see section \ref{sec-eval}), by the tautology and
monadic deciders (see the commands {\tt tauteq} and {\tt monadeq} commands in
section \ref{sec-decide}).


\[
\begin{array}{|c|c||c|c|} \hline 
	&&& \\
	sub_l &
	\fraz{A(t_1) \ \ \ \ t_1 = t_2}
	     {A(t_2)}
	&
	sub_r &
	\fraz{A(t_1) \ \ \ \ t_2 = t_1}
	     {A(t_2)}
	\\
	&&& \\ \hline
\end{array}
\]


% user command for substitution
\gfcommand{subst}{equality substitution}
\index{subst}

\gfsyntax{
  subst \ARG{fact1} \OPT{with} \ARG{fact2};\\
  subst \ARG{fact1} \OPT{with} \ARG{fact2} \OPT{right \ALT left};\\
  subst \ARG{fact1} \OPT{with} \ARG{fact2} 
                    \OPT{occ \ARG{n1} \ARG{n2} \SEQ} \OPT{right \ALT left};
}

\gfdescription{
  \[
  \begin{array}{cc}
    sub_l \ \ 
    \fraz{A(t_1) \ \ \ \ t_1 = t_2}
    {A(t_2)}
    \ \ \ 
    &
    \ \ \ 
    sub_r \ \ 
    \fraz{A(t_1) \ \ \ \ t_2 = t_1}
    {A(t_2)}
  \end{array}
  \]

  The rule substitutes a term in \ARG{fact1}'s wff with another one proved to 
  be equal to the former.
  \ARG{fact2} must be the equality {\tt t1 = t2} and \ARG{fact1} may contain 
  one or more occurrences of {\tt t1}.
  The conclusion is the result of the substitution of {\tt t1} with {\tt t2} in
  \ARG{fact1}'s wff. 
  Dependencies of derived fact are the union of those of \ARG{fact1} and
  \ARG{fact2}.
  If \ARG{fact1} does not contain {\tt t1}, then the conclusion has the same 
  wff and dependencies as \ARG{fact1}.

  The default is the substitution of {\tt t2} with {\tt t1}, corresponding to 
  the option {\bf left}. If {\bf right} is indicated, then {\tt t1} is 
  substituted with {\tt t2}.

  Individual occurrences can be substituted by specifying the optional
  \OPT{{\tt occ} \ARG{n1} \ARG{n2} \SEQ}, where \ARG{n1}, \ARG{n2}, \SEQ are
  the occurrences to be substituted.
  Without this option, all occurrences are substituted. 
}

\gfrecap{
The rule substitutes a term in the formula of `fact1' with another one
proved to be equal to the former.
`fact2' must be the equality `t1 = t2' and `fact1' may contain one or more
occurrences of `t1'.
The conclusion is the result of the substitution of `t1' with `t2' in the
formula of `fact1'. 
Dependencies of derived fact are the union of those of `fact1' and `fact2'.
If `fact1' does not contain `t1', then the conclusion has the same wff and
dependencies as `fact1'.
The default is the substitution of `t2' with `t1', corresponding to the option
`left'.
If `right' is indicated, then `t1' is substituted with `t2'.
Individual occurrences can be substituted by specifying the optional
`[occ n1 n2 ...]', where `n1', `n2', ... are the occurrences to be substituted.
Without this option, all occurrences are substituted. 
}

\gfexample+
   ***** declare predconst P Q 2;
   ***** declare funconst f 1; 
   ***** declare indvar x; declare indvar y;
   [...]

   ***** assume P(x y) imp Q(y x);
   1   P(x,y) imp Q(y,x)     (1)

   ***** assume x = f(x);
   2   x = f(x)     (2)

   ***** subst 1 2;
   3   P(f(x),y) imp Q(y,f(x))     (1 2)

   ***** subst 3 2 right;
   4   P(x,y) imp Q(y,x)     (1 2)

   ***** subst 1 2 occ 1;
   5   P(f(x),y) imp Q(y,x)     (1 2)
+




	% introduction to the section
\newpage
\section{Other rules}

\subsection{Conditional rules}
\label{sec-cond}

Conditional rules allow us to introduce and eliminate the conditional wffs
{\wffif} and conditional terms {\termif}.

\begin{bnf}
	{\wffif}  \sep {\bf wffif} {\wff}$_1$ {\bf then} {\wff}$_2$ {\bf else}
				   {\wff}$_3$ \\
	{\termif} \sep {\bf trmif} {\wff} {\bf then} {\term}$_1$ {\bf else}
				   {\term}$_2$
\end{bnf}

Note that the {\termif} construct is not first order.
However, it can easily be shown that {\termif} can be defined as a conservative
extension of first order logic using an induction on the length of deductions.

\renewcommand{\arraystretch}{0.5}
\[
\begin{array}{|c|c||c|c|} \hline 
\multicolumn{2}{|c||}{\mbox{\bf Introduction rules}} &
\multicolumn{2}{c|}{\mbox{\bf Elimination rules}} \\ \hline
\begin{array}{l}
\\ \\ \\ \\ \\
\mbox{{\em wif}} \  I  
\end{array}
&
\begin{array}{cc}
\\ \\
\begin{array}{c}
{[A]}\\
\vdots\\
B
\end{array}
\ \ \ 
\begin{array}{c}
{[\neg A]}\\
\vdots\\
C
\end{array}
\\
\hline\\
\begin{array}{c}
\mbox{{\em wffif}} \ A \ \mbox{{\em then}} \ B \ \mbox{{\em else}} \ C 
\end{array}
\end{array}
&
\begin{array}{l}
\\ \\ \\ \\ \\
\mbox{{\em wif}} \  E
\end{array}
&
\begin{array}{cc}
\\ \\
\begin{array}{c}
\\ \\ \\ \\
A
\end{array}
\ \ \ 
\begin{array}{c}
\\ \\ \\ \\
\mbox{{\em wffif}} \ A \ \mbox{{\em then}} \ B \ \mbox{{\em else}} \ C 
\end{array}
\\
\hline\\
\begin{array}{c}
B
\end{array}
\end{array}
%\fraz{\Gamma \vdash A \ \ \Delta \vdash \mbox{{\em wffif}} \ A \ \mbox{{\em then}} \ B \ 
%\mbox{{\em else}} \ C}
%     {\Gamma,\Delta \vdash B}
\\ %\hline 
& &
\begin{array}{l}
\\ \\ \\ \\ \\
\mbox{{\em wif}} \  E_{\neg}
\end{array}
&
\begin{array}{cc}
\\ \\
\begin{array}{c}
\\ \\ \\ \\
\neg A
\end{array}
\ \ \ 
\begin{array}{c}
\\ \\ \\ \\
\mbox{{\em wffif}} \ A \ \mbox{{\em then}} \ B \ \mbox{{\em else}} \ C 
\end{array}
\\
\hline\\
\begin{array}{c}
C
\end{array}
\end{array}
\\ %\hline
\begin{array}{l}
\\ \\ \\ \\ \\
\mbox{{\em tif}} \  I    
\end{array}
&
\begin{array}{cc}
\\ \\
\begin{array}{c}
{[A]}\\
\vdots\\
B(t_1)
\end{array}
\ \ \ 
\begin{array}{c}
{[\neg A]}\\
\vdots\\
B(t_2)
\end{array}
\\
\hline\\
\begin{array}{c}
B(\mbox{{\em termif}} \ A \ \mbox{{\em then}} \ t_1 \ \mbox{{\em else}} \ t_2)
\end{array}
\end{array}
&
\begin{array}{l}
\\ \\ \\ \\ \\
\mbox{{\em tif}} \  E
\end{array}
&
\begin{array}{cc}
\\ \\
\begin{array}{c}
\\ \\ \\ \\
A
\end{array}
\ \ \ 
\begin{array}{c}
\\ \\ \\ \\
B(\mbox{{\em termif}} \ A \ \mbox{{\em then}} \ t_1 \ \mbox{{\em else}} \ t_2)
\end{array}
\\
\hline\\
\begin{array}{c}
B(t_1)
\end{array}
\end{array}
\\ %\hline 
& &
\begin{array}{l}
\\ \\ \\ \\ \\
\mbox{{\em tif}} \  E_{\neg}
\end{array}
&
\begin{array}{cc}
\\ \\
\begin{array}{c}
\\ \\ \\ \\
\neg A
\end{array}
\begin{array}{c}
\\ \\ \\ \\
B(\mbox{{\em termif}} \ A \ \mbox{{\em then}} \ t_1 \ \mbox{{\em else}} \ t_2)
\end{array}
\\
\hline\\
\begin{array}{c}
B(t_2)
\end{array}
\end{array}
\\ 
& & & \\ \hline 
\end{array}
\]
\renewcommand{\arraystretch}{1}


\subsection{Structural rules}

Structural rules are useful when performing theorem proving.

% user commands for conditional rules
\gfcommand{termife}{term conditional elimination}
\index{termife}

\gfsyntax{
	termife \ARG{fact1} \ARG{fact2} \ARG{termif}; \\
	termife \ARG{fact1} \ARG{fact2} \ARG{termif} \OPT{occ \ARG{n1} \ARG{n2}
	\SEQ}; 
}

\gfdescription{
	\renewcommand{\arraystretch}{0.5}
	\[
	\mbox{{\em tif}} \  E \ \ 
	\fraz{A \ \ \ \ \ B(\mbox{{\em termif}} \ A \ \mbox{{\em then}} \ t_1 \
	\mbox{{\em else}} \ t_2)} 
	{B(t_1)}
	\]
	\renewcommand{\arraystretch}{1}

	If \ARG{termif} is {\tt iftrm A then t1 else t2}, \ARG{fact1}'s wff is
	{\tt W(iftrm A then t1 else t2)} and {\em fact}$_2$'s wff is {\tt A},  then
	the rule deduces {\tt W(t1)}.
	If {\em termif} is not a subexpression of \ARG{fact1}'s wff, then no
	substitution is performed.
	Individual occurrences can be substituted by specifying the optional 
	\ARG{n1} \ARG{n2}, \SEQ, where \ARG{n1}, \ARG{n2}, \SEQ are the occurrences
	to be substituted.
	Without this option, all occurrences are substituted. 
	The dependencies of the derived wff are the union of those of \ARG{fact1}
	and \ARG{fact2}. 
}

\gfrecap{
Term conditional elimination.
}

\gfexample+
   ***** declare sentconst A;
   ***** declare predconst P 1;
   ***** declare indpar a b;
   ***** assume P(trmif A then a else b);
   1   P(trmif A then a else b)     (1)
   ***** assume A;
   2   A     (2)
   ***** termife 1 2 trmif A then a else b;
   3   P(a)     (1 2)
+

\gfnotes{}

\gfcommand{termifen}{term conditional elimination (with negation)}
\index{termifen}

\gfsyntax{
	termifen \ARG{fact1} \ARG{fact2} \ARG{termif}; \\
	termifen \ARG{fact1} \ARG{fact2} \ARG{termif} \OPT{occ \ARG{n1} \ARG{n2}
	\SEQ}; 
}

\gfdescription{
	\renewcommand{\arraystretch}{0.5}
	\[
	\mbox{{\em tif}} \  E_{\neg} \ \ 
	\fraz{\neg A \ \ \ \ \ B(\mbox{{\em termif}} \ A \ \mbox{{\em then}} \ t_1 \
	\mbox{{\em else}} \ t_2)} 
	{B(t_2)}
	\]
	\renewcommand{\arraystretch}{1}
	
	If {\em termif} is {\tt iftrm A then t1 else t2}, {\em fact}$_1$'s wff is
	{\tt W(iftrm A then t1 else t2)} and \ARG{fact2}'s wff is {\tt not A},
	then the rule deduces {\tt W(t2)}.
	If \ARG{termif} is not a subexpression of \ARG{fact1}'s wff, then no
	substitution is performed. 
	Individual occurrences can be substituted by specifying the optional 
	\ARG{n1}, \ARG{n2}, \SEQ, where \ARG{n1}, \ARG{n2}, \SEQ are the occurrences
	to be substituted.
	Without this option, all occurrences are substituted. 
	The dependencies of the derived wff are the union of those of \ARG{fact1}
	and \ARG{fact2}.
}

\gfrecap{
Term conditional elimination (with negation).
}

\gfexample+
   ***** declare sentconst A;
   ***** declare predconst P 1;
   ***** declare indpar a b;
   ***** assume P(trmif A then a else b);
   1   P(trmif A then a else b)     (1)
   ***** assume not A;
   2   not A     (2)
   ***** termifen 1 2 trmif A then a else b;
   3   P(b)     (1 2)
   \end{verbatim}
+

\gfnotes{
	\ARG{fact2}'s wff must be negation of \ARG{termif}'s condition and not
	viceversa.
	If \ARG{termif} = {\tt trmif not A then t1 else t2} and \ARG{fact2}'s wff =
	{\tt A}, then the rule is not applicable. 
}

\gfcommand{termifi}{term conditional introduction}
\index{termifi}

\gfsyntax{
	termifi \ARG{fact1} \ARG{fact2} \ARG{wff} \ARG{term1} \ARG{term2};
}

\gfdescription{
\renewcommand{\arraystretch}{0.5}
\[
\begin{array}{l}
\\ \\ \\ 
%\\ \\
\mbox{{\em tif}} \  I    
\end{array}
\ \
\begin{array}{cc}
%\\ \\
\begin{array}{c}
{[A]}\\
\vdots\\
B(t_1)
\end{array}
\ \ \ 
\begin{array}{c}
{[\neg A]}\\
\vdots\\
B(t_2)
\end{array}
\\
\hline\\
\begin{array}{c}
B(\mbox{{\em termif}} \ A \ \mbox{{\em then}} \ t_1 \ \mbox{{\em else}} \ t_2)
\end{array}
\end{array}
\]
\renewcommand{\arraystretch}{1}

If \ARG{fact1}'s wff is {\tt B({\em term}$_1$)} and \ARG{fact2}'s wff is {\tt
B(\ARG{term2})}, then the proof line's wff is
{\tt B(iftrm wff then \ARG{term1} else \ARG{term2})}.
All assumptions in \ARG{fact1}'s dependencies whose formula is \ARG{wff} and all
assumptions in \ARG{fact2}'s dependencies whose formula is the negation of
\ARG{wff} are discharged.
}

\gfrecap{
Term conditional introduction.
}

\gfexample+
   ***** declare sentconst A;
   ***** declare predconst P 1;
   ***** declare funconst f 1;
   ***** declare indpar a b;
   ***** assume P(trmif A then a else b);
   1   P(trmif A then a else b)     (1)
   ***** assume A;
   2   A     (2)
   ***** termife 1 2 trmif A then a else b;
   3   P(a)     (1 2)
   ***** assume not A;
   4   not A     (4)
   ***** termifen 1 4 trmif A then a else b;
   5   P(b)     (1 4)
   ***** termifi 3 5 A a b;
   6   P(trmif A then a else b)     (1)
   ***** COMMENT | Dependencies are discharged |
   ***** termifi 3 5 not A a b;
   7   P(trmif (not A) then a else b)     (1 2 4)
   ***** COMMENT | Dependencies are NOT discharged |
   ***** termifi 3 5 B a b;
   8   P(trmif B then a else b)     (1 2 4)
   ***** COMMENT | TERMIFI with function symbols |
   ***** assume P(trmif A then f(a) else f(b));
   9   P(trmif A then f(a) else f(b))     (9)
   ***** termife 9 2 trmif A then f(a) else f(b);
   10   P(f(a))     (2 9)
   ***** termifen 9 4 trmif A then f(a) else f(b);
   11   P(f(b))     (4 9)
   ***** termifen 9 4 trmif A then f(a) else f(b);
   12   P(f(b))     (4 9)
   ***** termifi 10 11  A a b;
   13   P(f(trmif A then a else b))     (9)
   ***** termifi 10 11 A f(a) f(b);
   14   P(trmif A then f(a) else f(b))     (9)
+   
   
   
\gfcommand{wffife}{wff conditional elimination}
\index{wffife}

\gfsyntax{
	wffife \ARG{fact1} \ARG{fact2};
}

\gfdescription{
\renewcommand{\arraystretch}{0.5}
\[
\mbox{{\em wif}} \  E \ \
\fraz{A \ \ \ \ \ \mbox{{\em wffif}} \ A \ \mbox{{\em then}} \ B \ \mbox{{\em else}} \ C }
{B}
\]
\renewcommand{\arraystretch}{1}

If \ARG{fact1}'s wff is {\tt wffif A then B else C} and \ARG{fact2}'s wff is
{\tt A}, then the derived wff is {\tt B} and its dependencies are the union of
those of \ARG{fact1} and \ARG{fact2}.
}

\gfrecap{
If the wff of `fact1' is `wffif A then B else C' and the wff of `fact2' is
`A', then the derived wff is `B' and its dependencies are the union of those of
`fact1' and `fact2'.
}

\gfexample+
   ***** declare sentconst A B C;
   ***** assume A;
   1   A     (1)
   ***** assume wffif A then B else C;
   2   wffif A then B else C     (2)
   ***** wffife 2 1;
   3   B     (1 2)
+

\gfnotes{}

\gfcommand{wffifen}{wff conditional elimination (with negation)}
\index{wffifen}

\gfsyntax{
	wffifen \ARG{fact1} \ARG{fact2};
}

\gfdescription{
	\renewcommand{\arraystretch}{0.5}
	\[
	\mbox{{\em wif}} \  E_{\neg} \ \
	\fraz{\neg A \ \ \ \ \ \mbox{{\em wffif}} \ A \ \mbox{{\em then}} \ B \
	\mbox{{\em else}} \ C } 
	{C}
	\]
	\renewcommand{\arraystretch}{1}
	
	If {\em fact}$_1$'s wff is {\tt wffif A then B else C} and {\em fact}$_2$'s
	wff is {\tt not A}, then the derived wff is {\tt C} and its dependencies are
	the union of those of {\em fact}$_1$ and {\em fact}$_2$.
}

\gfrecap{
If the wff of `fact1' is `wffif A then B else C' and the wff of `fact' wff is
`not A', then the derived wff is `C' and its dependencies are the union of those
of `fact1' and `fact2'.
}

\gfexample+
   ***** declare sentconst A B C; 
   ...

   ***** assume not A;
   1   not A     (1)
   ***** assume wffif A then B else C;
   2   wffif A then B else C     (2)
   ***** wffifen 2 1;
   3   C     (1 2)
+

\gfnotes{}

\gfcommand{wffifi}{wff conditional introduction}
\index{wffifi}

\gfsyntax{
	wffifi \ARG{wff} \ARG{fact1} \ARG{fact2};
}

\gfdescription{
\renewcommand{\arraystretch}{0.5}
\[
\begin{array}{l}
\\ \\ \\ 
%\\ \\
\mbox{{\em wif}} \  I  
\end{array}
\ \ 
\begin{array}{cc}
%\\ \\
\begin{array}{c}
{[A]}\\
\vdots\\
B
\end{array}
\ \ \ 
\begin{array}{c}
{[\neg A]}\\
\vdots\\
C
\end{array}
\\
\hline\\
\begin{array}{c}
\mbox{{\em wffif}} \ A \ \mbox{{\em then}} \ B \ \mbox{{\em else}} \ C 
\end{array}
\end{array}
\]
\renewcommand{\arraystretch}{1}

If \ARG{wff} is {\tt A}, \ARG{fact1}'s wff is {\tt B} and \ARG{fact2}'s wff is
{\tt C}, then the conclusion is {\tt wffif A then B else C}.
The rule discharges all the dependencies of {\em fact}$_1$ whose wff is {\tt A}
and of {\em fact}$_2$ whose wff is {\tt not A}.
}

\gfrecap{
If `wff' is `A', the wff of `fact1' is `B' and the wff of `fact2' is `C', then
the conclusion is `wffif A then B else C'.
The rule discharges all the dependencies of `fact1' whose wff is `A' and of
`fact2' whose wff is `not A'.
}

\gfexample+
   ***** declare sentconst A;
   ...
   ***** assume A;
   1   A     (1)
   ***** assume not A;
   2   not A     (2)
   ***** wffifi A 1 2;
   3   wffif A then A else (not A)
   ***** wffifi A 2 1;
   4   wffif A then (not A) else A     (1 2)
   ***** wffifi not A 2 1;
   5   wffif (not A) then (not A) else A     (1)
   ***** wffifi not A 1 2;
   6   wffif (not A) then A else (not A)     (1 2)
+

\gfnotes{}


% introduction to the sort's section
\gfcommand{contract}{contraction rule}
\index{contract}

\gfsyntax{
	contract \ALT ctc  \ARG{fact} by \ARG{assumption1} \SEQ  \ARG{assumptionN}; 
}

\gfdescription{
	Every \ARG{assumptionI} must occur in the list of dependencies of \ARG{fact}.

	The proof line's dependencies are those of \ARG{fact} less all the dependencies 
	with the same wff of some \ARG{assumptionI}s.
}

\gfrecap{
	Every `assumptionI' must occur in the list of dependencies of `fact'.

	The proof line's dependencies are those of `fact' less all the dependencies 
	with the same wff of some `assumptionI's.
}

\gfexample+
   ***** declare sentconst A B C D;
   ***** assume A A A A B B C D;
   1   A     (1)
   2   A     (2)
   3   A     (3)
   4   A     (4)
   5   B     (5)
   6   B     (6)
   7   C     (7)
   8   D     (8)
   ***** wk 8 by 1 2 3 4 5 6 7;
   9   D     (1 2 3 4 5 6 7 8)
   ***** ctc 9 by 1 5;
   10   D     (1 5 7 8)
   ***** ctc 9 by 1 2 3 5 7;
   11   D     (1 2 3 5 7 8)
   ***** ctc 9 by 10;
   I can only contract facts which occur in the assumption list!
   ***** ctc 11 by 4;
   I cannot contract using a fact not occurring in the assumption list!
+

\gfnotes{}



\gfcommand{cut}{cut rule}
\index{cut}

\gfsyntax{
	cut \ARG{fact1} \ARG{fact2};\\
	cut \ARG{fact1} \ARG{fact2} \OPT{keep \ARG{assumption1} \SEQ
	\ARG{assumptionN}};
}

\gfdescription{
	This command generates a new proof line obtained from \ARG{fact2} 
	by eliminating dependencies of all facts whose wff is equal to \ARG{fact1}'s
	wff, and then adding the dependencies of \ARG{fact1}.
	In the second form, every \ARG{assumptionI} in the list of dependencies
	is kept. 
	Every \ARG{assumptionI} must occur in the list of dependencies of
	\ARG{fact2} . 
}

\gfrecap{
This command generates a new proof line obtained from `fact2'
by eliminating dependencies of all facts whose wff is equal to `fact1''s
wff, and then adding the dependencies of `fact1'.
In the second form, every `assumptionI' in the list of dependencies is kept. 
Every `assumptionI' must occur in the list of dependencies of `fact2'. 
}


\gfexample+
   ***** declare sentconst A B C;
   ***** axiom AAA : A;
   AAA : A
   ***** assume A A A A B C;
   1   A     (1)
   2   A     (2)
   3   A     (3)
   4   A     (4)
   5   B     (5)
   6   C     (6)
   ***** wk 5 by 1 2 3 4;
   7   B     (1 2 3 4 5)
   ***** wk AAA by 6;
   8   A     (6)
   ***** cut 8 7;
   9   B     (5 6)
   ***** wk 7 by 6;
   10   B     (1 2 3 4 5 6)
   ***** cut AAA 10;
   11   B     (5 6)
   ***** cut 8 7 keep 3 2;
   12   B     (2 3 5 6)
+

\gfcommand{weaken}{weakening rule}
\index{weaken}

\gfsyntax{
	weaken \ALT wk \ARG{fact} by \ARG{fact1} \SEQ \ARG{factN};
}

\gfdescription{
	Each \ARG{factI} can be an assumption or, more generally, a fact.
	In the first case, the derived fact depends on the dependencies of 
	\ARG{fact} plus those of \ARG{factI}.
	In the second case, the dependencies of the \ARG{factI} are added to the
	list of dependencies of \ARG{fact}. 
}

\gfrecap{
Each `factI' can be an assumption or, more generally, a fact.
In the first case, the derived fact depends on the dependencies of `fact' plus
those of `factI'.
In the second case, the dependencies of the `factI' are added to the list of
dependencies of `fact'. 
}

\gfexample+
   ***** declare sentconst A B C;
   ***** assume A A A B B C;
   1   A     (1)
   2   A     (2)
   3   A     (3)
   4   B     (4)
   5   B     (5)
   6   C     (6)
   ***** axiom AAAA : A;
   AAAA : A
   ***** ori 3 4;
   7   A or B     (3 4)
   ***** ori 5 6;
   8   B or C     (5 6)
   ***** wk 7 by 6 2 1 3;
   9   A or B     (1 2 3 4 6)
   ***** wk 1 by 7 8;
   10   A     (1 3 4 5 6)
   ***** wk AAAA by 3 7 2 8;
   11   A     (2 3 4 5 6)
+

\gfnotes{}

 	% introduction to the deciders
\newpage
\newcommand{\eg}{{\em e.g.~}}
\newcommand{\ie}{{\em i.e.~}}
\newcommand{\wrt}{w.r.t.~}
\newcommand{\co}[2]{\langle #1, \: #2 \rangle}


\section{Decision procedures}
\label{sec-decide}
\label{system:sec}
A detailed description of the main decision procedures of {\tt GETFOL}
is given in \cite{armando5}.

The set of procedures of the {\tt GETFOL} system is depicted in figure
\ref{system:fig}.
Each box represents either a decider ({\tt PTAUT}, {\tt PTAUTEQ},
{\tt FOLTAUT}) or a rewriting procedure ({\tt tautren}, {\tt  phexp},
{\tt  reduce}).

\begin{figure}
\begin{center}
\makebox[3.375in][l]{
  \vbox to 2.750in{
    \vfill
    \special{psfile=decide/NEWFIG.PS}
  }
  \vspace{-\baselineskip}
}
\end{center}
\caption{The system of deciders}
\label{system:fig}
\end{figure}

\subsubsection*{{\tt PTAUT} and {\tt PTAUTEQ}}
{\tt PTAUT} and {\tt PTAUTEQ}
are deciders working on a quantifier-free first order language (hereon by
abuse of language we call them propositional deciders).
{\tt PTAUT} decides the set of first order formulas provable using
only the propositional deductive machinery (moreover it returns a
falsifying assignment whenever the input formula is not a tautology).
For instance, the formula $(P(x)\con R(x,b))\imp (P(x)\dis R(x,b))$ can be
easily inferred by a single application of {\tt PTAUT}.
{\tt PTAUT} is a generalization of the Davis-Putnam-Loveland procedure
(hereon DPL) \cite{davis2,davis6} to non clausal formulas.
The core of {\tt PTAUT} is a procedure capable of partially evaluating
the input formula \wrt a partial assignment of truth-values to the atomic
subformulas. 
A step of statistical analysis (of polynomial time complexity) collects
information about the {\em polarity} of the subformulas and the existence
of {\em Top-Level Disjunctive Occurrences} (TLDO) of atomic subformulas.
A formula $\alpha$ occurs as a TLDO in $\beta$ 
if and only if $\beta$ can be rewritten into a formula either of the form
$(\alpha\dis\gamma)$ or $(\neg\alpha\dis\gamma)$ by means of rules
expressing  the usual properties of the propositional connectives such as
the associativity, commutativity and distributivity of the propositional
connectives.
The notion of positive (negative) subformula occurrence
is inductively defined as follows: $\alpha$ occurs positively in $\alpha$, 
$\alpha$ occurs negatively in $\neg\alpha$;
$\alpha$ and $\beta$ occur positively in $(\alpha\con\beta)$ and
$(\alpha\dis\beta)$;
$\alpha$ occurs negatively and $\beta$ occurs positively in $(\alpha\imp\beta)$;
finally $\alpha$ and $\beta$ occur both positively and negatively in
$(\alpha\liff\beta)$.
A subformula $\alpha$ is positive (negative) in $\beta$ if and only if
each occurrence of $\alpha$ occurs positively (negatively) in $\beta$.

The statistical analysis may suggest a partial assignment $\mu$ (\wrt which
the formula can be simplified) according to the following criteria:
\begin{itemize}
\item for each positive (negative) atomic formula $\alpha$ occurring
in $\beta$, $\mu(\alpha)=F$ ($\mu(\alpha)=T$);
\item if $\beta$ contains a positive (negative) TLDO of $\alpha$ and there
are no negative (positive) TLDO of $\alpha$, then $\mu(\alpha)=F$
($\mu(\alpha)=T$).
\end{itemize}

If $\mu$ is not completely undefined, then {\tt PTAUT} simplifies the
formula in input \wrt $\mu$ and recurs on the resulting (simplified) formula.
If the input formula contains both a positive and a negative TLDO of an
atomic formula the input formula is a tautology.
These optimizations generalize the {\em Affirmative-Negative Rule} and the
{\em Rule for the Elimination of One-Literal Clauses} of DPL.
If $\mu$ is totally undefined, then
an atomic formula is chosen, two partial assignments are created
(one assigning $T$, the other $F$ to the chosen atomic formula),
the formula is partially evaluated \wrt such assignments and finally
the procedure branches recurring on the two simplified formulas.
This last step generalizes the {\em Splitting Rule} of DPL.

{\tt PTAUTEQ} is the result of adapting {\tt PTAUT}
to take into account the properties of equality.
The main difference is that, before a formula is simplified \wrt some
assignment, the assignment is tested to check whether it is model of the
quantifier-free theory of equality.
The formula $(P(x)\con x=y)\imp (P(y)\con y=x)$ can be
easily inferred by a single application of {\tt PTAUTEQ}.

\subsubsection*{{\tt nnf} and {\tt skolemize}}
{\tt nnf} rewrites the input formula into a logically equivalent one in
{\em negative normal form}.

{\tt skolemize} computes the skolemization of the input formula.

\subsubsection*{{\tt tautren} and {\tt phexp}}
The procedures on top of the propositional deciders (namely {\tt tautren}
and {\tt phexp}) map the first-order formula in input into a quantifier-free
formula.
The mappings are such that the decision problem of the input (first-order)
formula is related to the decision problem of the returned (quantifier-free)
formula in a useful way.
In particular, {\tt tautren} atomizes equal (modulo renaming of bound
variables) quantified subformulas into newly introduced propositional
letters.
For instance the formula
\begin{equation}\label{pb29-reduced}
%\setlength{\templength}{\arraycolsep}
\setlength{\arraycolsep}{0cm}
\begin{array}{rl}
(\exists x.F(x) \con \exists x.G(x)) \imp (&( \forall x.(F(x) \imp H(x)) \con \forall x.(G(x) \imp J(x))) \liff \\
& ((\exists y.G(y) \imp \forall x.(F(x) \imp H(x))) \con\\
&\ (\exists x.F(x) \imp \forall y.(G(y) \imp J(y)))))
\end{array}
\end{equation}
is mapped into the propositional formula
\begin{equation}\label{pb29-prop}
(A \con B) \imp ((C \con D) \liff ((B \imp C) \con (A \imp D)))
\end{equation}
The relation between the decision problems of the input formula (say $\alpha$)
and of the output formula (say $\alpha'$) is that
$\der{}\alpha'$ only if $\der{}\alpha$.

A more careful reduction to the quantifier-free fragment is performed by
{\tt phexp}.
{\tt phexp} maps an existential formula $\alpha$ into a quantifier-free formula
$\alpha'$ such that $\der{}{\alpha'}$ if and only if $\der{}{\alpha}$.%
\footnote{The set of existential formualas is the class of prenex
universal-existential formulas without function symbols.}

The formula $\alpha'$ is an improved version of the Herbrand's expansion
of $\alpha$ \cite{dreben1}.
An application of {\tt phexp} to the following formula:
\begin{equation}\label{pb28}\small
    (((P(x) \con \neg Q(y)) \dis
     ((Q(a) \dis R(a)) \con (\neg Q(b) \dis \neg S(b)))) \dis
     ((F(z) \con \neg G(z)) \con S(v))) \dis
       ((\neg P(c) \dis \neg F(c)) \dis G(c))
\end{equation}
yields
\begin{equation}\label{pb28-exp}\small
\begin{array}{l}
((((P(a) \dis P(b) \dis P(c)) \con (\neg Q(a) \dis \neg Q(b) \dis
\neg Q(c)))\dis\\
((Q(a) \dis R(a)) \con (\neg Q(b) \dis \neg S(b)))) \dis \\
(((F(a) \con \neg G(a)) \dis (F(b) \con \neg G(b)) \dis
(F(c) \con \neg G(c)))\con\\
(S(a) \dis S(b) \dis S(c)))) \dis ((\neg P(c) \dis \neg F(c)) \dis G(c))\\
\end{array}
\end{equation}
In \cite{armando5} it is shown that, the size of (\ref{pb28-exp})
is 44 times smaller than the size of the standard Herbrand's expansion of
(\ref{pb28}).

\subsubsection*{{\tt reduce}}
{\tt reduce} tries a set of rewriting rules on the input
formula aiming at rewriting it into a logically equivalent formula that
can be easily turned into an existential one via skolemization.
The rewriting rules employed by {\tt reduce} are the usual rules
expressing the distributivity of quantifiers through propositional connectives
and the commutativity and associativity of propositional connectives
listed in the following table.\\

    \renewcommand{\arraystretch}{1.5}
    {\small
      $$
      \begin{array}{|c|rcl|} \hline
        (1) & Q x. \alpha[x] & \mapsto & \alpha \\ \hline
        %(2) & Q x. (\neg \alpha(x)) & \mapsto & (\neg \hat{Q} x. \alpha(x)) \\ \hline
        (2) & Q x. (\alpha \circ \beta)(x) & \mapsto & (Q x. \alpha \circ Q x. \beta) 
        \\ \hline
        (3) & Q x. (\alpha[x] + \beta(x)) & \mapsto & (\alpha[x] + Q x. \beta(x)) 
        \\ \hline \hline
        (4) & (\alpha(x) + \beta[x]) & \mapsto & (\beta[x] + \alpha(x)) \\ \hline
        (5) & ((\alpha[x] + \beta(x)) + \gamma(x)) & \mapsto & 
        (\alpha[x] + (\beta(x) + \gamma(x))) \\ \hline
        (6) & ((\alpha \circ \beta)(x) + \gamma(x)) & \mapsto & 
        ((\alpha + \gamma(x)) \circ (\beta + \gamma(x))) \\ \hline
        (7) & (\alpha(x) + (\beta[x] + \gamma(x))) & \mapsto &
        (\beta[x] + (\alpha(x) + \gamma(x))) \\ \hline
        (8) & ((\alpha(x) + (\beta \circ \gamma)(x))) & \mapsto &
        ((\alpha(x) + \beta) \circ (\alpha(x) + \gamma)) \\ \hline
      \end{array}
      $$
      }
    \renewcommand{\arraystretch}{1}
{\small
{\em Restrictions}: 
\begin{itemize}
\item In rules $\{(4)-(8)\}$ the left hand side must be a top normalizable
formula.
\item In rules $\{(7),(8)\}$ $\alpha$ must be minimal \wrt $\co{Q}{x}$.
%\item Rules $\{(4)-(8)\}$ can be applied only to subformulae (say $\alpha$)
%of a formula $Qx.\beta$ in which there is no proper
%subformula $Qy.\gamma$ of which $\alpha$ is a subformula.
\end{itemize}}

Where
$\alpha(x)$ denotes a formula in which there is at least one free occurrence
of the variable $x$.
$\alpha{[x]}$ denotes a formula in which there is no free occurrences of $x$.
$Q$ and $Q'$ stand either for $\forall$ or for $\exists$.
If $Q = \forall$, then $\circ = \con$ and $+ = \dis$.
If $Q = \exists$, then $\circ = \dis$ and $+ = \con$.
%$\con$ is said to be $\forall$-compatible and $\exists$-incompatible,
%$\dis$ is said to be $\exists$-compatible and $\forall$-incompatible.
%If $\cal S$ is a set of rewriting rules then $\mapsto_{\cal S}$ is the
%reducibility relation induced by $\cal S$ and
%$\stackrel{*}{\mapsto}_{\cal S}$ is the reflexive and transitive closure
%of $\mapsto_{\cal S}$.
The definition of {\em top normalizable formula} and of {\em minimal
formula} are given in \cite{armando5}.

For instance, a single application of {\tt reduce} turns the formula
\begin{equation}\label{pb29}
%\setlength{\templength}{\arraycolsep}
\setlength{\arraycolsep}{0cm}
\begin{array}{rl}
(\exists x.F(x) \con \exists x.G(x)) \imp (&( \forall x.(F(x) \imp H(x)) \con \forall x.(G(x) \imp J(x))) \liff \\
&(\forall x.\forall y.((F(x) \con G(y)) \imp (H(x) \con J(y)))))
\end{array}
\end{equation}
into (\ref{pb29-reduced}).
{\tt reduce} considerably enlarges the set of formulas which can be solved
by using the system of deciders.
In particular, any prenex first order formula 
$$
\forall \vec{y}_n \exists \vec{x}_n \ldots
\forall \vec{y}_i \exists \vec{x}_i \ldots
\forall \vec{y}_1 \exists \vec{x}_1 . \Phi
$$
such that each literal in $\Phi$ contains no variables in $\vec{y}_k$ and
in $\vec{x}_l$ with $k < l$, or in $\vec{x}_k$ and in $\vec{x}_l$ with
$k \neq l$ can be ``reduced" to an existential formula.
On the basis of the previous result it is a trivial consequence
that the {\em monadic calculus} can be reduced to the existential fragment
by means of {\tt reduce}.


% user commands for deciders
\gfcommand{decide}{general purpose decider}
\index{decide}

\gfsyntax{
  decide \ARG{wff}  \OPT{by \ARG{fact1} \ARG{fact2} \SEQ} using 
  \OPT{\{ \ARG{rewriter} \SEQ \}} \ARG{decider};
}

\gfdescription{
 {\em rewriter} and {\em decider} are defined by the following grammar:
 %
 \begin{bnf}
    \T{rewriter} \sep nnf | reduce | skolemize |
                      phexp | tautren \\ 
    \T{decider}  \sep ptaut | ptauteq
 \end{bnf}
 %
 Let {\it wff}$_i$ be the formula of {\it fact}$_{i}$, $i = 1, 2, ...\ .$
 The command tries to verify whether the formula 
 %
 \begin{equation}
   {\it wff_1} \con ... \con {\it wff}_n \imp {\it wff} \label{eq-dec}
 \end{equation}
 %
 is a theorem by applying the specified rewriters in the same order they
appear on the command line and finally applying the decider to the resulting
formula.
 If this succeeds, it asserts a fact whose formula is \ARG{wff}
 and whose dependencies are the union of the dependencies of 
 \ARG{fact1}, \ARG{fact2}, \SEQ.
Otherwise failure is reported.\\
 %
 Whenever the formula in input to one of the specified routines does not  
 match the syntactic  restrictions of the routine, an error  message is
 printed, together with hints on how to modify the strategy. 
}

\gfrecap{
  `rewriter' is one of the following: nnf, reduce, skolemize, phexp, tautren.
  `decider'  is either ptaut or taut.
The command tries to verify whether the formula:
   +-------------------------------------------------------+
   | (wff1 and ... and wffN) imp wff                   (1) |
   +-------------------------------------------------------+
(where `wffI' if the formula of `factI') is a theorem by applying
the rewriters `rewriter ...' (in the order in which they appear) and 
then the decider `decider' to (1).
}

\gfexample+
   ***** declare indvar x,y,z;
   ***** declare predconst P,Q 1;
   ***** decide exists x.forall y.forall z.((P(y) imp Q(z) imp
               (P(x) imp Q(x))))  using nnf,phexp,ptaut;
   
   PHEXP requires the formula to be in existential form.
   Try using SKOLEMIZE before .... 
   
   ***** decide exists x.forall y.forall z.((P(y) imp Q(z) imp
               (P(x) imp Q(x)))) using nnf,skolemize,phexp,ptaut;
   
   decide couldn't prove that exists x. forall y z. ((P(y) imp Q(z))
    imp (P(x) imp Q(x))) is a tautology using nnf,skolemize,phexp,ptaut.
   
   ***** decide exists x.forall y.forall z.((P(y) imp Q(z) imp
               (P(x) imp Q(x)))) using nnf,reduce,skolemize,phexp,ptaut;
   
   1   exists x. forall y z. ((P(y) imp Q(z)) imp (P(x) imp Q(x)))     
+

\gfcommand{monad}{first order decider for monadic formulas}
\index{monad}

\gfsyntax{
  monad \ARG{wff} \OPT{by \ARG{fact1} \ARG{fact2} \SEQ};
}

\gfdescription{
It tries to establish whether the input formula is deducible from the
specified facts by using {\tt nnf}, {\tt reduce}, {\tt phexp} and finally
{\tt ptaut}.
}

\gfrecap{
It tries to establish whether the input formula is deducible from the
specified facts by using nnf, reduce, phexp and finally ptaut.
}

\gfexample+
   ***** declare predconst P 1;
   ***** declare funconst f 3;
   ***** declare indvar x y z;
   ***** declare indpar a b;
   
   *****comment | *** MONAD EXAMPLES *** |
   ***** monad forall x. exists y. (P(x) imp P(y));
   1   forall x. exists y. (P(x) imp P(y))
   ***** monad exists y. forall x. (P(x) imp P(y));
   2   exists y. forall x. (P(x) imp P(y))
   ***** monad exists y. forall x. ((P(x) imp P(y)) or P(x));
   3   exists y. forall x. ((P(x) imp P(y)) or P(x))
   
   ***** monad forall x. exists y.
    wffif P(trmif P(y) then x else y)
     then P(trmif P(y) 
             then trmif P(y) 
                   then x 
                   else trmif P(y) 
                         then x 
                         else y 
             else y) 
          or TRUE
     else P(y) or TRUE;
   4   forall x. exists y. 
       (wffif P(trmif P(y) then x else y) 
         then (P(trmif P(y) 
                 then 
                 (trmif P(y) 
                   then x 
                   else (trmif P(y) then x else y)) 
               else y) or TRUE) else (P(y) or TRUE))
   ***** monad forall x. exists y. wffif P(x) then P(y) else not P(y);
   5   forall x. exists y. (wffif P(x) then P(y) else (not P(y)))  
   ***** monad forall y. exists x. (P(f(a,b,x)) or not P(f(a,b,y)));
   6   forall y. exists x. (P(f(a,b,x)) or (not P(f(a,b,y))))
   ***** monad exists z. forall y. exists x. (P(f(z,b,x)) or not P(f(x,b,z)));
   7   exists z. forall y. exists x. (P(f(z,b,x)) or (not P(f(x,b,z)))) 
   ***** monad exists z. forall y. exists x. (P(f(z,b,x)) or not P(f(x,b,y)));
   7   exists z. forall y. exists x. (P(f(z,b,x)) or (not P(f(x,b,y)))) 
+
   
\gfnotes{
   The name ``{\tt monad}" is due to the fact that the monadic predicate
calculus is subset of the class of formulas decided by such a decider.
}   

\gfcommand{monadeq}{first order decider for monadic formulas with equality}
\index{monadeq}

\gfsyntax{
  monadeq \ARG{wff} \OPT{by \ARG{fact1} \ARG{fact2} \SEQ};
}


\gfdescription{
It tries to establish whether the input formula is deducible from the
specified facts by using {\tt nnf}, {\tt reduce}, {\tt phexp} and finally
{\tt ptauteq}.
}

\gfrecap{
It tries to establish whether the input formula is deducible from the
specified facts by using nnf, reduce, phexp and finally ptauteq.
}

\gfexample+
   ***** declare predconst P 1;
   ***** declare funconst f 3;
   ***** declare indvar x y z;
   ***** declare indpar a b;

   *****comment | *** MONADEQ EXAMPLES *** |
   ***** monadeq forall x. exists y. (x=y);
   1   forall x. exists y. (x=y)
   ***** monadeq forall x.  forall y. (x=y imp (P(x) imp P(y)));
   2   forall x.  forall y. (x=y imp (P(x) imp P(y)))
   ***** monad (x=y imp (P(x) imp P(y))) by 2;
   3   (x=y imp (P(x) imp P(y)))  (2)
+

\gfcommand{ptaut}{tautological decider}
\index{ptaut}
\label{sec-decproc}

\gfsyntax{
  ptaut \ARG{wff} \OPT{by \ARG{fact1} \ARG{fact2} \SEQ};
}

\gfdescription{
It decides the quantifier-free formulas provable only by means of the
propositional deductive machinery.
}

\gfrecap{
It decides the quantifier-free formulas provable only by means of the
propositional deductive machinery.
}

\gfexample+
   ***** declare sentconst A B;
   ***** ptaut (A imp (B imp A));
   1   A imp (B imp A)
   ***** assume A B;
   2   A     (2)
   3   B     (3)
   ***** ptaut (A imp (B imp A)) by 2 3;
   4   A imp (B imp A)  (2 3)
   ***** ptaut (A and B) by 2 3;
   5   A and B  (2 3)
   ***** ptaut A by 3;
   PTAUT couldn't prove that A
   is a logical consequence of facts.
   
   ***** declare predconst P 1;
   ***** declare indconst c;
   ***** declare indvar x;

   ***** ptaut (A imp (P(c) imp A));
   6   A imp (P(c) imp A)     
   ***** ptaut (A imp (forall x.P(x) imp A));
   The formula passed to PTAUT is not propositional !
+   

\gfcommand{taut}{tautological first order decider}
\index{taut}

\gfsyntax{
  taut \ARG{wff} \OPT{by \ARG{fact1} \ARG{fact2} \SEQ};
}

\gfdescription{
  The class of formulae decided by {\tt taut} is the set of first order
  formulas provable using only the introduction and elimination rules for 
  the sentential connectives plus the following rule ({\em congruence-rule}):\\
  %
  \[
      \fraz{\forall x A(x)}{\forall y A(y)}
  \]
}

\gfrecap{
The class of formulae decided by `taut' is the set of first order formulas
provable using only the introduction and elimination rules for the 
sentencial connectives plus the following rule (congruence rule)::
              +------------------------------------+
              | forall x. A(x) imp forall y. A(y)  |
              +------------------------------------+
}

\gfexample+
   ***** declare sentconst A;
   ***** declare predconst P 1;
   ***** declare indconst c;

   ***** taut (A imp (P(c) imp A));
   1   A imp (P(c) imp A)

   ***** declare indvar x y [S1];
   ***** declare indvar z [S2];

   ***** taut forall x.P(x) iff forall y.P(y);
   2   forall x.P(x) iff forall y.P(y);
   ***** taut forall x.P(x) iff forall z.P(z);
   TAUT couldn't prove that forall x. P(x) iff forall z. P(z)
   is a tautology.
+

\gfcommand{tauteq}{tautological decider with equality}
\index{tauteq}

\gfsyntax{
  tauteq \ARG{wff} \OPT{by \ARG{fact1} \ARG{fact2} \SEQ};
}

\gfdescription{
  The class of formulae decided by {\tt tauteq} is the set of formulas provable
  using:
  \begin{itemize}
  \item the introduction and elimination rules for the sentential connectives;
  \item the {\em congruence-rule};
  \item the following axioms schemata for equality:
  $$
  \begin{array}{l}
    x=x\\
    (x=y\ \imp\ y=x)\\
    ((x=y\ \con\ y=z)\ \imp\ x=z)\\
    ((x_1=y_1\ \con \ldots \con\ x_n=y_n)\ \imp\ 
    (P(x_1, \ldots \ ,x_n)\ \liff\ P(y_1, \ldots \ ,y_n)))
  \end{array}
  $$
  %
  corresponding to {\em reflexivity, symmetry, transitivity} and {\em substitution 
  into predicates}.
  \end{itemize}
}

\gfrecap{
The class of formulae decided by {\tt tauteq} is the set of formulas provable
using:
* the introduction and elimination rules for the sentential connectives;
* the ``congruence-rule'';
* the following axioms schemata for equality:
     +-------------------------------------------------------------------------+
     |   x = x                                                                 |
     |   x = y imp y = x                                                       |
     |   (x = y and y = z) imp x = z                                           |
     |   (x1 =y1 and ... and xN = yN) imp (P(x1, ..., xN) iff P(y1, ..., yN))  |
     +-------------------------------------------------------------------------+
  corresponding to ``reflexivity'', ``symmetry'', ``transitivity'' and
  ``substitution into predicates''.
}

\gfexample+
   ***** declare predconst P 1;
   ***** declare funconst f 1;
   ***** declare indvar x y;
   ***** declare indvar z;
   ***** tauteq x=x;
   1   x=x
   ***** tauteq x=y imp y=x; 
   2   x=y imp y=x
   ***** tauteq ((x=y and y=z) imp x=z);
   3   (x=y and y=z) imp x=z
   ***** tauteq (x=y imp (P(x) or not P(y)));
   4   x=y imp (P(x) or not P(y))
   ***** tauteq (f(x)=f(y) imp (P(f(x)) iff P(f(y))));
   5   f(x)=f(y) imp (P(f(x)) iff P(f(y)))
   ***** tauteq x=y imp f(x)=f(y);
   TAUTEQ couldn't prove that (x = y) imp (f(x) = f(y))
   is a tautology.
+

 	% introduction to the semantic simplification's section
\newpage
\section{Semantic simplification}
\label{sec-comp}

The subsections \ref{sec-ss-intro}, \ref{sec-ss-model} and \ref{sec-ss-repr} 
of this section have been taken from \cite{rww3}.


\subsection{Introduction}
\label{sec-ss-intro}

{\GF} is intended to express a variety of methods of human reasoning.
Though the word "reasoning" usually connotes a logical deductive process of 
using facts and assertions to obtain conclusions, much of human intelligence 
relies more upon observation than upon deduction.
We look at a book. The book is seen to be "green", as an immediate observation,
not as a deduction involving, say, analysis of wavelengths of light and 
sensory receptors  in the eye. Similarly, humans cross streets without 
conscious analysis of the traffic flow, add numbers without resorting to basic
set theory, and play chess without considering each move in terms of the 
geometry of the board. 

Any system which hopes to express a variety of reasoning processes, therefore 
needs a method of doing purely computational tasks.
In {\GF}, the {\bf semantic interpretation mechanism}, which provides this 
ability, consists of two parts:
\begin{itemize}
\item {\GF}'s {\bf semantic attachment mechanism} permits the user to define a
      ``correspondence'' between the various constants (function symbols,
      predicate constants, individual constants) of the language and
      corresponding objects of the programming language {\HG}.
\item facts about the {\HG} structure can be used directly in the proof
      via the {\bf semantic simplification mechanism}, eliminating the 
      necessity of a possibly complicated deduction.
\end{itemize}
For example, obvious attachments to the function symbol $+$ and to the 
individual constants $17$, $34$, $51$ would allow to conclude $17+34=51$ in 
one step, instead of computing $34$ successors of $17$.
In order to explain this more clearly we first give an informal account of the 
technical details.

\subsection{``Intended'' and ``computational'' models}
\label{sec-ss-model}

The declarations made by a {\GF} user specify a first order language 
$L=\langle P,F,C\rangle$, where $P$ is the list of {\predconst}s, $F$ the list 
of {\funconst}s, and $C$ the list of {\indconst}s (see section \ref{sec-decl}).

A model for such a language is a structure $M=\langle D,P',F',C'\rangle$ where
$D$ is a set and $P'$,$F'$ and $C'$ are lists of predicates over $D$, functions 
over $D$, and individuals of $D$ such that the arities of the symbols in $P$ and 
$F$ match the arities of the predicates and functions at the correspondent 
positions in $P'$ and $F'$.
The idea here is that the language $L$ is used for making statements about 
structures such as $M$ (what we call {\bf ``intended'' or ``standard'' model}). 
In particular, when the user writes down a theory in {\GF}, he generally has 
in mind some particular model for his language, and the axioms of his theory 
are intended to express the properties of this particular model.

The fact that {\GF} is really a {\HG} program running in a LISP 
environment, inspires the following idea: some parts of a model for a {\GF} 
language can often be expressed computationally in the sense that the elements 
of $D$ can be represented by s-expressions, and the predicates and functions 
on $D$ can be represented by {\HG} functions and predicates.
It should then be possible to use the computational representation to aid 
{\GF} deductions concerning the model.
For example, suppose the theory we are interested in, is first order number 
theory, and the model that we have in mind is the set of natural numbers 
together with the operations of successor, addition and multiplication.
The numerals have natural representations as {\HG} numbers, and the 
functions in question have {\tt PLUS1}, {\tt PLUS}, {\tt TIMES} as their {\HG}
counterparts.
As mentioned above it should then be possible to use the computational 
representation to provide swift deductions of such statements as $25+37=52$.

The semantic attachment mechanism in {\GF} allows the user to set up these 
computational representations of his subject matter, and the semantic
interpretation mechanism allows to use these representations to aid deduction 
in {\GF}.

With the above overview in mind, let us proceed to the details.

Given a language $L=\langle P,F,C\rangle$ and a model 
$M=\langle D,P',F',C'\rangle$, we define an interpretation function $I$.
For each {\term} $t$ of $L$ in which no free variable occurs, $I(t)$ is the 
individual in $D$ which $t$ denotes.
In particular we define the interpretation of an {\indconst} $c$ to be the 
individual $c'$ in $D$, and where $f$ is a {\funconst}, and the interpretation 
of {\term} $t_1,\ldots,t_n$ are defined, we inductively define the 
interpretation of the {\term} 
$f(t_1,\ldots,t_n)$ to be $f'(I(t_1),\ldots,I(t_n))$.
We may extend the interpretation function to formulas (again without free 
variables) over $L$ by defining $I(w)$ to be the object {\tt TRUE} exactly 
when the formula $w$ is true of the model (for a technical definition 
see \cite{kleene2}).

When $f'$ is the function in a model corresponding to the {\funconst} $f$ in 
$L$, we will also say that $f'$ is the interpretation of $f$, and similarly 
for predconsts.

Now we define a {\bf computational model} to be an object
$K=\langle D',P'',F'',C''\rangle$, where it is understood that $D'$ is a set
of s-expressions, and $P''$, $F''$ and $C''$ are lists of {\HG} predicates,
functions and s-expressions respectively, with the appropriate restrictions on 
arities.

From the extensional point of view, a computational model is for a language
just like a set-theoretic model for a language, except that we do not require
that the functions and predicates concerned be total; that is functions and
predicates may be undefined (non-terminating) for some elements
of $D'$.

We define an {\bf attachment map} $att$ from terms and formulas of $L$ into
$K$ in a manner exactly analogous to the definition of $I$ given above.

We have one last map to worry about, the map {\bf $rep$} which gives, for each
object in the domain $D'$ of the computational model $K$, the object it
represents in the domain $D$ of the model $M$.

Now we may define precisely the meaning of attachments made in the {\GF}
system: the attachment of an {\indconst} $c$ to an s-expression $c''$
signifies that $c$ and $c''$ represent the same object in the model, that is
to say, $I(c)=rep(c'')$.
Similarly, the attachment of a {\funconst} $f$ to a {\HG} function $f''$
signifies that the result of applying $f''$ to an s-expression $c''$ which
represents an individual $c'$ in the model, is a s-expression which represents
the individual $f'(c')$ in the model.
The analogous statements hold for attachments to {\predconst}s.

The above conditions are equivalent to the statement that the 
diagram in figure \ref{fig-ss} commutes.

\begin{figure}[htb]
\begin{center}
\setlength{\unitlength}{0.0125in}%
\begin{picture}(288,260)(92,540)
\thicklines
\put(340,580){\oval(80,80)}
\put(160,580){\oval(80,80)}
\put(160,760){\oval(80,80)}
\put(200,580){\vector( 1, 0){100}}
\put(194,726){\vector( 1,-1){112}}
\put(160,720){\vector( 0,-1){100}}
\put(340,559){\makebox(0,0)[b]{\raisebox{0pt}[0pt][0pt]{\twlrm model}}}
\put(340,577){\makebox(0,0)[b]{\raisebox{0pt}[0pt][0pt]{\twlrm of intended}}}
\put(340,595){\makebox(0,0)[b]{\raisebox{0pt}[0pt][0pt]{\twlrm Domain }}}
\put(160,562){\makebox(0,0)[b]{\raisebox{0pt}[0pt][0pt]{\twlrm sexpr}}}
\put(160,580){\makebox(0,0)[b]{\raisebox{0pt}[0pt][0pt]{\twlrm {\HG}}}}
\put(160,752){\makebox(0,0)[b]{\raisebox{0pt}[0pt][0pt]{\twlrm Terms}}}
\put(160,770){\makebox(0,0)[b]{\raisebox{0pt}[0pt][0pt]{\twlrm {\GF}}}}
\put(250,560){\makebox(0,0)[b]{\raisebox{0pt}[0pt][0pt]{\twlrm Representation}}}
\put(280,660){\makebox(0,0)[b]{\raisebox{0pt}[0pt][0pt]{\twlrm I}}}
\put(120,660){\makebox(0,0)[b]{\raisebox{0pt}[0pt][0pt]{\twlrm attachment}}}
\end{picture}
\end{center}
\label{fig-ss}
\caption{intended model - computational model mappings}
\end{figure}


\subsection{Multiple representation functions}
\label{sec-ss-repr} 

The semantic attachment mechanism allows several representation of the model 
by {\HG} s-expressions to be in force at the same time.
We will seek to motivate this aspect of the semantic attachment mechanism 
by means of an example: consider a theory of chess with includes a general 
theory of lists as a subtheory (this subtheory would be applied in arguments 
about lists of pieces, lists of game positions and so on).
The intended model of such a theory includes at least two kinds of objects: 
chess positions and lists.
Lists and positions form disjoint domains in the model, though it may be 
possible to build lists of chess position.
If we are going to build a computational representation of this model, we will
need to represent positions and lists by s-expressions in such a way that no 
s-expression represents both a list and a position.
The natural representation of a chess position as an s-expression is as a 
list of eight lists, each of which is a list of eight piece names (one of 
which is "empty" or some such), and the natural representation of lists as 
s-expressions is the direct representation as {\HG} lists.
This representation scheme cannot be used, since it will not be possible to 
decide whether a given list of eight lists of eight piece names represents a 
chess board or a list of list of pieces. 
That is to say, the map $rep$ will not be well defined. 
It is of course not hard to solve this problem by the use of some slightly 
fancier coding, but a general solution to the problem of disambiguating 
computational representations is available.
Suppose that the intended model of a {\GF} theory $T$ includes the disjoint 
domains $D_1,\ldots,D_n$, and suppose further that we have a different coding 
function for each of these domains.
That is we have $n$ different {\bf representation functions} $rep_i$ which map 
the domain of s-expressions into domain of the model, with the property that 
the range of $rep_i$ is a subset of $D_i$.
Then it is possible that a single s-expression codes two different objects 
$d_i$, $d_j$ in the model, but as long as we know what coding function $rep_i$
to apply, there is no ambiguity. 


Then the definition of the $att$ map may be extended to take account of the 
possibility of multiple representations in the following way: the domain of 
the $att$ map will still consist of the set of {\GF} terms and formulas, but
its range will now lie in the set of pairs of the form $\langle$ representation
function, s-expression $\rangle$.

The soundness condition for the $att$ map is now that, when 
$att(t)=\langle rep, c'' \rangle$, we have $rep(c'')=I(t)$.
In order to specify this new more complicated $att$ map, the user of the {\GF}
system must give representation information concerning his attachments.

Specifically, each representation function must be given a name and when the 
attachment to an {\indconst} is given, the name of the associate 
representation function must be given as well.
Similarly, when the attachment $f''$ to a {\funconst} $f$ is specified, the 
(names of the) representations of its arguments and of the value it returns 
must be given, and when the attachment to a {\predconst} is specified, the 
representations of its arguments must also be specified.

The significance of specifying that the representations of the arguments and 
value of the attachment $f''$ to a {\funconst} $f$ are 
$R_1,\ldots,R_n$ and $R_{n+1}$ respectively, is that 
$R_{n+1}(f''(c''_1,\ldots,c''_n))=f'(R_1(c''_1),\ldots,R_n(c''_n))$ 
where $f'$ is the interpretation of $f$, whenever $c''_1$,..,$c''_n$ are 
s-expressions in the domains of $R_1$,..,$R_n$.
The same holds for attachments to {\predconst}, mutatis mutandis.
Given the attachments with representation information for individual symbols,
the map $att$ on the domain of terms and formulas is defined inductively in 
the obvious way: if $f$ is attached to $f''$ and the declared representation 
of the arguments of $f''$ are $R_1,\ldots,R_n$ and terms $t_1$,..,$t_n$ have 
attachments with representations $R_1$,..,$R_n$ then 
$att(f(t_1,\ldots,t_n))=f''(att(t_1),\ldots,att(t_n))$.
Under this definition the diagram above commutes for each individual 
representation function.

Note that if the representation of the attachment of any term $t_i$ does not 
match that of its place in the argument list, then 
$f''(att(t_1),\ldots,att(t_n))$ cannot be expected to represent the 
interpretation of $f(t_1,\ldots,t_n)$.
The reason for this is that the correctness of a computation which purports to
represent a mathematical function depends on the representation of the 
arguments of the function as data objects.
For example, no one would expect a floating point multiplication algorithm to 
behave correctly if its arguments were encoded as integers rather than 
floating point numbers.

Finally, note that the attachment map, as well as the s-expressions which 
represent functions, may be partial.
The user is never required to provide an attachment for any {\GF} symbol, nor is
any attachment to a {\funconst} or {\predconst} required to be complete.

The semantic simplification mechanism will use whatever information is 
available and if there will be insufficient information, it will return this 
fact to the user. 




% user commands for semantic simplification
\gfcommand{attach}{semantic attachment}
\index{attach}

\gfsyntax{
  attach \ARG{indconst}  to \ALT dar [ rep ] \ARG{sexpr};\\
  attach \ARG{sentconst} to T \ALT NIL \ALT UNDEF;\\
  attach \ARG{funconst} \ALT \ARG{predconst} to \ARG{atom};\\
  attach \ARG{funconst}  to [ \ARG{rep1}, \SEQ, \ARG{repN} = \ARG{repM} ]
  \ARG{atom};\\
  attach \ARG{predconst} to [ \ARG{rep1}, \SEQ, \ARG{repN} ] \ARG{atom};
}

\gfdescription{
  Defines the attachment for the {\GF} constants.
  \ARG{repI} can be a representation function or an asterisk; if it is an
  asterisk or no representation is specified, then the default representation
  function {\tt UNIVERSALREP} is taken.
  \ARG{indconst}s can be attached either ``one way'' (using {\tt to}) or ``two ways''
  (using {\tt dar}).
  The ``two ways'' attachment tells the semantic interpretation mechanism 
  that whenever \ARG{sexpr} is computed as the {\HG} representation of 
  a term $t$, then the attached {\GF} \ARG{indconst} should be returned as the 
  simplified version of $t$.
  That is, not only \ARG{sexpr} is the {\HG} representation of \ARG{indconst}, but 
  \ARG{indconst} is the preferred {\GF} name of (the intended model value denoted 
  by) the {\HG} object \ARG{sexpr}.
  \ARG{sentconst}s can be attached to the three possible truth values
  corresponding to true, false and undefined\cite{kleene1}.
  This is done by attaching a sentconst to {\tt T}, {\tt NIL}, {\tt UNDEF}
  respectively.
  \ARG{funconst}s and \ARG{predconst}s can be attached to a {\HG} \ARG{atom}
  \cite{giunchiglia35}.
  \ARG{atom} will be used as the identifier of a {\HG} function, whose number of 
  arguments is supposed to match the arity of the {\GF} symbol.
}

\gfrecap{
Defines the attachment for the GETFOL constants.
`repI' can be a representation function or an asterisk; if it is an
asterisk or no representation is specified, then the default representation
function `UNIVERSALREP' is taken.
`indconst's can be attached either ``one way'' (using `to') or ``two ways''
(using `dar').
The ``two ways'' attachment tells the semantic interpretation mechanism 
that whenever `sexpr' is computed as the HGKM representation of 
a term `t', then the attached GETFOL `indconst' should be returned as the 
simplified version of `t'.
That is, not only `sexpr' is the HGKM representation of `indconst', but 
`indconst' is the preferred GETFOL name of (the intended model value denoted 
by) the HGKM object `sexpr'.
`sentconst's can be attached to the three possible truth values
corresponding to true, false and undefined.
This is done by attaching a sentconst to `T', `NIL', `UNDEF'
respectively.
`funconst's and `predconst's can be attached to a HGKM `atom'
`atom' will be used as the identifier of a HGKM function, whose number of 
arguments is supposed to match the arity of the GETFOL symbol.
}

\gfexample+
   ***** declare indconst a b;
   ***** declare sentconst s;
   ***** declare funconst f 1;
   ***** declare predconst p 1;
   ***** decrep rep;

   ***** attach a to a;
   a attached to 'a
   ***** attach a dar a;
   a attached to 'a
   a is the preferred name of a
   ***** attach a dar [rep]a;
   a attached to 'a
   ***** attach a dar [rep]b;
   a is already an preferred name in this representation
   ***** attach b dar [rep]a;
   a has already a preferred name in this representation

   ***** COMMENT | deflam defines an HGKM function |;
   ***** deflam f(x) x;
   ***** attach f to f;
   f attached to f
   ***** deflam p(x) (IF (EQUAL x (QUOTE a))TRUE FALSE);
   ***** attach p to p;
   p attached to p
   attach f to [repin=repout]f;
   f attached to f
   ***** deflam p1(x) (IF (EQUAL x (QUOTE b))TRUE FALSE);
   ***** attach p to [repin]p1;
   p attached to p1
   ***** attach p to [repin]p1;
   p has already an attachment with these representation informations
   ***** attach s to T;
   s attached to 'T
   ***** attach s to NIL;
   s has already an attachment
+

\gfcommand{decrep}{representation declaration}
\index{decrep}

\gfsyntax{
  decrep \ARG{replabel1} \OPT{\SEQ \ARG{replabelN}};
}

\gfdescription{
  It declares \ARG{replabelI} to be representation functions.\\
  The only builtin representation functions are {\tt NATNUMREP}, {\tt TRUTHREP}
  and {\tt UNIVERSALREP}, the representation functions for natural numbers, for 
  truth values and for default representation respectively.
  Numerals have a builtin attachment to {\HG} numbers in the representation 
  function {\tt NUMERALREP}.
}

\gfrecap{
It declares `replabelI' to be representation functions.
The only builtin representation functions are `NATNUMREP', `TRUTHREP'
and `UNIVERSALREP', the representation functions for natural numbers, for 
truth values and for default representation respectively.
Numerals have a builtin attachment to HGKM numbers in the representation 
function `NUMERALREP'.
}

\gfexample+
   ***** decrep rep1 rep2;
+

\gfnotes{
  Since the intended model itself appears nowhere in the {{\GF}} system, there
  is no need for the user to give any detailed information about the nature of 
  the representation maps which he has in mind.
  {\tt NATNUMREP} is known by {\GF} only after typing {\tt know natnums}.
}

\gfcommand{hardware}{semantic attachment to values of a s-expression}
\index{hardware}

\gfsyntax{
  hardware \ARG{indconst} to \ALT dar \ARG{sexpr};\\
  hardware \ARG{indconst} to \ALT dar [ \ARG{rep} ] \ARG{sexpr};
}

\gfdescription{
  This command is similar to the attach command for \ARG{indconst}.
  The difference is that, if \ARG{sexpr} changes value over time, then so does
  the value of the attachment.
  It is a "dynamic attachment" in the sense that it is attached to the values the
  \ARG{sexpr} assumes over time.
}

\gfrecap{
This command is similar to the attach command for `indconst'.
The difference is that, if `sexpr' changes value over time, then so does
the value of the attachment.
It is a "dynamic attachment" in the sense that it is attached to the values the
`sexpr' assumes over time.
}

\gfexample+
   ***** declare indconst clock t0 t1;
   ***** attach t0 to 0;
   t0 attached to '0
   ***** attach t1 to 1;
   t1 attached to '1
   ***** hardware clock to time;
   clock attached to time
   ***** done;
   >(SETQ time 0)
   0
   >(GETFOL)
   Hi!  Glad your back.  What would you like to talk about now?
   ***** simplify clock = t0;
   1   clock = t0     
   ***** done;
   >(SETQ time 1)
   1
   >(GETFOL)
   Hi!  Glad your back.  What would you like to talk about now?
   ***** simplify clock = t1;
   2   clock = t1     
+

\gfnotes{
  This command gives the possibility of changing the intended model. 
}

\gfcommand{represent}{default representation for sorts}
\index{represent}
\label{sec-rep-sort}

\gfsyntax{
  represent \{ \ARG{sort1}, \SEQ, \ARG{sortN} \} as \ARG{rep} \ALT \ARG{*};
}

\gfdescription{
  Sets the default representation for \ARG{sort1}, \SEQ, \ARG{sortN} to be
  \ARG{rep} ({\tt UNIVERSALREP} if \ARG{*} is specified).
  The default representation is used by the reflect command (see section
  \ref{sec-refl}).
}

\gfrecap{
Sets the default representation for `sort1', ..., `sortN' to be `rep'
(`UNIVERSALREP' if `*' is specified).
The default representation is used by the `reflect' command.
}

\gfexample+
   ***** declare sort s t;
   ***** decrep rep;
   ***** represent {s t} as rep;
   ***** represent {s t} as rep;
   s has already a default representation
+

\gfcommand{simplify}{semantic simplification}
\index{simplify}
\label{sec-simplify}

\gfsyntax{
  simplify \ARG{wff} \ALT \ARG{fact} \ALT \ARG{term};
}

\gfdescription{
  If \ARG{term} is provided as argument, three steps are performed:\\
  %
  \begin{itemize}
  \item 
    the interpretation of \ARG{term} in the computational model is computed;
  \item a preferred name for the interpretation is found;
  \item the equality of \ARG{term} with the preferred name is asserted as the next
    line of the proof.
    No action is taken if \ARG{term} has no interpretation in the computational
    model (has an undefined interpretation) or a preferred name for its
    interpretation does not exist.
  \end{itemize}
  %
  If {\em wff} is provided as an argument, two steps are performed:
  %
  \begin{itemize}
  \item the interpretation of {\em wff} in the computational model is computed 
    (in this case the interpretation will be {\tt TRUE}, {\tt FALSE} or an 
    undefined truth value)
  \item if the interpretation of {\em wff} is {\tt TRUE}, {\em wff} 
    is asserted as the next line of the proof, if it is {\tt FALSE} 
    the negation of {\em wff} is asserted.
  \end{itemize}
  %
  When \ARG{fact} is provided as argument, the simplify command works on the wff
  of \ARG{fact}.
}

\gfrecap{
  If `term' is provided as argument, three steps are performed:
      * the interpretation of `term' in the computational model is computed;
      * a preferred name for the interpretation is found;
      * the equality of $term$ with the preferred name is asserted as the next
        line of the proof. No action is taken if `term' has no interpretation
        in the computational model (has an undefined interpretation) or a
        preferred name for its interpretation does not exist.
  If `wff' is provided as an argument, two steps are performed:
      * the interpretation of `wff' in the computational model is computed 
        (in this case the interpretation will be `TRUE', `FALSE' or an 
        undefined truth value)
      * if the interpretation of `wff' is `TRUE', `wff' is asserted as the next
        line of the proof, if it is `FALSE' the negation of `wff' is asserted.
  When `fact' is provided as argument, the simplify command works on the wff
  of `fact'.
}

\gfexample+
   ***** declare indconst a b c;
   ***** decrep REP;
   ***** attach a dar [REP]a;
   a attached to 'a
   a is the preferred name of a
   ***** attach b dar [REP]b;
   b attached to 'b
   b is the preferred name of b
   ***** attach c to [REP]c;
   c attached to 'c
   ***** declare funconst F 1;
   F has been declared to be a Funconst
   ***** DEFLAM F(x) (IF (EQ x (QUOTE a)) (QUOTE b) 
                         (IF (EQ x (QUOTE b)) (QUOTE c)
                          (QUOTE UNDEF&)));
   ***** attach F to [REP=REP]F;
   F attached to F
   ***** simplify F(a);
   1   F(a) = b    
   ***** simplify F(b);
   F(b) : No simplification is possible.
   ***** simplify F(c);
   F(c) : No simplification is possible.
   ***** declare predconst P 1;
   P has been declared to be a Predconst
   ***** DEFLAM P(x) (IF (EQ x (QUOTE a)) TRUE 
                         (IF (EQ x (QUOTE b)) FALSE 
                          (QUOTE UNDEF&)));
   ***** attach P to [REP]P;
   P attached to P
   ***** simplify P(a);
   2   P(a)  
   ***** simplify P(b);
   3   not P(b)    
   ***** simplify P(c);
   P(c) : No simplification is possible.
   ***** extension UNIVERSAL by {a b c};
   Now the extension of UNIVERSAL is fixed to be : (a b c)
   ***** declare indvar x;
   UNIVERSAL is a sort
   x has been declared to be an Indvar
   ***** simplify exists x.F(x)=b;
   4   exists x. (F(x) = b)
   ***** simplify forall x.P(x);
   5   not forall x. P(x) 
+

\gfnotes{
  In the case of sorts with extensions (see the command {\tt extension} in section 
  \ref{sec-sort}) quantification is considered as {\bf bounded quantification}.
  In other words, let $P$ be a predicate and $x$ an indvar of sort $S$, where
  $S$ has extension $\{s_1,\ldots,s_n\}$.
  Then the following equivalences hold:
  $$ 
  \forall x P(x)\liff (P(s_1) \con \ldots \con P(s_n))
  $$
  $$
  \exists x P(x)\liff (P(s_1) \dis \ldots \dis P(s_n))
  $$
  %
  The command explicitly unfolds universal/existential statements into 
  their propositional equivalents. 
}



% introduction to the syntactic simplification's section
\newpage
\section{Syntactic simplification}
\label{sec-rew}

The subsections \ref{sec-rew-intro} and \ref{sec-rew-simpset} 
of this section have been taken from \cite{rww3}.


\subsection{Introduction}
\label{sec-rew-intro}

The basic idea  of syntactic simplification is repeated substitution of 
selected equalities and equivalences into a given expression.
More precisely, let $E$ be a set of universally quantified equations and 
equivalences ("rewrite rules"), so members of $E$ look like:
\begin{itemize}
\item $\forall\ {\vec x}.(t_1=t_2)$
\item $\forall\ {\vec y}.(F_1\ \liff \ F_2)$
\end{itemize}
where ${\vec x}$ and ${\vec y}$ are the {\indvar} sequences $x_1$...$x_n$
and $y_1$..,$y_m$, $t_1$ and $t_2$ are {\term}s, and $F_1$, $F_2$ are {\wff}s.

A match, or an immediate simplification, of a {{\GF}} expression $exp$ consists
 of replacing an occurrence of $t_1[x \leftarrow u]$($F_1[y \leftarrow v]$) 
in $exp$ by $t_2[x \leftarrow u]$($F_2[y \leftarrow v]$), where $u$($v$) is a 
sequence of terms and where $\leftarrow$ indicates substitution.

There are two problems to solve:
\begin{enumerate}
\item There may be more than one equation (or equivalence) whose left half 
      matches a given expression, so one has to establish a precedence 
      hierarchy for matching.

\item The order used by the algorithm to consider the subexpressions of a 
      given expression.
\end{enumerate}

{{\GF}}'s solution to the first problem is the following ordering expression:
each simplification expression (i.e., left half of a rewrite rule) is 
regarded as a linear string of atoms.
Each atom is either:

\begin{itemize}
\item a {\bf constant} (which is not bound by the universal quantifier in the
                       prefix);

\item an {\bf old variable} (which is bound by the universal quantifier in 
                            the prefix and which has occurred before in the 
                            linear string);

\item a {\bf new variable} (which is bound by the universal quantifiers in the
                           prefix and which has not occurred before in the 
                           linear string);
\end{itemize}

If we think of concatenating different atoms to a given initial string, then 
the atoms have this precedence ordering:
\begin{center}
constants $<$ old variables $<$ new variables
\end{center}
and expressions are ordered lexicographically in accordance with this 
ordering on atoms.

Let's consider, for example, the precedence relations among the simplification
 expressions :
$f(a,b,b)$, $f(a,b,c)$, $f(a,a,x)$, $f(a,x,x)$, $f(a,x,y)$, $f(x,x,x)$,
$f(x,x,y)$, 
where $f$, $a$, $b$, $c$ are constants and $x$, $y$ are variables.

The last four expressions are linearly ordered:
$$
f(a,x,x)<f(a,x,y)<f(x,x,x)f<(x,x,y)
$$
and each of the first three expressions is less than $f(a,x,x)$ and 
incomparable to the other two of the first three expressions:
$$
f(a,b,b)<f(a,x,x)
$$
$$
f(a,b,c)<f(a,x,x)
$$
$$
f(a,a,x)<f(a,x,x)
$$
Together with transitivity, these inequalities completely define the 
precedence relation.

As far as regard the second problem, {{\GF}}'s syntactic simplification code 
basically considers subexpressions of $exp$ in the usual left-to-right order.
The exceptions occur after a subexpression $exp'$ has been matched (and 
substituted for).
The algorithm then begins again at the subexpression one level above $exp'$.

The syntactic simplification algorithm has the usual problems of rewrite rules.
A typical difficulty is the infinitely recurring substitutions: 
for example if one uses $\forall\ x.x+y=y+x$ as simplification equation, 
the algorithm will attempt to make this substitution without end.

\subsection{Simplification sets}
\label{sec-rew-simpset}

Syntactic simplification in {\GF} is performed by using 
{\bf syntactic simplification sets} (called {\bf simpsets} from now on).
Simpsets contain a label ({\em simplabel}) to identify the
rewrite rules used to rewrite expressions.
{\GF} has built-in simpsets, but the user can define his own ones: he can specify a 
set of formulae or facts as rewrite rules in a {\bf basic simspset}
or he can compose already defined simpsets in {\bf compound simpsets}.

The {{\GF}} builtin simpsets (see figure \ref{fig-simpset}) are
{\bf \tt LPROPTREE}, {\bf \tt LQUANTREE}, {\bf \tt LARGIFTREE} and 
{\tt LOGICTREE}.
{\tt LPROPTREE} contains a set of rewrite rules 
corresponding to basic logical equivalences (e.g. $P\con\neg P\liff \bot$). 
{\tt LQUANTREE} contains a set of rewrite rules 
corresponding to logic equivalences for quantified formulas.
{\tt LARGIFTREE} contains a set of rewrite rules corresponding to logic
equalities and equivalences for conditional terms and formulas.
{\tt LOGICTREE} is the union of all the previous builtin simpsets.

\newpage

\begin{figure}[htbp]
\begin{center}
\fbox{
\parbox{16cm}{
$$
\begin{array}{ll}
\neg \neg P \liff P    & \ \ \                         \\
\neg {\tt TRUE} \liff \bot   & \ \ \  \neg \bot \liff {\tt TRUE}    \\
P \con \bot \liff \bot & \ \ \  \bot \con P\liff \bot   \\
P \con {\tt TRUE}  \liff  P  & \ \ \  {\tt TRUE}  \con P\liff P     \\
\neg P\con P\liff\bot  & \ \ \  P \con \neg P\liff \bot \\
P\con P\liff P         & \ \ \                         \\
P     \dis \bot \liff P  & \ \ \  \bot \dis P \liff P       \\
P \dis {\tt TRUE} \liff {\tt TRUE}   & \ \ \  {\tt TRUE}  \dis P \liff {\tt TRUE}   \\
\neg P \dis P\liff {\tt TRUE}  & \ \ \  P \dis \neg P \liff {\tt TRUE}  \\
P     \dis P     \liff P & \ \ \                         \\
P \imp \bot\liff\neg P & \ \ \  \bot \imp P\liff {\tt TRUE}   \\
P\imp {\tt TRUE}\liff {\tt TRUE}   & \ \ \  {\tt TRUE}  \imp P     \liff P\\
\neg P \imp P \liff P  & \ \ \  P \imp \neg P\liff\neg p\\
P \imp P\liff {\tt TRUE}     & \ \ \                         \\
P\liff \bot\liff\neg P  & \ \ \  \bot \liff P\liff\neg P  \\
P\liff {\tt TRUE}\liff P      & \ \ \  {\tt TRUE}  \liff P\liff P     \\
\neg P \liff P\liff\bot & \ \ \  P\liff\neg P\liff\neg\bot\\
P\liff P\liff {\tt TRUE}      & \ \ \                         \\
\end{array}
$$
}}
\fbox{
\parbox{16cm}{
$$
\begin{array}{ll}
\forall x.{\tt TRUE}  \liff {\tt TRUE} & \ \ \  \forall x.\bot \liff \bot    \\
\exists x.{\tt TRUE}  \liff {\tt TRUE} & \ \ \  \exists x.\bot \liff \bot    \\
\end{array}
$$
}}
\vspace{0.3cm} 
\fbox{
\parbox{16cm}{
$$
\begin{array}{ll}
\forall x y.{\em trmif}\ \bot{\em then}\ x\ {\em else}\ y = y    & \ \ \ 
\forall x y.{\em trmif}\ {\tt TRUE}\ {\em then}\ x\ {\em else}\ y\ = x \\
\forall x.  {\em trmif}\ P\ {\em then}\ x\ {\em else}\ x = x & \ \ \  \\
{\em wffif}\ \bot\ {\em then}\ P1\ {\em else}\ P2\ \liff P2 & \ \ \ 
{\em wffif}\ {\tt TRUE}\ {\em then}\ P1\ {\em else}\ P2\ \liff\ P1 \\
{\em wffif}\ P\ {\em then}\ P1\ {\em else}\ P1\ \liff\ P1 & \ \ \  \\
\end{array}
\\
$$
}}
\caption{The rewrite rules of {\tt LPROPTREE}, {\tt LQUANTREE}
\label{fig-simpset}
and {\tt LARGIFTREE}.}
\end{center}
\end{figure}


% user commands for syntactic simplification
\gfcommand{assertsimp}{simpset command}
\index{assertsimp}

\gfsyntax{
  assertsimp \ARG{simplabel}; 
}

\gfdescription{
  It generates a proof step for each formula contained in the rewrite rules
  of \ARG{simplabel}, which cannot be a builtin simpset.
}

\gfrecap{
It generates a proof step for each formula contained in the rewrite rules
of `simplabel', which cannot be a builtin simpset.
}

\gfexample+
   ***** declare indconst a b;
   ***** declare predconst q r 1;
   ***** declare indvar x;
   ***** setbasicsimp s1 at wffs {q(a), forall x.(q(x) iff r(x))};
   ***** assume a=b;
   1   a = b     (1)
   ***** assume q(b);
   2   q(b)     (2)
   ***** setbasicsimp s2 at facts {1,2};
   ***** setbasicsimp s2 at facts {1};
   s2 is already the label of a simpset
   ***** setcompsimp s4 at s1 uni s2;
   ***** assertsimp s1;
   3   q(a)     
   4   forall x. (q(x) iff r(x))     
   ***** assertsimp s2;
   s2 does not contain any wff to assert.
   ***** assertsimp s4;
   5   q(a)
   6   forall x. (q(x) iff r(x))
   ***** assertsimp LOGICTREE;
   LOGICTREE is the label of a builtin simpset, you can't assert it.
+
\gfcommand{rewrite}{syntactic simplifier command}
\index{rewrite}

\gfsyntax{
  rewrite \ARG{wff} \ALT \ARG{fact} \ALT \ARG{term} \OPT{by \ARG{simpexpr}};
}

\gfdescription{
  It rewrites the given expression by using the union of the rewrite rules 
  indicated by the \ARG{simpexpr}.
  If \ARG{simpexpr} is not specified, {\tt LOGICTREE} is used.
  If \ARG{term} is provided as an argument, two steps are performed:
  %
  \begin{itemize}
  \item \ARG{term} is rewritten by using the set of rewrite rules 
    indicated by \ARG{simpexpr}.
  \item the equality of \ARG{term} with its rewritten form is asserted as the next 
    line of the proof. The dependencies depend on the simpsets actually used
    during the syntactic simplification.
  \end{itemize}
  %
  If \ARG{wff} is provided as argument, also two steps are performed:
  %
  \begin{itemize}
  \item \ARG{wff} is rewritten by the set of rewrite rules of the
    \ARG{simpexpr} result.
  \item if \ARG{wff} is rewritten to {\tt TRUE}, \ARG{wff} is asserted as the
    next line in the proof, if \ARG{wff} is rewritten to {\tt FALSE},
    $\neg$ \ARG{wff} is asserted, otherwise the equivalence of \ARG{wff} with 
    its rewritten form is asserted.
    The dependencies depend on the simpsets used during the syntactic
    simplification.
  \end{itemize}
  %
  When \ARG{fact} is provided as an argument, the rewrite command works on the wff
  of the \ARG{fact}.
}

\gfrecap{
It rewrites the given expression by using the union of the rewrite rules 
indicated by the `simpexpr'.
If `simpexpr' is not specified, LOGICTREE is used.
}

\gfexample+
   ***** declare indconst A,B;
   ***** declare indvar X,Y;
   ***** declare funconst F 2;
   ***** declare funconst G 1;
   ***** declare sentconst P;
   ***** assume forall X . F(X,A) = A;
   1   forall X. (F(X,A) = A)     (1)
   ***** assume forall X . F(X,X) = G(X);
   2   forall X. (F(X,X) = G(X))     (2)
   ***** assume forall X Y . F(X,Y) =Y;
   3   forall X Y. (F(X,Y) = Y)     (3)
   ***** axiom F1:forall X . F(X,A) = A;
   F1 : forall X. (F(X,A) = A)
   ***** axiom F2:forall X . F(X,X) = G(X);
   F2 : forall X. (F(X,X) = G(X))
   ***** axiom F3:forall X Y. F(X,Y) = Y;
   F3 : forall X Y. (F(X,Y) = Y)
   ***** setbasicsimp S1 at facts {1};
   ***** setbasicsimp S2 at facts {2};
   ***** setbasicsimp S3 at facts {3};
   ***** setbasicsimp S4 at facts {F1};
   ***** setbasicsimp S5 at facts {F2};
   ***** setbasicsimp S6 at facts {F3};
   ***** setbasicsimp SIMPEQ at wffs {forall X.(X=X iff TRUE)};
   ***** setcompsimp S7 at S1 uni S2 uni S3;
   ***** rewrite F(A,A) by S6;
   4   F(A,A) = A     
   ***** rewrite F(A,A) by S5;
   5   F(A,A) = G(A)     
   ***** rewrite F(A,A) by S4;
   6   F(A,A) = A     
   ***** rewrite F(A,A) by S1;
   7   F(A,A) = A     (1)
   ***** rewrite F(A,A) by S2;
   8   F(A,A) = G(A)     (2)
   ***** rewrite F(A,A) by S3;
   9   F(A,A) = A     (3)
   ***** rewrite F(A,A) by S7;
   10   F(A,A) = A     (1)
   ***** rewrite F(B,B) by S1 uni S3;
   11   F(B,B) = B     (3)
   ***** rewrite F(B,B) by S1;
   F(B,B): No simplification is possible
   ***** rewrite not TRUE by S1;
   not TRUE: No simplification is possible
   ***** rewrite not TRUE;
   12   not (not TRUE)     
   ***** rewrite TRUE imp (P imp X=X);
   13   (TRUE imp (P imp (X = X))) iff (P imp (X = X)) 
   ***** rewrite TRUE imp (P imp X=X) by SIMPEQ uni LOGICTREE;
   14   TRUE imp (P imp (X = X))
   ***** rewrite F(A,A) by S7;
   15   F(A,A) = A     (1)
   ***** rewrite F(A,A)=A by S7;
   16   F(A,A) = A  iff (A = A)   (1)
   ***** rewrite F(A,A)=A by S7 uni SIMPEQ;
   17   F(A,A) = A     (1)
   ***** rewrite F(A,A)=G(A) by S7;
   18   (F(A,A) = G(A)) iff (A = G(A))     (1)
   ***** rewrite F(B,B) by S7;
   19   F(B,B) = G(B)     (2)
   ***** rewrite F(B,B)=G(B) by S7 uni SIMPEQ;
   20   F(B,B) = G(B)     (2)
   ***** rewrite F(B,B)=G(B) and F(A,A)=A by S7 uni SIMPEQ uni LOGICTREE;
   21   (F(B,B) = G(B)) and (F(A,A) = A)     (1 2)
   ***** rewrite F(A,A) by S7 dif S1 ;
   22   F(A,A) = G(A)     (2)
   ***** rewrite F(A,A) by S7 dif (S1 uni S2);
   23   F(A,A) = A     (3)
   ***** rewrite F(A,A)=A by S3 dif (S1 uni S2) uni SIMPEQ;
   24   F(A,A) = A     (3)
+
   


% introduction to the syntactic/semantic simplification's section
\newpage
\section{Syntactic and semantic simplification}

Some of the commands perform both syntactic and semantic simplifications.


% user commands for the syntactic/semantic simplification's section
\gfcommand{eval}{mixed simplifier command}
\index{eval}
\label{sec-eval}

\gfsyntax{
  eval \ARG{wff} \ALT \ARG{fact} \ALT \ARG{term} \OPT{by \ARG{simpexpr}};
}

\gfdescription{
  This command evaluates the expression (\ARG{wff}, the wff of \ARG{fact}
  or \ARG{term} respectively) by combining the semantic evaluation of the
  expression in the simulation structure and the syntactical rewriting
  performed by using the union of the rewrite rules indicated by 
  \ARG{simpexpr}.
}

\gfrecap{
  This command evaluates the expression (`wff', the wff of `fact'
  or `term' respectively) by combining the semantic evaluation of the
  expression in the simulation structure and the syntactical rewriting
  performed by using the union of the rewrite rules indicated by 
  `simpexpr'.
}

\gfexample+
   ***** declare indconst a b c;
   ***** decrep REP;
   ***** attach a dar [REP]a;
   a attached to 'a
   a is the preferred name of a
   ***** attach b dar [REP]b;
   b attached to 'b
   b is the preferred name of b
   ***** attach c dar [REP]c;
   c attached to 'c
   c is the preferred name of c
   ***** declare funconst G 2;
   ***** declare indvar x y;
   ***** setbasicsimp S at wffs {forall x y.G(x y)=x};
   ***** declare predconst P 1;
   ***** DEFLAM P(x) (IF (EQ x (QUOTE a)) TRUE 
                         (IF (EQ x (QUOTE b)) FALSE 
                          (QUOTE UNDEF&)));
   ***** attach P to [REP]P;
   P attached to P
   ***** eval P(G(a,G(b,c))) by S;
   1   P(G(a,G(b,c)))     
   ***** eval P(G(b,c)) and P(c) by S;
   2   not (P(G(b,c)) and P(c))
   ***** eval P(G(c,a)) by S;
   3   P(G(c,a)) iff P(c)   
   ***** eval forall x.P(G(x x)) by S;
   4   forall x. P(G(x,x)) iff forall x. P(x)     
   ***** extension UNIVERSAL by {a b c};
   Now the extension of UNIVERSAL is fixed to be : (a b c)
   ***** eval forall x.P(G(x x)) by S;
   5   not forall x. P(G(x,x)) 
+

\gfnotes{
  In the case of sorts with extensions (see the command {\tt extension} in section 
  \ref{sec-sort}) quantification is considered as  {\bf bounded quantification}.
  In other words, let $P$ be a predicate and $x$ an indvar of sort $S$, where
  $S$ has extension $\{s_1,\ldots,s_n\}$.
  Then the following equivalences hold:
  $$ 
  \forall x(P(x)\liff P(s_1) \con \ldots P(s_n))
  $$
  $$
  \exists x(P(x)\liff P(s_1) \dis \ldots P(s_n))
  $$
  %
  The command explicitly unfolds universal/existential statements into 
  their propositional equivalents. The expansion is performed 
  syntactically, that is the formula $\forall x P(x)$ [$\exists x P(x)$]
  is rewritten as $P(s_1) \con \ldots \con P(s_n)$
  [$P(s_1) \dis \ldots \dis P(s_n)$].
  The expansion mechanism embedded in {\tt simplify} is not used by {\tt eval}.
}


\gfcommand{let}{evaluation plus attachment}
\index{let}

\gfsyntax{
  let \ARG{\indconst} to \ALT dar [ \ARG{rep} ] \ARG{term};
}

\gfdescription{
  This command evaluates the \ARG{term} using the mixed evaluation mechanism
  of the {\tt eval} command.
  If the evaluation returns a {\HG} representation for \ARG{term}, then it is
  attached to  \ARG{indconst} with representation function \ARG{rep}. Then the 
  equality of \ARG{indconst} with \ARG{term} is asserted as the next line in the
  proof, otherwise an error message is given. If \ARG{rep} is not specified the
  representation is the default representation. If this does not happen {\GF}
  outputs an error message.
}

\gfrecap{
This command evaluates the `term' using the mixed evaluation mechanism
of the {\tt eval} command.
If the evaluation returns a HGKM representation for `term', then it is
attached to  `indconst' with representation function `rep'. Then the 
equality of `indconst' with `term' is asserted as the next line in the
proof, otherwise an error message is given. If `rep' is not specified the
representation is the default representation. If this does not happen GETFOL
outputs an error message.
}

\gfexample+
   ***** declare indconst a b c;
   ***** attach b to b;
   b attached to 'b
   ***** attach c to c;
   c attached to 'c
   ***** declare funconst h 2;
   ***** DEFLAM h(x y) (QUOTE d);
   ***** attach h to h;
   h attached to h
   ***** let a dar h(b c);
   a attached to 'd
   a is the preferred name of d
   1   a = h(b,c)     
+



 	% loading introduction to the section
\newpage
\section{Multiple contexts}
\label{sec-cxt}

\subsection{Introduction}

\begin{quote}\em
  ... When reasoning, people seem to be able to switch focus of their
  attention and make always some sort of local reasoning ...
  \cite{giunchiglia2}
\end{quote}

Structuring the knowledge into {\em distinguished partial descriptions of the
world}, has been hinted as a cognitively plausible hypotheses. 
Distinct partial descriptions can be represented by {\GF} {\bf contexts}.
A {\GF} context contains its own language defined by a set of declarations, 
its own axioms and definitions, and its own computational model.
Reasoning can be performed within a context. 
You can type any command defined so far within any context in {\GF}.
Multiple proofs can be performed within a context.
When you work in {\GF} you are always in one context.
The context in which you are working in is called the {\em current context}.
When you enter the system the current context is empty and without name.
If you want to leave the context to work in another one, you have to give the
context a name to refer to it later (by the command {\tt namecontext}).
You can create a new context by using {\tt makecontext}, and  switch to it
by using {\tt switchcontext}.
The context you switch to then becomes the current context.


% loading explanation of commands
\gfcommand{copycontext}{Multiple contexts manipulation}
\index{copycontext}

\gfsyntax{
  copycontext \ARG{ctx-name};
}

\gfdescription{
  A new context with name \ARG{ctx-name} is created, and the current context
  is copied in it.
}

\gfrecap{
  A new context with name `ctx-name' is created, and the current context
  is copied in it.
}

\gfexample+
   ***** show whereami;
   You are now using an unnamed context.
   You are now using an unnamed proof.
   ***** namecontext C1;
   You have named the current context: C1
   ***** show whereami;
   You are now using context: C1
   You are now using an unnamed proof.
   ***** declare sentconst A;
   ***** makecontext C2;
   You have created the empty context: C2
   ***** switchcontext C2;
   You are now using context: C2
   ***** declare indvar A;
+
\gfcommand{copylex}{Language declaration through contexts}
\index{copylex}

\gfsyntax{
	copylex	\ARG{ctx-name};
}

\gfdescription{
	This command copies in the current context all the symbols and sorts
	declared in the context {\em ctx-name}.
	The command has no effects in the case there is at least a symbol in the
	current context that has the same name as a symbol in the context
	{\em ctx-name}.
}

\gfrecap{
This command copies in the current context all the symbols and sorts
declared in the context `ctx-name'.
The command has no effects in the case there is at least a symbol in the
current context that has the same name as a symbol in the context `ctx-name'.
}


\gfexample+
   ***** declare indconst a b;
   ***** declare sentconst A B;
   ***** declare sort S1 S2;
   ***** namecontext C1;
   You have named the current context: C1
   ***** makecontext C2;
   You have created the empty context: C2
   ***** switchcontext C2;
   You are now using context: C2
   You are switching to a proof with no name.
   ***** probe declare;
   ***** copylex C1;
   S1 has been declared to be a sort
   S2 has been declared to be a sort
   A has been declared to be a Sentconst
   B has been declared to be a Sentconst
   a has been declared to be an Indconst
   b has been declared to be an Indconst
   ***** copylex C1;
   COPYLEX cannot be done: A has already been declared
   ***** copylex C2;
   You cannot copy the lex of the current context
+

\gfcommand{makecontext}{Multiple contexts' manipulation}
\index{makecontext}

\gfsyntax{
  makecontext \ARG{ctx-name};
}

\gfdescription{
  A new empty context  with name  \ARG{ctx-name} is created.
}

\gfrecap{
  A new empty context  with name  `ctx-name' is created.
}

\gfexample+
   ***** show whereami;
   You are now using an unnamed context.
   You are now using an unnamed proof.
   ***** namecontext C1;
   You have named the current context: C1
   ***** show whereami;
   You are now using context: C1
   You are now using an unnamed proof.
   ***** declare sentconst A;
   ***** makecontext C2;
   You have created the empty context: C2
   ***** switchcontext C2;
   You are now using context: C2
   ***** declare indvar A;
+

\gfcommand{namecontext}{Multiple contexts' manipulation}
\index{namecontext}

\gfsyntax{
  namecontext \ARG{ctx-name};
}

\gfdescription{
  If the current context has no  name,  it is named with \ARG{ctx-name}.
}

\gfrecap{
  If the current context has no  name,  it is named with `ctx-name'.
}

\gfexample+
   ***** show whereami;
   You are now using an unnamed context.
   You are now using an unnamed proof.
   ***** namecontext C1;
   You have named the current context: C1
   ***** show whereami;
   You are now using context: C1
   You are now using an unnamed proof.
   ***** declare sentconst A;
   ***** makecontext C2;
   You have created the empty context: C2
   ***** switchcontext C2;
   You are now using context: C2
   ***** declare indvar A;
+
\gfcommand{reset}{{\GF} reset}
\index{reset}

\gfsyntax{
  reset;
}

\gfdescription{
  Resets the whole {\GF} system.
}

\gfrecap{
  Resets the whole GETFOL system.
}

\gfexample+
   ***** namecontext c; nameproof p;
   You have named the current context: c
   You have named the current proof: p
   
   ***** show whereami;
   You are now using context: c
   You are now using the proof: p
   
   ***** declare sentconst A;
   A has been declared to be a Sentconst
   ***** assume A;
   1   A     (1)
   ***** show proof;
   1   A     (1)
   ***** makecontext c1;
   You have created the empty context: c1
   ***** switchcontext c1;
   You are now using context: c1
   You are switching to a proof with no name.
   
   ***** reset;
   Resetting the whole GETFOL-system
   
   ***** show whereami;
   You are now using an unnamed context.
   You are now using an unnamed proof.
   
   ***** show proof;
+

\gfcommand{switchcontext}{Multiple contexts' manipulation}
\index{switchcontext}

\gfsyntax{
  switchcontext \ARG{ctx-name};
}

\gfdescription{ 
  Switches from the current context to the context \ARG{ctx-name} which
  becomes the current context.
}

\gfrecap{ 
  Switches from the current context to the context `ctx-name' which
  becomes the current context.
}

\gfexample+
   ***** show whereami;
   You are now using an unnamed context.
   You are now using an unnamed proof.
   ***** namecontext C1;
   You have named the current context: C1
   ***** show whereami;
   You are now using context: C1
   You are now using an unnamed proof.
   ***** declare sentconst A;
   ***** makecontext C2;
   You have created the empty context: C2
   ***** switchcontext C2;
   You are now using context: C2
   ***** declare indvar A;
+

 	% loading introduction to the section
\newpage
\section{Metareasoning}
\label{sec-meta}

\subsection{Introduction}

A special context is {\meta}.
{\GF} recognizes {\meta} as a metatheory of all the other contexts.
The context {\meta} can be used to perform {\em metareasoning}, that is to
describe other contexts and to reason about them. 
Metareasoning in {\meta} is performed by employing the following novel
features:
%
\begin{itemize}
  \item
    the metatheory is, in general, distinct from the object theories it
    describes; 
  \item
    the link between the metalanguage and the object language is not performed
    by encoding, but rather by naming \cite{giunchiglia3}.
    Naming is implemented by using the commands which implement reasoning in
    the computational model of a context (see section \ref{sec-comp}).
    These features are available to the user by the commands \C{attach},
    \C{simplify}, \C{eval} etc.  
  \item
    Metareasoning and object reasoning can be mixed via the reflection rule 
    \cite{giunchiglia3}:

    \begin{equation}
      R_{down}
      \fraz{\der{M} Theorem(``w'')}{\der{O} w}
      \label{refl}
    \end{equation}

    where $M$ and $O$ stand for {\meta} and object theory respectively. 
    This rule is implemented in the {\GF} command \C{reflect}.
    The command knows that some form of metareasoning must be performed in
    {\meta} to deduce the metastatement $Theorem(``w'')$.
    The command can use the reflection rule (\ref{refl}) to assert a new proof
    line in the object level context (the context in which object level 
    reasoning is performed and where the command \C{reflect} is typed in).
  \item
    Any context can be the object level context, {\meta} itself.
    The amalgamation of the object and meta level is a particular case of {\GF}
    metareasoning. 
\end{itemize}

In {\meta}, the user is free to declare any language, any set of axioms and to 
define any computational model. This amounts to say that {\meta} is the 
``metatheory'' of a theory represented in a context as far as the user defines
the appropriate attachments and axioms.
A special unary predicate symbol which can be declared in {\meta} is
{\tt THEOREM}: this is the predicate recognized as meaning theoremhood by the
the reflect rule (\ref{refl}) in the command {\tt reflect}.



% loading explanation of commands
\gfcommand{mattach}{semantic meta attachment}
\index{mattach}

\gfsyntax{
   mattach \ARG{indconst} to \ALT dar \OPT{[rep]}
   \ARG{cname}:\ARG{pname}:\ARG{sort}:\ARG{object};
}

\gfdescription{
   Defines an attachment for a constant of the context {\em meta}.

   \ARG{rep}, if present, can be a representation function or \verb+*+.
   If it is \verb+*+ or no representation is specified, then the default representation function 
   {\tt UNIVERSALREP} is taken.
   \ARG{indconst} is a symbol declared to be an INDCONST in {\em meta}; \ARG{cname} is the name
   of the context to which \ARG{object} belongs; \ARG{pname} is the name of the proof in which
   \ARG{object} is present; \ARG{sort} is a sort of the meta-context associated to one of the
   syntactic categories reported with the {\tt reflect} command; \ARG{object} is an object of
   type \ARG{sort}.

   This command implements the mechanism of ``naming'' symbols or objects belonging to the 
   context \ARG{cname}, {\em ie.} the creation of names denoting objects of \ARG{cname}.
   \ARG{indconst} can be attached ``one way'' (using {\tt to}) or ``two ways'' (using {\tt dar}).
   The ``one way'' attachment tells the semantic interpretation mechanism that \ARG{indconst} is
   the name in {\meta} of the {\GF} object in the context \ARG{cname} corresponding to
   \ARG{object}.
   The two ways attachment tells the semantic interpretation mechanism that whenever the 
   (data structure representing) the \ARG{object} is computed as the representation of a 
   term {\em t}, then \ARG{indconst} should be returned as the simplified version of {\em t}.
}

\gfrecap{
This command implements the mechanism of ``naming'' symbols or objects belonging to the 
context `cname', ie. the creation of names denoting objects of `cname'.
`indconst' can be attached ``one way'' (using `to') or ``two ways'' (using `dar').
The ``one way'' attachment tells the semantic interpretation mechanism that `indconst' is
the name in meta of the GETFOL object in the context `cname' corresponding to
`object'.
The two ways attachment tells the semantic interpretation mechanism that whenever the 
(data structure representing) the `object' is computed as the representation of a 
term `t', then `indconst' should be returned as the simplified version of `t'.
}

\gfexample+
   ***** namecontext META;
   ***** nameproof P1;

   ***** declare indconst sc [SENTCONST];
   ***** declare indconst ic [INDCONST];
   ***** declare indconst vl [FACT];
   ***** declare indconst f1 [FACT];

   ***** DECREP  SENTCONST INDCONST FACT;

   ***** represent { SENTCONST } as SENTCONST;
   ***** represent { INDCONST } as INDCONST;
   ***** represent { FACT } as FACT;

   ***** makecontext C;
   ***** switchcontext C;
   ***** declare indconst c;
   ***** declare sentconst A;
   ***** nameproof P1;
   You have named the current proof: P1

   ***** assume c=c;
   1   c = c     (1)

   ***** makeproof P2;
   You have created the empty proof: P2

   ***** switchproof P2;
   You are now using the proof: P2

   ***** assume A imp A;
   1   A imp A     (1)

   ***** label fact ax = 1;

   ***** switchcontext META;

   ***** MATTACH sc TO  C::SENTCONST:A;
   ctext-get: I changed context to: C
   ctext-get: I changed context to: META
   sc attached to 'A

   ***** MATTACH ic DAR C:P2:INDCONST:c;
   ctext-get: I changed context to: C
   ctext-get: I changed context to: META
   ic attached to 'c
   ic is the preferred name of c

   ***** MATTACH vl DAR [SENTCONST] C:P1:FACT:1;
   ctext-get: I changed context to: C
   proof-get: I changed proof to: P1
   proof-get: I changed proof to: P2
   ctext-get: I changed context to: META
   vl attached to '(1 (= c c) (1) ASSUME (%WFF% = c c))
   vl is the preferred name of (1 (= c c) (1) ASSUME (%WFF% = c c))

   ***** MATTACH f1 TO  C:P2:FACT:1;
   ctext-get: I changed context to: C
   ctext-get: I changed context to: META
   f1 attached to '(1 (imp A A) (1) ASSUME (%WFF% imp A A))

   ***** MATTACH f1 DAR C:P2:FACT:ax;
   ctext-get: I changed context to: C
   ctext-get: I changed context to: META
   f1 attached to '(1 (imp A A) (1) ASSUME (%WFF% imp A A))
   f1 is the preferred name of (1 (imp A A) (1) ASSUME (%WFF% imp A A))

+

\gfnotes{}


\gfcommand{reflect}{reflection}
\index{reflect}
\label{sec-refl}

\gfsyntax{
  reflect \ARG{M-fact} \ARG{arg1} \ARG{arg2} \SEQ \ARG{argN};
}

\gfdescription{
  In the   following description we  call ``object context''  the 
  context where the {\tt reflect} command is executed.

  \ARG{M-fact} is any fact of  the context {\meta} whose wff is of the
  form $\forall x_1 x_2\ldots x_n A(x_1, x_2,\ldots,x_n)$, $(n \geq 0)$,
  where the sorts of the variables $x_1, x_2,\ldots,x_n$ 
  correspond to some {\GF} syntactic category ({\em term, wff, fact} ... ).
  For any syntactic category corresponding to a sort in {\meta},
  {\GF} provides the necessary parsing routine.
  This parsing routine is necessary to run the reflect command.
  The relation between sorts in {\meta} and the associated parsed syntactic
  category is the following:

  \begin{figure}
    \begin{tabular}{|l|l|}
   \hline
   {\bf sort}&  {\bf syntactic category} \hspace{7cm} \\ \hline \hline
   SENTCONST &  a {\em sentconst}; \\
   QUANT     &  a {\em quant} (quantifier: {\tt forall} or {\tt exists}); \\
   SORT      &  a symbol declared as a sort; \\
   DECSYM    &  any declared symbol: {\em sym};\\
   FACT      &  a {\em fact};\\
   WFF       &  a {\em wff}; \\
   WFFIF     &  a {\em wffif};\\
   QUANTWFF  &  a {\em quantwff} (of the form  {\tt forall ... }  or 
                {\tt exists  ... });\\
   AWFF      &  an atomic wff (a wff of the form {\tt P( ... ))};\\
   TERM      &  a {\em term}; \\
   TERMIF    &  a {\em termif};\\
   INDSYM    &  a symbol declared as an {\em indconst} or {\em indvar} or
                {\em indpar}; \\
   INDVAR    &  a symbol declared as an {\em indvar};\\
   INDPAR    &  a symbol declared as an {\em indpar};\\
   INDCONST  &  a symbol declared as an {\em indconst};\\
   SENTSYM   &  a symbol declared as a {\em sentconst} or a {\em sentpar}; \\
   SENTPAR   &  a symbol declared as a {\em sentpar}; \\
   SENTCONST &  a symbol declared as a {\em sentconst}; \\
   APPLSYM   &  a symbol declared as a {\em funconst, funpar, predconst, 
                predpar} \\
             &  or a boolean connective; \\ 
   PREDSYM   &  a symbol declared as a {\em predconst} or {\em predpar}; \\
   PREDPAR   &  a symbol declared as a {\em predpar}; \\
   PREDCONST &  a symbol declared as a {\em predconst}; \\
   FUNSYM    &  a symbol declared as a {\em funconst} or {\em funpar}; \\
   FUNPAR    &  a symbol declared as a {\em funpar}; \\
   FUNCONST  &  a symbol declared as a {\em funconst}; \\ \hline 
   \end{tabular}
   \caption{
	Sorts for {\GF} syntactic categories.
   }
   \end{figure}

   If, for example,  the sort of the variable $x_1$  is SENTCONST,  we
   expect $x_1$ to denote any object that is declared  as  a {\em sentconst}
   in the object  context .  Therefore \ARG{arg1}, \ARG{arg2}, \SEQ, \ARG{argN}
   are objects in the object context, such that all \ARG{arg$_i$} are of
   the syntactic category corresponding to the sort of the variables $x_i$
   in \ARG{M-fact}.

   In the following we give an explanation of the major steps performed during
   the execution of the command {\tt reflect} \cite{giunchiglia3}~\footnote{
   This description is the generalization of the example reported below
   and copied from \cite{giunchiglia3}.}:

   \begin{enumerate}
   \item 
     In the object context, when \C{REFLECT} is executed, {\GF}, after parsing 
     \C{REFLECT}, knows that the next argument is the label of a fact in 
     {\meta}. 
     Thus, {\GF} switches context automatically and goes to {\meta}.
   \item
     In {\meta}, the first argument of the command (\ARG{M-fact}) is parsed 
     and the formula of the fact whose label is \ARG{M-fact} is returned:
     $\forall x_1 x_2\ldots x_n A(x_1, x_2,\ldots,x_n)$.
     The variables $x_1, x_2,\ldots,x_n$ must be instantiated to constants
     in {\meta} which will be the names of the objects in the object context
     \ARG{arg1}, \ARG{arg2}, \SEQ, \ARG{argN}.
     The syntactic type of these objects must correspond to the sort of the 
     variables in {\meta}.
     For instance, if the sort of $x_1$ is WFF, then it must be instantiated
     to a constant denoting a {\it wff} in the object context. At this point
     {\GF} knows that \ARG{arg1} must be a {\it wff} in the object context,
     and so {\em arg}$_1$ can be parsed.
     This step provides {\GF} with the information needed to parse \ARG{arg1},
     \SEQ, \ARG{argN} in the object context.
   \item 
     {\GF} switches to the object context and parses  \ARG{arg1}, \ARG{arg2}, 
     \SEQ, \ARG{argN}, using the parsing functions of the syntactic categories
     corresponding to the variables $x_1, x_2,\ldots,x_n$ respectively.
   \item 
     {\GF} switches to {\meta} and automatically declares $n$ constants in 
     {\meta}. Let them be $c_1, c_2, ... c_n$, with the sorts of $x_1, x_2, 
     ... x_n$ respectively.
     Any constant in $c_1, ... c_n$ is automatically ``attached'' to the objects
     returned by parsing \ARG{arg1}, \ARG{arg2}, \SEQ, \ARG{argN} respectively.
     The representation associated to these constants is the representation
     declared for their sorts (see the command {\bf represent}, section
     \ref{sec-rep-sort}).
   \item 
     Still in {\meta}, an universal elimination is performed on 
     $\forall x_1 x_2\ldots x_n A(x_1, x_2,\ldots,x_n)$ obtaining
     $A(c_1, \ldots ,c_n)$.
   \item 
     Still in {\meta}, the formula $A(c_1,...,c_n)$ is automatically evaluated
     by the command {\tt eval} (see section \ref{sec-eval}).
     In this step {\GF}, by using the command {\bf eval}, performs metalevel
     reasoning by computation in the model.
     This metalevel reasoning is used to compute a formula of the form 
     $Theorem(``w'')$.
     If the metalevel reasoning does not lead to $Theorem(``w'')$, then an error
     message is returned. Otherwise the evaluation of the ground term ``$w$''
     gives $w$, the formula of a fact to be asserted in the object context by
     $R_{down}$
   \item
     At this point the reflection rule $R_{down}$ can be applied.
     Thus {\GF} forgets the constants $c_1 ... c_n$ declared in {\meta} and the
     attachments, switches back to the object context and asserts a new fact
     whose formula is $w$ and whose dependencies are the union of the 
     dependencies of the facts in {\em arg$_1$ ... arg$_n$} if there are any.
   \end{enumerate}

	Let us define two context {\tt OBJ} and {\tt META} as follows:

	\gfsourcefile{example.tst}{
	  NAMECONTEXT META; \\
	  DECLARE SORT TERM WFF;\\
	  DECLARE PREDCONST THEOREM 1;\\
	  DECLARE FUNCONST mkequal (INDVAR, INDVAR) = WFF;\\
	  DECLARE INDVAR x [TERM];\\
	  AXIOM M1: forall x. THEOREM(mkequal(x,x));\\
	  DECREP TERM;\\
	  DECREP WFF;\\
	  REPRESENT \{TERM\} AS TERM;\\
	  REPRESENT \{WFF\} AS WFF;\\
	  ATTACH mkequal TO [TERM,TERM=WFF] mkequ;\\
	  MAKECONTEXT OBJ;\\
	  SWITCHCONTEXT OBJ;\\
	  DECLARE INDCONST c;\\
	  DECLARE INDVAR   x;\\
	  DECLARE FUNCONST f 2;
	}

	Let's now type the following lines in {\GF}:

	\begin{quote}\tt
	   ***** FETCH example.tst;
	   ...

	   ***** REFLECT M1 c;
	   1  c = c;
	   ***** REFLECT M1 f(x,f(c,c));
	   2  f(x,f(c,c)) = f(x,f(c,c))
	\end{quote}

	Let us describe  step by step what happened during  the execution of
	the two  previous command \C{REFLECT}.

	\begin{enumerate}
	\item
    In {\tt OBJ} the word \C{REFLECT}  is parsed.
    The next argument must be  the name of a fact  in {\meta}.  Thus {\GF}
    automatically switches context and goes to {\meta}.
	\item
    In  {\meta}, {\tt M1} is parsed  and the axiom
    {\tt forall  x.THEOREM(mkequ(x,x))} is returned.  The variable {\tt x}
    in {\tt M1} must be instantiated  to a constant  in {\meta} which will
    be the name of a symbol of the language  of {\tt OBJ}.   Since the
    sort of  $x$ is {\em TERM}, then  the symbol of  the language  of {\tt
    OBJ}  must have  syntactic  type {\em term}   (it must be  a term).
	\item
    {\GF} switches to the context {\tt OBJ}.
    In  {\tt OBJ} the   term  {\tt  c} [{\tt f(x,f(c,c))} in  the second case]
    is  parsed.  
	\item
    Since no more arguments are needed, {\GF} switches back to {\meta}.
    In {\meta} a new constant, say {\tt C1} of sort {\tt TERM} is created and
    added to the language of {\meta}.
    {\tt   C1} is {\em attached} to the term {\tt c}  [{\tt f(x,f(c,c))}] 
    of the language of {\tt OBJ}.  
    The representation associated to {\tt C1} is {\tt TERM} which is associated
    to the sort {\tt TERM} by the command {\tt REPRESENT} in example.tst.
    In this way,  {\tt C1} is defined  as the name in  {\meta} 
    of {\tt c} [{\tt f(x,f(c,c))}].   
	\item
    Still in {\meta}, an   universal elimination is performed on {\tt  M1}
    obtaining\\
    {\tt THEOREM(mkequ(C1,C1))}.
	\item
    Still in {\meta}, {\tt  THEOREM(mkequ(C1,C1))} is evaluated in {\meta}'s 
    model.
    {\tt THEOREM} has no interpretation, {\tt mkequ(C1,C1)} evaluates to 
    {\tt c = c} [{\tt f(x,f(c,c)) = f(x,f(c,c))}], namely {\tt mkequ(C1,C1)}
    turns out to be the name of {\tt c = c} [{\tt f(x,f(c,c)) = f(x,f(c,c))}].
    So the result of this step is something that could be written as 
    {\tt THEOREM(``c =  c'')}.
    [{\tt THEOREM(``f(x,f(c,c)) = f(x,f(c,c))'')}],  where {\tt ``c = c''}
    [{\tt ``f(x,f(c,c)) = f(x,f(c,c))''}]  should be read as  the  name  of
    {\tt c = c} [{\tt (f(f(c,c),c),f(x,c)) = f(x,f(c,c))}]
	\item
    At  this point the  reflection rule can be applied.
    {\GF} forgets everything in {\meta}, (in this case {\tt C1} from the
    language and {\tt c} [{\tt f(x,f(c,c))}] from the domain of the
   	interpretation).
    {\GF} switches back to the context {\tt OBJ}, and assert a new fact
    with wff  {\tt  c = c} [{\tt  f(x,f(c,c))}] and with the empty deplist.
    \end{enumerate}

    The following example shows how {\GF} computes the deplist of a fact derived 
    by a {\tt reflect} command whose arguments contain a fact.
}

\gfrecap{
	Reflection
}

\gfexample+
   <host prompt> cat example.tst
   NAMECONTEXT META;\\
   DECLARE SORT FACT WFF;\\
   DECLARE INDVAR fc [FACT];\\
   DECLARE FUNCONST wffof (FACT) = WFF;\\
   DECLARE PREDCONST THEOREM 1;\\
   DECREP FACT;\\
   DECREP WFF; \\
   REPRESENT \{WFF\} AS WFF; \\
   REPRESENT \{FACT\} AS FACT;\\
   ATTACH wffof TO [FACT = WFF] fact\-get\-wff;\\
   AXIOM M2: forall fc. THEOREM(wffof(fc));\\
   MAKECONTEXT OBJ;\\
   SWITCHCONTEXT OBJ;\\
   DECLARE SENTCONST A;

   <host prompt> GETFOL
   ...

   ***** FETCH example.tst;
   ***** ASSUME A;
   1   A     (1)
   ***** REFLECT M2 1;
   2   A     (1)
+



 
	% ........................... QUICK REFERENCE ..........................
	\newpage
	\appendix
	\section{Open problems}
\label{app-op}

If a {\tt reflect} command is used when a context {\tt META} does not
exist the error produced is that the relevant fact can not be found.
The error message should say that {\tt META} does not exist.

\gap
The {\tt subst} command has problems distinguishing
proof line labels from numbers in equalities.

\gap
The {\tt label} command does not check for overlaps between
labels for proof lines, axioms and theories.



	\newpage
	\newcommand{\syndes}[2]{\item[\parbox{\textwidth}{#1}]\hfill #2}
\newcommand{\module}[1]{\subsection{#1}}

\section{Syntax of commands}

\module{admin}

\begin{description}
\syndes{
   comment \ARG{separator} \OPT{\ARG{text}} \ARG{separator}
}
{}
%3rd-begin%3rd-end

\syndes{
   deflam \ARG{funname} \ARG{var-list} \ARG{form};
}
{}
%3rd-begin%3rd-end

\syndes{
   deflam \ARG{funname} \ARG{var-list} \ARG{form};
}
{}
%3rd-begin%3rd-end

\syndes{
   echo \ARG{separator} \OPT{\ARG{text}} \ARG{separator}
}
{}
%3rd-begin%3rd-end

\syndes{
   hgk \ARG{s-expr};
}
{}
%3rd-begin%3rd-end

\syndes{
   hgk \ARG{s-expr};
}
{}
%3rd-begin%3rd-end

\syndes{
   know natnums \OPT{\ARG{natnum1}, \SEQ \ARG{natnumN}};
}
{}
%3rd-begin%3rd-end

\syndes{
   load \ARG{file};
}
{}
%3rd-begin%3rd-end

\syndes{
   resetprompt;
}
{}
%3rd-begin%3rd-end

\syndes{
   setprompt to \ARG{s-expr};
}
{}
%3rd-begin%3rd-end

\syndes{
   setprompt to \ARG{s-expr};
}
{}
%3rd-begin%3rd-end

\syndes{
   show \ARG{option};
}
{}
%3rd-begin%3rd-end

\end{description}

\module{context}

\begin{description}
\syndes{
  copycontext \ARG{ctx-name};
}
{}
%3rd-begin%3rd-end

\syndes{
	copylex	\ARG{ctx-name};
}
{}
%3rd-begin%3rd-end

\syndes{
  makecontext \ARG{ctx-name};
}
{}
%3rd-begin%3rd-end

\syndes{
  namecontext \ARG{ctx-name};
}
{}
%3rd-begin%3rd-end

\syndes{
  reset;
}
{}
%3rd-begin%3rd-end

\syndes{
  switchcontext \ARG{ctx-name};
}
{}
%3rd-begin%3rd-end

\end{description}

\module{decide}

\begin{description}
\syndes{
  decide \ARG{wff}  \OPT{by \ARG{fact1} \ARG{fact2} \SEQ} using 
  \OPT{\{ \ARG{rewriter} \SEQ \}} \ARG{decider};
}
{}
%3rd-begin%3rd-end

\syndes{
  monad \ARG{wff} \OPT{by \ARG{fact1} \ARG{fact2} \SEQ};
}
{}
%3rd-begin%3rd-end

\syndes{
  monadeq \ARG{wff} \OPT{by \ARG{fact1} \ARG{fact2} \SEQ};
}
{}
%3rd-begin%3rd-end

\syndes{
  ptaut \ARG{wff} \OPT{by \ARG{fact1} \ARG{fact2} \SEQ};
}
{}
%3rd-begin%3rd-end

\syndes{
  taut \ARG{wff} \OPT{by \ARG{fact1} \ARG{fact2} \SEQ};
}
{}
%3rd-begin%3rd-end

\syndes{
  tauteq \ARG{wff} \OPT{by \ARG{fact1} \ARG{fact2} \SEQ};
}
{}
%3rd-begin%3rd-end

\end{description}

\module{eval}

\begin{description}
\syndes{
  assertsimp \ARG{simplabel}; 
}
{}
%3rd-begin%3rd-end

\syndes{
  attach \ARG{indconst}  to \ALT dar [ rep ] \ARG{sexpr};\\
  attach \ARG{sentconst} to T \ALT NIL \ALT UNDEF;\\
  attach \ARG{funconst} \ALT \ARG{predconst} to \ARG{atom};\\
  attach \ARG{funconst}  to [ \ARG{rep1}, \SEQ, \ARG{repN} = \ARG{repM} ]
  \ARG{atom};\\
  attach \ARG{predconst} to [ \ARG{rep1}, \SEQ, \ARG{repN} ] \ARG{atom};
}
{}
%3rd-begin%3rd-end

\syndes{
  decrep \ARG{replabel1} \OPT{\SEQ \ARG{replabelN}};
}
{}
%3rd-begin%3rd-end

\syndes{
  eval \ARG{wff} \ALT \ARG{fact} \ALT \ARG{term} \OPT{by \ARG{simpexpr}};
}
{}
%3rd-begin%3rd-end

\syndes{
  hardware \ARG{indconst} to \ALT dar \ARG{sexpr};\\
  hardware \ARG{indconst} to \ALT dar [ \ARG{rep} ] \ARG{sexpr};
}
{}
%3rd-begin%3rd-end

\syndes{
  let \ARG{\indconst} to \ALT dar [ \ARG{rep} ] \ARG{term};
}
{}
%3rd-begin%3rd-end

\syndes{
  represent \{ \ARG{sort1}, \SEQ, \ARG{sortN} \} as \ARG{rep} \ALT \ARG{*};
}
{}
%3rd-begin%3rd-end

\syndes{
  rewrite \ARG{wff} \ALT \ARG{fact} \ALT \ARG{term} \OPT{by \ARG{simpexpr}};
}
{}
%3rd-begin%3rd-end

\syndes{
  setbasicsimp \ARG{simplabel} at wffs \{ \ARG{wff1} \SEQ \ARG{wffN} \};\\
  setbasicsimp \ARG{simplabel} at facts \{ \ARG{fact1} \SEQ \ARG{factN} \};
}
{}
%3rd-begin%3rd-end

\syndes{
  simplify \ARG{wff} \ALT \ARG{fact} \ALT \ARG{term};
}
{}
%3rd-begin%3rd-end

\end{description}

\module{language}

\begin{description}
\syndes{
   awff \ARG{awff};
}
{}
%3rd-begin%3rd-end

\syndes{
   declare \OPT{\ARG{sentsym} \ALT \ARG{indsym}} \ARG{sym1} \SEQ \ARG{symN}; \\
   declare \OPT{\ARG{funsym}  \ALT \ARG{predsym}} \ARG{sym1} \SEQ \ARG{symN}
   \ARG{arity};\\
   declare \OPT{\ARG{funsym} \ALT \ARG{predsym}} \ARG{sym1} \SEQ \ARG{symN}
   1 \OPT{[ pre \OPT{= \ARG{prbp}} ]};\\
   declare \OPT{\ARG{funsym} \ALT \ARG{predsym}} \ARG{sym1} \SEQ \ARG{symN}
   2 \OPT{[ inf \OPT{= \ARG{lbp} \ARG{rbp}} ] };
}
{}
%3rd-begin%3rd-end

\syndes{
  declare sort \ARG{sym1} \SEQ \ARG{symN};
}
{}
%3rd-begin%3rd-end

\syndes{
  declare indconst \ALT indpar \ALT indvar \ARG{sym1} \SEQ \ARG{symN}
  [ \ARG{sortsym} ]; \\
  declare funconst \ALT funpar \ARG{sym1} \SEQ \ARG{symN}
  ( \ARG{sortsym1} \SEQ \ARG{sortsymN} ) = \ARG{sortsym};\\
}
{}
%3rd-begin%3rd-end

\syndes{
  extension \ARG{sort} by \ARG{extexpr};
}
{}
%3rd-begin%3rd-end

\syndes{
  moregeneral \ARG{sort1} < \ARG{sort2}, \SEQ, \ARG{sortN} >;
}
{}
%3rd-begin%3rd-end

\syndes{
  mostgeneral \ARG{sym};
}
{}
%3rd-begin%3rd-end

\syndes{
  setfmap \ARG{funsym} ( \ARG{sym1} \SEQ \ARG{symN} ) = \ARG{sym};
}
{}
%3rd-begin%3rd-end

\syndes{
   term \ARG{term};
}
{}
%3rd-begin%3rd-end

\syndes{
   wff \ARG{wff};
}
{}
%3rd-begin%3rd-end

\end{description}

\module{meta}

\begin{description}
\syndes{
   mattach \ARG{indconst} to \ALT dar \OPT{[rep]}
   \ARG{cname}:\ARG{pname}:\ARG{sort}:\ARG{object};
}
{}
%3rd-begin%3rd-end

\syndes{
  reflect \ARG{M-fact} \ARG{arg1} \ARG{arg2} \SEQ \ARG{argN};
}
{}
%3rd-begin%3rd-end

\end{description}

\module{nd}

\begin{description}
\syndes{
   alle \ALT us \ARG{fact} \OPT{,} \ARG{term1} \ARG{term2} \SEQ;
}
{}
%3rd-begin%3rd-end

\syndes{
   alli \ALT ug \ARG{fact} \OPT{\OPT{,} \ARG{indvar1} \ALT \ARG{indpar1} :} 
                           \ARG{indvar11}
                           \OPT{\OPT{,} \ARG{indvar2} \ALT \ARG{indpar2} :}
                           \ARG{indvar22} \SEQ;
}
{}
%3rd-begin%3rd-end

\syndes{
   ande \ALT ae \ARG{fact} \OPT{,} 1 \ALT 2; \\
   ande \ALT ae \ARG{fact} \OPT{,} 1 \ALT 2 1 \ALT 2 \SEQ;
}
{}
%3rd-begin%3rd-end

\syndes{
   andi \ALT ai \ARG{fact1} \OPT{,} \ARG{fact2}; \\
   andi \ALT ai \ARG{fact11}
                \OPT{conj \ALT cj \ARG{fact12} conj \ALT cj \ARG{fact13} \SEQ}
                \OPT{,}
                \ARG{fact21}
                \OPT{conj \ALT cj \ARG{fact22} conj \ALT cj \ARG{fact23} \SEQ};
}
{}
%3rd-begin%3rd-end

\syndes{
   assume \ARG{wff1} \OPT{\OPT{,} \ARG{wff2} \SEQ};
}
{}
%3rd-begin%3rd-end

\syndes{
   existe \ALT es \ARG{fact}  \OPT{,} \ARG{indvar1} \ALT \ARG{indpar1}
                  \OPT{,} \ARG{indvar2} \ALT \ARG{indpar2} \SEQ;
}
{}
%3rd-begin%3rd-end

\syndes{
  existi \ARG{fact}
   \OPT{\OPT{,} \ARG{term1} :} \ARG{indvar1} \OPT{occ \ARG{n11} \ARG{n12} \SEQ}
   \OPT{\OPT{,} \ARG{term2} :} \ARG{indvar2} \OPT{occ \ARG{n21} \ARG{n22} \SEQ}
   \SEQ;
}
{}
%3rd-begin%3rd-end

\syndes{
   falsee \ALT fe \ARG{fact1} \OPT{,} \ARG{wff};\\
   falsee \ALT fe \ARG{fact1} \OPT{,} \ARG{fact2};
}
{}
%3rd-begin%3rd-end

\syndes{
   falsei \ALT fi \ARG{fact1} \OPT{,} \ARG{fact2}; 
}
{}
%3rd-begin%3rd-end

\syndes{
   iffe \ALT ie \ARG{fact} \OPT{,} 1 \ALT 2;
}
{}
%3rd-begin%3rd-end

\syndes{
   iffi \ALT ii \ARG{fact1} \OPT{,} \ARG{fact2};
}
{}
%3rd-begin%3rd-end

\syndes{
   impe \ALT mp \ARG{fact1} \OPT{,} \ARG{fact2};
}
{}
%3rd-begin%3rd-end

\syndes{
   impi \ALT ded  \ARG{fact1} \OPT{, \ALT imp} \ARG{fact};\\
   impi \ALT ded  \ARG{wff} \OPT{, \ALT imp} \ARG{fact};
}
{}
%3rd-begin%3rd-end

\syndes{
   note \ALT ne \ARG{fact1} \OPT{,} \ARG{wff}; \\
   note \ALT ne \ARG{fact1} \OPT{,} \ARG{fact2};
}
{}
%3rd-begin%3rd-end

\syndes{
   noti \ALT ni \ARG{fact1} \OPT{,} \ARG{wff}; \\
   noti \ALT ni \ARG{fact1} \OPT{,} \ARG{fact2};
}
{}
%3rd-begin%3rd-end

\syndes{
   ore \ALT oe \ARG{fact1} \OPT{,} \ARG{fact2} \OPT{,} \ARG{fact2};
}
{}
%3rd-begin%3rd-end

\syndes{
   ori \ALT oi \ARG{fact} \OPT{,} \ARG{wff} \OPT{,} \OPT{lr \ALT rl};\\
   ori \ALT oi \ARG{fact} \OPT{,} \ARG{fact1} \ALT \ARG{wff1} 
       disj \ALT dj \ARG{fact2} \ALT \ARG{wff2} disj \ALT dj \SEQ
       \OPT{,} \OPT{lr \ALT rl};
}
{}
%3rd-begin%3rd-end

\syndes{
  subst \ARG{fact1} \OPT{with} \ARG{fact2};\\
  subst \ARG{fact1} \OPT{with} \ARG{fact2} \OPT{right \ALT left};\\
  subst \ARG{fact1} \OPT{with} \ARG{fact2} 
                    \OPT{occ \ARG{n1} \ARG{n2} \SEQ} \OPT{right \ALT left};
}
{}
%3rd-begin%3rd-end

\end{description}

\module{parser}

\begin{description}
\syndes{
  backup \ARG{file} open;\\
  backup \ARG{file} close;  
}
{}
%3rd-begin%3rd-end

\syndes{done;}
{}
%3rd-begin%3rd-end

\syndes{
   fetch \ARG{file} \OPT{from \ARG{mark1}} \OPT{to \ARG{mark2}};
}
{}
%3rd-begin%3rd-end

\syndes{
   mark \ARG{sym};
}
{}
%3rd-begin%3rd-end

\syndes{
   probe;\\
   probe \ARG{activity};\\
   probe all;\\
}
{}
%3rd-begin%3rd-end

\syndes{
   unprobe \ARG{activity};\\
   unprobe all;
}
{}
%3rd-begin%3rd-end

\end{description}

\module{proof}

\begin{description}
\syndes{
  axiom \ARG{sym} : \ARG{wff};
}
{}
%3rd-begin%3rd-end

\syndes{
   cancel \OPT{\ARG{label}};
}
{}
%3rd-begin%3rd-end

\syndes{
  copyproof \ARG{prf-name};
}
{}
%3rd-begin%3rd-end

\syndes{
   label fact \ARG{sym};\\
   label fact \ARG{sym} = \ARG{label};
}
{}
%3rd-begin%3rd-end

\syndes{
   makeproof \ARG{prf-name};
}
{}
%3rd-begin%3rd-end

\syndes{
   nameproof \ARG{prf-name};
}
{}
%3rd-begin%3rd-end

\syndes{
  switchproof \ARG{prf-name};
}
{}
%3rd-begin%3rd-end

\syndes{
  theorem \ARG{sym} \ARG{hook};
}
{}
%3rd-begin%3rd-end

\syndes{
  theory \ARG{thlabel} : \ARG{wff1} \OPT{\ARG{wff2} \SEQ};\\
  theory \ARG{thlabel} : \ARG{axlabel1} : \ARG{wff}
                         \OPT{\ARG{axlabel2} : \ARG{wff} \SEQ};
}
{}
%3rd-begin%3rd-end

\end{description}

\module{rules}

\begin{description}
\syndes{
	contract \ALT ctc  \ARG{fact} by \ARG{assumption1} \SEQ  \ARG{assumptionN}; 
}
{}
%3rd-begin%3rd-end

\syndes{
	cut \ARG{fact1} \ARG{fact2};\\
	cut \ARG{fact1} \ARG{fact2} \OPT{keep \ARG{assumption1} \SEQ
	\ARG{assumptionN}};
}
{}
%3rd-begin%3rd-end

\syndes{
	termife \ARG{fact1} \ARG{fact2} \ARG{termif}; \\
	termife \ARG{fact1} \ARG{fact2} \ARG{termif} \OPT{occ \ARG{n1} \ARG{n2}
	\SEQ}; 
}
{}
%3rd-begin%3rd-end

\syndes{
	termifen \ARG{fact1} \ARG{fact2} \ARG{termif}; \\
	termifen \ARG{fact1} \ARG{fact2} \ARG{termif} \OPT{occ \ARG{n1} \ARG{n2}
	\SEQ}; 
}
{}
%3rd-begin%3rd-end

\syndes{
	termifi \ARG{fact1} \ARG{fact2} \ARG{wff} \ARG{term1} \ARG{term2};
}
{}
%3rd-begin%3rd-end

\syndes{
	weaken \ALT wk \ARG{fact} by \ARG{fact1} \SEQ \ARG{factN};
}
{}
%3rd-begin%3rd-end

\syndes{
	wffife \ARG{fact1} \ARG{fact2};
}
{}
%3rd-begin%3rd-end

\syndes{
	wffifen \ARG{fact1} \ARG{fact2};
}
{}
%3rd-begin%3rd-end

\syndes{
	wffifi \ARG{wff} \ARG{fact1} \ARG{fact2};
}
{}
%3rd-begin%3rd-end

\end{description}



	% ............................. BIBLIOGRAPHY ...........................
	\newpage
	\bibliographystyle{alpha}
    \gfbibliography
	
	% ................................ INDEX ...............................
	\newpage
	%............................... USER MANUAL .................................
%.............................................................................

\documentstyle[12pt]{../styfiles/GFmanual}

\title{GETFOL Manual}
\author{\bf Fausto Giunchiglia}
\date{7 March 1994}
\version{2.0}
\abstract{
	  {\GF} is an interactive reasoning system.
	  We use it as an environment for studying epistemological issues.
	  We try to look at questions like: which notions are
	  important for the development of mechanized reasoning systems?
	  What kind of conversations do we want to have with them?
	  What parts of logic should we use to represent such notions?
	  How should logic be embedded in a conversational reasoning system?
	}
\addresses{
     \begin{tabular}[c]{l}
       {\bf Fausto Giunchiglia}           \\
       Mechanized Reasoning Group		  \\
		 IRST, Povo, 38050 Trento, Italy  \\
		 e-mail: {\tt fausto@irst.it}     \\
		 phone: +39 461 314436
	\end{tabular}	
}
\published{
  \begin{tabular}{l}
	  DIST Technical Report No. 92-0010 (1994). \\
	  DIST -- University of Genoa,\\
	  Via Opera Pia 11A, 16145 Genova, Italy.\\ \\
  \end{tabular}
}

%% \newcommand{\gfbibliography}{%
%% \bibliography{/home/tarski/staff/mrg/biblio/bib/a-l,%
%% /home/tarski/staff/mrg/biblio/bib/m-z,userman}}
\newcommand{\gfbibliography}{\bibliography{}}

%% \makeindex
\begin{document}
	%  ............................. COVER ..................................
	\thispagestyle{empty}
	\maketitle

	%  ........................ TABLE OF CONTENTS ...........................
	\newpage
	\pagenumbering{roman}
	\tableofcontents

	\newpage
	\pagenumbering{arabic}
	\pagestyle{headings}

	%  ........................... INTRODUCTION ............................
	\newcommand{\eg}{{\em e.g.~}}
\newcommand{\ie}{{\em i.e.~}}
\newcommand{\wrt}{w.r.t.~}
\newcommand{\co}[2]{\langle #1, \: #2 \rangle}


\section{Decision procedures}
\label{sec-decide}
\label{system:sec}
A detailed description of the main decision procedures of {\tt GETFOL}
is given in \cite{armando5}.

The set of procedures of the {\tt GETFOL} system is depicted in figure
\ref{system:fig}.
Each box represents either a decider ({\tt PTAUT}, {\tt PTAUTEQ},
{\tt FOLTAUT}) or a rewriting procedure ({\tt tautren}, {\tt  phexp},
{\tt  reduce}).

\begin{figure}
\begin{center}
\makebox[3.375in][l]{
  \vbox to 2.750in{
    \vfill
    \special{psfile=decide/NEWFIG.PS}
  }
  \vspace{-\baselineskip}
}
\end{center}
\caption{The system of deciders}
\label{system:fig}
\end{figure}

\subsubsection*{{\tt PTAUT} and {\tt PTAUTEQ}}
{\tt PTAUT} and {\tt PTAUTEQ}
are deciders working on a quantifier-free first order language (hereon by
abuse of language we call them propositional deciders).
{\tt PTAUT} decides the set of first order formulas provable using
only the propositional deductive machinery (moreover it returns a
falsifying assignment whenever the input formula is not a tautology).
For instance, the formula $(P(x)\con R(x,b))\imp (P(x)\dis R(x,b))$ can be
easily inferred by a single application of {\tt PTAUT}.
{\tt PTAUT} is a generalization of the Davis-Putnam-Loveland procedure
(hereon DPL) \cite{davis2,davis6} to non clausal formulas.
The core of {\tt PTAUT} is a procedure capable of partially evaluating
the input formula \wrt a partial assignment of truth-values to the atomic
subformulas. 
A step of statistical analysis (of polynomial time complexity) collects
information about the {\em polarity} of the subformulas and the existence
of {\em Top-Level Disjunctive Occurrences} (TLDO) of atomic subformulas.
A formula $\alpha$ occurs as a TLDO in $\beta$ 
if and only if $\beta$ can be rewritten into a formula either of the form
$(\alpha\dis\gamma)$ or $(\neg\alpha\dis\gamma)$ by means of rules
expressing  the usual properties of the propositional connectives such as
the associativity, commutativity and distributivity of the propositional
connectives.
The notion of positive (negative) subformula occurrence
is inductively defined as follows: $\alpha$ occurs positively in $\alpha$, 
$\alpha$ occurs negatively in $\neg\alpha$;
$\alpha$ and $\beta$ occur positively in $(\alpha\con\beta)$ and
$(\alpha\dis\beta)$;
$\alpha$ occurs negatively and $\beta$ occurs positively in $(\alpha\imp\beta)$;
finally $\alpha$ and $\beta$ occur both positively and negatively in
$(\alpha\liff\beta)$.
A subformula $\alpha$ is positive (negative) in $\beta$ if and only if
each occurrence of $\alpha$ occurs positively (negatively) in $\beta$.

The statistical analysis may suggest a partial assignment $\mu$ (\wrt which
the formula can be simplified) according to the following criteria:
\begin{itemize}
\item for each positive (negative) atomic formula $\alpha$ occurring
in $\beta$, $\mu(\alpha)=F$ ($\mu(\alpha)=T$);
\item if $\beta$ contains a positive (negative) TLDO of $\alpha$ and there
are no negative (positive) TLDO of $\alpha$, then $\mu(\alpha)=F$
($\mu(\alpha)=T$).
\end{itemize}

If $\mu$ is not completely undefined, then {\tt PTAUT} simplifies the
formula in input \wrt $\mu$ and recurs on the resulting (simplified) formula.
If the input formula contains both a positive and a negative TLDO of an
atomic formula the input formula is a tautology.
These optimizations generalize the {\em Affirmative-Negative Rule} and the
{\em Rule for the Elimination of One-Literal Clauses} of DPL.
If $\mu$ is totally undefined, then
an atomic formula is chosen, two partial assignments are created
(one assigning $T$, the other $F$ to the chosen atomic formula),
the formula is partially evaluated \wrt such assignments and finally
the procedure branches recurring on the two simplified formulas.
This last step generalizes the {\em Splitting Rule} of DPL.

{\tt PTAUTEQ} is the result of adapting {\tt PTAUT}
to take into account the properties of equality.
The main difference is that, before a formula is simplified \wrt some
assignment, the assignment is tested to check whether it is model of the
quantifier-free theory of equality.
The formula $(P(x)\con x=y)\imp (P(y)\con y=x)$ can be
easily inferred by a single application of {\tt PTAUTEQ}.

\subsubsection*{{\tt nnf} and {\tt skolemize}}
{\tt nnf} rewrites the input formula into a logically equivalent one in
{\em negative normal form}.

{\tt skolemize} computes the skolemization of the input formula.

\subsubsection*{{\tt tautren} and {\tt phexp}}
The procedures on top of the propositional deciders (namely {\tt tautren}
and {\tt phexp}) map the first-order formula in input into a quantifier-free
formula.
The mappings are such that the decision problem of the input (first-order)
formula is related to the decision problem of the returned (quantifier-free)
formula in a useful way.
In particular, {\tt tautren} atomizes equal (modulo renaming of bound
variables) quantified subformulas into newly introduced propositional
letters.
For instance the formula
\begin{equation}\label{pb29-reduced}
%\setlength{\templength}{\arraycolsep}
\setlength{\arraycolsep}{0cm}
\begin{array}{rl}
(\exists x.F(x) \con \exists x.G(x)) \imp (&( \forall x.(F(x) \imp H(x)) \con \forall x.(G(x) \imp J(x))) \liff \\
& ((\exists y.G(y) \imp \forall x.(F(x) \imp H(x))) \con\\
&\ (\exists x.F(x) \imp \forall y.(G(y) \imp J(y)))))
\end{array}
\end{equation}
is mapped into the propositional formula
\begin{equation}\label{pb29-prop}
(A \con B) \imp ((C \con D) \liff ((B \imp C) \con (A \imp D)))
\end{equation}
The relation between the decision problems of the input formula (say $\alpha$)
and of the output formula (say $\alpha'$) is that
$\der{}\alpha'$ only if $\der{}\alpha$.

A more careful reduction to the quantifier-free fragment is performed by
{\tt phexp}.
{\tt phexp} maps an existential formula $\alpha$ into a quantifier-free formula
$\alpha'$ such that $\der{}{\alpha'}$ if and only if $\der{}{\alpha}$.%
\footnote{The set of existential formualas is the class of prenex
universal-existential formulas without function symbols.}

The formula $\alpha'$ is an improved version of the Herbrand's expansion
of $\alpha$ \cite{dreben1}.
An application of {\tt phexp} to the following formula:
\begin{equation}\label{pb28}\small
    (((P(x) \con \neg Q(y)) \dis
     ((Q(a) \dis R(a)) \con (\neg Q(b) \dis \neg S(b)))) \dis
     ((F(z) \con \neg G(z)) \con S(v))) \dis
       ((\neg P(c) \dis \neg F(c)) \dis G(c))
\end{equation}
yields
\begin{equation}\label{pb28-exp}\small
\begin{array}{l}
((((P(a) \dis P(b) \dis P(c)) \con (\neg Q(a) \dis \neg Q(b) \dis
\neg Q(c)))\dis\\
((Q(a) \dis R(a)) \con (\neg Q(b) \dis \neg S(b)))) \dis \\
(((F(a) \con \neg G(a)) \dis (F(b) \con \neg G(b)) \dis
(F(c) \con \neg G(c)))\con\\
(S(a) \dis S(b) \dis S(c)))) \dis ((\neg P(c) \dis \neg F(c)) \dis G(c))\\
\end{array}
\end{equation}
In \cite{armando5} it is shown that, the size of (\ref{pb28-exp})
is 44 times smaller than the size of the standard Herbrand's expansion of
(\ref{pb28}).

\subsubsection*{{\tt reduce}}
{\tt reduce} tries a set of rewriting rules on the input
formula aiming at rewriting it into a logically equivalent formula that
can be easily turned into an existential one via skolemization.
The rewriting rules employed by {\tt reduce} are the usual rules
expressing the distributivity of quantifiers through propositional connectives
and the commutativity and associativity of propositional connectives
listed in the following table.\\

    \renewcommand{\arraystretch}{1.5}
    {\small
      $$
      \begin{array}{|c|rcl|} \hline
        (1) & Q x. \alpha[x] & \mapsto & \alpha \\ \hline
        %(2) & Q x. (\neg \alpha(x)) & \mapsto & (\neg \hat{Q} x. \alpha(x)) \\ \hline
        (2) & Q x. (\alpha \circ \beta)(x) & \mapsto & (Q x. \alpha \circ Q x. \beta) 
        \\ \hline
        (3) & Q x. (\alpha[x] + \beta(x)) & \mapsto & (\alpha[x] + Q x. \beta(x)) 
        \\ \hline \hline
        (4) & (\alpha(x) + \beta[x]) & \mapsto & (\beta[x] + \alpha(x)) \\ \hline
        (5) & ((\alpha[x] + \beta(x)) + \gamma(x)) & \mapsto & 
        (\alpha[x] + (\beta(x) + \gamma(x))) \\ \hline
        (6) & ((\alpha \circ \beta)(x) + \gamma(x)) & \mapsto & 
        ((\alpha + \gamma(x)) \circ (\beta + \gamma(x))) \\ \hline
        (7) & (\alpha(x) + (\beta[x] + \gamma(x))) & \mapsto &
        (\beta[x] + (\alpha(x) + \gamma(x))) \\ \hline
        (8) & ((\alpha(x) + (\beta \circ \gamma)(x))) & \mapsto &
        ((\alpha(x) + \beta) \circ (\alpha(x) + \gamma)) \\ \hline
      \end{array}
      $$
      }
    \renewcommand{\arraystretch}{1}
{\small
{\em Restrictions}: 
\begin{itemize}
\item In rules $\{(4)-(8)\}$ the left hand side must be a top normalizable
formula.
\item In rules $\{(7),(8)\}$ $\alpha$ must be minimal \wrt $\co{Q}{x}$.
%\item Rules $\{(4)-(8)\}$ can be applied only to subformulae (say $\alpha$)
%of a formula $Qx.\beta$ in which there is no proper
%subformula $Qy.\gamma$ of which $\alpha$ is a subformula.
\end{itemize}}

Where
$\alpha(x)$ denotes a formula in which there is at least one free occurrence
of the variable $x$.
$\alpha{[x]}$ denotes a formula in which there is no free occurrences of $x$.
$Q$ and $Q'$ stand either for $\forall$ or for $\exists$.
If $Q = \forall$, then $\circ = \con$ and $+ = \dis$.
If $Q = \exists$, then $\circ = \dis$ and $+ = \con$.
%$\con$ is said to be $\forall$-compatible and $\exists$-incompatible,
%$\dis$ is said to be $\exists$-compatible and $\forall$-incompatible.
%If $\cal S$ is a set of rewriting rules then $\mapsto_{\cal S}$ is the
%reducibility relation induced by $\cal S$ and
%$\stackrel{*}{\mapsto}_{\cal S}$ is the reflexive and transitive closure
%of $\mapsto_{\cal S}$.
The definition of {\em top normalizable formula} and of {\em minimal
formula} are given in \cite{armando5}.

For instance, a single application of {\tt reduce} turns the formula
\begin{equation}\label{pb29}
%\setlength{\templength}{\arraycolsep}
\setlength{\arraycolsep}{0cm}
\begin{array}{rl}
(\exists x.F(x) \con \exists x.G(x)) \imp (&( \forall x.(F(x) \imp H(x)) \con \forall x.(G(x) \imp J(x))) \liff \\
&(\forall x.\forall y.((F(x) \con G(y)) \imp (H(x) \con J(y)))))
\end{array}
\end{equation}
into (\ref{pb29-reduced}).
{\tt reduce} considerably enlarges the set of formulas which can be solved
by using the system of deciders.
In particular, any prenex first order formula 
$$
\forall \vec{y}_n \exists \vec{x}_n \ldots
\forall \vec{y}_i \exists \vec{x}_i \ldots
\forall \vec{y}_1 \exists \vec{x}_1 . \Phi
$$
such that each literal in $\Phi$ contains no variables in $\vec{y}_k$ and
in $\vec{x}_l$ with $k < l$, or in $\vec{x}_k$ and in $\vec{x}_l$ with
$k \neq l$ can be ``reduced" to an existential formula.
On the basis of the previous result it is a trivial consequence
that the {\em monadic calculus} can be reduced to the existential fragment
by means of {\tt reduce}.


	%  .............................. MODULES ..............................
	% loading introduction to the section
\input{parser/intropar}

% loading command files
\input{parser/backup}
\input{parser/done}
\input{parser/fetch}
\input{parser/mark}
\input{parser/probe}
\input{parser/unprobe}

	% loading introduction to the section
\newpage
\input{admin/introadm.tex}

% loading explanation of commands
\input{admin/comment}
\input{admin/deflam}
\input{admin/echo}
\input{admin/hgk}
\input{admin/know}
\input{admin/load}
\input{admin/resprmpt}
\input{admin/setprmpt}
\input{admin/show}

 	% introduction to the language's section
\newpage
\input{language/introlang}

% user commands for language
\input{language/awff}
\input{language/declare1}
\input{language/term}
\input{language/wff}

% introduction to the sort's section
\newpage
\input{language/introsort}

% user commands for sorts
\input{language/declare2}
\input{language/declare3}
\input{language/extension}
\input{language/moregeneral}
\input{language/mostgeneral}
\input{language/setfmap}

 	% introduction to facts: the reasoning's building blocks!
\newpage
\input{proof/introfct}

% user command: axioms, theories, theorems 
\input{proof/axiom}
\input{proof/theorem}
\input{proof/theory}

% introduction to the multiple proof's section
\newpage
\input{proof/introprf}

% user commands for multiple proofs
\input{proof/cancel}
\input{proof/copyproof}
\input{proof/label}
\input{proof/makeproof}
\input{proof/nameproof}
\input{proof/switchproof}


 	% introduction to the natural deduction's section
\newpage
\input{nd/intrond}

% user commands for nd
\input{nd/assume}
\input{nd/ande}
\input{nd/andi}
\input{nd/falsee}
\input{nd/falsei}
\input{nd/iffe}
\input{nd/iffi}
\input{nd/impe}
\input{nd/impi}
\input{nd/note}
\input{nd/noti}
\input{nd/ore}
\input{nd/ori}
\input{nd/alle}
\input{nd/alli}
\input{nd/existe}
\input{nd/existi}

% introduction to the substitution rule
\newpage
\input{nd/introsub}

% user command for substitution
\input{nd/subst}

	% introduction to the section
\newpage
\input{rules/introrul}

% user commands for conditional rules
\input{rules/termife}
\input{rules/termifen}
\input{rules/termifi}
\input{rules/wffife}
\input{rules/wffifen}
\input{rules/wffifi}

% introduction to the sort's section
\input{rules/contract}
\input{rules/cut}
\input{rules/weaken}

 	% introduction to the deciders
\newpage
\input{decide/intro}

% user commands for deciders
\input{decide/decide}
\input{decide/monad}
\input{decide/monadeq}
\input{decide/ptaut}
\input{decide/taut}
\input{decide/tauteq}

 	% introduction to the semantic simplification's section
\newpage
\input{eval/introsema}

% user commands for semantic simplification
\input{eval/attach}
\input{eval/decrep}
\input{eval/hardware}
\input{eval/represent}
\input{eval/simplify}

% introduction to the syntactic simplification's section
\newpage
\input{eval/introsynt}

% user commands for syntactic simplification
\input{eval/assertsimp}
\input{eval/rewrite}


% introduction to the syntactic/semantic simplification's section
\newpage
\input{eval/introsynsema.tex}

% user commands for the syntactic/semantic simplification's section
\input{eval/eval}
\input{eval/let}


 	% loading introduction to the section
\newpage
\input{context/introcon}

% loading explanation of commands
\input{context/copycontext}
\input{context/copylex}
\input{context/makecontext}
\input{context/namecontext}
\input{context/reset}
\input{context/switchcontext}

 	% loading introduction to the section
\newpage
\input{meta/intromet.tex}

% loading explanation of commands
\input{meta/mattach}
\input{meta/reflect}

 
	% ........................... QUICK REFERENCE ..........................
	\newpage
	\appendix
	\section{Open problems}
\label{app-op}

If a {\tt reflect} command is used when a context {\tt META} does not
exist the error produced is that the relevant fact can not be found.
The error message should say that {\tt META} does not exist.

\gap
The {\tt subst} command has problems distinguishing
proof line labels from numbers in equalities.

\gap
The {\tt label} command does not check for overlaps between
labels for proof lines, axioms and theories.



	\newpage
	\newcommand{\syndes}[2]{\item[\parbox{\textwidth}{#1}]\hfill #2}
\newcommand{\module}[1]{\subsection{#1}}

\section{Syntax of commands}

\module{admin}

\begin{description}
\syndes{
   comment \ARG{separator} \OPT{\ARG{text}} \ARG{separator}
}
{}
%3rd-begin%3rd-end

\syndes{
   deflam \ARG{funname} \ARG{var-list} \ARG{form};
}
{}
%3rd-begin%3rd-end

\syndes{
   deflam \ARG{funname} \ARG{var-list} \ARG{form};
}
{}
%3rd-begin%3rd-end

\syndes{
   echo \ARG{separator} \OPT{\ARG{text}} \ARG{separator}
}
{}
%3rd-begin%3rd-end

\syndes{
   hgk \ARG{s-expr};
}
{}
%3rd-begin%3rd-end

\syndes{
   hgk \ARG{s-expr};
}
{}
%3rd-begin%3rd-end

\syndes{
   know natnums \OPT{\ARG{natnum1}, \SEQ \ARG{natnumN}};
}
{}
%3rd-begin%3rd-end

\syndes{
   load \ARG{file};
}
{}
%3rd-begin%3rd-end

\syndes{
   resetprompt;
}
{}
%3rd-begin%3rd-end

\syndes{
   setprompt to \ARG{s-expr};
}
{}
%3rd-begin%3rd-end

\syndes{
   setprompt to \ARG{s-expr};
}
{}
%3rd-begin%3rd-end

\syndes{
   show \ARG{option};
}
{}
%3rd-begin%3rd-end

\end{description}

\module{context}

\begin{description}
\syndes{
  copycontext \ARG{ctx-name};
}
{}
%3rd-begin%3rd-end

\syndes{
	copylex	\ARG{ctx-name};
}
{}
%3rd-begin%3rd-end

\syndes{
  makecontext \ARG{ctx-name};
}
{}
%3rd-begin%3rd-end

\syndes{
  namecontext \ARG{ctx-name};
}
{}
%3rd-begin%3rd-end

\syndes{
  reset;
}
{}
%3rd-begin%3rd-end

\syndes{
  switchcontext \ARG{ctx-name};
}
{}
%3rd-begin%3rd-end

\end{description}

\module{decide}

\begin{description}
\syndes{
  decide \ARG{wff}  \OPT{by \ARG{fact1} \ARG{fact2} \SEQ} using 
  \OPT{\{ \ARG{rewriter} \SEQ \}} \ARG{decider};
}
{}
%3rd-begin%3rd-end

\syndes{
  monad \ARG{wff} \OPT{by \ARG{fact1} \ARG{fact2} \SEQ};
}
{}
%3rd-begin%3rd-end

\syndes{
  monadeq \ARG{wff} \OPT{by \ARG{fact1} \ARG{fact2} \SEQ};
}
{}
%3rd-begin%3rd-end

\syndes{
  ptaut \ARG{wff} \OPT{by \ARG{fact1} \ARG{fact2} \SEQ};
}
{}
%3rd-begin%3rd-end

\syndes{
  taut \ARG{wff} \OPT{by \ARG{fact1} \ARG{fact2} \SEQ};
}
{}
%3rd-begin%3rd-end

\syndes{
  tauteq \ARG{wff} \OPT{by \ARG{fact1} \ARG{fact2} \SEQ};
}
{}
%3rd-begin%3rd-end

\end{description}

\module{eval}

\begin{description}
\syndes{
  assertsimp \ARG{simplabel}; 
}
{}
%3rd-begin%3rd-end

\syndes{
  attach \ARG{indconst}  to \ALT dar [ rep ] \ARG{sexpr};\\
  attach \ARG{sentconst} to T \ALT NIL \ALT UNDEF;\\
  attach \ARG{funconst} \ALT \ARG{predconst} to \ARG{atom};\\
  attach \ARG{funconst}  to [ \ARG{rep1}, \SEQ, \ARG{repN} = \ARG{repM} ]
  \ARG{atom};\\
  attach \ARG{predconst} to [ \ARG{rep1}, \SEQ, \ARG{repN} ] \ARG{atom};
}
{}
%3rd-begin%3rd-end

\syndes{
  decrep \ARG{replabel1} \OPT{\SEQ \ARG{replabelN}};
}
{}
%3rd-begin%3rd-end

\syndes{
  eval \ARG{wff} \ALT \ARG{fact} \ALT \ARG{term} \OPT{by \ARG{simpexpr}};
}
{}
%3rd-begin%3rd-end

\syndes{
  hardware \ARG{indconst} to \ALT dar \ARG{sexpr};\\
  hardware \ARG{indconst} to \ALT dar [ \ARG{rep} ] \ARG{sexpr};
}
{}
%3rd-begin%3rd-end

\syndes{
  let \ARG{\indconst} to \ALT dar [ \ARG{rep} ] \ARG{term};
}
{}
%3rd-begin%3rd-end

\syndes{
  represent \{ \ARG{sort1}, \SEQ, \ARG{sortN} \} as \ARG{rep} \ALT \ARG{*};
}
{}
%3rd-begin%3rd-end

\syndes{
  rewrite \ARG{wff} \ALT \ARG{fact} \ALT \ARG{term} \OPT{by \ARG{simpexpr}};
}
{}
%3rd-begin%3rd-end

\syndes{
  setbasicsimp \ARG{simplabel} at wffs \{ \ARG{wff1} \SEQ \ARG{wffN} \};\\
  setbasicsimp \ARG{simplabel} at facts \{ \ARG{fact1} \SEQ \ARG{factN} \};
}
{}
%3rd-begin%3rd-end

\syndes{
  simplify \ARG{wff} \ALT \ARG{fact} \ALT \ARG{term};
}
{}
%3rd-begin%3rd-end

\end{description}

\module{language}

\begin{description}
\syndes{
   awff \ARG{awff};
}
{}
%3rd-begin%3rd-end

\syndes{
   declare \OPT{\ARG{sentsym} \ALT \ARG{indsym}} \ARG{sym1} \SEQ \ARG{symN}; \\
   declare \OPT{\ARG{funsym}  \ALT \ARG{predsym}} \ARG{sym1} \SEQ \ARG{symN}
   \ARG{arity};\\
   declare \OPT{\ARG{funsym} \ALT \ARG{predsym}} \ARG{sym1} \SEQ \ARG{symN}
   1 \OPT{[ pre \OPT{= \ARG{prbp}} ]};\\
   declare \OPT{\ARG{funsym} \ALT \ARG{predsym}} \ARG{sym1} \SEQ \ARG{symN}
   2 \OPT{[ inf \OPT{= \ARG{lbp} \ARG{rbp}} ] };
}
{}
%3rd-begin%3rd-end

\syndes{
  declare sort \ARG{sym1} \SEQ \ARG{symN};
}
{}
%3rd-begin%3rd-end

\syndes{
  declare indconst \ALT indpar \ALT indvar \ARG{sym1} \SEQ \ARG{symN}
  [ \ARG{sortsym} ]; \\
  declare funconst \ALT funpar \ARG{sym1} \SEQ \ARG{symN}
  ( \ARG{sortsym1} \SEQ \ARG{sortsymN} ) = \ARG{sortsym};\\
}
{}
%3rd-begin%3rd-end

\syndes{
  extension \ARG{sort} by \ARG{extexpr};
}
{}
%3rd-begin%3rd-end

\syndes{
  moregeneral \ARG{sort1} < \ARG{sort2}, \SEQ, \ARG{sortN} >;
}
{}
%3rd-begin%3rd-end

\syndes{
  mostgeneral \ARG{sym};
}
{}
%3rd-begin%3rd-end

\syndes{
  setfmap \ARG{funsym} ( \ARG{sym1} \SEQ \ARG{symN} ) = \ARG{sym};
}
{}
%3rd-begin%3rd-end

\syndes{
   term \ARG{term};
}
{}
%3rd-begin%3rd-end

\syndes{
   wff \ARG{wff};
}
{}
%3rd-begin%3rd-end

\end{description}

\module{meta}

\begin{description}
\syndes{
   mattach \ARG{indconst} to \ALT dar \OPT{[rep]}
   \ARG{cname}:\ARG{pname}:\ARG{sort}:\ARG{object};
}
{}
%3rd-begin%3rd-end

\syndes{
  reflect \ARG{M-fact} \ARG{arg1} \ARG{arg2} \SEQ \ARG{argN};
}
{}
%3rd-begin%3rd-end

\end{description}

\module{nd}

\begin{description}
\syndes{
   alle \ALT us \ARG{fact} \OPT{,} \ARG{term1} \ARG{term2} \SEQ;
}
{}
%3rd-begin%3rd-end

\syndes{
   alli \ALT ug \ARG{fact} \OPT{\OPT{,} \ARG{indvar1} \ALT \ARG{indpar1} :} 
                           \ARG{indvar11}
                           \OPT{\OPT{,} \ARG{indvar2} \ALT \ARG{indpar2} :}
                           \ARG{indvar22} \SEQ;
}
{}
%3rd-begin%3rd-end

\syndes{
   ande \ALT ae \ARG{fact} \OPT{,} 1 \ALT 2; \\
   ande \ALT ae \ARG{fact} \OPT{,} 1 \ALT 2 1 \ALT 2 \SEQ;
}
{}
%3rd-begin%3rd-end

\syndes{
   andi \ALT ai \ARG{fact1} \OPT{,} \ARG{fact2}; \\
   andi \ALT ai \ARG{fact11}
                \OPT{conj \ALT cj \ARG{fact12} conj \ALT cj \ARG{fact13} \SEQ}
                \OPT{,}
                \ARG{fact21}
                \OPT{conj \ALT cj \ARG{fact22} conj \ALT cj \ARG{fact23} \SEQ};
}
{}
%3rd-begin%3rd-end

\syndes{
   assume \ARG{wff1} \OPT{\OPT{,} \ARG{wff2} \SEQ};
}
{}
%3rd-begin%3rd-end

\syndes{
   existe \ALT es \ARG{fact}  \OPT{,} \ARG{indvar1} \ALT \ARG{indpar1}
                  \OPT{,} \ARG{indvar2} \ALT \ARG{indpar2} \SEQ;
}
{}
%3rd-begin%3rd-end

\syndes{
  existi \ARG{fact}
   \OPT{\OPT{,} \ARG{term1} :} \ARG{indvar1} \OPT{occ \ARG{n11} \ARG{n12} \SEQ}
   \OPT{\OPT{,} \ARG{term2} :} \ARG{indvar2} \OPT{occ \ARG{n21} \ARG{n22} \SEQ}
   \SEQ;
}
{}
%3rd-begin%3rd-end

\syndes{
   falsee \ALT fe \ARG{fact1} \OPT{,} \ARG{wff};\\
   falsee \ALT fe \ARG{fact1} \OPT{,} \ARG{fact2};
}
{}
%3rd-begin%3rd-end

\syndes{
   falsei \ALT fi \ARG{fact1} \OPT{,} \ARG{fact2}; 
}
{}
%3rd-begin%3rd-end

\syndes{
   iffe \ALT ie \ARG{fact} \OPT{,} 1 \ALT 2;
}
{}
%3rd-begin%3rd-end

\syndes{
   iffi \ALT ii \ARG{fact1} \OPT{,} \ARG{fact2};
}
{}
%3rd-begin%3rd-end

\syndes{
   impe \ALT mp \ARG{fact1} \OPT{,} \ARG{fact2};
}
{}
%3rd-begin%3rd-end

\syndes{
   impi \ALT ded  \ARG{fact1} \OPT{, \ALT imp} \ARG{fact};\\
   impi \ALT ded  \ARG{wff} \OPT{, \ALT imp} \ARG{fact};
}
{}
%3rd-begin%3rd-end

\syndes{
   note \ALT ne \ARG{fact1} \OPT{,} \ARG{wff}; \\
   note \ALT ne \ARG{fact1} \OPT{,} \ARG{fact2};
}
{}
%3rd-begin%3rd-end

\syndes{
   noti \ALT ni \ARG{fact1} \OPT{,} \ARG{wff}; \\
   noti \ALT ni \ARG{fact1} \OPT{,} \ARG{fact2};
}
{}
%3rd-begin%3rd-end

\syndes{
   ore \ALT oe \ARG{fact1} \OPT{,} \ARG{fact2} \OPT{,} \ARG{fact2};
}
{}
%3rd-begin%3rd-end

\syndes{
   ori \ALT oi \ARG{fact} \OPT{,} \ARG{wff} \OPT{,} \OPT{lr \ALT rl};\\
   ori \ALT oi \ARG{fact} \OPT{,} \ARG{fact1} \ALT \ARG{wff1} 
       disj \ALT dj \ARG{fact2} \ALT \ARG{wff2} disj \ALT dj \SEQ
       \OPT{,} \OPT{lr \ALT rl};
}
{}
%3rd-begin%3rd-end

\syndes{
  subst \ARG{fact1} \OPT{with} \ARG{fact2};\\
  subst \ARG{fact1} \OPT{with} \ARG{fact2} \OPT{right \ALT left};\\
  subst \ARG{fact1} \OPT{with} \ARG{fact2} 
                    \OPT{occ \ARG{n1} \ARG{n2} \SEQ} \OPT{right \ALT left};
}
{}
%3rd-begin%3rd-end

\end{description}

\module{parser}

\begin{description}
\syndes{
  backup \ARG{file} open;\\
  backup \ARG{file} close;  
}
{}
%3rd-begin%3rd-end

\syndes{done;}
{}
%3rd-begin%3rd-end

\syndes{
   fetch \ARG{file} \OPT{from \ARG{mark1}} \OPT{to \ARG{mark2}};
}
{}
%3rd-begin%3rd-end

\syndes{
   mark \ARG{sym};
}
{}
%3rd-begin%3rd-end

\syndes{
   probe;\\
   probe \ARG{activity};\\
   probe all;\\
}
{}
%3rd-begin%3rd-end

\syndes{
   unprobe \ARG{activity};\\
   unprobe all;
}
{}
%3rd-begin%3rd-end

\end{description}

\module{proof}

\begin{description}
\syndes{
  axiom \ARG{sym} : \ARG{wff};
}
{}
%3rd-begin%3rd-end

\syndes{
   cancel \OPT{\ARG{label}};
}
{}
%3rd-begin%3rd-end

\syndes{
  copyproof \ARG{prf-name};
}
{}
%3rd-begin%3rd-end

\syndes{
   label fact \ARG{sym};\\
   label fact \ARG{sym} = \ARG{label};
}
{}
%3rd-begin%3rd-end

\syndes{
   makeproof \ARG{prf-name};
}
{}
%3rd-begin%3rd-end

\syndes{
   nameproof \ARG{prf-name};
}
{}
%3rd-begin%3rd-end

\syndes{
  switchproof \ARG{prf-name};
}
{}
%3rd-begin%3rd-end

\syndes{
  theorem \ARG{sym} \ARG{hook};
}
{}
%3rd-begin%3rd-end

\syndes{
  theory \ARG{thlabel} : \ARG{wff1} \OPT{\ARG{wff2} \SEQ};\\
  theory \ARG{thlabel} : \ARG{axlabel1} : \ARG{wff}
                         \OPT{\ARG{axlabel2} : \ARG{wff} \SEQ};
}
{}
%3rd-begin%3rd-end

\end{description}

\module{rules}

\begin{description}
\syndes{
	contract \ALT ctc  \ARG{fact} by \ARG{assumption1} \SEQ  \ARG{assumptionN}; 
}
{}
%3rd-begin%3rd-end

\syndes{
	cut \ARG{fact1} \ARG{fact2};\\
	cut \ARG{fact1} \ARG{fact2} \OPT{keep \ARG{assumption1} \SEQ
	\ARG{assumptionN}};
}
{}
%3rd-begin%3rd-end

\syndes{
	termife \ARG{fact1} \ARG{fact2} \ARG{termif}; \\
	termife \ARG{fact1} \ARG{fact2} \ARG{termif} \OPT{occ \ARG{n1} \ARG{n2}
	\SEQ}; 
}
{}
%3rd-begin%3rd-end

\syndes{
	termifen \ARG{fact1} \ARG{fact2} \ARG{termif}; \\
	termifen \ARG{fact1} \ARG{fact2} \ARG{termif} \OPT{occ \ARG{n1} \ARG{n2}
	\SEQ}; 
}
{}
%3rd-begin%3rd-end

\syndes{
	termifi \ARG{fact1} \ARG{fact2} \ARG{wff} \ARG{term1} \ARG{term2};
}
{}
%3rd-begin%3rd-end

\syndes{
	weaken \ALT wk \ARG{fact} by \ARG{fact1} \SEQ \ARG{factN};
}
{}
%3rd-begin%3rd-end

\syndes{
	wffife \ARG{fact1} \ARG{fact2};
}
{}
%3rd-begin%3rd-end

\syndes{
	wffifen \ARG{fact1} \ARG{fact2};
}
{}
%3rd-begin%3rd-end

\syndes{
	wffifi \ARG{wff} \ARG{fact1} \ARG{fact2};
}
{}
%3rd-begin%3rd-end

\end{description}



	% ............................. BIBLIOGRAPHY ...........................
	\newpage
	\bibliographystyle{alpha}
    \gfbibliography
	
	% ................................ INDEX ...............................
	\newpage
	%............................... USER MANUAL .................................
%.............................................................................

\documentstyle[12pt]{../styfiles/GFmanual}

\title{GETFOL Manual}
\author{\bf Fausto Giunchiglia}
\date{7 March 1994}
\version{2.0}
\abstract{
	  {\GF} is an interactive reasoning system.
	  We use it as an environment for studying epistemological issues.
	  We try to look at questions like: which notions are
	  important for the development of mechanized reasoning systems?
	  What kind of conversations do we want to have with them?
	  What parts of logic should we use to represent such notions?
	  How should logic be embedded in a conversational reasoning system?
	}
\addresses{
     \begin{tabular}[c]{l}
       {\bf Fausto Giunchiglia}           \\
       Mechanized Reasoning Group		  \\
		 IRST, Povo, 38050 Trento, Italy  \\
		 e-mail: {\tt fausto@irst.it}     \\
		 phone: +39 461 314436
	\end{tabular}	
}
\published{
  \begin{tabular}{l}
	  DIST Technical Report No. 92-0010 (1994). \\
	  DIST -- University of Genoa,\\
	  Via Opera Pia 11A, 16145 Genova, Italy.\\ \\
  \end{tabular}
}

%% \newcommand{\gfbibliography}{%
%% \bibliography{/home/tarski/staff/mrg/biblio/bib/a-l,%
%% /home/tarski/staff/mrg/biblio/bib/m-z,userman}}
\newcommand{\gfbibliography}{\bibliography{}}

%% \makeindex
\begin{document}
	%  ............................. COVER ..................................
	\thispagestyle{empty}
	\maketitle

	%  ........................ TABLE OF CONTENTS ...........................
	\newpage
	\pagenumbering{roman}
	\tableofcontents

	\newpage
	\pagenumbering{arabic}
	\pagestyle{headings}

	%  ........................... INTRODUCTION ............................
	\input{intro/intro.tex}

	%  .............................. MODULES ..............................
	\input{parser/parser}
	\input{admin/admin}
 	\input{language/language}
 	\input{proof/proof}
 	\input{nd/nd}
	\input{rules/rules}
 	\input{decide/deciders}
 	\input{eval/evaluator}
 	\input{context/context}
 	\input{meta/meta}
 
	% ........................... QUICK REFERENCE ..........................
	\newpage
	\appendix
	\input{appendix/openprob.tex}

	\newpage
	\input{appendix/commands.tex}

	% ............................. BIBLIOGRAPHY ...........................
	\newpage
	\bibliographystyle{alpha}
    \gfbibliography
	
	% ................................ INDEX ...............................
	\newpage
	\input{userman.ind}
\end{document}

\end{document}

\end{document}

\end{document}
