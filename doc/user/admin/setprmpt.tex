\gfcommand{setprompt}{Redefines the prompt}
\index{setprompt}

\gfsyntax{
   setprompt to \ARG{s-expr};
}

\gfdescription{
   Sets {\GF}'s prompt  to the value of the s-expression
   \ARG{s-expr}  followed by  ":: ". It is particularly  useful when  you are
   working on multiple contexts as you can set the prompt to the value of
   the current context.
}

\gfrecap{
Sets GETFOL's prompt to `s-expr'.
}

\gfexample+
   ***** setprompt to (QUOTE myprompt);

   myprompt:: setprompt to (QUOTE Tweedledee\&Tweedledum);

   Tweedledee&Tweedledum:: setprompt to (CAPITALIZE (curcname\-get));

   NOTNAMED&:: namecontext Disneyland;
   You have named the current context: Disneyland

   DISNEYLAND:: makecontext Cartoonia;
   You have created the empty context: Cartoonia

   DISNEYLAND:: switchcontext Cartoonia;
   You are now using context: Cartoonia
   You are switching to a proof with no name.

   CARTOONIA:: 
+

\gfnotes{
	The command tries to evaluate the {\em s-expression} passed as argument.
	Failure of the evaluation causes a crash to the {\HG} evalautor.
}
