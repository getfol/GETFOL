\gfcommand{show}{shows {\GF} information}

\gfsyntax{
   show \ARG{option};
}

\gfdescription{
   Shows {\GF} information. In the example we show some of the
   options implemented in a {\GF} version.
   Notice that ``\verb+show show;+" lists the options supported by the show
   command.
}

\gfrecap{
Shows GETFOL information.
}

\gfexample+
   ***** comment | ******* SHOW SHOW ******* |
   ***** show show;
   The list of show options is the following:

   CONTEXT : WHEREAMI 

   REWRITER : SIMPSET 

   SIMPLIFIER : INT REP 

   DEFINITION : DEFINITION 

   PROOF : PREMISES FACT AXIOM 

   LANGUAGE : LGS MGS MEM SORT TYP SYM 

   SYSTEM : COM SHOW 
   
   ***** comment | ******* SHOW WHEREAMI ******* |
   
   ***** show whereami;
   You are now using an unnamed context.
   You are now using an unnamed proof.
   
   ***** comment | ******* SHOW REP and INT ******* |
   
   ***** decrep PIPPOREP;
   ***** attach A dar [PIPPOREP] a;
   ***** attach C dar [PIPPOREP] c;
   ***** attach B to  [PIPPOREP] b;
   
   ***** show rep PIPPOREP;
   The designators for the representation: PIPPOREP are:
   (c . C) (a . A)
   
   ***** show int A;
   The Indconst A is attached to 'a
   with representation PIPPOREP
   
   ***** comment | ******* SHOW SIMPSET ******* |
   
   ***** setbasicsimp simp1 at wffs
   {forall x1.F1(x1)=x1,
   forall x1 x2.F2(x1 x2)=x1,
   forall x1 x2 x3.F3(x1 x2 x3)=x1};
   
   ***** show simpset simp1;
   Wffs :
   forall x1. (F1(x1) = x1)
   forall x1 x2. (F2(x1,x2) = x1)
   forall x1 x2 x3. (F3(x1,x2,x3) = x1)
   
   ***** setbasicsimp simp2 at facts {1 2};
   
   ***** show simpset simp2;
   Proof lines :  1 2
   
   ***** comment | AX1 and AX2 are two axioms |
   ***** setbasicsimp simp3 at facts {AX1 AX2};
   ***** show simpset simp3;
   Axioms :  AX1 AX2
   
   ***** setcompsimp simpA at simp1 uni simp2 uni simp3;
   ***** show simpset simpA;
   simpA is compound by this list of basic simpsets :
   (simp1 simp2 simp3)
   
   ***** SETCOMPSIMP simpB at simpA dif simp1;
   ***** show simpset simpB;
   simpB is compound by this list of basic simpsets :(simp2 simp3)
   
   
   ***** comment | ******* SHOW PROOF, FACT and AXIOM ******* |
   
   ***** show proof;
   1   forall x1. (F1(x1) = x1)     (1)
   2   forall x1 x2. (F2(x1,x2) = x1)     (2)
   
   ***** label fact identity = 1;
   ***** show fact;
   identity   1
   
   ***** show axiom;
   AX1 : forall x1. (F1(x1) = x1)
   AX2 : forall x1 x2. (F2(x1,x2) = x1)
   
   ***** comment | ******* SHOW LSG, MGS , MEM and SORT ******* |
   
   ***** declare sort a b c d e f g h;
   moregeneral a < b c d e f g h >;
   moregeneral b < c d e f g h >;
   
   ***** show lgs a;
   No sort is strictly lessgeneral than a.
   ***** show lgs b;
   No sort is strictly lessgeneral than b.
   
   ***** show mgs a;
   The only sort strictly moregeneral than a is UNIVERSAL
   ***** show mgs f;
   The only sort strictly moregeneral than f is UNIVERSAL
   
   ***** show mem a;
   No <indsym> is declared to be of sort a.
   
   ***** declare indvar x [a];
   a is a sort
   x has been declared to be an Indvar
   ***** declare indvar y [a];
   a is a sort
   y has been declared to be an Indvar
   ***** show mem a;
   The <indsym>'s declared to be of sort a are
       y  x  
   
   ***** show sort;
   The symbols declared to be sorts are
       b  a  UNIVERSAL  
   
   ***** comment | ******* SHOW TYP, SYM ******* |
   
   ***** show typ INDVAR;
   The symbols declared to be Indvars are
       x  y  
   
   ***** show sym a;
   a is declared to be a sort.
   ***** show sym x;
   x is declared to be an Indvar of sort a.
   
   
   ***** comment | **************** SHOW PREMISES *************** |
   
   ***** declare sentconst A B C;
   ***** assume A B;
   1   A     (1)
   2   B     (2)
   ***** andi 1 2;
   3   A and B     (1 2)
   ***** ori 3 C;
   4   (A and B) or C     (1 2)
   ***** andi 3 4;
   5   (A and B) and ((A and B) or C)     (1 2)
   ***** show premises 5;
   5  (A and B) and ((A and B) or C)  (1 2)
      3  A and B  (1 2)
      4  (A and B) or C  (1 2)
   ***** show premises 5 2;
   5  (A and B) and ((A and B) or C)  (1 2)
      3  A and B  (1 2)
         1  A  (1)
         2  B  (2)
      4  (A and B) or C  (1 2)
         3  A and B  (1 2)
   *****  show premises 5 all;
   5  (A and B) and ((A and B) or C)  (1 2)
      3  A and B  (1 2)
         1  A  (1)
         2  B  (2)
      4  (A and B) or C  (1 2)
         3  A and B  (1 2)
            1  A  (1)
            2  B  (2)
   
   
   ***** comment | ******* SHOW COM ******* |
   
   ***** show com;
   The list of commands is the following:

   META : REFLECT MATTACH 

   CONTEXT : COPYLEX SWITCHCONTEXT COPYCONTEXT NAMECONTEXT MAKECONTEXT 

   DECIDER : DECIDE MONADEQ MONAD TAUTEQ TAUT PTAUT 

   EVAL : EVAL 

   SIMPLIFIER : SIMPLIFY LET HARDWARE REPRESENT ATTACH DECREP 

   REWRITER : REWRITE ASSERTSIMP SETCOMPSIMP SETBASICSIMP UNFOLD FOLD CUT
   CTC CONTRACT WK WEAKEN WFFIFI WFFIFEN WFFIFE TERMIFI TERMIFEN TERMIFE  

   Natural-Deduction : ES EXISTE EXISTI US ALLE UG ALLI IE IFFE II IFFI
   NE NOTE NI NOTI FE FALSEE FI FALSEI OE ORE OI ORI AE ANDE AI ANDI MP
   IMPE DED IMPI SUBST THEOREM ASSUME   

   DEFINITION : DEFINE 

   PROOF : LABEL CANCEL AXIOM THEORY 

   LANGUAGE : SWITCHPROOF COPYPROOF NAMEPROOF MAKEPROOF EXTENSION WFF
   AWFF TERM MOREGENERAL MOSTGENERAL SETFMAP DECLARE  

   SYSTEM : COPYLEX RESET PAGER RESETPROMPT SETPROMPT KNOW HGK SHOW ECHO
   COMMENT UNPROBE PROBE DEFLAM LOAD MARK FETCH DONE BACKUP  
+