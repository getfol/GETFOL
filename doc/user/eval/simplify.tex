\gfcommand{simplify}{semantic simplification}
\index{simplify}
\label{sec-simplify}

\gfsyntax{
  simplify \ARG{wff} \ALT \ARG{fact} \ALT \ARG{term};
}

\gfdescription{
  If \ARG{term} is provided as argument, three steps are performed:\\
  %
  \begin{itemize}
  \item 
    the interpretation of \ARG{term} in the computational model is computed;
  \item a preferred name for the interpretation is found;
  \item the equality of \ARG{term} with the preferred name is asserted as the next
    line of the proof.
    No action is taken if \ARG{term} has no interpretation in the computational
    model (has an undefined interpretation) or a preferred name for its
    interpretation does not exist.
  \end{itemize}
  %
  If {\em wff} is provided as an argument, two steps are performed:
  %
  \begin{itemize}
  \item the interpretation of {\em wff} in the computational model is computed 
    (in this case the interpretation will be {\tt TRUE}, {\tt FALSE} or an 
    undefined truth value)
  \item if the interpretation of {\em wff} is {\tt TRUE}, {\em wff} 
    is asserted as the next line of the proof, if it is {\tt FALSE} 
    the negation of {\em wff} is asserted.
  \end{itemize}
  %
  When \ARG{fact} is provided as argument, the simplify command works on the wff
  of \ARG{fact}.
}

\gfrecap{
  If `term' is provided as argument, three steps are performed:
      * the interpretation of `term' in the computational model is computed;
      * a preferred name for the interpretation is found;
      * the equality of $term$ with the preferred name is asserted as the next
        line of the proof. No action is taken if `term' has no interpretation
        in the computational model (has an undefined interpretation) or a
        preferred name for its interpretation does not exist.
  If `wff' is provided as an argument, two steps are performed:
      * the interpretation of `wff' in the computational model is computed 
        (in this case the interpretation will be `TRUE', `FALSE' or an 
        undefined truth value)
      * if the interpretation of `wff' is `TRUE', `wff' is asserted as the next
        line of the proof, if it is `FALSE' the negation of `wff' is asserted.
  When `fact' is provided as argument, the simplify command works on the wff
  of `fact'.
}

\gfexample+
   ***** declare indconst a b c;
   ***** decrep REP;
   ***** attach a dar [REP]a;
   a attached to 'a
   a is the preferred name of a
   ***** attach b dar [REP]b;
   b attached to 'b
   b is the preferred name of b
   ***** attach c to [REP]c;
   c attached to 'c
   ***** declare funconst F 1;
   F has been declared to be a Funconst
   ***** DEFLAM F(x) (IF (EQ x (QUOTE a)) (QUOTE b) 
                         (IF (EQ x (QUOTE b)) (QUOTE c)
                          (QUOTE UNDEF&)));
   ***** attach F to [REP=REP]F;
   F attached to F
   ***** simplify F(a);
   1   F(a) = b    
   ***** simplify F(b);
   F(b) : No simplification is possible.
   ***** simplify F(c);
   F(c) : No simplification is possible.
   ***** declare predconst P 1;
   P has been declared to be a Predconst
   ***** DEFLAM P(x) (IF (EQ x (QUOTE a)) TRUE 
                         (IF (EQ x (QUOTE b)) FALSE 
                          (QUOTE UNDEF&)));
   ***** attach P to [REP]P;
   P attached to P
   ***** simplify P(a);
   2   P(a)  
   ***** simplify P(b);
   3   not P(b)    
   ***** simplify P(c);
   P(c) : No simplification is possible.
   ***** extension UNIVERSAL by {a b c};
   Now the extension of UNIVERSAL is fixed to be : (a b c)
   ***** declare indvar x;
   UNIVERSAL is a sort
   x has been declared to be an Indvar
   ***** simplify exists x.F(x)=b;
   4   exists x. (F(x) = b)
   ***** simplify forall x.P(x);
   5   not forall x. P(x) 
+

\gfnotes{
  In the case of sorts with extensions (see the command {\tt extension} in section 
  \ref{sec-sort}) quantification is considered as {\bf bounded quantification}.
  In other words, let $P$ be a predicate and $x$ an indvar of sort $S$, where
  $S$ has extension $\{s_1,\ldots,s_n\}$.
  Then the following equivalences hold:
  $$ 
  \forall x P(x)\liff (P(s_1) \con \ldots \con P(s_n))
  $$
  $$
  \exists x P(x)\liff (P(s_1) \dis \ldots \dis P(s_n))
  $$
  %
  The command explicitly unfolds universal/existential statements into 
  their propositional equivalents. 
}

