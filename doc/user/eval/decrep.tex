\gfcommand{decrep}{representation declaration}
\index{decrep}

\gfsyntax{
  decrep \ARG{replabel1} \OPT{\SEQ \ARG{replabelN}};
}

\gfdescription{
  It declares \ARG{replabelI} to be representation functions.\\
  The only builtin representation functions are {\tt NATNUMREP}, {\tt TRUTHREP}
  and {\tt UNIVERSALREP}, the representation functions for natural numbers, for 
  truth values and for default representation respectively.
  Numerals have a builtin attachment to {\HG} numbers in the representation 
  function {\tt NUMERALREP}.
}

\gfrecap{
It declares `replabelI' to be representation functions.
The only builtin representation functions are `NATNUMREP', `TRUTHREP'
and `UNIVERSALREP', the representation functions for natural numbers, for 
truth values and for default representation respectively.
Numerals have a builtin attachment to HGKM numbers in the representation 
function `NUMERALREP'.
}

\gfexample+
   ***** decrep rep1 rep2;
+

\gfnotes{
  Since the intended model itself appears nowhere in the {{\GF}} system, there
  is no need for the user to give any detailed information about the nature of 
  the representation maps which he has in mind.
  {\tt NATNUMREP} is known by {\GF} only after typing {\tt know natnums}.
}
