\gfcommand{attach}{semantic attachment}
\index{attach}

\gfsyntax{
  attach \ARG{indconst}  to \ALT dar [ rep ] \ARG{sexpr};\\
  attach \ARG{sentconst} to T \ALT NIL \ALT UNDEF;\\
  attach \ARG{funconst} \ALT \ARG{predconst} to \ARG{atom};\\
  attach \ARG{funconst}  to [ \ARG{rep1}, \SEQ, \ARG{repN} = \ARG{repM} ]
  \ARG{atom};\\
  attach \ARG{predconst} to [ \ARG{rep1}, \SEQ, \ARG{repN} ] \ARG{atom};
}

\gfdescription{
  Defines the attachment for the {\GF} constants.
  \ARG{repI} can be a representation function or an asterisk; if it is an
  asterisk or no representation is specified, then the default representation
  function {\tt UNIVERSALREP} is taken.
  \ARG{indconst}s can be attached either ``one way'' (using {\tt to}) or ``two ways''
  (using {\tt dar}).
  The ``two ways'' attachment tells the semantic interpretation mechanism 
  that whenever \ARG{sexpr} is computed as the {\HG} representation of 
  a term $t$, then the attached {\GF} \ARG{indconst} should be returned as the 
  simplified version of $t$.
  That is, not only \ARG{sexpr} is the {\HG} representation of \ARG{indconst}, but 
  \ARG{indconst} is the preferred {\GF} name of (the intended model value denoted 
  by) the {\HG} object \ARG{sexpr}.
  \ARG{sentconst}s can be attached to the three possible truth values
  corresponding to true, false and undefined\cite{kleene1}.
  This is done by attaching a sentconst to {\tt T}, {\tt NIL}, {\tt UNDEF}
  respectively.
  \ARG{funconst}s and \ARG{predconst}s can be attached to a {\HG} \ARG{atom}
  \cite{giunchiglia35}.
  \ARG{atom} will be used as the identifier of a {\HG} function, whose number of 
  arguments is supposed to match the arity of the {\GF} symbol.
}

\gfrecap{
Defines the attachment for the GETFOL constants.
`repI' can be a representation function or an asterisk; if it is an
asterisk or no representation is specified, then the default representation
function `UNIVERSALREP' is taken.
`indconst's can be attached either ``one way'' (using `to') or ``two ways''
(using `dar').
The ``two ways'' attachment tells the semantic interpretation mechanism 
that whenever `sexpr' is computed as the HGKM representation of 
a term `t', then the attached GETFOL `indconst' should be returned as the 
simplified version of `t'.
That is, not only `sexpr' is the HGKM representation of `indconst', but 
`indconst' is the preferred GETFOL name of (the intended model value denoted 
by) the HGKM object `sexpr'.
`sentconst's can be attached to the three possible truth values
corresponding to true, false and undefined.
This is done by attaching a sentconst to `T', `NIL', `UNDEF'
respectively.
`funconst's and `predconst's can be attached to a HGKM `atom'
`atom' will be used as the identifier of a HGKM function, whose number of 
arguments is supposed to match the arity of the GETFOL symbol.
}

\gfexample+
   ***** declare indconst a b;
   ***** declare sentconst s;
   ***** declare funconst f 1;
   ***** declare predconst p 1;
   ***** decrep rep;

   ***** attach a to a;
   a attached to 'a
   ***** attach a dar a;
   a attached to 'a
   a is the preferred name of a
   ***** attach a dar [rep]a;
   a attached to 'a
   ***** attach a dar [rep]b;
   a is already an preferred name in this representation
   ***** attach b dar [rep]a;
   a has already a preferred name in this representation

   ***** COMMENT | deflam defines an HGKM function |;
   ***** deflam f(x) x;
   ***** attach f to f;
   f attached to f
   ***** deflam p(x) (IF (EQUAL x (QUOTE a))TRUE FALSE);
   ***** attach p to p;
   p attached to p
   attach f to [repin=repout]f;
   f attached to f
   ***** deflam p1(x) (IF (EQUAL x (QUOTE b))TRUE FALSE);
   ***** attach p to [repin]p1;
   p attached to p1
   ***** attach p to [repin]p1;
   p has already an attachment with these representation informations
   ***** attach s to T;
   s attached to 'T
   ***** attach s to NIL;
   s has already an attachment
+
