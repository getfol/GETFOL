\gfcommand{rewrite}{syntactic simplifier command}
\index{rewrite}

\gfsyntax{
  rewrite \ARG{wff} \ALT \ARG{fact} \ALT \ARG{term} \OPT{by \ARG{simpexpr}};
}

\gfdescription{
  It rewrites the given expression by using the union of the rewrite rules 
  indicated by the \ARG{simpexpr}.
  If \ARG{simpexpr} is not specified, {\tt LOGICTREE} is used.
  If \ARG{term} is provided as an argument, two steps are performed:
  %
  \begin{itemize}
  \item \ARG{term} is rewritten by using the set of rewrite rules 
    indicated by \ARG{simpexpr}.
  \item the equality of \ARG{term} with its rewritten form is asserted as the next 
    line of the proof. The dependencies depend on the simpsets actually used
    during the syntactic simplification.
  \end{itemize}
  %
  If \ARG{wff} is provided as argument, also two steps are performed:
  %
  \begin{itemize}
  \item \ARG{wff} is rewritten by the set of rewrite rules of the
    \ARG{simpexpr} result.
  \item if \ARG{wff} is rewritten to {\tt TRUE}, \ARG{wff} is asserted as the
    next line in the proof, if \ARG{wff} is rewritten to {\tt FALSE},
    $\neg$ \ARG{wff} is asserted, otherwise the equivalence of \ARG{wff} with 
    its rewritten form is asserted.
    The dependencies depend on the simpsets used during the syntactic
    simplification.
  \end{itemize}
  %
  When \ARG{fact} is provided as an argument, the rewrite command works on the wff
  of the \ARG{fact}.
}

\gfrecap{
It rewrites the given expression by using the union of the rewrite rules 
indicated by the `simpexpr'.
If `simpexpr' is not specified, LOGICTREE is used.
}

\gfexample+
   ***** declare indconst A,B;
   ***** declare indvar X,Y;
   ***** declare funconst F 2;
   ***** declare funconst G 1;
   ***** declare sentconst P;
   ***** assume forall X . F(X,A) = A;
   1   forall X. (F(X,A) = A)     (1)
   ***** assume forall X . F(X,X) = G(X);
   2   forall X. (F(X,X) = G(X))     (2)
   ***** assume forall X Y . F(X,Y) =Y;
   3   forall X Y. (F(X,Y) = Y)     (3)
   ***** axiom F1:forall X . F(X,A) = A;
   F1 : forall X. (F(X,A) = A)
   ***** axiom F2:forall X . F(X,X) = G(X);
   F2 : forall X. (F(X,X) = G(X))
   ***** axiom F3:forall X Y. F(X,Y) = Y;
   F3 : forall X Y. (F(X,Y) = Y)
   ***** setbasicsimp S1 at facts {1};
   ***** setbasicsimp S2 at facts {2};
   ***** setbasicsimp S3 at facts {3};
   ***** setbasicsimp S4 at facts {F1};
   ***** setbasicsimp S5 at facts {F2};
   ***** setbasicsimp S6 at facts {F3};
   ***** setbasicsimp SIMPEQ at wffs {forall X.(X=X iff TRUE)};
   ***** setcompsimp S7 at S1 uni S2 uni S3;
   ***** rewrite F(A,A) by S6;
   4   F(A,A) = A     
   ***** rewrite F(A,A) by S5;
   5   F(A,A) = G(A)     
   ***** rewrite F(A,A) by S4;
   6   F(A,A) = A     
   ***** rewrite F(A,A) by S1;
   7   F(A,A) = A     (1)
   ***** rewrite F(A,A) by S2;
   8   F(A,A) = G(A)     (2)
   ***** rewrite F(A,A) by S3;
   9   F(A,A) = A     (3)
   ***** rewrite F(A,A) by S7;
   10   F(A,A) = A     (1)
   ***** rewrite F(B,B) by S1 uni S3;
   11   F(B,B) = B     (3)
   ***** rewrite F(B,B) by S1;
   F(B,B): No simplification is possible
   ***** rewrite not TRUE by S1;
   not TRUE: No simplification is possible
   ***** rewrite not TRUE;
   12   not (not TRUE)     
   ***** rewrite TRUE imp (P imp X=X);
   13   (TRUE imp (P imp (X = X))) iff (P imp (X = X)) 
   ***** rewrite TRUE imp (P imp X=X) by SIMPEQ uni LOGICTREE;
   14   TRUE imp (P imp (X = X))
   ***** rewrite F(A,A) by S7;
   15   F(A,A) = A     (1)
   ***** rewrite F(A,A)=A by S7;
   16   F(A,A) = A  iff (A = A)   (1)
   ***** rewrite F(A,A)=A by S7 uni SIMPEQ;
   17   F(A,A) = A     (1)
   ***** rewrite F(A,A)=G(A) by S7;
   18   (F(A,A) = G(A)) iff (A = G(A))     (1)
   ***** rewrite F(B,B) by S7;
   19   F(B,B) = G(B)     (2)
   ***** rewrite F(B,B)=G(B) by S7 uni SIMPEQ;
   20   F(B,B) = G(B)     (2)
   ***** rewrite F(B,B)=G(B) and F(A,A)=A by S7 uni SIMPEQ uni LOGICTREE;
   21   (F(B,B) = G(B)) and (F(A,A) = A)     (1 2)
   ***** rewrite F(A,A) by S7 dif S1 ;
   22   F(A,A) = G(A)     (2)
   ***** rewrite F(A,A) by S7 dif (S1 uni S2);
   23   F(A,A) = A     (3)
   ***** rewrite F(A,A)=A by S3 dif (S1 uni S2) uni SIMPEQ;
   24   F(A,A) = A     (3)
+
   