\section{Syntactic simplification}
\label{sec-rew}

The subsections \ref{sec-rew-intro} and \ref{sec-rew-simpset} 
of this section have been taken from \cite{rww3}.


\subsection{Introduction}
\label{sec-rew-intro}

The basic idea  of syntactic simplification is repeated substitution of 
selected equalities and equivalences into a given expression.
More precisely, let $E$ be a set of universally quantified equations and 
equivalences ("rewrite rules"), so members of $E$ look like:
\begin{itemize}
\item $\forall\ {\vec x}.(t_1=t_2)$
\item $\forall\ {\vec y}.(F_1\ \liff \ F_2)$
\end{itemize}
where ${\vec x}$ and ${\vec y}$ are the {\indvar} sequences $x_1$...$x_n$
and $y_1$..,$y_m$, $t_1$ and $t_2$ are {\term}s, and $F_1$, $F_2$ are {\wff}s.

A match, or an immediate simplification, of a {{\GF}} expression $exp$ consists
 of replacing an occurrence of $t_1[x \leftarrow u]$($F_1[y \leftarrow v]$) 
in $exp$ by $t_2[x \leftarrow u]$($F_2[y \leftarrow v]$), where $u$($v$) is a 
sequence of terms and where $\leftarrow$ indicates substitution.

There are two problems to solve:
\begin{enumerate}
\item There may be more than one equation (or equivalence) whose left half 
      matches a given expression, so one has to establish a precedence 
      hierarchy for matching.

\item The order used by the algorithm to consider the subexpressions of a 
      given expression.
\end{enumerate}

{{\GF}}'s solution to the first problem is the following ordering expression:
each simplification expression (i.e., left half of a rewrite rule) is 
regarded as a linear string of atoms.
Each atom is either:

\begin{itemize}
\item a {\bf constant} (which is not bound by the universal quantifier in the
                       prefix);

\item an {\bf old variable} (which is bound by the universal quantifier in 
                            the prefix and which has occurred before in the 
                            linear string);

\item a {\bf new variable} (which is bound by the universal quantifiers in the
                           prefix and which has not occurred before in the 
                           linear string);
\end{itemize}

If we think of concatenating different atoms to a given initial string, then 
the atoms have this precedence ordering:
\begin{center}
constants $<$ old variables $<$ new variables
\end{center}
and expressions are ordered lexicographically in accordance with this 
ordering on atoms.

Let's consider, for example, the precedence relations among the simplification
 expressions :
$f(a,b,b)$, $f(a,b,c)$, $f(a,a,x)$, $f(a,x,x)$, $f(a,x,y)$, $f(x,x,x)$,
$f(x,x,y)$, 
where $f$, $a$, $b$, $c$ are constants and $x$, $y$ are variables.

The last four expressions are linearly ordered:
$$
f(a,x,x)<f(a,x,y)<f(x,x,x)f<(x,x,y)
$$
and each of the first three expressions is less than $f(a,x,x)$ and 
incomparable to the other two of the first three expressions:
$$
f(a,b,b)<f(a,x,x)
$$
$$
f(a,b,c)<f(a,x,x)
$$
$$
f(a,a,x)<f(a,x,x)
$$
Together with transitivity, these inequalities completely define the 
precedence relation.

As far as regard the second problem, {{\GF}}'s syntactic simplification code 
basically considers subexpressions of $exp$ in the usual left-to-right order.
The exceptions occur after a subexpression $exp'$ has been matched (and 
substituted for).
The algorithm then begins again at the subexpression one level above $exp'$.

The syntactic simplification algorithm has the usual problems of rewrite rules.
A typical difficulty is the infinitely recurring substitutions: 
for example if one uses $\forall\ x.x+y=y+x$ as simplification equation, 
the algorithm will attempt to make this substitution without end.

\subsection{Simplification sets}
\label{sec-rew-simpset}

Syntactic simplification in {\GF} is performed by using 
{\bf syntactic simplification sets} (called {\bf simpsets} from now on).
Simpsets contain a label ({\em simplabel}) to identify the
rewrite rules used to rewrite expressions.
{\GF} has built-in simpsets, but the user can define his own ones: he can specify a 
set of formulae or facts as rewrite rules in a {\bf basic simspset}
or he can compose already defined simpsets in {\bf compound simpsets}.

The {{\GF}} builtin simpsets (see figure \ref{fig-simpset}) are
{\bf \tt LPROPTREE}, {\bf \tt LQUANTREE}, {\bf \tt LARGIFTREE} and 
{\tt LOGICTREE}.
{\tt LPROPTREE} contains a set of rewrite rules 
corresponding to basic logical equivalences (e.g. $P\con\neg P\liff \bot$). 
{\tt LQUANTREE} contains a set of rewrite rules 
corresponding to logic equivalences for quantified formulas.
{\tt LARGIFTREE} contains a set of rewrite rules corresponding to logic
equalities and equivalences for conditional terms and formulas.
{\tt LOGICTREE} is the union of all the previous builtin simpsets.

\newpage

\begin{figure}[htbp]
\begin{center}
\fbox{
\parbox{16cm}{
$$
\begin{array}{ll}
\neg \neg P \liff P    & \ \ \                         \\
\neg {\tt TRUE} \liff \bot   & \ \ \  \neg \bot \liff {\tt TRUE}    \\
P \con \bot \liff \bot & \ \ \  \bot \con P\liff \bot   \\
P \con {\tt TRUE}  \liff  P  & \ \ \  {\tt TRUE}  \con P\liff P     \\
\neg P\con P\liff\bot  & \ \ \  P \con \neg P\liff \bot \\
P\con P\liff P         & \ \ \                         \\
P     \dis \bot \liff P  & \ \ \  \bot \dis P \liff P       \\
P \dis {\tt TRUE} \liff {\tt TRUE}   & \ \ \  {\tt TRUE}  \dis P \liff {\tt TRUE}   \\
\neg P \dis P\liff {\tt TRUE}  & \ \ \  P \dis \neg P \liff {\tt TRUE}  \\
P     \dis P     \liff P & \ \ \                         \\
P \imp \bot\liff\neg P & \ \ \  \bot \imp P\liff {\tt TRUE}   \\
P\imp {\tt TRUE}\liff {\tt TRUE}   & \ \ \  {\tt TRUE}  \imp P     \liff P\\
\neg P \imp P \liff P  & \ \ \  P \imp \neg P\liff\neg p\\
P \imp P\liff {\tt TRUE}     & \ \ \                         \\
P\liff \bot\liff\neg P  & \ \ \  \bot \liff P\liff\neg P  \\
P\liff {\tt TRUE}\liff P      & \ \ \  {\tt TRUE}  \liff P\liff P     \\
\neg P \liff P\liff\bot & \ \ \  P\liff\neg P\liff\neg\bot\\
P\liff P\liff {\tt TRUE}      & \ \ \                         \\
\end{array}
$$
}}
\fbox{
\parbox{16cm}{
$$
\begin{array}{ll}
\forall x.{\tt TRUE}  \liff {\tt TRUE} & \ \ \  \forall x.\bot \liff \bot    \\
\exists x.{\tt TRUE}  \liff {\tt TRUE} & \ \ \  \exists x.\bot \liff \bot    \\
\end{array}
$$
}}
\vspace{0.3cm} 
\fbox{
\parbox{16cm}{
$$
\begin{array}{ll}
\forall x y.{\em trmif}\ \bot{\em then}\ x\ {\em else}\ y = y    & \ \ \ 
\forall x y.{\em trmif}\ {\tt TRUE}\ {\em then}\ x\ {\em else}\ y\ = x \\
\forall x.  {\em trmif}\ P\ {\em then}\ x\ {\em else}\ x = x & \ \ \  \\
{\em wffif}\ \bot\ {\em then}\ P1\ {\em else}\ P2\ \liff P2 & \ \ \ 
{\em wffif}\ {\tt TRUE}\ {\em then}\ P1\ {\em else}\ P2\ \liff\ P1 \\
{\em wffif}\ P\ {\em then}\ P1\ {\em else}\ P1\ \liff\ P1 & \ \ \  \\
\end{array}
\\
$$
}}
\caption{The rewrite rules of {\tt LPROPTREE}, {\tt LQUANTREE}
\label{fig-simpset}
and {\tt LARGIFTREE}.}
\end{center}
\end{figure}
