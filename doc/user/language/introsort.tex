\section{Sorts}
\label{sec-sort}

\subsection{Introduction}

The {\GF} logic is sorted, although you are not forced to make any
sort declarations.
If you never use any of the system's sort commands, you can use the 
logic as if it were unsorted. {\GF} makes sure that everything works correctly
with appropriate default sorts
(see later in this section).

Sorts in {\GF} are specially-handled unary predicates: that is, aside from the
usual features of unary predicates, they possess some additional ones.
This choice won't be deeply justified here.
The resulting logic can, however, be proved to be 
consistent and to allow more compact axiomatizations and proofs
than the unsorted one.

\subsection{Defining the sort hierarchy}
We present here the commands for the declaration of new sorts, for the
definition of the generality relations between sorts, for the definition
of a most general sort and for the declaration sort extensions.


The generality relations can be intuitively specified in this way:
stating that sort $S_1$ is {\it weakly more general} than sort $S_2$
is equivalent to
specifying an axiom of the form $\forall x(S_2(x) \imp S_1(x))$.
In this case we also say that $S_2$ is {\it weakly less general} than $S_1$.
Two sorts are said to be {\it equivalent} when they are mutually
(weakly) more general.
When $S_1$ is {\it weakly more (less) general} than $S_2$ and it is not
equivalent to it, then we say that $S_1$ is {\it strictly more (less)
general} than $S_2$.
Finally, we say that $S$ is a {\em most general sort} if it is weakly more
general than any sort. {\GF} has a default most general sort, {\tt
UNIVERSAL}, that is used when no explicit sort information is provided to
the system.

\subsection{Sorted declarations}

A sort is associated to every {\indsym} at the moment of its declaration.
In unsorted
declarations, when no sort is specified, the default most general sort
{\tt UNIVERSAL} is taken to be the sort of the symbol.

The sort information associated with a {\funsym} is a set of so-called 
{\it fmaps},
that is of $n+1$-tuples of sorts, where $n$ is the arity of the \funconst.
The fmaps for $f$ specify the sort of a term whose external {\funsym} is 
$f$, depending on the sort of its arguments. When a {\funsym} is declared
the $n+1$-tuple ({\tt UNIVERSAL} \ldots {\tt UNIVERSAL}) is always added to its
{\it fmaps}.

According to the sort information of symbols we give now the
definition of the useful notion of {\it sort of a term}:
we say that $t$ is a term of sort $S$ 
if and only if
%
\begin{itemize}
\item
  $t$ is an individual symbol, and its sort is weakly less general than $S$,
  or
\item
  $t=f(t_1,\dots,t_n)$, $f$ has an {\it fmap} of the form $(S_1,\dots,S_n,
  S_f)$, $S_f$ is weakly less general than $S$ and all $t_i$ are terms of sort
  $S_i$.
\end{itemize}
