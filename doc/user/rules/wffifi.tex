\gfcommand{wffifi}{wff conditional introduction}
\index{wffifi}

\gfsyntax{
	wffifi \ARG{wff} \ARG{fact1} \ARG{fact2};
}

\gfdescription{
\renewcommand{\arraystretch}{0.5}
\[
\begin{array}{l}
\\ \\ \\ 
%\\ \\
\mbox{{\em wif}} \  I  
\end{array}
\ \ 
\begin{array}{cc}
%\\ \\
\begin{array}{c}
{[A]}\\
\vdots\\
B
\end{array}
\ \ \ 
\begin{array}{c}
{[\neg A]}\\
\vdots\\
C
\end{array}
\\
\hline\\
\begin{array}{c}
\mbox{{\em wffif}} \ A \ \mbox{{\em then}} \ B \ \mbox{{\em else}} \ C 
\end{array}
\end{array}
\]
\renewcommand{\arraystretch}{1}

If \ARG{wff} is {\tt A}, \ARG{fact1}'s wff is {\tt B} and \ARG{fact2}'s wff is
{\tt C}, then the conclusion is {\tt wffif A then B else C}.
The rule discharges all the dependencies of {\em fact}$_1$ whose wff is {\tt A}
and of {\em fact}$_2$ whose wff is {\tt not A}.
}

\gfrecap{
If `wff' is `A', the wff of `fact1' is `B' and the wff of `fact2' is `C', then
the conclusion is `wffif A then B else C'.
The rule discharges all the dependencies of `fact1' whose wff is `A' and of
`fact2' whose wff is `not A'.
}

\gfexample+
   ***** declare sentconst A;
   ...
   ***** assume A;
   1   A     (1)
   ***** assume not A;
   2   not A     (2)
   ***** wffifi A 1 2;
   3   wffif A then A else (not A)
   ***** wffifi A 2 1;
   4   wffif A then (not A) else A     (1 2)
   ***** wffifi not A 2 1;
   5   wffif (not A) then (not A) else A     (1)
   ***** wffifi not A 1 2;
   6   wffif (not A) then A else (not A)     (1 2)
+

\gfnotes{}
