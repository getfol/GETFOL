\section{Metareasoning}
\label{sec-meta}

\subsection{Introduction}

A special context is {\meta}.
{\GF} recognizes {\meta} as a metatheory of all the other contexts.
The context {\meta} can be used to perform {\em metareasoning}, that is to
describe other contexts and to reason about them. 
Metareasoning in {\meta} is performed by employing the following novel
features:
%
\begin{itemize}
  \item
    the metatheory is, in general, distinct from the object theories it
    describes; 
  \item
    the link between the metalanguage and the object language is not performed
    by encoding, but rather by naming \cite{giunchiglia3}.
    Naming is implemented by using the commands which implement reasoning in
    the computational model of a context (see section \ref{sec-comp}).
    These features are available to the user by the commands \C{attach},
    \C{simplify}, \C{eval} etc.  
  \item
    Metareasoning and object reasoning can be mixed via the reflection rule 
    \cite{giunchiglia3}:

    \begin{equation}
      R_{down}
      \fraz{\der{M} Theorem(``w'')}{\der{O} w}
      \label{refl}
    \end{equation}

    where $M$ and $O$ stand for {\meta} and object theory respectively. 
    This rule is implemented in the {\GF} command \C{reflect}.
    The command knows that some form of metareasoning must be performed in
    {\meta} to deduce the metastatement $Theorem(``w'')$.
    The command can use the reflection rule (\ref{refl}) to assert a new proof
    line in the object level context (the context in which object level 
    reasoning is performed and where the command \C{reflect} is typed in).
  \item
    Any context can be the object level context, {\meta} itself.
    The amalgamation of the object and meta level is a particular case of {\GF}
    metareasoning. 
\end{itemize}

In {\meta}, the user is free to declare any language, any set of axioms and to 
define any computational model. This amounts to say that {\meta} is the 
``metatheory'' of a theory represented in a context as far as the user defines
the appropriate attachments and axioms.
A special unary predicate symbol which can be declared in {\meta} is
{\tt THEOREM}: this is the predicate recognized as meaning theoremhood by the
the reflect rule (\ref{refl}) in the command {\tt reflect}.

