\gfcommand{reflect}{reflection}
\index{reflect}
\label{sec-refl}

\gfsyntax{
  reflect \ARG{M-fact} \ARG{arg1} \ARG{arg2} \SEQ \ARG{argN};
}

\gfdescription{
  In the   following description we  call ``object context''  the 
  context where the {\tt reflect} command is executed.

  \ARG{M-fact} is any fact of  the context {\meta} whose wff is of the
  form $\forall x_1 x_2\ldots x_n A(x_1, x_2,\ldots,x_n)$, $(n \geq 0)$,
  where the sorts of the variables $x_1, x_2,\ldots,x_n$ 
  correspond to some {\GF} syntactic category ({\em term, wff, fact} ... ).
  For any syntactic category corresponding to a sort in {\meta},
  {\GF} provides the necessary parsing routine.
  This parsing routine is necessary to run the reflect command.
  The relation between sorts in {\meta} and the associated parsed syntactic
  category is the following:

  \begin{figure}
    \begin{tabular}{|l|l|}
   \hline
   {\bf sort}&  {\bf syntactic category} \hspace{7cm} \\ \hline \hline
   SENTCONST &  a {\em sentconst}; \\
   QUANT     &  a {\em quant} (quantifier: {\tt forall} or {\tt exists}); \\
   SORT      &  a symbol declared as a sort; \\
   DECSYM    &  any declared symbol: {\em sym};\\
   FACT      &  a {\em fact};\\
   WFF       &  a {\em wff}; \\
   WFFIF     &  a {\em wffif};\\
   QUANTWFF  &  a {\em quantwff} (of the form  {\tt forall ... }  or 
                {\tt exists  ... });\\
   AWFF      &  an atomic wff (a wff of the form {\tt P( ... ))};\\
   TERM      &  a {\em term}; \\
   TERMIF    &  a {\em termif};\\
   INDSYM    &  a symbol declared as an {\em indconst} or {\em indvar} or
                {\em indpar}; \\
   INDVAR    &  a symbol declared as an {\em indvar};\\
   INDPAR    &  a symbol declared as an {\em indpar};\\
   INDCONST  &  a symbol declared as an {\em indconst};\\
   SENTSYM   &  a symbol declared as a {\em sentconst} or a {\em sentpar}; \\
   SENTPAR   &  a symbol declared as a {\em sentpar}; \\
   SENTCONST &  a symbol declared as a {\em sentconst}; \\
   APPLSYM   &  a symbol declared as a {\em funconst, funpar, predconst, 
                predpar} \\
             &  or a boolean connective; \\ 
   PREDSYM   &  a symbol declared as a {\em predconst} or {\em predpar}; \\
   PREDPAR   &  a symbol declared as a {\em predpar}; \\
   PREDCONST &  a symbol declared as a {\em predconst}; \\
   FUNSYM    &  a symbol declared as a {\em funconst} or {\em funpar}; \\
   FUNPAR    &  a symbol declared as a {\em funpar}; \\
   FUNCONST  &  a symbol declared as a {\em funconst}; \\ \hline 
   \end{tabular}
   \caption{
	Sorts for {\GF} syntactic categories.
   }
   \end{figure}

   If, for example,  the sort of the variable $x_1$  is SENTCONST,  we
   expect $x_1$ to denote any object that is declared  as  a {\em sentconst}
   in the object  context .  Therefore \ARG{arg1}, \ARG{arg2}, \SEQ, \ARG{argN}
   are objects in the object context, such that all \ARG{arg$_i$} are of
   the syntactic category corresponding to the sort of the variables $x_i$
   in \ARG{M-fact}.

   In the following we give an explanation of the major steps performed during
   the execution of the command {\tt reflect} \cite{giunchiglia3}~\footnote{
   This description is the generalization of the example reported below
   and copied from \cite{giunchiglia3}.}:

   \begin{enumerate}
   \item 
     In the object context, when \C{REFLECT} is executed, {\GF}, after parsing 
     \C{REFLECT}, knows that the next argument is the label of a fact in 
     {\meta}. 
     Thus, {\GF} switches context automatically and goes to {\meta}.
   \item
     In {\meta}, the first argument of the command (\ARG{M-fact}) is parsed 
     and the formula of the fact whose label is \ARG{M-fact} is returned:
     $\forall x_1 x_2\ldots x_n A(x_1, x_2,\ldots,x_n)$.
     The variables $x_1, x_2,\ldots,x_n$ must be instantiated to constants
     in {\meta} which will be the names of the objects in the object context
     \ARG{arg1}, \ARG{arg2}, \SEQ, \ARG{argN}.
     The syntactic type of these objects must correspond to the sort of the 
     variables in {\meta}.
     For instance, if the sort of $x_1$ is WFF, then it must be instantiated
     to a constant denoting a {\it wff} in the object context. At this point
     {\GF} knows that \ARG{arg1} must be a {\it wff} in the object context,
     and so {\em arg}$_1$ can be parsed.
     This step provides {\GF} with the information needed to parse \ARG{arg1},
     \SEQ, \ARG{argN} in the object context.
   \item 
     {\GF} switches to the object context and parses  \ARG{arg1}, \ARG{arg2}, 
     \SEQ, \ARG{argN}, using the parsing functions of the syntactic categories
     corresponding to the variables $x_1, x_2,\ldots,x_n$ respectively.
   \item 
     {\GF} switches to {\meta} and automatically declares $n$ constants in 
     {\meta}. Let them be $c_1, c_2, ... c_n$, with the sorts of $x_1, x_2, 
     ... x_n$ respectively.
     Any constant in $c_1, ... c_n$ is automatically ``attached'' to the objects
     returned by parsing \ARG{arg1}, \ARG{arg2}, \SEQ, \ARG{argN} respectively.
     The representation associated to these constants is the representation
     declared for their sorts (see the command {\bf represent}, section
     \ref{sec-rep-sort}).
   \item 
     Still in {\meta}, an universal elimination is performed on 
     $\forall x_1 x_2\ldots x_n A(x_1, x_2,\ldots,x_n)$ obtaining
     $A(c_1, \ldots ,c_n)$.
   \item 
     Still in {\meta}, the formula $A(c_1,...,c_n)$ is automatically evaluated
     by the command {\tt eval} (see section \ref{sec-eval}).
     In this step {\GF}, by using the command {\bf eval}, performs metalevel
     reasoning by computation in the model.
     This metalevel reasoning is used to compute a formula of the form 
     $Theorem(``w'')$.
     If the metalevel reasoning does not lead to $Theorem(``w'')$, then an error
     message is returned. Otherwise the evaluation of the ground term ``$w$''
     gives $w$, the formula of a fact to be asserted in the object context by
     $R_{down}$
   \item
     At this point the reflection rule $R_{down}$ can be applied.
     Thus {\GF} forgets the constants $c_1 ... c_n$ declared in {\meta} and the
     attachments, switches back to the object context and asserts a new fact
     whose formula is $w$ and whose dependencies are the union of the 
     dependencies of the facts in {\em arg$_1$ ... arg$_n$} if there are any.
   \end{enumerate}

	Let us define two context {\tt OBJ} and {\tt META} as follows:

	\gfsourcefile{example.tst}{
	  NAMECONTEXT META; \\
	  DECLARE SORT TERM WFF;\\
	  DECLARE PREDCONST THEOREM 1;\\
	  DECLARE FUNCONST mkequal (INDVAR, INDVAR) = WFF;\\
	  DECLARE INDVAR x [TERM];\\
	  AXIOM M1: forall x. THEOREM(mkequal(x,x));\\
	  DECREP TERM;\\
	  DECREP WFF;\\
	  REPRESENT \{TERM\} AS TERM;\\
	  REPRESENT \{WFF\} AS WFF;\\
	  ATTACH mkequal TO [TERM,TERM=WFF] mkequ;\\
	  MAKECONTEXT OBJ;\\
	  SWITCHCONTEXT OBJ;\\
	  DECLARE INDCONST c;\\
	  DECLARE INDVAR   x;\\
	  DECLARE FUNCONST f 2;
	}

	Let's now type the following lines in {\GF}:

	\begin{quote}\tt
	   ***** FETCH example.tst;
	   ...

	   ***** REFLECT M1 c;
	   1  c = c;
	   ***** REFLECT M1 f(x,f(c,c));
	   2  f(x,f(c,c)) = f(x,f(c,c))
	\end{quote}

	Let us describe  step by step what happened during  the execution of
	the two  previous command \C{REFLECT}.

	\begin{enumerate}
	\item
    In {\tt OBJ} the word \C{REFLECT}  is parsed.
    The next argument must be  the name of a fact  in {\meta}.  Thus {\GF}
    automatically switches context and goes to {\meta}.
	\item
    In  {\meta}, {\tt M1} is parsed  and the axiom
    {\tt forall  x.THEOREM(mkequ(x,x))} is returned.  The variable {\tt x}
    in {\tt M1} must be instantiated  to a constant  in {\meta} which will
    be the name of a symbol of the language  of {\tt OBJ}.   Since the
    sort of  $x$ is {\em TERM}, then  the symbol of  the language  of {\tt
    OBJ}  must have  syntactic  type {\em term}   (it must be  a term).
	\item
    {\GF} switches to the context {\tt OBJ}.
    In  {\tt OBJ} the   term  {\tt  c} [{\tt f(x,f(c,c))} in  the second case]
    is  parsed.  
	\item
    Since no more arguments are needed, {\GF} switches back to {\meta}.
    In {\meta} a new constant, say {\tt C1} of sort {\tt TERM} is created and
    added to the language of {\meta}.
    {\tt   C1} is {\em attached} to the term {\tt c}  [{\tt f(x,f(c,c))}] 
    of the language of {\tt OBJ}.  
    The representation associated to {\tt C1} is {\tt TERM} which is associated
    to the sort {\tt TERM} by the command {\tt REPRESENT} in example.tst.
    In this way,  {\tt C1} is defined  as the name in  {\meta} 
    of {\tt c} [{\tt f(x,f(c,c))}].   
	\item
    Still in {\meta}, an   universal elimination is performed on {\tt  M1}
    obtaining\\
    {\tt THEOREM(mkequ(C1,C1))}.
	\item
    Still in {\meta}, {\tt  THEOREM(mkequ(C1,C1))} is evaluated in {\meta}'s 
    model.
    {\tt THEOREM} has no interpretation, {\tt mkequ(C1,C1)} evaluates to 
    {\tt c = c} [{\tt f(x,f(c,c)) = f(x,f(c,c))}], namely {\tt mkequ(C1,C1)}
    turns out to be the name of {\tt c = c} [{\tt f(x,f(c,c)) = f(x,f(c,c))}].
    So the result of this step is something that could be written as 
    {\tt THEOREM(``c =  c'')}.
    [{\tt THEOREM(``f(x,f(c,c)) = f(x,f(c,c))'')}],  where {\tt ``c = c''}
    [{\tt ``f(x,f(c,c)) = f(x,f(c,c))''}]  should be read as  the  name  of
    {\tt c = c} [{\tt (f(f(c,c),c),f(x,c)) = f(x,f(c,c))}]
	\item
    At  this point the  reflection rule can be applied.
    {\GF} forgets everything in {\meta}, (in this case {\tt C1} from the
    language and {\tt c} [{\tt f(x,f(c,c))}] from the domain of the
   	interpretation).
    {\GF} switches back to the context {\tt OBJ}, and assert a new fact
    with wff  {\tt  c = c} [{\tt  f(x,f(c,c))}] and with the empty deplist.
    \end{enumerate}

    The following example shows how {\GF} computes the deplist of a fact derived 
    by a {\tt reflect} command whose arguments contain a fact.
}

\gfrecap{
	Reflection
}

\gfexample+
   <host prompt> cat example.tst
   NAMECONTEXT META;\\
   DECLARE SORT FACT WFF;\\
   DECLARE INDVAR fc [FACT];\\
   DECLARE FUNCONST wffof (FACT) = WFF;\\
   DECLARE PREDCONST THEOREM 1;\\
   DECREP FACT;\\
   DECREP WFF; \\
   REPRESENT \{WFF\} AS WFF; \\
   REPRESENT \{FACT\} AS FACT;\\
   ATTACH wffof TO [FACT = WFF] fact\-get\-wff;\\
   AXIOM M2: forall fc. THEOREM(wffof(fc));\\
   MAKECONTEXT OBJ;\\
   SWITCHCONTEXT OBJ;\\
   DECLARE SENTCONST A;

   <host prompt> GETFOL
   ...

   ***** FETCH example.tst;
   ***** ASSUME A;
   1   A     (1)
   ***** REFLECT M2 1;
   2   A     (1)
+

