\section{Systems and Modules}
\label{sysmod}

It is common practice during the development of a software package
to design it as set of interacting components.
The main advantages are the ease of development and maintenance
as well as the flexibility of the resulting system.
Modularity allows both the development and the maintenance activities
to be confined inside the modules.
Furthermore, having encapsulated the functionalities of the system inside
modules, it is possible to build different applications simply by assembling
in different ways the available modules.

In the following we will refer to the application to be built by means
of the term {\it system} and to the components to be assembled
with the term {\it modules}.

\subsection{Modules}

A module consists of a set of files containing source code.
The configuration of a module is kept in a data structure containing the
following slots:

\begin{itemize}
	\item {\bf name}:
		a string containing the name of the module.
	\item {\bf files}:
		a list of data structures containing the following information for each
		file of the module: 
		%
		\begin{itemize}
			\item the name of the source file,
			\item the name of the object file,
			\item the mode flag for the file.
		\end{itemize}
		%
		{\it mode} is a flag indicating whether the interpreted or the compiled
		version of the file is to be loaded while assembling the module.
%
% The S-expressions are added to the existing FILES
% list by means of the action:
% (MODULE-ADD-FILE MODULE SRCFILE OBJFILE MODE)
% The module configuration file usually contains a
% sequence of such actions.
%
	\item {\bf srcdir}:
		the name of the directory containing the source files of the module.
	\item {\bf objdir}:
		the name of the directory containing the compiled files of the module.
	\item {\bf docdir}
		the name of the directory containing the documentation of the module.
	\item {\bf docfile}
		the name of the file containing the documentation of the module.
	\item {\bf mode}
		is a flag. If it is set to {\tt INTERPRETED} then all
		the source files of the module are loaded ignoring the {\em modes}
		associated to the files.
		If it is set to {\tt COMPILED} then the modes associated to the files are
		taken into account.
\end{itemize}

Usually all the instructions needed to configure a module are stored in a
file ({\em module configuration file}).
An example of module configuration file (for the {\em decide} module)
is given in figure~\ref{modcfg}.

\begin{figure}
 \begin{boxverbatim}
\begin{verbatim}
;************************************************************************;
;                                                                        ;
;                    "DECIDE" MODULE CONFIGURATION FILE                  ;
;                                                                        ;
;************************************************************************;

(MODULE-INIT 'DECIDE)
(MODULE-SET-NAME 'DECIDE "DECIDE")
(MODULE-SET-SRCDIR 'DECIDE (PATHNAME-CONCAT %HOME-DIR% "fol" "decide"))
(MODULE-SET-OBJDIR 'DECIDE (PATHNAME-CONCAT %HOME-DIR% "o"))
(MODULE-SET-MODE 'DECIDE 'COMPILED)

; adding the files to the module
;
(MODULE-ADD-FILE 'DECIDE "vdecide.cl"   ""         'INTERPRETED)
(MODULE-ADD-FILE 'DECIDE "decide.hgk"   "decideh"  'COMPILED)
(MODULE-ADD-FILE 'DECIDE "taut.hgk"     "tauth"    'COMPILED)
(MODULE-ADD-FILE 'DECIDE "tauteq.hgk"   "tauteqh"  'COMPILED)
(MODULE-ADD-FILE 'DECIDE "monadic.hgk"  "monadich" 'COMPILED)
(MODULE-ADD-FILE 'DECIDE "cmdecide.fol" "cmdecidf" 'COMPILED)
(MODULE-ADD-FILE 'DECIDE "decide.fol"   "decidef"  'COMPILED)
(MODULE-ADD-FILE 'DECIDE "ptaut.fol"    "ptautf"   'COMPILED)
(MODULE-ADD-FILE 'DECIDE "taut.fol"     "tautf"    'COMPILED)
(MODULE-ADD-FILE 'DECIDE "tauteq.fol"   "tauteqf"  'COMPILED)
(MODULE-ADD-FILE 'DECIDE "monadic.fol"  "monadicf" 'COMPILED)
(MODULE-ADD-FILE 'DECIDE "idecide.fol"  ""         'INTERPRETED)
\end{verbatim}
\end{boxverbatim}

 \caption{
	Configuration file for the {\em decide} module
 }
 \label{modcfg}
\end{figure}


\subsection{Systems}

A system is a whole application.
A system is built by assembling a set of modules.
The configuration of a system is kept in a data structure containing the
following slots:

\begin{itemize}
	\item {\bf name}:
		a string containing the name of the system.
	\item {\bf version}:
		a string containing the version number of the system.
	\item {\bf release}:
		 a string containing the month and the year.
	\item {\bf banner}:
		a string containing the banner of the system.
	\item {\bf modules}:
		list of the modules of the system.
	\item {\bf mode}:
		a flag such that if it is set to {\tt INTERPRETED} then all
		the source files of the system are loaded indipendently of the {\em
		modes} associated to the modules and to the files.
		If it is set to {\tt COMPILED} then the modes associated to the modules
		and to the files are taken into account.
	\item {\bf fixfile}:
		the path name of the file containing the source code that has been
		modified since the system has been compiled.
		This file is loaded each time the system is run.
	\item {\bf debugfile}:
		path name of the file containing the code that the user wants to be
		loaded in each time the system is run. 
		Useful for debugging.
	\item {\bf startfile}:
		path name of the file containing GETFOL instructions which are to be
		executed at start-up.
	\item {\bf scrdir}:
		path name of the directory containing source code.
	\item {\bf docdir}:
		path name of the directory containing documentation files.
	\item {\bf objdir}:
		path name of the directory containing object files;
	\item {\bf docfile}:
		path name of the file containing the documentation of  the system.
\end{itemize}

An example of system configuration file (for the {\tt GETFOL} system)
is depicted in figure~\ref{syscfg}.

\begin{figure}
 \begin{boxverbatim}
\begin{verbatim}
;************************************************************************;
;                                                                        ;
;                      FOL SYSTEM CONFIGURATION FILE                     ;
;                                                                        ;
;************************************************************************;

(SYS-INIT 'FOL)
(SYS-SET-NAME 'FOL "FOL")
(SYS-SET-RELEASE 'FOL "July 1991")
(SYS-SET-VERSION 'FOL "2")
(SYS-SET-FIXFILE 'FOL (PATHNAME-CONCAT %HOME-DIR% "fol" "fix.fol"))
(SYS-SET-DEBUGFILE 'FOL (PATHNAME-CONCAT %HOME-DIR% "fol" "debug.fol"))
(SYS-SET-MODE 'FOL 'COMPILED)

; loading the module configuration files
(LOAD (PATHNAME-CONCAT %HOME-DIR% "fol" "system"   "system.cfg"))
(LOAD (PATHNAME-CONCAT %HOME-DIR% "fol" "language" "language.cfg"))
(LOAD (PATHNAME-CONCAT %HOME-DIR% "fol" "proof"    "proof.cfg"))
                                .
                                .
; adding the modules to the system
(SYS-ADD-MODULE 'FOL 'SYSTEM)
(SYS-ADD-MODULE 'FOL 'LANGUAGE)
(SYS-ADD-MODULE 'FOL 'PROOF)
                                .
                                .
                                .
(SYS-SET-BANNER 'FOL "
;*************************************************************************;
;                                                                         ;
;                    +------------------------------+                     ;
;                    |             FOL              |                     ;
;                    +------------------------------+                     ;
;                                                                         ;
                                    .
                                    .
                                    .
\end{verbatim}
\end{boxverbatim}

 \caption{Configuration file for the {\tt GETFOL} system}
 \label{syscfg}
\end{figure}

A system is a hierarchical structure that can be regarded as a tree
whose root branches into the modules which, in turn, branch into a set
of leaves consisting of the files containing the source code.

Task of the installation facilities is to run across such a tree and to
build up the corresponding system.
The primitive operations are the loading of the source or the compiled files
for each module of the system.
Before loading a compiled file, the installation facilities verify that
it is up to date with respect to the corresponding source file.
If it is not the case the source file is automatically compiled and the old
compiled file is substituted with the new one that is finally loaded.

{\it Mode flags} are particularly useful for debugging.
Mode flags can be associated to the source files,
to the modules and to the system itself.
The mode flag associated to the files indicates whether the source file
or the corresponding compiled file is to be loaded.
The mode flags of the system and of the modules allows you to alter this
situation in the following sense.
If the mode flag of a module is {\tt INTERPRETED} then all the source files
of the module, regardless of their own flag mode, are loaded.
Analogously the mode flag of a system set to {\tt INTERPRETED} forces the
loading of all the source files of all its modules (i.e. no compiled file
is loaded).
The mode flag of a file set to {\tt COMPILED} forces the compiled file
to be loaded, while the mode flag of the system or of a module set to
{\tt COMPILED} delegates the choice to the mode flags of the corresponding 
components.
