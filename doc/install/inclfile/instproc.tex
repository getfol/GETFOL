\section{The installation procedure}
\label{instproc}
In order to install a system the following steps are to be performed:
\begin{enumerate}
	\item
		Go in the {\tt make} directory.
	\item {\bf [optional]}
		Edit the file {\tt host.lsp} and set the values of host-dependent
		variables.
	\item
		Run the Common Lisp you are provided of.
	\item
		Load the file {\tt install.lsp}.
	\item {\bf [optional]}
		If you didn't edit the file {\tt host.lsp} (step 2), call the form:
		%
		\begin{itemize}
			\item
				{\tt (configure)}
		\end{itemize}
		%
		to set the values of host-dependent variables.
	\item 
		At LISP prompt type one of the following three forms:
		%
      \begin{itemize}
	      \item {\tt (MAKE-HGKM)}
	      \item {\tt (MAKE-GETFOL)}
			\item {\tt (MAKE-AGETFOL)}
      \end{itemize}
		%
		if you want to install {\tt HGKM}, {\tt GETFOL} or {\tt AGETFOL}
		respectively. 
	\item
		After all the files have been successfully loaded type at the LISP
		prompt the following form:
		%
      \begin{itemize}
	      \item {\tt (SYSTEM-SAVE {\it filename})}
      \end{itemize}
		%
		This saves the {\it image} of the whole environment onto a file named
		{\it filename}\footnote{
			{\it filename} must be a string.
		}: this allows you to recall the system simply by running the executable
		file stored in {\it filename} without the need of reloading all the
		object files into the Common Lisp environment.
\end{enumerate}

The configuration files for the {\tt HGKM}, {\tt GETFOL} and {\tt AGETFOL}
systems are provided in a standard form.
Before altering the configuration file of a system or a module
we recommend that you read section~\ref{sysmod} of this manual carefully.
