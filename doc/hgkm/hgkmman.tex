%............................... HGKM MANUAL .................................
%.............................................................................


\documentstyle[11pt]{../styfiles/GFmanual}

%................................. macros ....................................
%.............................................................................

\setcounter{totalnumber}{1}

\def\actionzero#1{\noindent ({\tt {#1}}) \hfill action}
\def\actionone#1#2{\noindent ({\tt {#1}} {\it [{#2}]}) \hfill action}
\def\actiontwo#1#2#3{\noindent ({\tt {#1}} {\it [{#2}]}{\it [{#3}]}) \hfill
action} 
\def\actionthree#1#2#3#4{\noindent ({\tt {#1}} {\it [{#2}]}{\it [{#3}]}{\it
[{#4}]}) \hfill action}

\def\functionzero#1#2{\noindent ({\tt {#1}}) $\Longrightarrow$ {#2} \hfill
function} 
\def\functionone#1#2#3{\noindent ({\tt {#1}} {\it [{#2}]}) $\Longrightarrow$
{#3} \hfill function}
\def\functiontwo#1#2#3#4{\noindent ({\tt {#1}} {\it [{#2}]}{\it [{#3}]})
$\Longrightarrow$ {#4} \hfill function} 
\def\functionthree#1#2#3#4#5{\noindent ({\tt {#1}} {\it [{#2}]}{\it [{#3}]}{\it
[{#4}]}) $\Longrightarrow$ {#5} \hfill function} 

\def\predicatezero#1{\noindent ({\tt {#1}}) \hfill predicate}
\def\predicateone#1#2{\noindent ({\tt {#1}} {\it [{#2}]}) \hfill predicate}
\def\predicatetwo#1#2#3{\noindent ({\tt {#1}} {\it [{#2}]}{\it [{#3}]}) \hfill
predicate}
\def\predicatethree#1#2#3#4{\noindent ({\tt {#1}} {\it [{#2}]}{\it [{#3}]}{\it
[{#4}]}) \hfill predicate}

\def\sysconst#1{\noindent{\tt {#1}} \hfill system constant}

\def\chan {{\it channel\/}}
\def\chr {{\it chr\/}}
\def\chrlist {{\it chr-list\/}}
\def\debmode {\%DE\-BUG\--MO\-DE\%}
\def\dev {{\it device\/}}
\def\dir {{\it direction\/}}
\def\filena {{\it filena\/}}
\def\flconst {first level constructor}
\def\flrec {first level recognizer}
\def\form {{\it form\/}}
\def\formlist {{\it form list\/}}
\def\funcdesc {{\it function description\/}}
\def\infmode {\%IN\-FOR\-MA\-TION-MO\-DE\%}
\def\inte {{\it int\/}}
\def\LIST {{\it list\/}}
\def\LISTone {{\it list\/$_1$\/}}
\def\LISTtwo {{\it list\/$_2$\/}}
\def\obj {{\it obj\/}}
\def\objn {{\it obj\/$_n$\/}}
\def\objone {{\it obj\/$_1$\/}}
\def\objtwo {{\it obj\/$_2$\/}}
\def\pair {{\it pair\/}}
\def\sbral {{\it argument list\/}}
\def\sbrbo {{\it body\/}}
\def\sbrna {{\it subroutine name\/}}
\def\str {{\it str\/}}
\def\sym {{\it sym\/}}
\def\symone {{\it sym\/$_1$\/}}
\def\symtwo {{\it sym\/$_2$\/}}
\def\value {{\it value\/}}

% ................................. document ..................................
% .............................................................................

% \makeindex
\begin{document}

\title{HGKM User Manual}
\author{{\bf Fausto Giunchiglia} and {\bf Alessandro Cimatti}}
\date{4 March 1994}
\version{2.1}
\abstract{
	This manual describes the implementation of a simple programming language,
	{\HG}. 
	{\HG} was originally defined and developed by Richard Weyhrauch and Caroline
	Talcott at the Department of Computer Science of Stanford University.
	The version described here sticks as much as possible the conventions and the
	specifications of the original version.
	The implementation is almost totally new and some new functionalities (e.g.
	{\tt DEFSUB}) have been added.
	Theoretical {\HG} was defined so that its semantics could be easily
	represented in first order logic.
	The version of {\HG} described in this manual acts as the implementation
	language for {\GF}.
	This manual has been written only to facilitate the
	implementation/modification of {\GF}.
}
\addresses{
   \begin{tabular}[c]{l@{\hskip 2em}l}
       {\bf Fausto Giunchiglia}:	    & {\bf Alessandro Cimatti}           \\
       Mechanized Reasoning Group       & Mechanized Reasoning Group,        \\
       IRST, Povo, 38050 Trento, Italy  & IRST, Povo, 38050 Trento, Italy    \\
       e-mail: {\tt fausto@irst.it}     & e-mail: {\tt cx@irst.it}           \\
       phone: +39 461 314359            & phone: +39 461 314351
   \end{tabular}
}
\published{
	\begin{tabular}{l}
		DIST Technical Report No. 9107-05 (1991). \\
		DIST -- University of Genoa,\\
		Via Opera Pia 11A, 16145 Genova, Italy.
    \end{tabular}
}

% ................................... cover ...................................

\thispagestyle{empty}
\maketitle

% ............................. table of contents ...............................

\newpage
\pagenumbering{roman}
\tableofcontents
\newpage
\pagenumbering{arabic}


\section{Introduction}

This manual describes an enhanced version of {\HG}, which has been revised by
the authors within the {\GF} project\cite{giunchiglia34,giunchiglia12,giunchiglia29}.
In effect, the most recent version of {\GF} has been implemented on this
version of {\HG}.

{\HG} is a programming language which has been implemented on top of Common
LISP; it is powerful and simple to use. {\HG} has been developed mainly for
three reasons: to increase {\GF}'s portability, to improve {\GF}'s performance
and to introduce a simple mechanism of data typing.

A language we will call {\it basic} {\HG} was developed including only a small
set of data types and a few built in primitives. This language is enough to
provide the full expressive power of {\HG}.

However, in order to facilitate the development of {\GF}, more data types and
primitives have been defined.
These additions form {\it extended} {\HG}.

This manual is not a tutorial.
It is a reference manual for people already familiar with LISP and
s-expressions notation.


\subsection{Basic concepts}

A {\HG} object belongs to a data type. For each data type three classes of
primitives are defined:

\begin{description}
    \item[predicates:]
        test if a condition is verified or not; predicates are forms which can
        be used as test for branching in conditional forms: that is, evaluating
		a predicate returns a truth value.
    \item[functions:]
        specify a correspondence between {\HG} data and returns an {\HG} object;
		the data type of returned object is specified too; a function produces
		no side effects.
    \item[actions:]
        specify a side effect, and do not return an {\HG} object.
\end{description}
%
\noindent
Furthermore for each data type we define some
%
\begin{description}
    \item[system constants:]
        symbols to which corresponds a fixed value
\end{description}


\subsection{Remarks about notation}

The syntax of primitives and system constants will be described using the
following notation:

\begin{description}
	\item \sysconst{<constant name>}
	\item \predicatethree{<predicate name>}{arg type}{arg type}{\ldots}
	\item \functionthree{<function name>}{arg type}{arg type}{\ldots}{res type}
	\item \actionthree{<action name>}{arg type}{arg type}{\ldots}
\end{description}

\begin{quote}\bf
	After the syntactic description we have a section in which the semantics are
	described.
	This section appears narrower than the rest of the text, in this way.
\end{quote}

The name of the primitive is printed in {\tt typewriter} style.
Its arguments are language forms which will be evaluated.
With {\it [arg type]} we mean the data type of the {\HG} object that the
evaluation of the argument form should return.
If the result of the evaluation does not match the specified data type a fatal
error is generated.

For instance, \inte\ denotes a form which evaluates to an integer.
If an operation requires more than one argument of a single type, the bracketed
type name will have a numbered subscript.
For example we write {\it str\/$_1$} and {\it str\/$_2$} to denote two different
forms that have strings as values.
In the case an operation does not care about the type of its arguments we write
\obj\ to indicate a form which can evaluate to any object in the {\HG} world.

In section 2 we present basic {\HG}; in section 3 extended {\HG}; finally, in
appendix we describe some aspects related to the implementation.


\newpage
\section{Basic {\HG}}

In this section we shall describe basic {\HG}: we present the structured data
types, the way in which the language forms are evaluated by the interpreter,
some special forms for subroutine definitions, the error handling and the
input/output primitives.


\subsection{Data structures}

Basic {\HG} data types are:

\begin{itemize}
	\item truth values {\it [trth];}
	\item symbols {\it [\sym];}
	\item integers {\it [\inte];}
	\item characters {\it [\chr];}
	\item strings {\it [\str];}
	\item pairs {\it [\pair];}
	\item lists {\it [list].}
\end{itemize}

In this chapter we shall refer to each data type by the corresponding name
enclosed in square brackets.

The syntax of data types is defined by this grammar:

\begin{bnf}
	\T{obj}			\sep  \T{atomic obj} | \T{compound obj} \\
	\T{atomic obj}	\sep  \T{trth} | \T{sym} | \T{int} | \T{chr} | \T{str} \\
	\T{compound obj}\sep  \T{pair} | \T{list} | \T{chr-list}\\
	\T{pair}	    \sep  ( \T{obj} . \T{obj}) \\
	\T{list}		\sep  empty-list | ( \T{obj} . \T{list} ) \\
	\T{chr-list}	\sep  empty-list | ( \T{chr} . \T{chr-list} )
\end{bnf}

In this grammar we refer explicitly to a data structure, the list of characters
{\it [chr-list],} which is not a primitive data type.
However it is very used in {\HG}, and it is interesting to define its syntax too.
Note that a list of characters is different from a string: in basic {\HG} a
string is an atomic object, while a list of characters has a compound structure.


\subsubsection{Truth values and equality predicate}

\sysconst {TRUE}\index{TRUE}
\begin{quote}
	{\tt TRUE} is a system constant which always evaluates to true.
\end{quote}

\sysconst {FALSE}\index{FALSE}
\begin{quote}
	{\tt FALSE} is a system constant which always evaluates to false.
\end{quote}

\predicatetwo{EQ}{\objone}{\objtwo}\index{EQ}
\begin{quote}
	{\tt EQ} is the equality predicate, and is defined for all data types.
	{\tt EQ} evaluates to true if both \objone\ and \objtwo\ are the same
	object, false otherwise.
\end{quote}


\subsubsection{Symbols}

{\HG} symbols are like Common LISP identifiers, with the only difference that
they are case sensitive.
{\tt abc}, {\tt CAR}, {\tt NIL}, {\tt THISISALONGATOM} are examples of {\HG}
symbols.

\bigskip

\predicateone{SYMBOL?}{\obj}\index{SYMBOL?}
\begin{quote}
	{\tt SYMBOL?} evaluates to true if \obj\ is a symbol, false otherwise.
\end{quote}

\actiontwo{SETVAL}{\sym}{\obj}\index{SETVAL}
\begin{quote}
	{\tt SETVAL} changes the value associated with the specified symbol to \obj.
\end{quote}

\functionone{GETVAL}{\sym}{\obj}\index{GETVAL}
\begin{quote}
	{\tt GETVAL} returns the value associated with \sym.
\end{quote}

\actionthree{SETPROP}{\symone}{\obj}{\symtwo}\index{SETPROP}
\begin{quote}
	{\tt SETPROP} sets the property of \symone\ with key \symtwo\ to have the
	value \obj.
\end{quote}

\functiontwo{GETPROP}{\symone}{\symtwo}{\obj}\index{GETPROP}
\begin{quote}
	{\tt GETPROP} returns the value of the property \symtwo\ of \symone.
	If there is no such property {\tt GETPROP} returns the false truth value.
\end{quote}
 
\functiontwo{REMPROP}{\symone}{\symtwo}{\obj}\index{REMPROP}
\begin{quote}
	{\tt REMPROP} removes from \symone\ the property with an indicator {\tt EQ}
	to \symtwo. 
	It returns the false truth value if there is no such indicator.
\end{quote}


\subsubsection{Integers}

Integers are the only type of numbers allowed in {\HG}.
\bigskip

\predicateone {INTEGER?}{\obj}\index{INTEGER?}
\begin{quote}
	{\tt INTEGER?} evaluates to true if \obj\ is an integer, false otherwise.
\end{quote}

\predicateone {ZERO?}{\obj}\index{ZERO?}
\begin{quote}
	{\tt ZERO?} evaluates  to true if \obj\ is the integer $0$, false otherwise.
\end{quote}

\predicateone {POSITIVE?}{\obj}\index{POSITIVE?}
\begin{quote}
	{\tt POSITIVE?} evaluates to true if \obj\ is a positive integer, false
	otherwise.
\end{quote}

\predicateone {NEGATIVE?}{\obj}\index{NEGATIVE?}
\begin{quote}
	{\tt NEGATIVE?} evaluates to true if \obj\ is a negative integer, false
	otherwise.
\end{quote}

\functionone {ADD1}{\inte}{\inte}\index{ADD1}
\begin{quote}
	{\tt ADD1} adds one to an integer.
\end{quote}

\functionone {SUB1}{\inte}{\inte}\index{SUB1}
\begin{quote}
	{\tt SUB1} subtracts one from an integer.
\end{quote}


\subsubsection{Characters}

{\HG} characters are all the characters to which is associated an ASCII code.
An {\HG} character is represented by the integer between $0$ and $255$
corresponding to its {\tt ASCII} code.

\bigskip

\predicateone {CHARACTER?}{\obj}\index{CHARACTER}
\begin{quote}
	{\tt CHARACTER?} evaluates  to true if \obj\ is a character, false otherwise.
\end{quote}

\functionone {CHR2SYM}{\chrlist}{\sym}\index{CHR2SYM}
\begin{quote}
	{\tt CHR2SYM} returns the symbol whose print name consists of the characters
	in \chrlist.

	For instance:
	%
	\begin{verbatim}
		> (CHR2SYM '(65 80 80 76 69))
		APPLE
	\end{verbatim}
\end{quote}

\functionone {SYM2CHR}{\sym}{\chrlist}\index{SYM2CHR}
\begin{quote}
	{\tt SYM2CHR} is the logical inverse of {\tt CHR2SYM}: it returns the list
	of the characters which compound the symbol.

	For instance:
	%
	\begin{verbatim}
		> (SYM2CHR 'APPLE)
		(65 80 80 76 69)
	\end{verbatim}
\end{quote}

\functionone {CHR2INT}{\chrlist}{\inte}\index{CHR2INT}
\begin{quote}
	{\tt CHR2INT} takes a list of characters which represent digits and returns
	the corresponding integer.
	If the list begins with a minus sign ``-'' then {\tt CHR2INT} returns the
	appropriate negative number.

	For instance:
	%
	\begin{verbatim}
		> (CHR2INT '(49 51 55)
		137
		> (CHR2INT '(45 49 51 55))
		-137
	\end{verbatim}
\end{quote}

\functionone {\tt INT2CHR}{\inte}{\chrlist}\index{INT2CHR}
\begin{quote}
	{\tt INT2CHR} is the logical inverse of {\tt CHR2INT}.
	{\tt INT2CHR} returns the list of characters corresponding to the digits of
	the integer.
	If the integer is negative then the first element of the list will contain
	the number $45$, which is the {\tt ASCII} code for the minus sign ``-''.

	Example:
	%
	\begin{verbatim}
		> (INT2CHR 137)
		(49 51 55)}
		> (INT2CHR -137)
		(45 49 51 55)
	\end{verbatim}
\end{quote}


\subsubsection{Strings}

A string is an arbitrary sequence of {\it n\/} characters, indexed from $1$ to
{\it n\/}.
A string is delimited by a pair of double quotes ({\tt "}) which do not belong
to it. {\HG} strings are case sensitive.
{\tt "Symbol"}, {\tt "number"}, {\tt "ThisIsAString"} are examples of {\HG}
strings.

\bigskip

\predicateone {STRING?}{\obj}\index{STRING?}
\begin{quote}
	{\tt STRING?} evaluates to true if \obj\ is a string, false otherwise.
\end{quote}

\functionone {LENSTR}{\str}{\inte}\index{LENSTR}
\begin{quote}
	{\tt LENSTR} returns an integer \inte\ equal to the number of characters in
	\str.
\end{quote}

\actionthree {SETSTR}{\str}{\inte}{\chr}\index{SETSTR}
\begin{quote}
	{\tt SETSTR} sets to \chr\ the \inte-th position of \str.
\end{quote}

\functiontwo {GETSTR}{\str}{\inte}{\chr}\index{GETSTR}
\begin{quote}
	{\tt GETSTR} returns the \inte-th character of \str.
\end{quote}

\functionone {\tt STR2CHR}{\str}{\chrlist}\index{STR2CHR}
\begin{quote}
	{\tt STR2CHR} returns a list of characters which appear in the string.
\end{quote}

\functionone {CHR2STR}{\chrlist}{\str}\index{CHR2STR}
\begin{quote}
	{\tt CHR2STR} returns a string of the length of \chrlist\ that contains the
	characters that appear in \chrlist.

	Example:
	%
	\begin{verbatim}
		> (STR2CHR "APPLE")
		(65 80 80 76 69)
		> (CHR2STR '(104 111 109 101)
		"home"
	\end{verbatim}
\end{quote}


\subsubsection{Pairs}

An {\HG} pair is the ordered collection of two {\HG} objects.

\bigskip


\predicateone{PAIR?}{\obj}\index{PAIR?}
\begin{quote}
	{\tt PAIR?} evaluates to true if \obj\ is a pair, false otherwise.
\end{quote}

\predicateone{ATOMIC?}{\obj}\index{ATOMIC?}
\begin{quote}
	{\tt ATOMIC?} evaluates to true if \obj\ is an atomic object, namely
	a symbol, an integer or a string, false otherwise.
\end{quote}

\functiontwo {CONS}{\objone}{\objtwo}{\pair}\index{CONS}
\begin{quote}
	{\tt CONS} returns a pair consisting of \objone\ and \objtwo.
\end{quote}

\functionone {CAR}{\pair}{\obj}\index{CAR}
\begin{quote}
	{\tt CAR} returns the first element of \pair.
\end{quote}

\functionone {CDR}{\pair}{\obj}\index{CDR}
\begin{quote}
	{\tt CDR} returns the second element of \pair.
\end{quote}

\actiontwo {SETCAR}{\pair}{\obj}\index{SETCAR}
\begin{quote}
	{\tt SETCAR} changes the first element of \pair\ to \obj.
\end{quote}

\actiontwo {SETCDR}{\pair}{\obj}\index{SETCDR}
\begin{quote}
	{\tt SETCDR} changes the second element of \pair\ to \obj.
\end{quote}


\subsubsection{Lists}

A list is either the empty list or a pair whose second element is a list.

\bigskip

\sysconst {MTL}\index{MTL}
\begin{quote}
	{\tt MTL} evaluates to the empty list.
\end{quote}

\predicateone {MTL?}{\obj}\index{MTL?}
\begin{quote}
	{\tt MTL?} evaluates to true if \obj\ is the empty list, false otherwise.
\end{quote}

\predicateone {LIST?}{\obj}\index{LIST?}
\begin{quote}
	{\tt LIST?} evaluates to true if \obj\ is a list, false otherwise.
\end{quote}

\functiontwo {LCONS}{\obj}{\LIST}{\LIST}\index{LCONS}
\begin{quote}
	{\tt LCONS} returns a list with \obj\ in the front of \LIST.
\end{quote}

\functionone {FIRST}{\LIST}{\obj}\index{FIRST}
\begin{quote}
	{\tt FIRST} returns the first element of \LIST.
\end{quote}

\functionone {REST}{\LIST}{\LIST}\index{REST}
\begin{quote}
	{\tt REST} returns the rest of \LIST.
\end{quote}

\actiontwo {SETFIRST}{\LIST}{\obj}\index{SETFIRST}
\begin{quote}
	{\tt SETFIRST} changes the first element of \LIST\ to \obj.
\end{quote}

\actiontwo {SETREST}{\LISTone}{\LISTtwo}\index{SETREST}
\begin{quote}
	{\tt SETREST} changes the rest of \LISTone\ to \LISTtwo.
\end{quote}


\subsection{Evaluation of forms}

This section describes how a form is evaluated by {\HG} interpreter.
The different types of forms are:

\begin{itemize}
	\item null operation {\tt <nop>}
	\item constant {\tt <const>}
	\item variable {\tt <var>}
	\item quotation {\tt <quote>}
	\item subroutine application {\tt <appl>}
	\item conditional evaluation {\tt <if>}
	\item operation sequence {\tt <seq>}
	\item local environment definition {\tt <let>}
	\item environment modification {\tt <setq>}
	\item program {\tt <pgm>}
\end{itemize}
%
In the rest of the chapter we shall refer to the different types of forms to be
evaluated by the corresponding names which are enclosed between angle brackets.
The syntax of the language forms is defined by the following grammar.

\begin{bnf}
	\T{form}           \sep \T{atomic form} | \T{compound form} \\
	\T{atomic form}	   \sep \T{nop} | \T{const} | \T{var} \\
	\T{compound form}  \sep \T{quote} | \T{appl} | \T{if} | \T{let} | \T{setq} |
							\T{seq} | \T{pgm} \\
	\T{nop}			   \sep ( NOP) \\
	\T{quote}		   \sep ( QUOTE \T{form}) \\
	\T{appl}		   \sep (\T{subroutine name} \{ \T{form} \})\\
	\T{if}			   \sep ( IF | IFN \T{predform} \T{form1} \T{form2}) \\
	\T{let}			   \sep ( LET ( \{ ( \T{var} \T{form} ) \} ) \{ \T{form} \})
							\\ 
	\T{seq}			   \sep ( SEQ \{ \T{form} \} )\\
	\T{pgm}			   \sep ( PGM ( \{ \T{var} \} ) \{ \T{pgm-stm} \} )\\
	\T{pgm-stm}		   \sep \T{form} | \T{rtn} | \T{jump} | \T{label}\\
	\T{rtn}			   \sep ( RETURN \T{form} ) \\
	\T{jump}		   \sep ( BRT | BRF \T{label} \T{predform} ) | ( GOTO
							\T{label} )\\
	\T{label}		   \sep \T{symbol}
\end{bnf}
%
Some explanation must be supplied: by constant ({\tt <const>}) we mean either
a truth value, an integer, a character or a string; by predicative form
({\tt <predform>}) we mean a particular form that evaluates to a truth value;
by program statements ({\tt <pgm-stm>}) we mean the special constructs that
compose a {\tt <pgm>} form statement list.
In the rest of this section we shall describe the semantics of each form,
referring to the grammar which has been presented.

\bigskip

{\bf Null operation}\index{NOP}
\begin{quote}
	The evaluation of the {\tt <nop>} form does nothing and does not return a
	value.
\end{quote}

{\bf Constant}
\begin{quote}
	The evaluation of a constant form (truth value, integer, character or
	string) returns the constant itself. Hence {\HG} constants are invariant
	with respect to the evaluation.
\end{quote}

{\bf Variable}
\begin{quote}
	A {\tt <var>} form is a symbol. This symbol must be the name of an
	initialized variable.
	In this case the result of the evaluation of the form is the value of the
	variable in the current environment; otherwise a fatal error occurs.
\end{quote}

{\bf Quotation}\index{QUOTE}
\begin{quote}
	The evaluation of a {\tt <quote>} form returns the argument form without
	evaluating it.
\end{quote}

{\bf Subroutine application}
\begin{quote}
	{\tt <subroutine name>} must be a symbol which is the name of an {\HG}
	function or predicate.
	It can be the name of a built-in primitive or a user-defined function.
	The argument {\tt <form>}s are evaluated and bound to the corresponding
	formal parameters of the function or predicate (call-by-value).
	The result of the application is the value obtained applying the specified
	subroutine to the arguments.
	The type of the values deriving from arguments must match properly the data
	type of the formal parameters, otherwise a fatal error occurs.

	Examples of subroutine applications are:
	%
	\begin{verbatim}
		> (CONS 1 13)
		(1 . 13)
		> (MTL? MTL)
		T
	\end{verbatim}
\end{quote}

{\bf Conditional expressions}\index{IF}\index{IFN}
\begin{quote}
	If the {\tt <predform>} evaluates to true, then the {\tt IF} conditional
	expression returns the evaluation of the first {\tt <form>}, otherwise of
	the second one.
	{\tt IFN} stands for {\tt IF NOT}; its semantics are trivial.

	For instance:
	%
	\begin{verbatim}
		> (IF (MTL? X) FALSE TRUE)
	\end{verbatim}
	%
	evaluates to false if {\tt X} evaluates to the empty list, true otherwise.
\end{quote}

{\bf Sequences of forms: \verb+<seq>+}\index{SEQ}
\begin{quote}
	The value of a {\tt SEQ} form is the value of its last form in the
	environment defined by the evaluation of previous forms.
\end{quote}

{\bf Local environment definition}\index{LET}
\begin{quote}
	{\tt LET} is used to create and initialize local variables.
	The result of evaluating a {\tt LET} is given by the evaluation of the last
	{\tt <form>} in a new environment in which each {\tt <var>} is associated
	with the value of the respective {\tt <form>}.
	If one of the initializing {\tt <form>}s contains any reference to any of
	the {\tt <var>}s, its value is taken from the lexically exterior
	environment.
	Hence the semantics of a {\tt <let>} form is the same as:
	%
	\begin{verbatim}
		(SEQ
		  (SETQ TEMP-1 <form-1>)
		  ...
		  (SETQ TEMP-n <form-n>)
		  (SETQ var-1 TEMP-1)
		  ...
		  (SETQ var-n TEMP-n)
		  <form>)
	\end{verbatim}
	%
	where {\tt TEMP-1} through {\tt TEMP-n} are symbols which do not appear in
	any {\tt <form-i>} or in the last {\tt <form>}.
\end{quote}

{\bf Environment modification}\index{SETQ}
\begin{quote}
	The evaluation of a {\tt <setq>} form produces a side effect on
	environments, and does not return a value.
	The evaluation of a {\tt <setq>} form sets the value associated with the
	variable {\tt <var>} to the evaluation of {\tt <form>}.
	If {\tt <var>} is a global variable the form {\tt (SETQ <var> <form>)} is
	equivalent to the form {\tt (SETVAL (QUOTE <var>) <form>)}.

	For instance:
	%
	\begin{verbatim}
		> (SETVAL (QUOTE X) 1)
		1
		> (LET ((X 5)) (SEQ (SETQ X 3) X)))
		3
	\end{verbatim}
	%
	evaluates to $3$ and leaves the value of {\tt X} to be $1$.
	%
	\begin{verbatim}
		(SEQ
		  (SETVAL (QUOTE X) 1)
		  (LET ((X 5)) (SEQ (SETVAL (QUOTE X) 3) X)))
	\end{verbatim}
	%
	evaluates to $5$ and leaves the value of {\tt X} to be $3$.	
\end{quote}

{\bf Program}\index{PGM}
\begin{quote}
	The {\tt <pgm>} form provides a facility for executing sequential programs,
	something like {\tt PROG}s in Common Lisp.
	The default order of executing {\tt <pgm-stm>}s is the sequential order.
	This can be modified by the use of {\tt <jump>} or of {\tt <rtn>}
	instructions.
	The {\tt <var>}s are local to the {\tt <pgm>} form and they have no initial
	value. 
\end{quote}

{\bf Jumps}\index{BRT}\index{BRF}\index{GOTO}
\begin{quote}
	The {\tt <jump>} statements can be either conditional ({\tt BRT} and
	{\tt BRF}) or unconditional ({\tt GOTO}).
	{\tt BRT} branches to the {\tt <label>} if the {\tt <predform>} evaluates to
	true.
	{\tt BRF} branches to the {\tt <label>} if the {\tt <predform>} evaluates to
	false.
	{\tt GOTO} branches to {\tt <label>}.
\end{quote}

{\bf Labels}
\begin{quote}
	{\tt <label>}s are statements which are simply symbols.
	They are not evaluated but provide markers for the {\tt <jump>} statements.
\end{quote}

{\bf Return}\index{RETURN}
\begin{quote}
	At the evaluation of a {\tt <rtn>} the {\tt <pgm>} statements evaluation is
	halted and the value of {\tt <form>} is returned.
	As shown by the grammar, a {\tt <rtn>} statement can only be used as
	statement into a {\tt <pgm>} form.
\end{quote}


\subsubsection{Eval and apply}

In this section we describe two special form for evaluating forms and applying
functions to forms.

\bigskip

\functionone {HGKM-EVAL}{\form}{\obj}\index{HGKM-EVAL}
\begin{quote}
	{\tt HGKM-EVAL} evaluates \form\ in the current environment and returns the
	resulting \obj.
\end{quote}

\functiontwo {HGKM-APPLY}{\funcdesc}{\formlist}{\value}\index{HGKM-APPLY}
\begin{quote}
	{\tt HGKM-APPLY} binds each element of the form list (without evaluating it)
	to the corresponding element of the bound variables list of the function
	description, and returns the value of evaluating the body of the function
	description in this new environment.

	Example:
	%
	\begin{verbatim}
		> (HGKM-APPLY (GET-FUN 'APPEND) '((A B) (C D)))
		(A B C D)
		> (HGKM-APPLY (GET-FUN 'APPEND) '(LIST '(B) '(C D)))
		(B C D)
	\end{verbatim}
\end{quote}


\subsection{Special forms}

In {\HG} special forms to define functions and macros are provided.
We call this special forms \flconst s, to distinguish them from Common LISP
constructors (level zero constructors) which have been used to define built-in
forms. 
In this section we also present {\HG} special forms to inspect and describe
built-in and user-defined subroutines: we shall refer to these special forms
using the term "\flrec s".


\subsubsection{First level constructors}

The \flconst s are

\begin{itemize}
	\item {\tt DEFLAM}
	\item {\tt DEFMAC}
	\item {\tt DEFSYS}
	\item {\tt DEFMACSYS}
	\item {\tt DEFSUB}
	\item {\tt SET-FUN}
\end{itemize}

A \flconst\ call has the usual form:

\begin{quote}\tt
	(<constructor name> {\it [\sbrna] [\sbral] [\sbrbo]})
\end{quote}

The result of a such call is the definition of a subroutine named {\it \sbrna},
in which {\it \sbral\/} is the list of formal parameters and {\it \sbrbo\/} is
the statement list.
The syntax of user defined {\HG} subroutines is the same as Common LISP
subroutines: hence are allowed all CL options, namely {\tt \&REST, \&OPTIONS,}
etc.

The way in which a subroutine is implemented depends on which is the \flconst
used in defining it, and on the value of two installation variables at the
moment of definition.
Anyway the implementation is almost transparent to the user, and so here we give
only a brief description of implementation features.

We shall explain now the meaning of an installation variable, which is named
{\tt \infmode}: the main idea is that when this variable evaluates to true some
system information is set up at the moment of installation, while when it
evaluates to false no information is available.
The following are special forms which {\it do not} evaluate their arguments.

\bigskip

\actionthree {DEFLAM}{\sbrna}{\sbral}{\sbrbo}\index{DEFLAM}
\begin{quote}
	{\tt DEFLAM} is a function constructor; which always allows the redefinition
	of previously defined functions.
	If {\tt\infmode} evaluates to false, then {\tt DEFLAM} just defines the
	function.
	If {\tt\infmode} evaluates to true, {\tt DEFLAM} also sets a property
	telling that {\it \sbrna\/} is a first level defined function and, in case
	of redefinition, prints a warning message.
\end{quote}

\actionthree {DEFMAC}{\sbrna}{\sbral}{\sbrbo}\index{DEFMAC}
\begin{quote}
	{\tt DEFMAC} is a macro definer: hence {\it \sbrbo\/} must be a macro
	statement list, that is must be ``expansible'' like Common LISP macros.
	{\tt DEFMAC} always allows to redefine already defined macros.
	If {\tt\infmode} evaluates to false then {\tt DEFMAC} just defines the
	macro; else {\tt DEFMAC} also sets a property telling that {\it \sbrna} is a
	first level defined macro and, in case of redefinition, prints a warning
	message. 
\end{quote}

\actionthree {\tt DEFSYS}{\sbrna}{\sbral}{\sbrbo}\index{DEFSYS}
\begin{quote}
	{\tt DEFSYS} is a function definer which allows to redefine already defined
	functions only when {\tt\infmode} evaluates to false; when {\tt\infmode} is
	true and {\it \sbrna} has already been defined, {\tt DEFSYS} will signal
	that it can not execute the redefinition.
	With respect to functions which have not been defined yet {\tt DEFSYS} is
	just like {\tt DEFLAM}.
\end{quote}

\actionthree {DEFMACSYS}{\sbrna}{\sbral}{\sbrbo}\index{DEFMACSYS}
\begin{quote}
	{\tt DEFMACSYS} is a macro definer which allows to redefine already defined
	macros only when {\tt\infmode} evaluates to false; when {\tt\infmode} is
	true and {\it \sbrna} has already been defined, {\tt DEFMACSYS} will signal
	that it can not execute the redefinition.
	With respect to macros which have not been defined yet {\tt DEFMACSYS} is
	just like {\tt DEFMAC}. 
\end{quote}

\actionthree {\tt SET-FUN}{\sbrna}{\sbral}{\sbrbo}\index{SET-FUN}
\begin{quote}
	{\tt SET-FUN} differs from {\tt DEFLAM} just because it evaluates its
	arguments,while {\tt DEFLAM} does not.
\end{quote}

\actionthree {DEFSUB}{\sbrna}{\sbral}{\sbrbo}\index{DEFSUB}
\begin{quote}
	{\tt DEFSUB} can work as {\tt DEFLAM} or as {\tt DEFMAC}.
	This subroutine constructor is included for efficiency reasons rather than
	theoretical reasons; hence its description is delayed to appendix A.
\end{quote}


\subsubsection{First level recognizers}

\predicateone {FUNCTION?}{\obj}\index{FUNCTION?}
\begin{quote}
	{\tt FUNCTION?} is a function recognizer, which evaluates to true if \obj\
	is the name of an {\HG} function, false otherwise. 
	{\tt FUNCTION?} can not tell a built-in function from a \flconst\ defined
	function, or a compiled function from an interpreted one.
	{\tt FUNCTION?} returns the correct information only if {\tt\infmode} is
	true; else it prints a warning message to signal that an un-allowable
	information has been required and executes an {\tt HGKM-ABORT} instruction: hence
	every {\tt FUNCTION?} call must be ``surrounded'' by a {\tt TRAP}
	call\footnote{Further information about {\tt TRAP} and {\tt HGKM-ABORT} can be
	found in ``Error handling'' section}.
\end{quote}

\predicateone {MACRO?}{\obj}\index{MACRO?}
\begin{quote}
	{\tt MACRO?} is a macro recognizer, which evaluates to true if \obj\ is the
	name of an {\HG} macro, false otherwise.
	Note that {\tt MACRO?} can not tell a system macro from a user defined
	macro.
	Furthermore, {\tt MACRO?} has the same behavior as {\tt FUNCTION?} with
	respect to {\tt\infmode}.
	Finally we want to mention a small problem: both {\tt MACRO?} and
	{\tt FUNCTION?} evaluate to true in case of built-in subroutines, because
	there is no information telling if a such subroutine has been defined as
	macro or as function.
\end{quote}

\functionone {GET-FUN}{\obj}{\funcdesc}\index{GET-FUN}
\begin{quote}
	{\tt GET-FUN} returns a function description, in one of the following forms:
	%
	\begin{itemize}
		\item
			{\tt (HGKM <function name> COMPILED)} if the \newline function is
			compiled; 
		\item
			{\tt (HOST <function name> <bound variables list> <body>)} if the
			\newline function is an un-compiled system function;
		\item
			{\tt (HGKM <function name> <bound variables list> <body>)} if the
			\newline function is an uncompiled \flconst\ defined function.
	\end{itemize}
\end{quote}


\subsection{Error handling}

\functionone {TRAP}{\form}{\value}\index{TRAP}
\begin{quote}
	{\tt TRAP} evaluates \form\ and if no error occurs during its evaluation
	{\tt TRAP} returns the same value as the form ({\tt LIST} \form).
	If a system error occurs during an evaluation then {\tt TRAP} returns an
	atomic object that can be tested for determining what the error was.
	If the function {\tt HGKM-ABORT} is called during evaluation of \form\ then {\tt
	TRAP} will return the value of the argument of {\tt HGKM-ABORT}.
\end{quote}

\actionone {HGKM-ABORT}{\form}\index{HGKM-ABORT}
\begin{quote}
	{\tt HGKM-ABORT} returns the value of \form\ to the most recent {\tt TRAP}; every
	{\tt HGKM-ABORT} call must be lexically surrounded at least by a {\tt TRAP}
	function call.
\end{quote}

\predicateone{ERROR-OCCURRED?}{\value}\index{ERROR-OCCURRED?}
\begin{quote}
	{\tt ERROR-OCCURRED?} checks whether a {\tt TRAP}ped \form\ {\tt HGKM-ABORT}ed.
\end{quote}

\functionone{TRAP-GET-RESULT}{\value}{\obj}\index{TRAP-GET-RESULT}
\begin{quote}
	{\tt TRAP-GET-RESULT} selects from \value\ the result of the evaluation of
	\form.
\end{quote}


\subsection{Input-output}

In this section {\HG} input-output handling primitives are presented.
{\HG} uses unidirectional channels implemented on Common LISP streams.
A channel is an object representing in {\HG} environment an external object,
i.e.\ a file or the keyboard.
A channel needs to be open to become active.
The system keeps track of open channels, storing information about each active
channel in a global data structure, the channel list.
Every channel list entry contains
%
\begin{itemize}
	\item
		the channel itself
	\item
		the data flow direction ({\tt IN} or {\tt OUT})
	\item
		the type of objects that can be communicated through the channel, namely
		s-expressions ({\tt SEXP}) or characters ({\tt CHAR}).
	\item
		the information on the object bound to the channel (the device on which
		is stored the physical file in case of file I/O, the keyboard in case of
		console I/O, etc).
\end{itemize}


\subsubsection{Opening and closing channels}

\functionone {HGKM-ISOPEN}{\filena}{\chan}\index{HGKM-ISOPEN}
\begin{quote}
	\filena\ is a form which evaluates to a string.
	{\tt HGKM-ISOPEN} creates and returns a channel for s-expressions
	corresponding to the file \filena, input direction.
	When trying to open a channel corresponding to a non existing file, {\tt
	HGKM-OPEN} executes an {\tt HGKM-ABORT} instruction, to signal that an illegal
	operation has been required: hence {\tt HGKM-OPEN} must be {\tt TRAP}-ped.
\end{quote}

\functionone {HGKM-ICOPEN}{\filena}{\chan}\index{HGKM-ICOPEN}
\begin{quote}
	\filena\ is a form which evaluates to a string.
	{\tt HGKM-ICOPEN} creates and returns a channel for characters corresponding
	to the file \filena, input direction.
	When trying to open a channel corresponding to a non existing file, {\tt
	HGKM-OPEN} executes an {\tt HGKM-ABORT} instruction, to signal that an illegal
	operation has been required: hence {\tt HGKM-OPEN} must be {\tt TRAP}-ped.
\end{quote}

\functionzero {TTY-IOPEN}{\chan}\index{TTY-IOPEN}
\begin{quote}
	{\tt TTY-IOPEN} returns a channel for console input.
\end{quote}

\functionone {HGKM-OSOPEN}{\filena}{\chan}\index{HGKM-OSOPEN}
\begin{quote}
	\filena\ is a form which evaluates to a string.
	{\tt HGKM-OSOPEN} creates and returns a channel for s-expressions
	corresponding to the file \filena, output direction.
\end{quote}

\functionone {HGKM-OCOPEN}{\filena}{\chan}\index{HGKM-OCOPEN}
\begin{quote}
	\filena\ is a form which evaluates to a string.
	{\tt HGKM-OCOPEN} creates and returns a channel for characters corresponding
	to the file \filena, output direction.
\end{quote}

\functionzero {TTY-OOPEN}{\chan}\index{TTY-OOPEN}
\begin{quote}
	{\tt TTY-OOPEN} returns a channel for console output.
\end{quote}

\actionone {HGKM-CLOSE}{\chan}\index{HGKM-CLOSE}
\begin{quote}
	{\tt HGKM-CLOSE} closes \chan\ and updates the channel list taking off the
	information about \chan.
	{\tt HGKM-CLOSE} must be trapped because, when \chan\ does not exist, an
	{\tt HGKM-ABORT} occurs to signal that an illegal operation has been required.
\end{quote}


\subsubsection{Input output specifying channels}

\functionone {CHAN-READ}{\chan}{\obj}\index{CHAN-READ}
\begin{quote}
	{\tt CHAN-READ} reads an s-expression from \chan\ and returns it; to handle
	the case in which the object read is an {\tt EOF} a {\tt TRAP} function call
	is required. 
	\chan\ must have been created by {\tt TTY-IOPEN} or by {\tt HGKM-ISOPEN}.
\end{quote}

\functionone {CHAN-CHI}{\chan}{\chr}\index{CHAN-CHI}
\begin{quote}
	{\tt CHAN-CHI} reads a character from \chan\ and returns it; to handle the
	case in which the character read is an {\tt EOF} a {\tt TRAP} function call
	is required. 
	\chan\ must have been created by {\tt TTY-IOPEN} or by {\tt HGKM-ICOPEN}.
\end{quote}

\actiontwo {CHAN-PRINC}{\obj}{\chan}\index{CHAN-PRINC}
\begin{quote}
	{\tt CHAN-PRINC} writes \obj\ on \chan.
	\chan\ must have been created by {\tt TTY-OOPEN} or by {\tt HGKM-OSOPEN}.
\end{quote}

\actionone {CHAN-CRLF}{\chan}\index{CHAN-CRLF}
\begin{quote}
	{\tt CHAN-CRLF} writes a carriage return on \chan.
	\chan\ must have been created by {\tt TTY-OOPEN} or by {\tt HGKM-OSOPEN}.
\end{quote}

\actiontwo {CHAN-CHO}{\chr}{\chan}\index{CHAN-CHO}
\begin{quote}
	{\tt CHAN-CHO} writes \chr\ on \chan.
	\chan\ must have been created by {\tt TTY-OOPEN} or by {\tt HGKM-OCOPEN}.
\end{quote}

\functionone{CHANNEL-GET-FNAME}{\chan}{\filena}\index{CHANNEL-GET-FNAME}
\begin{quote}
	{\tt CHANNEL-GET-FNAME} returns the name of the file corresponding to the
	channel \chan.
\end{quote}


\subsubsection{Input output with implicit channels}

\actionone {SELECT}{\chan}\index{SELECT}
\begin{quote}
	{\tt SELECT} is the primitive to set the values of default channels.
	If into the channel list the direction associated with \chan\ is {\tt IN},
	then \chan\ will be selected as default input channel, else as default
	output channel.
\end{quote}

\functionzero {HGKM-READ}{\obj}\index{HGKM-READ}
\begin{quote}
	{\tt HGKM-CHI} reads an s-expression from the selected input and returns it.
	to handle the case in which the object read is an {\tt EOF} a {\tt TRAP}
	function call is required.
	The selected input must have been created by {\tt TTY-IOPEN} or by {\tt
	HGKM-ISOPEN}. 
\end{quote}

\functionzero {HGKM-CHI}{\chr}\index{HGKM-CHI}
\begin{quote}
	{\tt HGKM-CHI} reads a character from the selected input and returns it; to
	handle the case in which the character read is an {\tt EOF} a {\tt TRAP}
	function call is required.
	The selected must have been created by {\tt TTY-IOPEN} or by
	{\tt HGKM-ICOPEN}. 
\end{quote}

\actionone {HGKM-PRINC}{\obj}\index{HGKM-PRINC}
\begin{quote}
	{\tt HGKM-PRINC} writes \obj\ on the selected output.
	The selected output must have been created by {\tt TTY-OOPEN} or by
	{\tt HGKM-OSOPEN}.
\end{quote}

\actionone {CRLF}{\chan}\index{CRLF}
\begin{quote}
	{\tt CRLF} writes a carriage return on the selected output.
	The selected output must have been created by {\tt TTY-OOPEN} or by
	{\tt HGKM-OSOPEN}.
\end{quote}

\actionone {HGKM-CHO}{\chan}\index{HGKM-CHO}
\begin{quote}
	{\tt HGKM-CHO} writes \chr\ on the selected output.
	The selected output must have been created by {\tt TTY-OOPEN} or by
	{\tt HGKM-OCOPEN}.
\end{quote}

\actionzero {HGKM-INITIO}\index{HGKM-INITIO}
\begin{quote}
	{\tt HGKM-INITIO} initializes I/O by setting the following I/O-related
	variables:  {\tt CHAN-LIST}, {\tt STANDARD-INPUT}, {\tt STANDARD-OUTPUT},
	{\tt SELECTED-INPUT}, {\tt SELECTED-OUTPUT}.
\end{quote}


\pagebreak
\section{Extended {\HG}}

In this section extended {\HG} is presented.
Extended {\HG} is a collection of primitives which operate on basic {\HG} data
structures or on newly defined ones.

Extended {\HG} is not founded on theoretical considerations like basic {\HG}.
Its semantics can't be axiomatized in first order logic.
Extended {\HG} has been developed mainly for sake of performance and programming
facility, and it has grown up ``on demand'': that is, when it has been realized
that a set of primitives was useful it has been implemented. This is why
extended {\HG} does not look homogeneous.

During this revision, an attempt has been made to keep extended {\HG} as small
as possible: hence, several primitives which are actually part of the source
code are not presented in this manual because they seem to be rather useless.
Further revisions of the source codes will face the problem of removing them.


\subsection{Truth values}

\begin{itemize}
	\item {\tt (NOT \ldots)}\index{NOT}: as in CL.
	\item {\tt (AND \ldots)}\index{AND}: as in CL.
	\item {\tt (OR \ldots)}\index{OR}: as in CL.
	\item {\tt (IST {\it [sym]})}\index{IST}:
	  {\tt IST} and its counterpart {\tt ISF} should theoretically allow GETFOL
	  to use real truth values instead that {\tt T} and {\tt NIL}. {\tt IST}
	  returns {\tt TVAL} when {\it sym} is not {\tt MTL}, {\tt FVAL} otherwise.
	\item {\tt (ISF {\it [sym]})}\index{ISF}:
	  {\tt ISF} returns {\tt FVAL} when {\it sym} is {\tt MTL}, {\tt TVAL}
	  otherwise.
\end{itemize}


\subsection{Symbols}

\begin{itemize}
    \item {\tt (CAPITALIZE {\it [sym {\tt | } str]})}:\index{CAPITALIZE}
        returns the argument with all its letters capitalized.
    \item {\tt (CAPITALIZE-SYM {\it [sym]})}:\index{CAPITALIZE-SYM}
        returns {\it sym} with all its letters capitalized.
    \item {\tt (CAPITALIZE-STR {\it [str]})}:\index{CAPITALIZE-STR}
        returns {\it str} with all its letters capitalized.
    \item {\tt (CONCAT-SYMS {\it [sym$_1$][sym$_2$]})}:\index{CONCAT-SYMS}
        returns the concatenation of {\it sym$_1$} and {\it sym$_2$}.
    \item {\tt  (PAIR2CHR {\it [pair]})}:\index{PAIR2CHR}
        makes {\it pair} into the corresponding symbol.
    \item {\tt  (INT2SYM {\it [int]})}:\index{INT2SYM}
        makes {\it int} into the corresponding symbol.
    \item {\tt  (NAT2SYM {\it [int]})}:\index{NAT2SYM}
        makes a natural {\it int} into the corresponding symbol.
    \item {\tt  (NAT2SYM {\it [nat]})}:\index{NAT2SYM}
        makes the natural {\it nat} into the corresponding symbol.
    \item {\tt (NUMERAL {\it [num]})}\index{NUMERAL}
        looks if {\it num} is a numeral.
\end{itemize}


\subsection{Integers}

These primitives are trivial, and do not require any explanation.
Note that {\tt INTEQ} (the integer equality predicate) must be used instead of
{\tt EQ} (the object equality) when compiling because some CL compiler
substitutes the variable n (where n goes from $-1024$ to $1023$), which
evaluates to the integer $n$, with its actual value. This can create unexpected
behavior which is very difficult to detect and fix.

\begin{itemize}
	\item {\tt (MINUS ...)}\index{MINUS}
	\item {\tt (PLUS ...)}\index{PLUS}
	\item {\tt (DIFFERENCE ...)}\index{DIFFERENCE}
	\item {\tt (TIMES ...)}\index{TIMES}
	\item {\tt (QUOTIENT ...)}\index{QUOTIENT}
	\item {\tt (REMAINDER ...)}\index{REMAINDER}
	\item {\tt (NATNUM ...)}\index{NATNUM}
	\item {\tt (INTEQ ...)}\index{INTEQ}
	\item {\tt (LT ...)}\index{LT}
	\item {\tt (LE ...)}\index{LE}
	\item {\tt (GT ...)}\index{GT}
	\item {\tt (GE ...)}\index{GE}
\end{itemize}


\subsection{Characters}

\begin{itemize}
	\item {\tt (ASCII {\it [chr]})}:\index{ASCII}
		  creates and returns the interned symbol corresponding to {\it chr}.
	\item {\tt (SYM-INTERN {\it [obj]})}:\index{SYM-INTERN}
		   creates and returns the interned symbol corresponding to {\it obj}.
	\item {\tt (MAKE-UNINTERNED {\it [obj]})}:\index{MAKE-UNINTERNED}
		   creates and returns an un-interned symbol corresponding to {\it obj}.
	\item {\tt (SYM-GETCHRFIRST {\it [sym]})}:\index{SYM-GETCHRFIRST}
		   returns the character corresponding to the first character of
		   {\it sym}.
\end{itemize}


\subsection{Strings}

\begin{itemize}
	\item {\tt (SUBSTR? {\it [str$_1$][str$_2$]})}:\index{SUBSTR?}
		   evaluates to true if {\it str$_2$} begins with {\it str$_1$}.
	\item {\tt (CONCAT-STRS {\it [str$_1$][str$_2$]})}:\index{CONCAT-STRS}
		  returns the concatenation of {\it str$_1$} and {\it str$_2$}.
	\item {\tt (SUBSTR {\it [str][int]})}:\index{SUBSTR}
		   returns the string of the first {\it int} characters of {\it str}. If
		   {\it int} is too big an error occurs.
	\item {\tt (INSTR? {\it [str$_1$][str$_2$]})}:\index{INSTR?}
		   evaluates to true if {\it str$_1$} appears in {\it str$_2$}.
\end{itemize}


\subsection{Lists}

\begin{itemize}
   	\item {\tt (LIST ...)}:\index{LIST}
        as in CL.
   	\item {\tt (EQUAL ...)}:\index{EQUAL}
        as in CL.
   	\item {\tt (COPY {\it [obj]})}:\index{COPY}
        returns an object {\tt EQUAL} and not  {\tt EQ} to {\it obj}.
    \item {\tt (LENGTH {\it [list]})}:\index{LENGTH}
        returns the length of {\it list}.
    \item {\tt (HGKM-MEMBER {\it [obj][list]})}:\index{HGKM-MEMBER}
        as in CL.
		The equality test is {\tt EQUAL}.
    \item {\tt (HGKM-MEMQ {\it [obj][list]})}:\index{HGKM-MEMQ}
	    like {\tt HGKM-MEMBER}, but with {\tt EQ} test.
		For instance:
		%
		\begin{verbatim}
			> (HGKM-MEMQ '(C D) ((A B)(C D) E))
			FALSE
		\end{verbatim}
    \item {\tt (APPEND ...)}:\index{APPEND}
        as in CL.
    \item {\tt (UNION ...)}:\index{UNION}
        as in CL.
    \item {\tt (REVERSE {\it [list]})}:\index{REVERSE}
        returns the reverse of the top level of {\it list}.
    \item {\tt (SUBST {\it [obj$_1$][obj$_2$][obj$_3$]})}:\index{SUBST}
        substitutes {\it obj$_1$} to every {\tt EQUAL} occurrence of
		{\it obj$_2$} in {\it obj$_3$}.
    \item {\tt (CL-ASSOC ...)}:\index{CL-ASSOC}
        as CL's {\tt ASSOC}: it tests with {\tt EQ}.
    \item {\tt (HGKM-ASSOC ...)}:\index{HGKM-ASSOC}
         the same as {\tt CL-ASSOC}, but it tests with {\tt EQUAL}.
    \item {\tt (SECOND {\it [list]})}:\index{SECOND}
         returns the second element of {\it list}.
    \item {\tt (LAST {\it [list]})}:\index{LAST}
         as in common lisp: it returns the last {\em cons} of {\it list}.
    \item {\tt (HGKM-NTH {\it [int][list]})}:\index{HGKM-NTH}
        returns the {\it int}-th element of {\it list}.
        Note that {\it int} must be positive, otherwise an error occurs.
    \item {\tt (NCONC ...)}:\index{NCONC}
         as in CL.
    \item {\tt  (POS {\it [sym][list]})}:\index{POS}
        returns the first position of {\it sym} in {\it list}.
        If it does not exist {\tt POS} returns {\tt 0}.
    \item {\tt  (TAIL {\it [list][obj]})}:\index{TAIL}
        if {\it obj} occurs in {\it list} the list of elements following
        {\it obj} (not including {\it obj}) is returned; otherwise {\tt MTL}
        is returned. 
    \item {\tt  (FLATTEN {\it [list]})}:\index{FLATTEN}
        {\it list} is flattened and returned.
    \item {\tt  (PRELIST {\it [list][int]})}:\index{PRELIST}
        creates a list of the first {\it int} elements of {\it list}.
    \item {\tt  (SUFLIST {\it [list][int]})}:\index{SUFLIST}
        creates a list by deleting the first {\it int} elements of {\it list}.
    \item {\tt  (INLIST? {\it [list$_1$][list$_2$]})}:\index{INLIST?}
        looks if {\it list$_1$} is in {\it list$_2$} (only at top level).
    \item {\tt  (SUBTRACT {\it [list$_1$][list$_2$]})}:\index{SUBTRACT}
        returns the list of the elements of {\it list$_1$} which are not in
        {\it list$_2$}. 
    \item {\tt  (INTERSECT {\it [list$_1$][list$_2$]})}:\index{INTERSECT}
         returns the intersection
    \item {\tt (SUBSET {\it [list$_1$][list$_2$]})}:\index{SUBSET}
        looks if {\it list$_1$} is a subset of {\it list$_2$}.
        of {\it list$_1$} and {\it list$_2$}.
    \item {\tt  (INSERT! {\it [obj][list]})}:\index{INSERT!}
        inserts {\it obj} in {\it list} making side effect.
        {\tt INSERT!} assumes that {\it obj} can only occur once in {\it
        list}. 
    \item {\tt  (HGKM-DELETE {\it [obj][list]})}:\index{HGKM-DELETE}
        deletes {\it obj} from {\it list} without doing side effect.
        {\tt HGKM-DELETE} assumes that {\it obj} only occurs once in {\it
        list}. 
    \item {\tt  (DELETE! {\it [obj][list]})}:\index{DELETE!}
         the same as {\tt HGKM-DELETE} but with side effect.
    \item {\tt  (OUTOF {\it [obj][list]})}:\index{OUTOF}
        removes all the occurrences of {\it obj} from {\it list} without
        doing side effect.
    \item {\tt  (OUTOF! {\it [obj][list]})}:\index{OUTOF!}
         the same as {\tt OUTOF} but with side effect.
    \item {\tt  (UNLIST {\it [list]})}:\index{UNLIST}
        substitutes the last cons of {\it list} with its {\tt CAR}.
        For instance:

        {\tt (UNLIST (QUOTE (A))) $\Longrightarrow$ A}

        {\tt (UNLIST (QUOTE (A B))) $\Longrightarrow$ (A . B)}

        {\tt (UNLIST (QUOTE (A B C))) $\Longrightarrow$ (A B . C)}
    \item {\tt  (REVMAP {\it [list]})}:\index{REVMAP}
        puts the last element of {\it list} at the top of
        {\it list} itself.
    \item {\tt  (REPETITION-GET {\it [list]})}:\index{REPETITION-GET}
         returns the first element occurring twice in {\it list}.
    \item {\tt  (LIST-N {\it [obj][int]})}:\index{LIST-N}
         returns the list of {\it int} occurrences of {\it obj}.
    \item {\tt  (PRINTLIST {\it [list]})}:\index{PRINTLIST}
         prints all the elements of {\it list}.
\end{itemize}


\subsection{Arrays}

\begin{itemize}
    \item {\tt (TUPLE {\em [sym]} {\em [len]})}:\index{TUPLE}
        sets the value of the variable {\em sym} to be an array of length
        {\em len}. 
    \item {\tt (SETTUP {\em [array]} {\em [index]} {\em [data]})}:
	\index{SETTUP}
        sets the {\em index}-th entry in {\em array} to {\em data} destroying
        the previous information.
    \item {\tt (GETTUP {[\em array]} {\em [index]})}:\index{GETTUP}
        returns the data stored in {\em array} at the position indexed by
		{\em index}.
\end{itemize}


\subsection{Ordered lists of integers}

\begin{itemize}
	\item {\tt (UNIONLE {\it [list$_1$][list$_2$]})}:\index{UNIONLE}
		given two lists of natural numbers, each in ascending order, creates the
		ascending ordered union. 
	\item {\tt (UNIONGE {\it [list$_1$][list$_2$]})}:\index{UNIONGE}
		given two lists of natural numbers, each in descending order, {\tt
		UNIONGE} creates the descending ordered union.
	\item {\tt (ORDER {\it [list]})}:\index{ORDER}
		returns the list of natural numbers occurring in {\it list} in
		increasing order.
	\item {\tt (INSERT {\it [int][list]})}:\index{INSERT}
		inserts {\it int} in {\it list}, an ascending-ordered list of natural
		numbers.
	\item {\tt (hole {\it [list]})}:\index{hole}
		being {\it list} argument a list of integers greater than or equal to
		$-1$, {\tt hole} returns the smallest number not appearing in
		{\it list}.
		
		Example:
		%
		\begin{verbatim}
			> (hole (QUOTE (-1 0 1 3)))
			2
			> (hole (QUOTE (0 2 3)))
			-1
		\end{verbatim}
\end{itemize}


\subsection{Associative lists}

Elements of associative lists are implemented as dotted pairs.
%
\begin{itemize}
    \item {\tt (alistelement-get-1st {\it [alist-elem]})}:
	\index{alistelement-get-1st}
        returns the {\tt CAR} of the dotted pair.
    \item {\tt (alistelement-get-2nd {\it [alist-elem]})}:
    \index{alistelement-get-2nd}
         returns the {\tt CDR} of the dotted pair.
    \item {\tt (alistelement-mak {\it [obj1][obj2]})}:
    \index{alistelement-mak}
        returns the pair {\tt ({\it obj1} . {\it obj2})}
\end{itemize}
%
Associative lists are associative addressing data structures; they are
implemented as lists of dotted pairs.
%
\begin{itemize}
    \item {\tt (alist-get-first {\it [alist]})}:\index{alist-get-first}
         returns the first element of {\it alist}
    \item {\tt (alist-get-rest {\it [alist]})}:\index{alist-get-rest}
         returns the rest of {\it alist} 
    \item {\tt (alist-cons {\it [alist][alist-elem]})}:\index{alist-cons}
         adds {\it alist-elem} to {\it alist}
    \item {\tt (HGKM-REPLACE {\it [obj$_1$][obj$_2$][alist]})}:
    \index{HGKM-REPLACE}
        if {\it obj$_1$} is the first element of an alist element, then
		{\tt HGKM-REPLACE} replaces the second element with  {\it obj$_2$};
        otherwise, {\tt HGKM-REPLACE} adds the {\tt CONS} of {\it obj$_1$} 
        and obj$_2$ to {\it alist}.
    \item {\tt Alist-empty}:\index{Alist-empty}
        is the constant representing an empty associative list
    \item {\tt (ALIST-EMPTY? {\it [alist]})}:\index{ALIST-EMPTY?}
        looks if {\it alist} is empty
    \item {\tt (alist-add {\it [alist][obj$_1$][obj$_2$]})}\index{alist-add}
        adds the pair {\tt ({\it obj$_1$} . {\it obj$_2$})} to {\it alist}
        if not already present
    \item {\tt (alist-del {\it [alist][obj]})}\index{alist-del}
        removes the pair pointed by {\it obj} from {\it alist}
    \item {\tt (alist-union {\it [alist$_1$][alist$_2$]})}\index{alist-union}
        adds to {\it alist$_2$} all the elements of {\it alist$_1$},
        overwriting those elements of {\it alist$_2$} which are also in
		{\it alist$_1$} 
    \item {\tt (alist-cons-shared {\it [alist][alist-elem]})}:
    \index{alist-cons-shared}
        same as {\tt alist-cons} but with side effect on {\it alist}.
\end{itemize}


\subsection{Extending the evaluation syntax}

\begin{itemize}
	\item {\tt (MAPCAR ...)}:\index{MAPCAR}
		  as in CL.
\end{itemize}


\subsection{Extending the input output}

\begin{itemize}
	\item {\tt (HGKM-LOAD {\it [str]})}:\index{HGKM-LOAD}
		  loads the file named {\it str}.
	\item {\tt (HGKM-LOAD-FROM-CHANNEL {\it [chan]})}:
	\index{HGKM-LOAD-FROM-CHANNEL}
		 loads s-expressions from the default channel until an {\tt EOF} is
		 found.
\end{itemize}


\appendix
\section{Some notes about implementation}

While dealing with big software systems, in phase of development it may be very
helpful to know if a subroutine is built-in or user-defined, and if it has been
defined as function or macro.
If {\tt \infmode}\ is true when {\HG} is created, all this information is made
available: for each subroutine declaration, zero-level constructors set a
property recording that the subroutine is built-in; first-level constructors
set a property recording that the subroutine is user-defined and whether it has
been defined as a function or as a macro; reading these properties, first level
recognizers can tell if a subroutine is built-in or system-defined, and if it is
a function or a macro.
If {\tt \infmode} is false no property is set by the zero and first level
constructors, and first level recognizers, when called, will signal that no
answer is possible because no information is available.
For a detailed description of each first-level constructor and recognizer, with
respect to {\tt \infmode}, the reader is referred to basic {\HG} chapter.

Of course, when {\tt \infmode} is false, a large amount of storage is saved, and
a more compact {\HG} executable file is obtained.
Finally we remark that, for sake of consistency, the value of {\tt \infmode}
should not change while {\HG} is being created.

A complex software system may have big performance problems, because of the
overhead due to the great number of function calls, even if it runs compiled.
Therefore, the only way to speed up the system is to define as many subroutines
as possible as macros. These macros are expanded at the moment of compilation,
obtaining in this way a code which is modular and well structured, and which
does not have the overhead of function calls.
There are two drawbacks to this approach: the former, and least important, is
that the program becomes big because of the macro expansion; the latter, and
critical, is that debugging becomes very hard, because all information related
to functional definition is lost.

In {\HG} this problem may be solved using the first-level constructor {\tt
DEFSUB}, that allows to define subroutines either like macros or like functions,
without changing source code.
As explained below, {\tt DEFSUB} definitions depend on the value of a system
constant, {\tt \debmode}.

A {\tt DEFSUB} call has a syntax as defined in basic {\HG} chapter: it requires
three arguments, that is a {\it subroutine name}, an {\it arguments list} and a
{\it body}.
The third argument, {\it body}, is a function body, as needed in {\tt DEFUN}
subroutines definitions.
If {\tt \debmode} is true, {\tt DEFSUB} works like {\tt DEFLAM}.
If {\tt \debmode} is false, {\tt DEFSUB} executes a {\tt DEFMAC} call, modifying
the third argument: that is, starting from {\it body}, {\tt DEFSUB} builds a
macro body that, when macro-expanded, returns {\it body}.

Hence {\tt DEFSUB} is a subroutine constructor that has a simple function
definition syntax and behaves either like a function or macro constructor,
depending on the value of {\tt \debmode}.
If {\tt \debmode}\ is true debugging is possible, because {\tt DEFSUB}
subroutines have been declared as functions; if {\tt \debmode}\ is false
debugging is very hard, but the execution is speed up because no function call
overhead is present.

Note that {\tt DEFSUB} is subject to some restriction, as remarked in basic
{\HG} chapter: this is due to the complexity of macro body creation.

Finally note that, to reach some performance improvements, a {\tt DEFSUB}
definition of a general subroutine {\tt SBR} must be loaded before compiling
subroutines calling {\tt SBR} itself; else, macro expansion can not be made at
compile time, and the {\tt SBR} is executed interpretively and expanded at run
time.


% ................................. bibliography ..............................
\bibliographystyle{alpha}
\bibliography{}

% .................................... index ..................................
%............................... HGKM MANUAL .................................
%.............................................................................


\documentstyle[11pt]{../styfiles/GFmanual}

%................................. macros ....................................
%.............................................................................

\setcounter{totalnumber}{1}

\def\actionzero#1{\noindent ({\tt {#1}}) \hfill action}
\def\actionone#1#2{\noindent ({\tt {#1}} {\it [{#2}]}) \hfill action}
\def\actiontwo#1#2#3{\noindent ({\tt {#1}} {\it [{#2}]}{\it [{#3}]}) \hfill
action} 
\def\actionthree#1#2#3#4{\noindent ({\tt {#1}} {\it [{#2}]}{\it [{#3}]}{\it
[{#4}]}) \hfill action}

\def\functionzero#1#2{\noindent ({\tt {#1}}) $\Longrightarrow$ {#2} \hfill
function} 
\def\functionone#1#2#3{\noindent ({\tt {#1}} {\it [{#2}]}) $\Longrightarrow$
{#3} \hfill function}
\def\functiontwo#1#2#3#4{\noindent ({\tt {#1}} {\it [{#2}]}{\it [{#3}]})
$\Longrightarrow$ {#4} \hfill function} 
\def\functionthree#1#2#3#4#5{\noindent ({\tt {#1}} {\it [{#2}]}{\it [{#3}]}{\it
[{#4}]}) $\Longrightarrow$ {#5} \hfill function} 

\def\predicatezero#1{\noindent ({\tt {#1}}) \hfill predicate}
\def\predicateone#1#2{\noindent ({\tt {#1}} {\it [{#2}]}) \hfill predicate}
\def\predicatetwo#1#2#3{\noindent ({\tt {#1}} {\it [{#2}]}{\it [{#3}]}) \hfill
predicate}
\def\predicatethree#1#2#3#4{\noindent ({\tt {#1}} {\it [{#2}]}{\it [{#3}]}{\it
[{#4}]}) \hfill predicate}

\def\sysconst#1{\noindent{\tt {#1}} \hfill system constant}

\def\chan {{\it channel\/}}
\def\chr {{\it chr\/}}
\def\chrlist {{\it chr-list\/}}
\def\debmode {\%DE\-BUG\--MO\-DE\%}
\def\dev {{\it device\/}}
\def\dir {{\it direction\/}}
\def\filena {{\it filena\/}}
\def\flconst {first level constructor}
\def\flrec {first level recognizer}
\def\form {{\it form\/}}
\def\formlist {{\it form list\/}}
\def\funcdesc {{\it function description\/}}
\def\infmode {\%IN\-FOR\-MA\-TION-MO\-DE\%}
\def\inte {{\it int\/}}
\def\LIST {{\it list\/}}
\def\LISTone {{\it list\/$_1$\/}}
\def\LISTtwo {{\it list\/$_2$\/}}
\def\obj {{\it obj\/}}
\def\objn {{\it obj\/$_n$\/}}
\def\objone {{\it obj\/$_1$\/}}
\def\objtwo {{\it obj\/$_2$\/}}
\def\pair {{\it pair\/}}
\def\sbral {{\it argument list\/}}
\def\sbrbo {{\it body\/}}
\def\sbrna {{\it subroutine name\/}}
\def\str {{\it str\/}}
\def\sym {{\it sym\/}}
\def\symone {{\it sym\/$_1$\/}}
\def\symtwo {{\it sym\/$_2$\/}}
\def\value {{\it value\/}}

% ................................. document ..................................
% .............................................................................

% \makeindex
\begin{document}

\title{HGKM User Manual}
\author{{\bf Fausto Giunchiglia} and {\bf Alessandro Cimatti}}
\date{4 March 1994}
\version{2.1}
\abstract{
	This manual describes the implementation of a simple programming language,
	{\HG}. 
	{\HG} was originally defined and developed by Richard Weyhrauch and Caroline
	Talcott at the Department of Computer Science of Stanford University.
	The version described here sticks as much as possible the conventions and the
	specifications of the original version.
	The implementation is almost totally new and some new functionalities (e.g.
	{\tt DEFSUB}) have been added.
	Theoretical {\HG} was defined so that its semantics could be easily
	represented in first order logic.
	The version of {\HG} described in this manual acts as the implementation
	language for {\GF}.
	This manual has been written only to facilitate the
	implementation/modification of {\GF}.
}
\addresses{
   \begin{tabular}[c]{l@{\hskip 2em}l}
       {\bf Fausto Giunchiglia}:	    & {\bf Alessandro Cimatti}           \\
       Mechanized Reasoning Group       & Mechanized Reasoning Group,        \\
       IRST, Povo, 38050 Trento, Italy  & IRST, Povo, 38050 Trento, Italy    \\
       e-mail: {\tt fausto@irst.it}     & e-mail: {\tt cx@irst.it}           \\
       phone: +39 461 314359            & phone: +39 461 314351
   \end{tabular}
}
\published{
	\begin{tabular}{l}
		DIST Technical Report No. 9107-05 (1991). \\
		DIST -- University of Genoa,\\
		Via Opera Pia 11A, 16145 Genova, Italy.
    \end{tabular}
}

% ................................... cover ...................................

\thispagestyle{empty}
\maketitle

% ............................. table of contents ...............................

\newpage
\pagenumbering{roman}
\tableofcontents
\newpage
\pagenumbering{arabic}


\section{Introduction}

This manual describes an enhanced version of {\HG}, which has been revised by
the authors within the {\GF} project\cite{giunchiglia34,giunchiglia12,giunchiglia29}.
In effect, the most recent version of {\GF} has been implemented on this
version of {\HG}.

{\HG} is a programming language which has been implemented on top of Common
LISP; it is powerful and simple to use. {\HG} has been developed mainly for
three reasons: to increase {\GF}'s portability, to improve {\GF}'s performance
and to introduce a simple mechanism of data typing.

A language we will call {\it basic} {\HG} was developed including only a small
set of data types and a few built in primitives. This language is enough to
provide the full expressive power of {\HG}.

However, in order to facilitate the development of {\GF}, more data types and
primitives have been defined.
These additions form {\it extended} {\HG}.

This manual is not a tutorial.
It is a reference manual for people already familiar with LISP and
s-expressions notation.


\subsection{Basic concepts}

A {\HG} object belongs to a data type. For each data type three classes of
primitives are defined:

\begin{description}
    \item[predicates:]
        test if a condition is verified or not; predicates are forms which can
        be used as test for branching in conditional forms: that is, evaluating
		a predicate returns a truth value.
    \item[functions:]
        specify a correspondence between {\HG} data and returns an {\HG} object;
		the data type of returned object is specified too; a function produces
		no side effects.
    \item[actions:]
        specify a side effect, and do not return an {\HG} object.
\end{description}
%
\noindent
Furthermore for each data type we define some
%
\begin{description}
    \item[system constants:]
        symbols to which corresponds a fixed value
\end{description}


\subsection{Remarks about notation}

The syntax of primitives and system constants will be described using the
following notation:

\begin{description}
	\item \sysconst{<constant name>}
	\item \predicatethree{<predicate name>}{arg type}{arg type}{\ldots}
	\item \functionthree{<function name>}{arg type}{arg type}{\ldots}{res type}
	\item \actionthree{<action name>}{arg type}{arg type}{\ldots}
\end{description}

\begin{quote}\bf
	After the syntactic description we have a section in which the semantics are
	described.
	This section appears narrower than the rest of the text, in this way.
\end{quote}

The name of the primitive is printed in {\tt typewriter} style.
Its arguments are language forms which will be evaluated.
With {\it [arg type]} we mean the data type of the {\HG} object that the
evaluation of the argument form should return.
If the result of the evaluation does not match the specified data type a fatal
error is generated.

For instance, \inte\ denotes a form which evaluates to an integer.
If an operation requires more than one argument of a single type, the bracketed
type name will have a numbered subscript.
For example we write {\it str\/$_1$} and {\it str\/$_2$} to denote two different
forms that have strings as values.
In the case an operation does not care about the type of its arguments we write
\obj\ to indicate a form which can evaluate to any object in the {\HG} world.

In section 2 we present basic {\HG}; in section 3 extended {\HG}; finally, in
appendix we describe some aspects related to the implementation.


\newpage
\section{Basic {\HG}}

In this section we shall describe basic {\HG}: we present the structured data
types, the way in which the language forms are evaluated by the interpreter,
some special forms for subroutine definitions, the error handling and the
input/output primitives.


\subsection{Data structures}

Basic {\HG} data types are:

\begin{itemize}
	\item truth values {\it [trth];}
	\item symbols {\it [\sym];}
	\item integers {\it [\inte];}
	\item characters {\it [\chr];}
	\item strings {\it [\str];}
	\item pairs {\it [\pair];}
	\item lists {\it [list].}
\end{itemize}

In this chapter we shall refer to each data type by the corresponding name
enclosed in square brackets.

The syntax of data types is defined by this grammar:

\begin{bnf}
	\T{obj}			\sep  \T{atomic obj} | \T{compound obj} \\
	\T{atomic obj}	\sep  \T{trth} | \T{sym} | \T{int} | \T{chr} | \T{str} \\
	\T{compound obj}\sep  \T{pair} | \T{list} | \T{chr-list}\\
	\T{pair}	    \sep  ( \T{obj} . \T{obj}) \\
	\T{list}		\sep  empty-list | ( \T{obj} . \T{list} ) \\
	\T{chr-list}	\sep  empty-list | ( \T{chr} . \T{chr-list} )
\end{bnf}

In this grammar we refer explicitly to a data structure, the list of characters
{\it [chr-list],} which is not a primitive data type.
However it is very used in {\HG}, and it is interesting to define its syntax too.
Note that a list of characters is different from a string: in basic {\HG} a
string is an atomic object, while a list of characters has a compound structure.


\subsubsection{Truth values and equality predicate}

\sysconst {TRUE}\index{TRUE}
\begin{quote}
	{\tt TRUE} is a system constant which always evaluates to true.
\end{quote}

\sysconst {FALSE}\index{FALSE}
\begin{quote}
	{\tt FALSE} is a system constant which always evaluates to false.
\end{quote}

\predicatetwo{EQ}{\objone}{\objtwo}\index{EQ}
\begin{quote}
	{\tt EQ} is the equality predicate, and is defined for all data types.
	{\tt EQ} evaluates to true if both \objone\ and \objtwo\ are the same
	object, false otherwise.
\end{quote}


\subsubsection{Symbols}

{\HG} symbols are like Common LISP identifiers, with the only difference that
they are case sensitive.
{\tt abc}, {\tt CAR}, {\tt NIL}, {\tt THISISALONGATOM} are examples of {\HG}
symbols.

\bigskip

\predicateone{SYMBOL?}{\obj}\index{SYMBOL?}
\begin{quote}
	{\tt SYMBOL?} evaluates to true if \obj\ is a symbol, false otherwise.
\end{quote}

\actiontwo{SETVAL}{\sym}{\obj}\index{SETVAL}
\begin{quote}
	{\tt SETVAL} changes the value associated with the specified symbol to \obj.
\end{quote}

\functionone{GETVAL}{\sym}{\obj}\index{GETVAL}
\begin{quote}
	{\tt GETVAL} returns the value associated with \sym.
\end{quote}

\actionthree{SETPROP}{\symone}{\obj}{\symtwo}\index{SETPROP}
\begin{quote}
	{\tt SETPROP} sets the property of \symone\ with key \symtwo\ to have the
	value \obj.
\end{quote}

\functiontwo{GETPROP}{\symone}{\symtwo}{\obj}\index{GETPROP}
\begin{quote}
	{\tt GETPROP} returns the value of the property \symtwo\ of \symone.
	If there is no such property {\tt GETPROP} returns the false truth value.
\end{quote}
 
\functiontwo{REMPROP}{\symone}{\symtwo}{\obj}\index{REMPROP}
\begin{quote}
	{\tt REMPROP} removes from \symone\ the property with an indicator {\tt EQ}
	to \symtwo. 
	It returns the false truth value if there is no such indicator.
\end{quote}


\subsubsection{Integers}

Integers are the only type of numbers allowed in {\HG}.
\bigskip

\predicateone {INTEGER?}{\obj}\index{INTEGER?}
\begin{quote}
	{\tt INTEGER?} evaluates to true if \obj\ is an integer, false otherwise.
\end{quote}

\predicateone {ZERO?}{\obj}\index{ZERO?}
\begin{quote}
	{\tt ZERO?} evaluates  to true if \obj\ is the integer $0$, false otherwise.
\end{quote}

\predicateone {POSITIVE?}{\obj}\index{POSITIVE?}
\begin{quote}
	{\tt POSITIVE?} evaluates to true if \obj\ is a positive integer, false
	otherwise.
\end{quote}

\predicateone {NEGATIVE?}{\obj}\index{NEGATIVE?}
\begin{quote}
	{\tt NEGATIVE?} evaluates to true if \obj\ is a negative integer, false
	otherwise.
\end{quote}

\functionone {ADD1}{\inte}{\inte}\index{ADD1}
\begin{quote}
	{\tt ADD1} adds one to an integer.
\end{quote}

\functionone {SUB1}{\inte}{\inte}\index{SUB1}
\begin{quote}
	{\tt SUB1} subtracts one from an integer.
\end{quote}


\subsubsection{Characters}

{\HG} characters are all the characters to which is associated an ASCII code.
An {\HG} character is represented by the integer between $0$ and $255$
corresponding to its {\tt ASCII} code.

\bigskip

\predicateone {CHARACTER?}{\obj}\index{CHARACTER}
\begin{quote}
	{\tt CHARACTER?} evaluates  to true if \obj\ is a character, false otherwise.
\end{quote}

\functionone {CHR2SYM}{\chrlist}{\sym}\index{CHR2SYM}
\begin{quote}
	{\tt CHR2SYM} returns the symbol whose print name consists of the characters
	in \chrlist.

	For instance:
	%
	\begin{verbatim}
		> (CHR2SYM '(65 80 80 76 69))
		APPLE
	\end{verbatim}
\end{quote}

\functionone {SYM2CHR}{\sym}{\chrlist}\index{SYM2CHR}
\begin{quote}
	{\tt SYM2CHR} is the logical inverse of {\tt CHR2SYM}: it returns the list
	of the characters which compound the symbol.

	For instance:
	%
	\begin{verbatim}
		> (SYM2CHR 'APPLE)
		(65 80 80 76 69)
	\end{verbatim}
\end{quote}

\functionone {CHR2INT}{\chrlist}{\inte}\index{CHR2INT}
\begin{quote}
	{\tt CHR2INT} takes a list of characters which represent digits and returns
	the corresponding integer.
	If the list begins with a minus sign ``-'' then {\tt CHR2INT} returns the
	appropriate negative number.

	For instance:
	%
	\begin{verbatim}
		> (CHR2INT '(49 51 55)
		137
		> (CHR2INT '(45 49 51 55))
		-137
	\end{verbatim}
\end{quote}

\functionone {\tt INT2CHR}{\inte}{\chrlist}\index{INT2CHR}
\begin{quote}
	{\tt INT2CHR} is the logical inverse of {\tt CHR2INT}.
	{\tt INT2CHR} returns the list of characters corresponding to the digits of
	the integer.
	If the integer is negative then the first element of the list will contain
	the number $45$, which is the {\tt ASCII} code for the minus sign ``-''.

	Example:
	%
	\begin{verbatim}
		> (INT2CHR 137)
		(49 51 55)}
		> (INT2CHR -137)
		(45 49 51 55)
	\end{verbatim}
\end{quote}


\subsubsection{Strings}

A string is an arbitrary sequence of {\it n\/} characters, indexed from $1$ to
{\it n\/}.
A string is delimited by a pair of double quotes ({\tt "}) which do not belong
to it. {\HG} strings are case sensitive.
{\tt "Symbol"}, {\tt "number"}, {\tt "ThisIsAString"} are examples of {\HG}
strings.

\bigskip

\predicateone {STRING?}{\obj}\index{STRING?}
\begin{quote}
	{\tt STRING?} evaluates to true if \obj\ is a string, false otherwise.
\end{quote}

\functionone {LENSTR}{\str}{\inte}\index{LENSTR}
\begin{quote}
	{\tt LENSTR} returns an integer \inte\ equal to the number of characters in
	\str.
\end{quote}

\actionthree {SETSTR}{\str}{\inte}{\chr}\index{SETSTR}
\begin{quote}
	{\tt SETSTR} sets to \chr\ the \inte-th position of \str.
\end{quote}

\functiontwo {GETSTR}{\str}{\inte}{\chr}\index{GETSTR}
\begin{quote}
	{\tt GETSTR} returns the \inte-th character of \str.
\end{quote}

\functionone {\tt STR2CHR}{\str}{\chrlist}\index{STR2CHR}
\begin{quote}
	{\tt STR2CHR} returns a list of characters which appear in the string.
\end{quote}

\functionone {CHR2STR}{\chrlist}{\str}\index{CHR2STR}
\begin{quote}
	{\tt CHR2STR} returns a string of the length of \chrlist\ that contains the
	characters that appear in \chrlist.

	Example:
	%
	\begin{verbatim}
		> (STR2CHR "APPLE")
		(65 80 80 76 69)
		> (CHR2STR '(104 111 109 101)
		"home"
	\end{verbatim}
\end{quote}


\subsubsection{Pairs}

An {\HG} pair is the ordered collection of two {\HG} objects.

\bigskip


\predicateone{PAIR?}{\obj}\index{PAIR?}
\begin{quote}
	{\tt PAIR?} evaluates to true if \obj\ is a pair, false otherwise.
\end{quote}

\predicateone{ATOMIC?}{\obj}\index{ATOMIC?}
\begin{quote}
	{\tt ATOMIC?} evaluates to true if \obj\ is an atomic object, namely
	a symbol, an integer or a string, false otherwise.
\end{quote}

\functiontwo {CONS}{\objone}{\objtwo}{\pair}\index{CONS}
\begin{quote}
	{\tt CONS} returns a pair consisting of \objone\ and \objtwo.
\end{quote}

\functionone {CAR}{\pair}{\obj}\index{CAR}
\begin{quote}
	{\tt CAR} returns the first element of \pair.
\end{quote}

\functionone {CDR}{\pair}{\obj}\index{CDR}
\begin{quote}
	{\tt CDR} returns the second element of \pair.
\end{quote}

\actiontwo {SETCAR}{\pair}{\obj}\index{SETCAR}
\begin{quote}
	{\tt SETCAR} changes the first element of \pair\ to \obj.
\end{quote}

\actiontwo {SETCDR}{\pair}{\obj}\index{SETCDR}
\begin{quote}
	{\tt SETCDR} changes the second element of \pair\ to \obj.
\end{quote}


\subsubsection{Lists}

A list is either the empty list or a pair whose second element is a list.

\bigskip

\sysconst {MTL}\index{MTL}
\begin{quote}
	{\tt MTL} evaluates to the empty list.
\end{quote}

\predicateone {MTL?}{\obj}\index{MTL?}
\begin{quote}
	{\tt MTL?} evaluates to true if \obj\ is the empty list, false otherwise.
\end{quote}

\predicateone {LIST?}{\obj}\index{LIST?}
\begin{quote}
	{\tt LIST?} evaluates to true if \obj\ is a list, false otherwise.
\end{quote}

\functiontwo {LCONS}{\obj}{\LIST}{\LIST}\index{LCONS}
\begin{quote}
	{\tt LCONS} returns a list with \obj\ in the front of \LIST.
\end{quote}

\functionone {FIRST}{\LIST}{\obj}\index{FIRST}
\begin{quote}
	{\tt FIRST} returns the first element of \LIST.
\end{quote}

\functionone {REST}{\LIST}{\LIST}\index{REST}
\begin{quote}
	{\tt REST} returns the rest of \LIST.
\end{quote}

\actiontwo {SETFIRST}{\LIST}{\obj}\index{SETFIRST}
\begin{quote}
	{\tt SETFIRST} changes the first element of \LIST\ to \obj.
\end{quote}

\actiontwo {SETREST}{\LISTone}{\LISTtwo}\index{SETREST}
\begin{quote}
	{\tt SETREST} changes the rest of \LISTone\ to \LISTtwo.
\end{quote}


\subsection{Evaluation of forms}

This section describes how a form is evaluated by {\HG} interpreter.
The different types of forms are:

\begin{itemize}
	\item null operation {\tt <nop>}
	\item constant {\tt <const>}
	\item variable {\tt <var>}
	\item quotation {\tt <quote>}
	\item subroutine application {\tt <appl>}
	\item conditional evaluation {\tt <if>}
	\item operation sequence {\tt <seq>}
	\item local environment definition {\tt <let>}
	\item environment modification {\tt <setq>}
	\item program {\tt <pgm>}
\end{itemize}
%
In the rest of the chapter we shall refer to the different types of forms to be
evaluated by the corresponding names which are enclosed between angle brackets.
The syntax of the language forms is defined by the following grammar.

\begin{bnf}
	\T{form}           \sep \T{atomic form} | \T{compound form} \\
	\T{atomic form}	   \sep \T{nop} | \T{const} | \T{var} \\
	\T{compound form}  \sep \T{quote} | \T{appl} | \T{if} | \T{let} | \T{setq} |
							\T{seq} | \T{pgm} \\
	\T{nop}			   \sep ( NOP) \\
	\T{quote}		   \sep ( QUOTE \T{form}) \\
	\T{appl}		   \sep (\T{subroutine name} \{ \T{form} \})\\
	\T{if}			   \sep ( IF | IFN \T{predform} \T{form1} \T{form2}) \\
	\T{let}			   \sep ( LET ( \{ ( \T{var} \T{form} ) \} ) \{ \T{form} \})
							\\ 
	\T{seq}			   \sep ( SEQ \{ \T{form} \} )\\
	\T{pgm}			   \sep ( PGM ( \{ \T{var} \} ) \{ \T{pgm-stm} \} )\\
	\T{pgm-stm}		   \sep \T{form} | \T{rtn} | \T{jump} | \T{label}\\
	\T{rtn}			   \sep ( RETURN \T{form} ) \\
	\T{jump}		   \sep ( BRT | BRF \T{label} \T{predform} ) | ( GOTO
							\T{label} )\\
	\T{label}		   \sep \T{symbol}
\end{bnf}
%
Some explanation must be supplied: by constant ({\tt <const>}) we mean either
a truth value, an integer, a character or a string; by predicative form
({\tt <predform>}) we mean a particular form that evaluates to a truth value;
by program statements ({\tt <pgm-stm>}) we mean the special constructs that
compose a {\tt <pgm>} form statement list.
In the rest of this section we shall describe the semantics of each form,
referring to the grammar which has been presented.

\bigskip

{\bf Null operation}\index{NOP}
\begin{quote}
	The evaluation of the {\tt <nop>} form does nothing and does not return a
	value.
\end{quote}

{\bf Constant}
\begin{quote}
	The evaluation of a constant form (truth value, integer, character or
	string) returns the constant itself. Hence {\HG} constants are invariant
	with respect to the evaluation.
\end{quote}

{\bf Variable}
\begin{quote}
	A {\tt <var>} form is a symbol. This symbol must be the name of an
	initialized variable.
	In this case the result of the evaluation of the form is the value of the
	variable in the current environment; otherwise a fatal error occurs.
\end{quote}

{\bf Quotation}\index{QUOTE}
\begin{quote}
	The evaluation of a {\tt <quote>} form returns the argument form without
	evaluating it.
\end{quote}

{\bf Subroutine application}
\begin{quote}
	{\tt <subroutine name>} must be a symbol which is the name of an {\HG}
	function or predicate.
	It can be the name of a built-in primitive or a user-defined function.
	The argument {\tt <form>}s are evaluated and bound to the corresponding
	formal parameters of the function or predicate (call-by-value).
	The result of the application is the value obtained applying the specified
	subroutine to the arguments.
	The type of the values deriving from arguments must match properly the data
	type of the formal parameters, otherwise a fatal error occurs.

	Examples of subroutine applications are:
	%
	\begin{verbatim}
		> (CONS 1 13)
		(1 . 13)
		> (MTL? MTL)
		T
	\end{verbatim}
\end{quote}

{\bf Conditional expressions}\index{IF}\index{IFN}
\begin{quote}
	If the {\tt <predform>} evaluates to true, then the {\tt IF} conditional
	expression returns the evaluation of the first {\tt <form>}, otherwise of
	the second one.
	{\tt IFN} stands for {\tt IF NOT}; its semantics are trivial.

	For instance:
	%
	\begin{verbatim}
		> (IF (MTL? X) FALSE TRUE)
	\end{verbatim}
	%
	evaluates to false if {\tt X} evaluates to the empty list, true otherwise.
\end{quote}

{\bf Sequences of forms: \verb+<seq>+}\index{SEQ}
\begin{quote}
	The value of a {\tt SEQ} form is the value of its last form in the
	environment defined by the evaluation of previous forms.
\end{quote}

{\bf Local environment definition}\index{LET}
\begin{quote}
	{\tt LET} is used to create and initialize local variables.
	The result of evaluating a {\tt LET} is given by the evaluation of the last
	{\tt <form>} in a new environment in which each {\tt <var>} is associated
	with the value of the respective {\tt <form>}.
	If one of the initializing {\tt <form>}s contains any reference to any of
	the {\tt <var>}s, its value is taken from the lexically exterior
	environment.
	Hence the semantics of a {\tt <let>} form is the same as:
	%
	\begin{verbatim}
		(SEQ
		  (SETQ TEMP-1 <form-1>)
		  ...
		  (SETQ TEMP-n <form-n>)
		  (SETQ var-1 TEMP-1)
		  ...
		  (SETQ var-n TEMP-n)
		  <form>)
	\end{verbatim}
	%
	where {\tt TEMP-1} through {\tt TEMP-n} are symbols which do not appear in
	any {\tt <form-i>} or in the last {\tt <form>}.
\end{quote}

{\bf Environment modification}\index{SETQ}
\begin{quote}
	The evaluation of a {\tt <setq>} form produces a side effect on
	environments, and does not return a value.
	The evaluation of a {\tt <setq>} form sets the value associated with the
	variable {\tt <var>} to the evaluation of {\tt <form>}.
	If {\tt <var>} is a global variable the form {\tt (SETQ <var> <form>)} is
	equivalent to the form {\tt (SETVAL (QUOTE <var>) <form>)}.

	For instance:
	%
	\begin{verbatim}
		> (SETVAL (QUOTE X) 1)
		1
		> (LET ((X 5)) (SEQ (SETQ X 3) X)))
		3
	\end{verbatim}
	%
	evaluates to $3$ and leaves the value of {\tt X} to be $1$.
	%
	\begin{verbatim}
		(SEQ
		  (SETVAL (QUOTE X) 1)
		  (LET ((X 5)) (SEQ (SETVAL (QUOTE X) 3) X)))
	\end{verbatim}
	%
	evaluates to $5$ and leaves the value of {\tt X} to be $3$.	
\end{quote}

{\bf Program}\index{PGM}
\begin{quote}
	The {\tt <pgm>} form provides a facility for executing sequential programs,
	something like {\tt PROG}s in Common Lisp.
	The default order of executing {\tt <pgm-stm>}s is the sequential order.
	This can be modified by the use of {\tt <jump>} or of {\tt <rtn>}
	instructions.
	The {\tt <var>}s are local to the {\tt <pgm>} form and they have no initial
	value. 
\end{quote}

{\bf Jumps}\index{BRT}\index{BRF}\index{GOTO}
\begin{quote}
	The {\tt <jump>} statements can be either conditional ({\tt BRT} and
	{\tt BRF}) or unconditional ({\tt GOTO}).
	{\tt BRT} branches to the {\tt <label>} if the {\tt <predform>} evaluates to
	true.
	{\tt BRF} branches to the {\tt <label>} if the {\tt <predform>} evaluates to
	false.
	{\tt GOTO} branches to {\tt <label>}.
\end{quote}

{\bf Labels}
\begin{quote}
	{\tt <label>}s are statements which are simply symbols.
	They are not evaluated but provide markers for the {\tt <jump>} statements.
\end{quote}

{\bf Return}\index{RETURN}
\begin{quote}
	At the evaluation of a {\tt <rtn>} the {\tt <pgm>} statements evaluation is
	halted and the value of {\tt <form>} is returned.
	As shown by the grammar, a {\tt <rtn>} statement can only be used as
	statement into a {\tt <pgm>} form.
\end{quote}


\subsubsection{Eval and apply}

In this section we describe two special form for evaluating forms and applying
functions to forms.

\bigskip

\functionone {HGKM-EVAL}{\form}{\obj}\index{HGKM-EVAL}
\begin{quote}
	{\tt HGKM-EVAL} evaluates \form\ in the current environment and returns the
	resulting \obj.
\end{quote}

\functiontwo {HGKM-APPLY}{\funcdesc}{\formlist}{\value}\index{HGKM-APPLY}
\begin{quote}
	{\tt HGKM-APPLY} binds each element of the form list (without evaluating it)
	to the corresponding element of the bound variables list of the function
	description, and returns the value of evaluating the body of the function
	description in this new environment.

	Example:
	%
	\begin{verbatim}
		> (HGKM-APPLY (GET-FUN 'APPEND) '((A B) (C D)))
		(A B C D)
		> (HGKM-APPLY (GET-FUN 'APPEND) '(LIST '(B) '(C D)))
		(B C D)
	\end{verbatim}
\end{quote}


\subsection{Special forms}

In {\HG} special forms to define functions and macros are provided.
We call this special forms \flconst s, to distinguish them from Common LISP
constructors (level zero constructors) which have been used to define built-in
forms. 
In this section we also present {\HG} special forms to inspect and describe
built-in and user-defined subroutines: we shall refer to these special forms
using the term "\flrec s".


\subsubsection{First level constructors}

The \flconst s are

\begin{itemize}
	\item {\tt DEFLAM}
	\item {\tt DEFMAC}
	\item {\tt DEFSYS}
	\item {\tt DEFMACSYS}
	\item {\tt DEFSUB}
	\item {\tt SET-FUN}
\end{itemize}

A \flconst\ call has the usual form:

\begin{quote}\tt
	(<constructor name> {\it [\sbrna] [\sbral] [\sbrbo]})
\end{quote}

The result of a such call is the definition of a subroutine named {\it \sbrna},
in which {\it \sbral\/} is the list of formal parameters and {\it \sbrbo\/} is
the statement list.
The syntax of user defined {\HG} subroutines is the same as Common LISP
subroutines: hence are allowed all CL options, namely {\tt \&REST, \&OPTIONS,}
etc.

The way in which a subroutine is implemented depends on which is the \flconst
used in defining it, and on the value of two installation variables at the
moment of definition.
Anyway the implementation is almost transparent to the user, and so here we give
only a brief description of implementation features.

We shall explain now the meaning of an installation variable, which is named
{\tt \infmode}: the main idea is that when this variable evaluates to true some
system information is set up at the moment of installation, while when it
evaluates to false no information is available.
The following are special forms which {\it do not} evaluate their arguments.

\bigskip

\actionthree {DEFLAM}{\sbrna}{\sbral}{\sbrbo}\index{DEFLAM}
\begin{quote}
	{\tt DEFLAM} is a function constructor; which always allows the redefinition
	of previously defined functions.
	If {\tt\infmode} evaluates to false, then {\tt DEFLAM} just defines the
	function.
	If {\tt\infmode} evaluates to true, {\tt DEFLAM} also sets a property
	telling that {\it \sbrna\/} is a first level defined function and, in case
	of redefinition, prints a warning message.
\end{quote}

\actionthree {DEFMAC}{\sbrna}{\sbral}{\sbrbo}\index{DEFMAC}
\begin{quote}
	{\tt DEFMAC} is a macro definer: hence {\it \sbrbo\/} must be a macro
	statement list, that is must be ``expansible'' like Common LISP macros.
	{\tt DEFMAC} always allows to redefine already defined macros.
	If {\tt\infmode} evaluates to false then {\tt DEFMAC} just defines the
	macro; else {\tt DEFMAC} also sets a property telling that {\it \sbrna} is a
	first level defined macro and, in case of redefinition, prints a warning
	message. 
\end{quote}

\actionthree {\tt DEFSYS}{\sbrna}{\sbral}{\sbrbo}\index{DEFSYS}
\begin{quote}
	{\tt DEFSYS} is a function definer which allows to redefine already defined
	functions only when {\tt\infmode} evaluates to false; when {\tt\infmode} is
	true and {\it \sbrna} has already been defined, {\tt DEFSYS} will signal
	that it can not execute the redefinition.
	With respect to functions which have not been defined yet {\tt DEFSYS} is
	just like {\tt DEFLAM}.
\end{quote}

\actionthree {DEFMACSYS}{\sbrna}{\sbral}{\sbrbo}\index{DEFMACSYS}
\begin{quote}
	{\tt DEFMACSYS} is a macro definer which allows to redefine already defined
	macros only when {\tt\infmode} evaluates to false; when {\tt\infmode} is
	true and {\it \sbrna} has already been defined, {\tt DEFMACSYS} will signal
	that it can not execute the redefinition.
	With respect to macros which have not been defined yet {\tt DEFMACSYS} is
	just like {\tt DEFMAC}. 
\end{quote}

\actionthree {\tt SET-FUN}{\sbrna}{\sbral}{\sbrbo}\index{SET-FUN}
\begin{quote}
	{\tt SET-FUN} differs from {\tt DEFLAM} just because it evaluates its
	arguments,while {\tt DEFLAM} does not.
\end{quote}

\actionthree {DEFSUB}{\sbrna}{\sbral}{\sbrbo}\index{DEFSUB}
\begin{quote}
	{\tt DEFSUB} can work as {\tt DEFLAM} or as {\tt DEFMAC}.
	This subroutine constructor is included for efficiency reasons rather than
	theoretical reasons; hence its description is delayed to appendix A.
\end{quote}


\subsubsection{First level recognizers}

\predicateone {FUNCTION?}{\obj}\index{FUNCTION?}
\begin{quote}
	{\tt FUNCTION?} is a function recognizer, which evaluates to true if \obj\
	is the name of an {\HG} function, false otherwise. 
	{\tt FUNCTION?} can not tell a built-in function from a \flconst\ defined
	function, or a compiled function from an interpreted one.
	{\tt FUNCTION?} returns the correct information only if {\tt\infmode} is
	true; else it prints a warning message to signal that an un-allowable
	information has been required and executes an {\tt HGKM-ABORT} instruction: hence
	every {\tt FUNCTION?} call must be ``surrounded'' by a {\tt TRAP}
	call\footnote{Further information about {\tt TRAP} and {\tt HGKM-ABORT} can be
	found in ``Error handling'' section}.
\end{quote}

\predicateone {MACRO?}{\obj}\index{MACRO?}
\begin{quote}
	{\tt MACRO?} is a macro recognizer, which evaluates to true if \obj\ is the
	name of an {\HG} macro, false otherwise.
	Note that {\tt MACRO?} can not tell a system macro from a user defined
	macro.
	Furthermore, {\tt MACRO?} has the same behavior as {\tt FUNCTION?} with
	respect to {\tt\infmode}.
	Finally we want to mention a small problem: both {\tt MACRO?} and
	{\tt FUNCTION?} evaluate to true in case of built-in subroutines, because
	there is no information telling if a such subroutine has been defined as
	macro or as function.
\end{quote}

\functionone {GET-FUN}{\obj}{\funcdesc}\index{GET-FUN}
\begin{quote}
	{\tt GET-FUN} returns a function description, in one of the following forms:
	%
	\begin{itemize}
		\item
			{\tt (HGKM <function name> COMPILED)} if the \newline function is
			compiled; 
		\item
			{\tt (HOST <function name> <bound variables list> <body>)} if the
			\newline function is an un-compiled system function;
		\item
			{\tt (HGKM <function name> <bound variables list> <body>)} if the
			\newline function is an uncompiled \flconst\ defined function.
	\end{itemize}
\end{quote}


\subsection{Error handling}

\functionone {TRAP}{\form}{\value}\index{TRAP}
\begin{quote}
	{\tt TRAP} evaluates \form\ and if no error occurs during its evaluation
	{\tt TRAP} returns the same value as the form ({\tt LIST} \form).
	If a system error occurs during an evaluation then {\tt TRAP} returns an
	atomic object that can be tested for determining what the error was.
	If the function {\tt HGKM-ABORT} is called during evaluation of \form\ then {\tt
	TRAP} will return the value of the argument of {\tt HGKM-ABORT}.
\end{quote}

\actionone {HGKM-ABORT}{\form}\index{HGKM-ABORT}
\begin{quote}
	{\tt HGKM-ABORT} returns the value of \form\ to the most recent {\tt TRAP}; every
	{\tt HGKM-ABORT} call must be lexically surrounded at least by a {\tt TRAP}
	function call.
\end{quote}

\predicateone{ERROR-OCCURRED?}{\value}\index{ERROR-OCCURRED?}
\begin{quote}
	{\tt ERROR-OCCURRED?} checks whether a {\tt TRAP}ped \form\ {\tt HGKM-ABORT}ed.
\end{quote}

\functionone{TRAP-GET-RESULT}{\value}{\obj}\index{TRAP-GET-RESULT}
\begin{quote}
	{\tt TRAP-GET-RESULT} selects from \value\ the result of the evaluation of
	\form.
\end{quote}


\subsection{Input-output}

In this section {\HG} input-output handling primitives are presented.
{\HG} uses unidirectional channels implemented on Common LISP streams.
A channel is an object representing in {\HG} environment an external object,
i.e.\ a file or the keyboard.
A channel needs to be open to become active.
The system keeps track of open channels, storing information about each active
channel in a global data structure, the channel list.
Every channel list entry contains
%
\begin{itemize}
	\item
		the channel itself
	\item
		the data flow direction ({\tt IN} or {\tt OUT})
	\item
		the type of objects that can be communicated through the channel, namely
		s-expressions ({\tt SEXP}) or characters ({\tt CHAR}).
	\item
		the information on the object bound to the channel (the device on which
		is stored the physical file in case of file I/O, the keyboard in case of
		console I/O, etc).
\end{itemize}


\subsubsection{Opening and closing channels}

\functionone {HGKM-ISOPEN}{\filena}{\chan}\index{HGKM-ISOPEN}
\begin{quote}
	\filena\ is a form which evaluates to a string.
	{\tt HGKM-ISOPEN} creates and returns a channel for s-expressions
	corresponding to the file \filena, input direction.
	When trying to open a channel corresponding to a non existing file, {\tt
	HGKM-OPEN} executes an {\tt HGKM-ABORT} instruction, to signal that an illegal
	operation has been required: hence {\tt HGKM-OPEN} must be {\tt TRAP}-ped.
\end{quote}

\functionone {HGKM-ICOPEN}{\filena}{\chan}\index{HGKM-ICOPEN}
\begin{quote}
	\filena\ is a form which evaluates to a string.
	{\tt HGKM-ICOPEN} creates and returns a channel for characters corresponding
	to the file \filena, input direction.
	When trying to open a channel corresponding to a non existing file, {\tt
	HGKM-OPEN} executes an {\tt HGKM-ABORT} instruction, to signal that an illegal
	operation has been required: hence {\tt HGKM-OPEN} must be {\tt TRAP}-ped.
\end{quote}

\functionzero {TTY-IOPEN}{\chan}\index{TTY-IOPEN}
\begin{quote}
	{\tt TTY-IOPEN} returns a channel for console input.
\end{quote}

\functionone {HGKM-OSOPEN}{\filena}{\chan}\index{HGKM-OSOPEN}
\begin{quote}
	\filena\ is a form which evaluates to a string.
	{\tt HGKM-OSOPEN} creates and returns a channel for s-expressions
	corresponding to the file \filena, output direction.
\end{quote}

\functionone {HGKM-OCOPEN}{\filena}{\chan}\index{HGKM-OCOPEN}
\begin{quote}
	\filena\ is a form which evaluates to a string.
	{\tt HGKM-OCOPEN} creates and returns a channel for characters corresponding
	to the file \filena, output direction.
\end{quote}

\functionzero {TTY-OOPEN}{\chan}\index{TTY-OOPEN}
\begin{quote}
	{\tt TTY-OOPEN} returns a channel for console output.
\end{quote}

\actionone {HGKM-CLOSE}{\chan}\index{HGKM-CLOSE}
\begin{quote}
	{\tt HGKM-CLOSE} closes \chan\ and updates the channel list taking off the
	information about \chan.
	{\tt HGKM-CLOSE} must be trapped because, when \chan\ does not exist, an
	{\tt HGKM-ABORT} occurs to signal that an illegal operation has been required.
\end{quote}


\subsubsection{Input output specifying channels}

\functionone {CHAN-READ}{\chan}{\obj}\index{CHAN-READ}
\begin{quote}
	{\tt CHAN-READ} reads an s-expression from \chan\ and returns it; to handle
	the case in which the object read is an {\tt EOF} a {\tt TRAP} function call
	is required. 
	\chan\ must have been created by {\tt TTY-IOPEN} or by {\tt HGKM-ISOPEN}.
\end{quote}

\functionone {CHAN-CHI}{\chan}{\chr}\index{CHAN-CHI}
\begin{quote}
	{\tt CHAN-CHI} reads a character from \chan\ and returns it; to handle the
	case in which the character read is an {\tt EOF} a {\tt TRAP} function call
	is required. 
	\chan\ must have been created by {\tt TTY-IOPEN} or by {\tt HGKM-ICOPEN}.
\end{quote}

\actiontwo {CHAN-PRINC}{\obj}{\chan}\index{CHAN-PRINC}
\begin{quote}
	{\tt CHAN-PRINC} writes \obj\ on \chan.
	\chan\ must have been created by {\tt TTY-OOPEN} or by {\tt HGKM-OSOPEN}.
\end{quote}

\actionone {CHAN-CRLF}{\chan}\index{CHAN-CRLF}
\begin{quote}
	{\tt CHAN-CRLF} writes a carriage return on \chan.
	\chan\ must have been created by {\tt TTY-OOPEN} or by {\tt HGKM-OSOPEN}.
\end{quote}

\actiontwo {CHAN-CHO}{\chr}{\chan}\index{CHAN-CHO}
\begin{quote}
	{\tt CHAN-CHO} writes \chr\ on \chan.
	\chan\ must have been created by {\tt TTY-OOPEN} or by {\tt HGKM-OCOPEN}.
\end{quote}

\functionone{CHANNEL-GET-FNAME}{\chan}{\filena}\index{CHANNEL-GET-FNAME}
\begin{quote}
	{\tt CHANNEL-GET-FNAME} returns the name of the file corresponding to the
	channel \chan.
\end{quote}


\subsubsection{Input output with implicit channels}

\actionone {SELECT}{\chan}\index{SELECT}
\begin{quote}
	{\tt SELECT} is the primitive to set the values of default channels.
	If into the channel list the direction associated with \chan\ is {\tt IN},
	then \chan\ will be selected as default input channel, else as default
	output channel.
\end{quote}

\functionzero {HGKM-READ}{\obj}\index{HGKM-READ}
\begin{quote}
	{\tt HGKM-CHI} reads an s-expression from the selected input and returns it.
	to handle the case in which the object read is an {\tt EOF} a {\tt TRAP}
	function call is required.
	The selected input must have been created by {\tt TTY-IOPEN} or by {\tt
	HGKM-ISOPEN}. 
\end{quote}

\functionzero {HGKM-CHI}{\chr}\index{HGKM-CHI}
\begin{quote}
	{\tt HGKM-CHI} reads a character from the selected input and returns it; to
	handle the case in which the character read is an {\tt EOF} a {\tt TRAP}
	function call is required.
	The selected must have been created by {\tt TTY-IOPEN} or by
	{\tt HGKM-ICOPEN}. 
\end{quote}

\actionone {HGKM-PRINC}{\obj}\index{HGKM-PRINC}
\begin{quote}
	{\tt HGKM-PRINC} writes \obj\ on the selected output.
	The selected output must have been created by {\tt TTY-OOPEN} or by
	{\tt HGKM-OSOPEN}.
\end{quote}

\actionone {CRLF}{\chan}\index{CRLF}
\begin{quote}
	{\tt CRLF} writes a carriage return on the selected output.
	The selected output must have been created by {\tt TTY-OOPEN} or by
	{\tt HGKM-OSOPEN}.
\end{quote}

\actionone {HGKM-CHO}{\chan}\index{HGKM-CHO}
\begin{quote}
	{\tt HGKM-CHO} writes \chr\ on the selected output.
	The selected output must have been created by {\tt TTY-OOPEN} or by
	{\tt HGKM-OCOPEN}.
\end{quote}

\actionzero {HGKM-INITIO}\index{HGKM-INITIO}
\begin{quote}
	{\tt HGKM-INITIO} initializes I/O by setting the following I/O-related
	variables:  {\tt CHAN-LIST}, {\tt STANDARD-INPUT}, {\tt STANDARD-OUTPUT},
	{\tt SELECTED-INPUT}, {\tt SELECTED-OUTPUT}.
\end{quote}


\pagebreak
\section{Extended {\HG}}

In this section extended {\HG} is presented.
Extended {\HG} is a collection of primitives which operate on basic {\HG} data
structures or on newly defined ones.

Extended {\HG} is not founded on theoretical considerations like basic {\HG}.
Its semantics can't be axiomatized in first order logic.
Extended {\HG} has been developed mainly for sake of performance and programming
facility, and it has grown up ``on demand'': that is, when it has been realized
that a set of primitives was useful it has been implemented. This is why
extended {\HG} does not look homogeneous.

During this revision, an attempt has been made to keep extended {\HG} as small
as possible: hence, several primitives which are actually part of the source
code are not presented in this manual because they seem to be rather useless.
Further revisions of the source codes will face the problem of removing them.


\subsection{Truth values}

\begin{itemize}
	\item {\tt (NOT \ldots)}\index{NOT}: as in CL.
	\item {\tt (AND \ldots)}\index{AND}: as in CL.
	\item {\tt (OR \ldots)}\index{OR}: as in CL.
	\item {\tt (IST {\it [sym]})}\index{IST}:
	  {\tt IST} and its counterpart {\tt ISF} should theoretically allow GETFOL
	  to use real truth values instead that {\tt T} and {\tt NIL}. {\tt IST}
	  returns {\tt TVAL} when {\it sym} is not {\tt MTL}, {\tt FVAL} otherwise.
	\item {\tt (ISF {\it [sym]})}\index{ISF}:
	  {\tt ISF} returns {\tt FVAL} when {\it sym} is {\tt MTL}, {\tt TVAL}
	  otherwise.
\end{itemize}


\subsection{Symbols}

\begin{itemize}
    \item {\tt (CAPITALIZE {\it [sym {\tt | } str]})}:\index{CAPITALIZE}
        returns the argument with all its letters capitalized.
    \item {\tt (CAPITALIZE-SYM {\it [sym]})}:\index{CAPITALIZE-SYM}
        returns {\it sym} with all its letters capitalized.
    \item {\tt (CAPITALIZE-STR {\it [str]})}:\index{CAPITALIZE-STR}
        returns {\it str} with all its letters capitalized.
    \item {\tt (CONCAT-SYMS {\it [sym$_1$][sym$_2$]})}:\index{CONCAT-SYMS}
        returns the concatenation of {\it sym$_1$} and {\it sym$_2$}.
    \item {\tt  (PAIR2CHR {\it [pair]})}:\index{PAIR2CHR}
        makes {\it pair} into the corresponding symbol.
    \item {\tt  (INT2SYM {\it [int]})}:\index{INT2SYM}
        makes {\it int} into the corresponding symbol.
    \item {\tt  (NAT2SYM {\it [int]})}:\index{NAT2SYM}
        makes a natural {\it int} into the corresponding symbol.
    \item {\tt  (NAT2SYM {\it [nat]})}:\index{NAT2SYM}
        makes the natural {\it nat} into the corresponding symbol.
    \item {\tt (NUMERAL {\it [num]})}\index{NUMERAL}
        looks if {\it num} is a numeral.
\end{itemize}


\subsection{Integers}

These primitives are trivial, and do not require any explanation.
Note that {\tt INTEQ} (the integer equality predicate) must be used instead of
{\tt EQ} (the object equality) when compiling because some CL compiler
substitutes the variable n (where n goes from $-1024$ to $1023$), which
evaluates to the integer $n$, with its actual value. This can create unexpected
behavior which is very difficult to detect and fix.

\begin{itemize}
	\item {\tt (MINUS ...)}\index{MINUS}
	\item {\tt (PLUS ...)}\index{PLUS}
	\item {\tt (DIFFERENCE ...)}\index{DIFFERENCE}
	\item {\tt (TIMES ...)}\index{TIMES}
	\item {\tt (QUOTIENT ...)}\index{QUOTIENT}
	\item {\tt (REMAINDER ...)}\index{REMAINDER}
	\item {\tt (NATNUM ...)}\index{NATNUM}
	\item {\tt (INTEQ ...)}\index{INTEQ}
	\item {\tt (LT ...)}\index{LT}
	\item {\tt (LE ...)}\index{LE}
	\item {\tt (GT ...)}\index{GT}
	\item {\tt (GE ...)}\index{GE}
\end{itemize}


\subsection{Characters}

\begin{itemize}
	\item {\tt (ASCII {\it [chr]})}:\index{ASCII}
		  creates and returns the interned symbol corresponding to {\it chr}.
	\item {\tt (SYM-INTERN {\it [obj]})}:\index{SYM-INTERN}
		   creates and returns the interned symbol corresponding to {\it obj}.
	\item {\tt (MAKE-UNINTERNED {\it [obj]})}:\index{MAKE-UNINTERNED}
		   creates and returns an un-interned symbol corresponding to {\it obj}.
	\item {\tt (SYM-GETCHRFIRST {\it [sym]})}:\index{SYM-GETCHRFIRST}
		   returns the character corresponding to the first character of
		   {\it sym}.
\end{itemize}


\subsection{Strings}

\begin{itemize}
	\item {\tt (SUBSTR? {\it [str$_1$][str$_2$]})}:\index{SUBSTR?}
		   evaluates to true if {\it str$_2$} begins with {\it str$_1$}.
	\item {\tt (CONCAT-STRS {\it [str$_1$][str$_2$]})}:\index{CONCAT-STRS}
		  returns the concatenation of {\it str$_1$} and {\it str$_2$}.
	\item {\tt (SUBSTR {\it [str][int]})}:\index{SUBSTR}
		   returns the string of the first {\it int} characters of {\it str}. If
		   {\it int} is too big an error occurs.
	\item {\tt (INSTR? {\it [str$_1$][str$_2$]})}:\index{INSTR?}
		   evaluates to true if {\it str$_1$} appears in {\it str$_2$}.
\end{itemize}


\subsection{Lists}

\begin{itemize}
   	\item {\tt (LIST ...)}:\index{LIST}
        as in CL.
   	\item {\tt (EQUAL ...)}:\index{EQUAL}
        as in CL.
   	\item {\tt (COPY {\it [obj]})}:\index{COPY}
        returns an object {\tt EQUAL} and not  {\tt EQ} to {\it obj}.
    \item {\tt (LENGTH {\it [list]})}:\index{LENGTH}
        returns the length of {\it list}.
    \item {\tt (HGKM-MEMBER {\it [obj][list]})}:\index{HGKM-MEMBER}
        as in CL.
		The equality test is {\tt EQUAL}.
    \item {\tt (HGKM-MEMQ {\it [obj][list]})}:\index{HGKM-MEMQ}
	    like {\tt HGKM-MEMBER}, but with {\tt EQ} test.
		For instance:
		%
		\begin{verbatim}
			> (HGKM-MEMQ '(C D) ((A B)(C D) E))
			FALSE
		\end{verbatim}
    \item {\tt (APPEND ...)}:\index{APPEND}
        as in CL.
    \item {\tt (UNION ...)}:\index{UNION}
        as in CL.
    \item {\tt (REVERSE {\it [list]})}:\index{REVERSE}
        returns the reverse of the top level of {\it list}.
    \item {\tt (SUBST {\it [obj$_1$][obj$_2$][obj$_3$]})}:\index{SUBST}
        substitutes {\it obj$_1$} to every {\tt EQUAL} occurrence of
		{\it obj$_2$} in {\it obj$_3$}.
    \item {\tt (CL-ASSOC ...)}:\index{CL-ASSOC}
        as CL's {\tt ASSOC}: it tests with {\tt EQ}.
    \item {\tt (HGKM-ASSOC ...)}:\index{HGKM-ASSOC}
         the same as {\tt CL-ASSOC}, but it tests with {\tt EQUAL}.
    \item {\tt (SECOND {\it [list]})}:\index{SECOND}
         returns the second element of {\it list}.
    \item {\tt (LAST {\it [list]})}:\index{LAST}
         as in common lisp: it returns the last {\em cons} of {\it list}.
    \item {\tt (HGKM-NTH {\it [int][list]})}:\index{HGKM-NTH}
        returns the {\it int}-th element of {\it list}.
        Note that {\it int} must be positive, otherwise an error occurs.
    \item {\tt (NCONC ...)}:\index{NCONC}
         as in CL.
    \item {\tt  (POS {\it [sym][list]})}:\index{POS}
        returns the first position of {\it sym} in {\it list}.
        If it does not exist {\tt POS} returns {\tt 0}.
    \item {\tt  (TAIL {\it [list][obj]})}:\index{TAIL}
        if {\it obj} occurs in {\it list} the list of elements following
        {\it obj} (not including {\it obj}) is returned; otherwise {\tt MTL}
        is returned. 
    \item {\tt  (FLATTEN {\it [list]})}:\index{FLATTEN}
        {\it list} is flattened and returned.
    \item {\tt  (PRELIST {\it [list][int]})}:\index{PRELIST}
        creates a list of the first {\it int} elements of {\it list}.
    \item {\tt  (SUFLIST {\it [list][int]})}:\index{SUFLIST}
        creates a list by deleting the first {\it int} elements of {\it list}.
    \item {\tt  (INLIST? {\it [list$_1$][list$_2$]})}:\index{INLIST?}
        looks if {\it list$_1$} is in {\it list$_2$} (only at top level).
    \item {\tt  (SUBTRACT {\it [list$_1$][list$_2$]})}:\index{SUBTRACT}
        returns the list of the elements of {\it list$_1$} which are not in
        {\it list$_2$}. 
    \item {\tt  (INTERSECT {\it [list$_1$][list$_2$]})}:\index{INTERSECT}
         returns the intersection
    \item {\tt (SUBSET {\it [list$_1$][list$_2$]})}:\index{SUBSET}
        looks if {\it list$_1$} is a subset of {\it list$_2$}.
        of {\it list$_1$} and {\it list$_2$}.
    \item {\tt  (INSERT! {\it [obj][list]})}:\index{INSERT!}
        inserts {\it obj} in {\it list} making side effect.
        {\tt INSERT!} assumes that {\it obj} can only occur once in {\it
        list}. 
    \item {\tt  (HGKM-DELETE {\it [obj][list]})}:\index{HGKM-DELETE}
        deletes {\it obj} from {\it list} without doing side effect.
        {\tt HGKM-DELETE} assumes that {\it obj} only occurs once in {\it
        list}. 
    \item {\tt  (DELETE! {\it [obj][list]})}:\index{DELETE!}
         the same as {\tt HGKM-DELETE} but with side effect.
    \item {\tt  (OUTOF {\it [obj][list]})}:\index{OUTOF}
        removes all the occurrences of {\it obj} from {\it list} without
        doing side effect.
    \item {\tt  (OUTOF! {\it [obj][list]})}:\index{OUTOF!}
         the same as {\tt OUTOF} but with side effect.
    \item {\tt  (UNLIST {\it [list]})}:\index{UNLIST}
        substitutes the last cons of {\it list} with its {\tt CAR}.
        For instance:

        {\tt (UNLIST (QUOTE (A))) $\Longrightarrow$ A}

        {\tt (UNLIST (QUOTE (A B))) $\Longrightarrow$ (A . B)}

        {\tt (UNLIST (QUOTE (A B C))) $\Longrightarrow$ (A B . C)}
    \item {\tt  (REVMAP {\it [list]})}:\index{REVMAP}
        puts the last element of {\it list} at the top of
        {\it list} itself.
    \item {\tt  (REPETITION-GET {\it [list]})}:\index{REPETITION-GET}
         returns the first element occurring twice in {\it list}.
    \item {\tt  (LIST-N {\it [obj][int]})}:\index{LIST-N}
         returns the list of {\it int} occurrences of {\it obj}.
    \item {\tt  (PRINTLIST {\it [list]})}:\index{PRINTLIST}
         prints all the elements of {\it list}.
\end{itemize}


\subsection{Arrays}

\begin{itemize}
    \item {\tt (TUPLE {\em [sym]} {\em [len]})}:\index{TUPLE}
        sets the value of the variable {\em sym} to be an array of length
        {\em len}. 
    \item {\tt (SETTUP {\em [array]} {\em [index]} {\em [data]})}:
	\index{SETTUP}
        sets the {\em index}-th entry in {\em array} to {\em data} destroying
        the previous information.
    \item {\tt (GETTUP {[\em array]} {\em [index]})}:\index{GETTUP}
        returns the data stored in {\em array} at the position indexed by
		{\em index}.
\end{itemize}


\subsection{Ordered lists of integers}

\begin{itemize}
	\item {\tt (UNIONLE {\it [list$_1$][list$_2$]})}:\index{UNIONLE}
		given two lists of natural numbers, each in ascending order, creates the
		ascending ordered union. 
	\item {\tt (UNIONGE {\it [list$_1$][list$_2$]})}:\index{UNIONGE}
		given two lists of natural numbers, each in descending order, {\tt
		UNIONGE} creates the descending ordered union.
	\item {\tt (ORDER {\it [list]})}:\index{ORDER}
		returns the list of natural numbers occurring in {\it list} in
		increasing order.
	\item {\tt (INSERT {\it [int][list]})}:\index{INSERT}
		inserts {\it int} in {\it list}, an ascending-ordered list of natural
		numbers.
	\item {\tt (hole {\it [list]})}:\index{hole}
		being {\it list} argument a list of integers greater than or equal to
		$-1$, {\tt hole} returns the smallest number not appearing in
		{\it list}.
		
		Example:
		%
		\begin{verbatim}
			> (hole (QUOTE (-1 0 1 3)))
			2
			> (hole (QUOTE (0 2 3)))
			-1
		\end{verbatim}
\end{itemize}


\subsection{Associative lists}

Elements of associative lists are implemented as dotted pairs.
%
\begin{itemize}
    \item {\tt (alistelement-get-1st {\it [alist-elem]})}:
	\index{alistelement-get-1st}
        returns the {\tt CAR} of the dotted pair.
    \item {\tt (alistelement-get-2nd {\it [alist-elem]})}:
    \index{alistelement-get-2nd}
         returns the {\tt CDR} of the dotted pair.
    \item {\tt (alistelement-mak {\it [obj1][obj2]})}:
    \index{alistelement-mak}
        returns the pair {\tt ({\it obj1} . {\it obj2})}
\end{itemize}
%
Associative lists are associative addressing data structures; they are
implemented as lists of dotted pairs.
%
\begin{itemize}
    \item {\tt (alist-get-first {\it [alist]})}:\index{alist-get-first}
         returns the first element of {\it alist}
    \item {\tt (alist-get-rest {\it [alist]})}:\index{alist-get-rest}
         returns the rest of {\it alist} 
    \item {\tt (alist-cons {\it [alist][alist-elem]})}:\index{alist-cons}
         adds {\it alist-elem} to {\it alist}
    \item {\tt (HGKM-REPLACE {\it [obj$_1$][obj$_2$][alist]})}:
    \index{HGKM-REPLACE}
        if {\it obj$_1$} is the first element of an alist element, then
		{\tt HGKM-REPLACE} replaces the second element with  {\it obj$_2$};
        otherwise, {\tt HGKM-REPLACE} adds the {\tt CONS} of {\it obj$_1$} 
        and obj$_2$ to {\it alist}.
    \item {\tt Alist-empty}:\index{Alist-empty}
        is the constant representing an empty associative list
    \item {\tt (ALIST-EMPTY? {\it [alist]})}:\index{ALIST-EMPTY?}
        looks if {\it alist} is empty
    \item {\tt (alist-add {\it [alist][obj$_1$][obj$_2$]})}\index{alist-add}
        adds the pair {\tt ({\it obj$_1$} . {\it obj$_2$})} to {\it alist}
        if not already present
    \item {\tt (alist-del {\it [alist][obj]})}\index{alist-del}
        removes the pair pointed by {\it obj} from {\it alist}
    \item {\tt (alist-union {\it [alist$_1$][alist$_2$]})}\index{alist-union}
        adds to {\it alist$_2$} all the elements of {\it alist$_1$},
        overwriting those elements of {\it alist$_2$} which are also in
		{\it alist$_1$} 
    \item {\tt (alist-cons-shared {\it [alist][alist-elem]})}:
    \index{alist-cons-shared}
        same as {\tt alist-cons} but with side effect on {\it alist}.
\end{itemize}


\subsection{Extending the evaluation syntax}

\begin{itemize}
	\item {\tt (MAPCAR ...)}:\index{MAPCAR}
		  as in CL.
\end{itemize}


\subsection{Extending the input output}

\begin{itemize}
	\item {\tt (HGKM-LOAD {\it [str]})}:\index{HGKM-LOAD}
		  loads the file named {\it str}.
	\item {\tt (HGKM-LOAD-FROM-CHANNEL {\it [chan]})}:
	\index{HGKM-LOAD-FROM-CHANNEL}
		 loads s-expressions from the default channel until an {\tt EOF} is
		 found.
\end{itemize}


\appendix
\section{Some notes about implementation}

While dealing with big software systems, in phase of development it may be very
helpful to know if a subroutine is built-in or user-defined, and if it has been
defined as function or macro.
If {\tt \infmode}\ is true when {\HG} is created, all this information is made
available: for each subroutine declaration, zero-level constructors set a
property recording that the subroutine is built-in; first-level constructors
set a property recording that the subroutine is user-defined and whether it has
been defined as a function or as a macro; reading these properties, first level
recognizers can tell if a subroutine is built-in or system-defined, and if it is
a function or a macro.
If {\tt \infmode} is false no property is set by the zero and first level
constructors, and first level recognizers, when called, will signal that no
answer is possible because no information is available.
For a detailed description of each first-level constructor and recognizer, with
respect to {\tt \infmode}, the reader is referred to basic {\HG} chapter.

Of course, when {\tt \infmode} is false, a large amount of storage is saved, and
a more compact {\HG} executable file is obtained.
Finally we remark that, for sake of consistency, the value of {\tt \infmode}
should not change while {\HG} is being created.

A complex software system may have big performance problems, because of the
overhead due to the great number of function calls, even if it runs compiled.
Therefore, the only way to speed up the system is to define as many subroutines
as possible as macros. These macros are expanded at the moment of compilation,
obtaining in this way a code which is modular and well structured, and which
does not have the overhead of function calls.
There are two drawbacks to this approach: the former, and least important, is
that the program becomes big because of the macro expansion; the latter, and
critical, is that debugging becomes very hard, because all information related
to functional definition is lost.

In {\HG} this problem may be solved using the first-level constructor {\tt
DEFSUB}, that allows to define subroutines either like macros or like functions,
without changing source code.
As explained below, {\tt DEFSUB} definitions depend on the value of a system
constant, {\tt \debmode}.

A {\tt DEFSUB} call has a syntax as defined in basic {\HG} chapter: it requires
three arguments, that is a {\it subroutine name}, an {\it arguments list} and a
{\it body}.
The third argument, {\it body}, is a function body, as needed in {\tt DEFUN}
subroutines definitions.
If {\tt \debmode} is true, {\tt DEFSUB} works like {\tt DEFLAM}.
If {\tt \debmode} is false, {\tt DEFSUB} executes a {\tt DEFMAC} call, modifying
the third argument: that is, starting from {\it body}, {\tt DEFSUB} builds a
macro body that, when macro-expanded, returns {\it body}.

Hence {\tt DEFSUB} is a subroutine constructor that has a simple function
definition syntax and behaves either like a function or macro constructor,
depending on the value of {\tt \debmode}.
If {\tt \debmode}\ is true debugging is possible, because {\tt DEFSUB}
subroutines have been declared as functions; if {\tt \debmode}\ is false
debugging is very hard, but the execution is speed up because no function call
overhead is present.

Note that {\tt DEFSUB} is subject to some restriction, as remarked in basic
{\HG} chapter: this is due to the complexity of macro body creation.

Finally note that, to reach some performance improvements, a {\tt DEFSUB}
definition of a general subroutine {\tt SBR} must be loaded before compiling
subroutines calling {\tt SBR} itself; else, macro expansion can not be made at
compile time, and the {\tt SBR} is executed interpretively and expanded at run
time.


% ................................. bibliography ..............................
\bibliographystyle{alpha}
\bibliography{}

% .................................... index ..................................
%............................... HGKM MANUAL .................................
%.............................................................................


\documentstyle[11pt]{../styfiles/GFmanual}

%................................. macros ....................................
%.............................................................................

\setcounter{totalnumber}{1}

\def\actionzero#1{\noindent ({\tt {#1}}) \hfill action}
\def\actionone#1#2{\noindent ({\tt {#1}} {\it [{#2}]}) \hfill action}
\def\actiontwo#1#2#3{\noindent ({\tt {#1}} {\it [{#2}]}{\it [{#3}]}) \hfill
action} 
\def\actionthree#1#2#3#4{\noindent ({\tt {#1}} {\it [{#2}]}{\it [{#3}]}{\it
[{#4}]}) \hfill action}

\def\functionzero#1#2{\noindent ({\tt {#1}}) $\Longrightarrow$ {#2} \hfill
function} 
\def\functionone#1#2#3{\noindent ({\tt {#1}} {\it [{#2}]}) $\Longrightarrow$
{#3} \hfill function}
\def\functiontwo#1#2#3#4{\noindent ({\tt {#1}} {\it [{#2}]}{\it [{#3}]})
$\Longrightarrow$ {#4} \hfill function} 
\def\functionthree#1#2#3#4#5{\noindent ({\tt {#1}} {\it [{#2}]}{\it [{#3}]}{\it
[{#4}]}) $\Longrightarrow$ {#5} \hfill function} 

\def\predicatezero#1{\noindent ({\tt {#1}}) \hfill predicate}
\def\predicateone#1#2{\noindent ({\tt {#1}} {\it [{#2}]}) \hfill predicate}
\def\predicatetwo#1#2#3{\noindent ({\tt {#1}} {\it [{#2}]}{\it [{#3}]}) \hfill
predicate}
\def\predicatethree#1#2#3#4{\noindent ({\tt {#1}} {\it [{#2}]}{\it [{#3}]}{\it
[{#4}]}) \hfill predicate}

\def\sysconst#1{\noindent{\tt {#1}} \hfill system constant}

\def\chan {{\it channel\/}}
\def\chr {{\it chr\/}}
\def\chrlist {{\it chr-list\/}}
\def\debmode {\%DE\-BUG\--MO\-DE\%}
\def\dev {{\it device\/}}
\def\dir {{\it direction\/}}
\def\filena {{\it filena\/}}
\def\flconst {first level constructor}
\def\flrec {first level recognizer}
\def\form {{\it form\/}}
\def\formlist {{\it form list\/}}
\def\funcdesc {{\it function description\/}}
\def\infmode {\%IN\-FOR\-MA\-TION-MO\-DE\%}
\def\inte {{\it int\/}}
\def\LIST {{\it list\/}}
\def\LISTone {{\it list\/$_1$\/}}
\def\LISTtwo {{\it list\/$_2$\/}}
\def\obj {{\it obj\/}}
\def\objn {{\it obj\/$_n$\/}}
\def\objone {{\it obj\/$_1$\/}}
\def\objtwo {{\it obj\/$_2$\/}}
\def\pair {{\it pair\/}}
\def\sbral {{\it argument list\/}}
\def\sbrbo {{\it body\/}}
\def\sbrna {{\it subroutine name\/}}
\def\str {{\it str\/}}
\def\sym {{\it sym\/}}
\def\symone {{\it sym\/$_1$\/}}
\def\symtwo {{\it sym\/$_2$\/}}
\def\value {{\it value\/}}

% ................................. document ..................................
% .............................................................................

% \makeindex
\begin{document}

\title{HGKM User Manual}
\author{{\bf Fausto Giunchiglia} and {\bf Alessandro Cimatti}}
\date{4 March 1994}
\version{2.1}
\abstract{
	This manual describes the implementation of a simple programming language,
	{\HG}. 
	{\HG} was originally defined and developed by Richard Weyhrauch and Caroline
	Talcott at the Department of Computer Science of Stanford University.
	The version described here sticks as much as possible the conventions and the
	specifications of the original version.
	The implementation is almost totally new and some new functionalities (e.g.
	{\tt DEFSUB}) have been added.
	Theoretical {\HG} was defined so that its semantics could be easily
	represented in first order logic.
	The version of {\HG} described in this manual acts as the implementation
	language for {\GF}.
	This manual has been written only to facilitate the
	implementation/modification of {\GF}.
}
\addresses{
   \begin{tabular}[c]{l@{\hskip 2em}l}
       {\bf Fausto Giunchiglia}:	    & {\bf Alessandro Cimatti}           \\
       Mechanized Reasoning Group       & Mechanized Reasoning Group,        \\
       IRST, Povo, 38050 Trento, Italy  & IRST, Povo, 38050 Trento, Italy    \\
       e-mail: {\tt fausto@irst.it}     & e-mail: {\tt cx@irst.it}           \\
       phone: +39 461 314359            & phone: +39 461 314351
   \end{tabular}
}
\published{
	\begin{tabular}{l}
		DIST Technical Report No. 9107-05 (1991). \\
		DIST -- University of Genoa,\\
		Via Opera Pia 11A, 16145 Genova, Italy.
    \end{tabular}
}

% ................................... cover ...................................

\thispagestyle{empty}
\maketitle

% ............................. table of contents ...............................

\newpage
\pagenumbering{roman}
\tableofcontents
\newpage
\pagenumbering{arabic}


\section{Introduction}

This manual describes an enhanced version of {\HG}, which has been revised by
the authors within the {\GF} project\cite{giunchiglia34,giunchiglia12,giunchiglia29}.
In effect, the most recent version of {\GF} has been implemented on this
version of {\HG}.

{\HG} is a programming language which has been implemented on top of Common
LISP; it is powerful and simple to use. {\HG} has been developed mainly for
three reasons: to increase {\GF}'s portability, to improve {\GF}'s performance
and to introduce a simple mechanism of data typing.

A language we will call {\it basic} {\HG} was developed including only a small
set of data types and a few built in primitives. This language is enough to
provide the full expressive power of {\HG}.

However, in order to facilitate the development of {\GF}, more data types and
primitives have been defined.
These additions form {\it extended} {\HG}.

This manual is not a tutorial.
It is a reference manual for people already familiar with LISP and
s-expressions notation.


\subsection{Basic concepts}

A {\HG} object belongs to a data type. For each data type three classes of
primitives are defined:

\begin{description}
    \item[predicates:]
        test if a condition is verified or not; predicates are forms which can
        be used as test for branching in conditional forms: that is, evaluating
		a predicate returns a truth value.
    \item[functions:]
        specify a correspondence between {\HG} data and returns an {\HG} object;
		the data type of returned object is specified too; a function produces
		no side effects.
    \item[actions:]
        specify a side effect, and do not return an {\HG} object.
\end{description}
%
\noindent
Furthermore for each data type we define some
%
\begin{description}
    \item[system constants:]
        symbols to which corresponds a fixed value
\end{description}


\subsection{Remarks about notation}

The syntax of primitives and system constants will be described using the
following notation:

\begin{description}
	\item \sysconst{<constant name>}
	\item \predicatethree{<predicate name>}{arg type}{arg type}{\ldots}
	\item \functionthree{<function name>}{arg type}{arg type}{\ldots}{res type}
	\item \actionthree{<action name>}{arg type}{arg type}{\ldots}
\end{description}

\begin{quote}\bf
	After the syntactic description we have a section in which the semantics are
	described.
	This section appears narrower than the rest of the text, in this way.
\end{quote}

The name of the primitive is printed in {\tt typewriter} style.
Its arguments are language forms which will be evaluated.
With {\it [arg type]} we mean the data type of the {\HG} object that the
evaluation of the argument form should return.
If the result of the evaluation does not match the specified data type a fatal
error is generated.

For instance, \inte\ denotes a form which evaluates to an integer.
If an operation requires more than one argument of a single type, the bracketed
type name will have a numbered subscript.
For example we write {\it str\/$_1$} and {\it str\/$_2$} to denote two different
forms that have strings as values.
In the case an operation does not care about the type of its arguments we write
\obj\ to indicate a form which can evaluate to any object in the {\HG} world.

In section 2 we present basic {\HG}; in section 3 extended {\HG}; finally, in
appendix we describe some aspects related to the implementation.


\newpage
\section{Basic {\HG}}

In this section we shall describe basic {\HG}: we present the structured data
types, the way in which the language forms are evaluated by the interpreter,
some special forms for subroutine definitions, the error handling and the
input/output primitives.


\subsection{Data structures}

Basic {\HG} data types are:

\begin{itemize}
	\item truth values {\it [trth];}
	\item symbols {\it [\sym];}
	\item integers {\it [\inte];}
	\item characters {\it [\chr];}
	\item strings {\it [\str];}
	\item pairs {\it [\pair];}
	\item lists {\it [list].}
\end{itemize}

In this chapter we shall refer to each data type by the corresponding name
enclosed in square brackets.

The syntax of data types is defined by this grammar:

\begin{bnf}
	\T{obj}			\sep  \T{atomic obj} | \T{compound obj} \\
	\T{atomic obj}	\sep  \T{trth} | \T{sym} | \T{int} | \T{chr} | \T{str} \\
	\T{compound obj}\sep  \T{pair} | \T{list} | \T{chr-list}\\
	\T{pair}	    \sep  ( \T{obj} . \T{obj}) \\
	\T{list}		\sep  empty-list | ( \T{obj} . \T{list} ) \\
	\T{chr-list}	\sep  empty-list | ( \T{chr} . \T{chr-list} )
\end{bnf}

In this grammar we refer explicitly to a data structure, the list of characters
{\it [chr-list],} which is not a primitive data type.
However it is very used in {\HG}, and it is interesting to define its syntax too.
Note that a list of characters is different from a string: in basic {\HG} a
string is an atomic object, while a list of characters has a compound structure.


\subsubsection{Truth values and equality predicate}

\sysconst {TRUE}\index{TRUE}
\begin{quote}
	{\tt TRUE} is a system constant which always evaluates to true.
\end{quote}

\sysconst {FALSE}\index{FALSE}
\begin{quote}
	{\tt FALSE} is a system constant which always evaluates to false.
\end{quote}

\predicatetwo{EQ}{\objone}{\objtwo}\index{EQ}
\begin{quote}
	{\tt EQ} is the equality predicate, and is defined for all data types.
	{\tt EQ} evaluates to true if both \objone\ and \objtwo\ are the same
	object, false otherwise.
\end{quote}


\subsubsection{Symbols}

{\HG} symbols are like Common LISP identifiers, with the only difference that
they are case sensitive.
{\tt abc}, {\tt CAR}, {\tt NIL}, {\tt THISISALONGATOM} are examples of {\HG}
symbols.

\bigskip

\predicateone{SYMBOL?}{\obj}\index{SYMBOL?}
\begin{quote}
	{\tt SYMBOL?} evaluates to true if \obj\ is a symbol, false otherwise.
\end{quote}

\actiontwo{SETVAL}{\sym}{\obj}\index{SETVAL}
\begin{quote}
	{\tt SETVAL} changes the value associated with the specified symbol to \obj.
\end{quote}

\functionone{GETVAL}{\sym}{\obj}\index{GETVAL}
\begin{quote}
	{\tt GETVAL} returns the value associated with \sym.
\end{quote}

\actionthree{SETPROP}{\symone}{\obj}{\symtwo}\index{SETPROP}
\begin{quote}
	{\tt SETPROP} sets the property of \symone\ with key \symtwo\ to have the
	value \obj.
\end{quote}

\functiontwo{GETPROP}{\symone}{\symtwo}{\obj}\index{GETPROP}
\begin{quote}
	{\tt GETPROP} returns the value of the property \symtwo\ of \symone.
	If there is no such property {\tt GETPROP} returns the false truth value.
\end{quote}
 
\functiontwo{REMPROP}{\symone}{\symtwo}{\obj}\index{REMPROP}
\begin{quote}
	{\tt REMPROP} removes from \symone\ the property with an indicator {\tt EQ}
	to \symtwo. 
	It returns the false truth value if there is no such indicator.
\end{quote}


\subsubsection{Integers}

Integers are the only type of numbers allowed in {\HG}.
\bigskip

\predicateone {INTEGER?}{\obj}\index{INTEGER?}
\begin{quote}
	{\tt INTEGER?} evaluates to true if \obj\ is an integer, false otherwise.
\end{quote}

\predicateone {ZERO?}{\obj}\index{ZERO?}
\begin{quote}
	{\tt ZERO?} evaluates  to true if \obj\ is the integer $0$, false otherwise.
\end{quote}

\predicateone {POSITIVE?}{\obj}\index{POSITIVE?}
\begin{quote}
	{\tt POSITIVE?} evaluates to true if \obj\ is a positive integer, false
	otherwise.
\end{quote}

\predicateone {NEGATIVE?}{\obj}\index{NEGATIVE?}
\begin{quote}
	{\tt NEGATIVE?} evaluates to true if \obj\ is a negative integer, false
	otherwise.
\end{quote}

\functionone {ADD1}{\inte}{\inte}\index{ADD1}
\begin{quote}
	{\tt ADD1} adds one to an integer.
\end{quote}

\functionone {SUB1}{\inte}{\inte}\index{SUB1}
\begin{quote}
	{\tt SUB1} subtracts one from an integer.
\end{quote}


\subsubsection{Characters}

{\HG} characters are all the characters to which is associated an ASCII code.
An {\HG} character is represented by the integer between $0$ and $255$
corresponding to its {\tt ASCII} code.

\bigskip

\predicateone {CHARACTER?}{\obj}\index{CHARACTER}
\begin{quote}
	{\tt CHARACTER?} evaluates  to true if \obj\ is a character, false otherwise.
\end{quote}

\functionone {CHR2SYM}{\chrlist}{\sym}\index{CHR2SYM}
\begin{quote}
	{\tt CHR2SYM} returns the symbol whose print name consists of the characters
	in \chrlist.

	For instance:
	%
	\begin{verbatim}
		> (CHR2SYM '(65 80 80 76 69))
		APPLE
	\end{verbatim}
\end{quote}

\functionone {SYM2CHR}{\sym}{\chrlist}\index{SYM2CHR}
\begin{quote}
	{\tt SYM2CHR} is the logical inverse of {\tt CHR2SYM}: it returns the list
	of the characters which compound the symbol.

	For instance:
	%
	\begin{verbatim}
		> (SYM2CHR 'APPLE)
		(65 80 80 76 69)
	\end{verbatim}
\end{quote}

\functionone {CHR2INT}{\chrlist}{\inte}\index{CHR2INT}
\begin{quote}
	{\tt CHR2INT} takes a list of characters which represent digits and returns
	the corresponding integer.
	If the list begins with a minus sign ``-'' then {\tt CHR2INT} returns the
	appropriate negative number.

	For instance:
	%
	\begin{verbatim}
		> (CHR2INT '(49 51 55)
		137
		> (CHR2INT '(45 49 51 55))
		-137
	\end{verbatim}
\end{quote}

\functionone {\tt INT2CHR}{\inte}{\chrlist}\index{INT2CHR}
\begin{quote}
	{\tt INT2CHR} is the logical inverse of {\tt CHR2INT}.
	{\tt INT2CHR} returns the list of characters corresponding to the digits of
	the integer.
	If the integer is negative then the first element of the list will contain
	the number $45$, which is the {\tt ASCII} code for the minus sign ``-''.

	Example:
	%
	\begin{verbatim}
		> (INT2CHR 137)
		(49 51 55)}
		> (INT2CHR -137)
		(45 49 51 55)
	\end{verbatim}
\end{quote}


\subsubsection{Strings}

A string is an arbitrary sequence of {\it n\/} characters, indexed from $1$ to
{\it n\/}.
A string is delimited by a pair of double quotes ({\tt "}) which do not belong
to it. {\HG} strings are case sensitive.
{\tt "Symbol"}, {\tt "number"}, {\tt "ThisIsAString"} are examples of {\HG}
strings.

\bigskip

\predicateone {STRING?}{\obj}\index{STRING?}
\begin{quote}
	{\tt STRING?} evaluates to true if \obj\ is a string, false otherwise.
\end{quote}

\functionone {LENSTR}{\str}{\inte}\index{LENSTR}
\begin{quote}
	{\tt LENSTR} returns an integer \inte\ equal to the number of characters in
	\str.
\end{quote}

\actionthree {SETSTR}{\str}{\inte}{\chr}\index{SETSTR}
\begin{quote}
	{\tt SETSTR} sets to \chr\ the \inte-th position of \str.
\end{quote}

\functiontwo {GETSTR}{\str}{\inte}{\chr}\index{GETSTR}
\begin{quote}
	{\tt GETSTR} returns the \inte-th character of \str.
\end{quote}

\functionone {\tt STR2CHR}{\str}{\chrlist}\index{STR2CHR}
\begin{quote}
	{\tt STR2CHR} returns a list of characters which appear in the string.
\end{quote}

\functionone {CHR2STR}{\chrlist}{\str}\index{CHR2STR}
\begin{quote}
	{\tt CHR2STR} returns a string of the length of \chrlist\ that contains the
	characters that appear in \chrlist.

	Example:
	%
	\begin{verbatim}
		> (STR2CHR "APPLE")
		(65 80 80 76 69)
		> (CHR2STR '(104 111 109 101)
		"home"
	\end{verbatim}
\end{quote}


\subsubsection{Pairs}

An {\HG} pair is the ordered collection of two {\HG} objects.

\bigskip


\predicateone{PAIR?}{\obj}\index{PAIR?}
\begin{quote}
	{\tt PAIR?} evaluates to true if \obj\ is a pair, false otherwise.
\end{quote}

\predicateone{ATOMIC?}{\obj}\index{ATOMIC?}
\begin{quote}
	{\tt ATOMIC?} evaluates to true if \obj\ is an atomic object, namely
	a symbol, an integer or a string, false otherwise.
\end{quote}

\functiontwo {CONS}{\objone}{\objtwo}{\pair}\index{CONS}
\begin{quote}
	{\tt CONS} returns a pair consisting of \objone\ and \objtwo.
\end{quote}

\functionone {CAR}{\pair}{\obj}\index{CAR}
\begin{quote}
	{\tt CAR} returns the first element of \pair.
\end{quote}

\functionone {CDR}{\pair}{\obj}\index{CDR}
\begin{quote}
	{\tt CDR} returns the second element of \pair.
\end{quote}

\actiontwo {SETCAR}{\pair}{\obj}\index{SETCAR}
\begin{quote}
	{\tt SETCAR} changes the first element of \pair\ to \obj.
\end{quote}

\actiontwo {SETCDR}{\pair}{\obj}\index{SETCDR}
\begin{quote}
	{\tt SETCDR} changes the second element of \pair\ to \obj.
\end{quote}


\subsubsection{Lists}

A list is either the empty list or a pair whose second element is a list.

\bigskip

\sysconst {MTL}\index{MTL}
\begin{quote}
	{\tt MTL} evaluates to the empty list.
\end{quote}

\predicateone {MTL?}{\obj}\index{MTL?}
\begin{quote}
	{\tt MTL?} evaluates to true if \obj\ is the empty list, false otherwise.
\end{quote}

\predicateone {LIST?}{\obj}\index{LIST?}
\begin{quote}
	{\tt LIST?} evaluates to true if \obj\ is a list, false otherwise.
\end{quote}

\functiontwo {LCONS}{\obj}{\LIST}{\LIST}\index{LCONS}
\begin{quote}
	{\tt LCONS} returns a list with \obj\ in the front of \LIST.
\end{quote}

\functionone {FIRST}{\LIST}{\obj}\index{FIRST}
\begin{quote}
	{\tt FIRST} returns the first element of \LIST.
\end{quote}

\functionone {REST}{\LIST}{\LIST}\index{REST}
\begin{quote}
	{\tt REST} returns the rest of \LIST.
\end{quote}

\actiontwo {SETFIRST}{\LIST}{\obj}\index{SETFIRST}
\begin{quote}
	{\tt SETFIRST} changes the first element of \LIST\ to \obj.
\end{quote}

\actiontwo {SETREST}{\LISTone}{\LISTtwo}\index{SETREST}
\begin{quote}
	{\tt SETREST} changes the rest of \LISTone\ to \LISTtwo.
\end{quote}


\subsection{Evaluation of forms}

This section describes how a form is evaluated by {\HG} interpreter.
The different types of forms are:

\begin{itemize}
	\item null operation {\tt <nop>}
	\item constant {\tt <const>}
	\item variable {\tt <var>}
	\item quotation {\tt <quote>}
	\item subroutine application {\tt <appl>}
	\item conditional evaluation {\tt <if>}
	\item operation sequence {\tt <seq>}
	\item local environment definition {\tt <let>}
	\item environment modification {\tt <setq>}
	\item program {\tt <pgm>}
\end{itemize}
%
In the rest of the chapter we shall refer to the different types of forms to be
evaluated by the corresponding names which are enclosed between angle brackets.
The syntax of the language forms is defined by the following grammar.

\begin{bnf}
	\T{form}           \sep \T{atomic form} | \T{compound form} \\
	\T{atomic form}	   \sep \T{nop} | \T{const} | \T{var} \\
	\T{compound form}  \sep \T{quote} | \T{appl} | \T{if} | \T{let} | \T{setq} |
							\T{seq} | \T{pgm} \\
	\T{nop}			   \sep ( NOP) \\
	\T{quote}		   \sep ( QUOTE \T{form}) \\
	\T{appl}		   \sep (\T{subroutine name} \{ \T{form} \})\\
	\T{if}			   \sep ( IF | IFN \T{predform} \T{form1} \T{form2}) \\
	\T{let}			   \sep ( LET ( \{ ( \T{var} \T{form} ) \} ) \{ \T{form} \})
							\\ 
	\T{seq}			   \sep ( SEQ \{ \T{form} \} )\\
	\T{pgm}			   \sep ( PGM ( \{ \T{var} \} ) \{ \T{pgm-stm} \} )\\
	\T{pgm-stm}		   \sep \T{form} | \T{rtn} | \T{jump} | \T{label}\\
	\T{rtn}			   \sep ( RETURN \T{form} ) \\
	\T{jump}		   \sep ( BRT | BRF \T{label} \T{predform} ) | ( GOTO
							\T{label} )\\
	\T{label}		   \sep \T{symbol}
\end{bnf}
%
Some explanation must be supplied: by constant ({\tt <const>}) we mean either
a truth value, an integer, a character or a string; by predicative form
({\tt <predform>}) we mean a particular form that evaluates to a truth value;
by program statements ({\tt <pgm-stm>}) we mean the special constructs that
compose a {\tt <pgm>} form statement list.
In the rest of this section we shall describe the semantics of each form,
referring to the grammar which has been presented.

\bigskip

{\bf Null operation}\index{NOP}
\begin{quote}
	The evaluation of the {\tt <nop>} form does nothing and does not return a
	value.
\end{quote}

{\bf Constant}
\begin{quote}
	The evaluation of a constant form (truth value, integer, character or
	string) returns the constant itself. Hence {\HG} constants are invariant
	with respect to the evaluation.
\end{quote}

{\bf Variable}
\begin{quote}
	A {\tt <var>} form is a symbol. This symbol must be the name of an
	initialized variable.
	In this case the result of the evaluation of the form is the value of the
	variable in the current environment; otherwise a fatal error occurs.
\end{quote}

{\bf Quotation}\index{QUOTE}
\begin{quote}
	The evaluation of a {\tt <quote>} form returns the argument form without
	evaluating it.
\end{quote}

{\bf Subroutine application}
\begin{quote}
	{\tt <subroutine name>} must be a symbol which is the name of an {\HG}
	function or predicate.
	It can be the name of a built-in primitive or a user-defined function.
	The argument {\tt <form>}s are evaluated and bound to the corresponding
	formal parameters of the function or predicate (call-by-value).
	The result of the application is the value obtained applying the specified
	subroutine to the arguments.
	The type of the values deriving from arguments must match properly the data
	type of the formal parameters, otherwise a fatal error occurs.

	Examples of subroutine applications are:
	%
	\begin{verbatim}
		> (CONS 1 13)
		(1 . 13)
		> (MTL? MTL)
		T
	\end{verbatim}
\end{quote}

{\bf Conditional expressions}\index{IF}\index{IFN}
\begin{quote}
	If the {\tt <predform>} evaluates to true, then the {\tt IF} conditional
	expression returns the evaluation of the first {\tt <form>}, otherwise of
	the second one.
	{\tt IFN} stands for {\tt IF NOT}; its semantics are trivial.

	For instance:
	%
	\begin{verbatim}
		> (IF (MTL? X) FALSE TRUE)
	\end{verbatim}
	%
	evaluates to false if {\tt X} evaluates to the empty list, true otherwise.
\end{quote}

{\bf Sequences of forms: \verb+<seq>+}\index{SEQ}
\begin{quote}
	The value of a {\tt SEQ} form is the value of its last form in the
	environment defined by the evaluation of previous forms.
\end{quote}

{\bf Local environment definition}\index{LET}
\begin{quote}
	{\tt LET} is used to create and initialize local variables.
	The result of evaluating a {\tt LET} is given by the evaluation of the last
	{\tt <form>} in a new environment in which each {\tt <var>} is associated
	with the value of the respective {\tt <form>}.
	If one of the initializing {\tt <form>}s contains any reference to any of
	the {\tt <var>}s, its value is taken from the lexically exterior
	environment.
	Hence the semantics of a {\tt <let>} form is the same as:
	%
	\begin{verbatim}
		(SEQ
		  (SETQ TEMP-1 <form-1>)
		  ...
		  (SETQ TEMP-n <form-n>)
		  (SETQ var-1 TEMP-1)
		  ...
		  (SETQ var-n TEMP-n)
		  <form>)
	\end{verbatim}
	%
	where {\tt TEMP-1} through {\tt TEMP-n} are symbols which do not appear in
	any {\tt <form-i>} or in the last {\tt <form>}.
\end{quote}

{\bf Environment modification}\index{SETQ}
\begin{quote}
	The evaluation of a {\tt <setq>} form produces a side effect on
	environments, and does not return a value.
	The evaluation of a {\tt <setq>} form sets the value associated with the
	variable {\tt <var>} to the evaluation of {\tt <form>}.
	If {\tt <var>} is a global variable the form {\tt (SETQ <var> <form>)} is
	equivalent to the form {\tt (SETVAL (QUOTE <var>) <form>)}.

	For instance:
	%
	\begin{verbatim}
		> (SETVAL (QUOTE X) 1)
		1
		> (LET ((X 5)) (SEQ (SETQ X 3) X)))
		3
	\end{verbatim}
	%
	evaluates to $3$ and leaves the value of {\tt X} to be $1$.
	%
	\begin{verbatim}
		(SEQ
		  (SETVAL (QUOTE X) 1)
		  (LET ((X 5)) (SEQ (SETVAL (QUOTE X) 3) X)))
	\end{verbatim}
	%
	evaluates to $5$ and leaves the value of {\tt X} to be $3$.	
\end{quote}

{\bf Program}\index{PGM}
\begin{quote}
	The {\tt <pgm>} form provides a facility for executing sequential programs,
	something like {\tt PROG}s in Common Lisp.
	The default order of executing {\tt <pgm-stm>}s is the sequential order.
	This can be modified by the use of {\tt <jump>} or of {\tt <rtn>}
	instructions.
	The {\tt <var>}s are local to the {\tt <pgm>} form and they have no initial
	value. 
\end{quote}

{\bf Jumps}\index{BRT}\index{BRF}\index{GOTO}
\begin{quote}
	The {\tt <jump>} statements can be either conditional ({\tt BRT} and
	{\tt BRF}) or unconditional ({\tt GOTO}).
	{\tt BRT} branches to the {\tt <label>} if the {\tt <predform>} evaluates to
	true.
	{\tt BRF} branches to the {\tt <label>} if the {\tt <predform>} evaluates to
	false.
	{\tt GOTO} branches to {\tt <label>}.
\end{quote}

{\bf Labels}
\begin{quote}
	{\tt <label>}s are statements which are simply symbols.
	They are not evaluated but provide markers for the {\tt <jump>} statements.
\end{quote}

{\bf Return}\index{RETURN}
\begin{quote}
	At the evaluation of a {\tt <rtn>} the {\tt <pgm>} statements evaluation is
	halted and the value of {\tt <form>} is returned.
	As shown by the grammar, a {\tt <rtn>} statement can only be used as
	statement into a {\tt <pgm>} form.
\end{quote}


\subsubsection{Eval and apply}

In this section we describe two special form for evaluating forms and applying
functions to forms.

\bigskip

\functionone {HGKM-EVAL}{\form}{\obj}\index{HGKM-EVAL}
\begin{quote}
	{\tt HGKM-EVAL} evaluates \form\ in the current environment and returns the
	resulting \obj.
\end{quote}

\functiontwo {HGKM-APPLY}{\funcdesc}{\formlist}{\value}\index{HGKM-APPLY}
\begin{quote}
	{\tt HGKM-APPLY} binds each element of the form list (without evaluating it)
	to the corresponding element of the bound variables list of the function
	description, and returns the value of evaluating the body of the function
	description in this new environment.

	Example:
	%
	\begin{verbatim}
		> (HGKM-APPLY (GET-FUN 'APPEND) '((A B) (C D)))
		(A B C D)
		> (HGKM-APPLY (GET-FUN 'APPEND) '(LIST '(B) '(C D)))
		(B C D)
	\end{verbatim}
\end{quote}


\subsection{Special forms}

In {\HG} special forms to define functions and macros are provided.
We call this special forms \flconst s, to distinguish them from Common LISP
constructors (level zero constructors) which have been used to define built-in
forms. 
In this section we also present {\HG} special forms to inspect and describe
built-in and user-defined subroutines: we shall refer to these special forms
using the term "\flrec s".


\subsubsection{First level constructors}

The \flconst s are

\begin{itemize}
	\item {\tt DEFLAM}
	\item {\tt DEFMAC}
	\item {\tt DEFSYS}
	\item {\tt DEFMACSYS}
	\item {\tt DEFSUB}
	\item {\tt SET-FUN}
\end{itemize}

A \flconst\ call has the usual form:

\begin{quote}\tt
	(<constructor name> {\it [\sbrna] [\sbral] [\sbrbo]})
\end{quote}

The result of a such call is the definition of a subroutine named {\it \sbrna},
in which {\it \sbral\/} is the list of formal parameters and {\it \sbrbo\/} is
the statement list.
The syntax of user defined {\HG} subroutines is the same as Common LISP
subroutines: hence are allowed all CL options, namely {\tt \&REST, \&OPTIONS,}
etc.

The way in which a subroutine is implemented depends on which is the \flconst
used in defining it, and on the value of two installation variables at the
moment of definition.
Anyway the implementation is almost transparent to the user, and so here we give
only a brief description of implementation features.

We shall explain now the meaning of an installation variable, which is named
{\tt \infmode}: the main idea is that when this variable evaluates to true some
system information is set up at the moment of installation, while when it
evaluates to false no information is available.
The following are special forms which {\it do not} evaluate their arguments.

\bigskip

\actionthree {DEFLAM}{\sbrna}{\sbral}{\sbrbo}\index{DEFLAM}
\begin{quote}
	{\tt DEFLAM} is a function constructor; which always allows the redefinition
	of previously defined functions.
	If {\tt\infmode} evaluates to false, then {\tt DEFLAM} just defines the
	function.
	If {\tt\infmode} evaluates to true, {\tt DEFLAM} also sets a property
	telling that {\it \sbrna\/} is a first level defined function and, in case
	of redefinition, prints a warning message.
\end{quote}

\actionthree {DEFMAC}{\sbrna}{\sbral}{\sbrbo}\index{DEFMAC}
\begin{quote}
	{\tt DEFMAC} is a macro definer: hence {\it \sbrbo\/} must be a macro
	statement list, that is must be ``expansible'' like Common LISP macros.
	{\tt DEFMAC} always allows to redefine already defined macros.
	If {\tt\infmode} evaluates to false then {\tt DEFMAC} just defines the
	macro; else {\tt DEFMAC} also sets a property telling that {\it \sbrna} is a
	first level defined macro and, in case of redefinition, prints a warning
	message. 
\end{quote}

\actionthree {\tt DEFSYS}{\sbrna}{\sbral}{\sbrbo}\index{DEFSYS}
\begin{quote}
	{\tt DEFSYS} is a function definer which allows to redefine already defined
	functions only when {\tt\infmode} evaluates to false; when {\tt\infmode} is
	true and {\it \sbrna} has already been defined, {\tt DEFSYS} will signal
	that it can not execute the redefinition.
	With respect to functions which have not been defined yet {\tt DEFSYS} is
	just like {\tt DEFLAM}.
\end{quote}

\actionthree {DEFMACSYS}{\sbrna}{\sbral}{\sbrbo}\index{DEFMACSYS}
\begin{quote}
	{\tt DEFMACSYS} is a macro definer which allows to redefine already defined
	macros only when {\tt\infmode} evaluates to false; when {\tt\infmode} is
	true and {\it \sbrna} has already been defined, {\tt DEFMACSYS} will signal
	that it can not execute the redefinition.
	With respect to macros which have not been defined yet {\tt DEFMACSYS} is
	just like {\tt DEFMAC}. 
\end{quote}

\actionthree {\tt SET-FUN}{\sbrna}{\sbral}{\sbrbo}\index{SET-FUN}
\begin{quote}
	{\tt SET-FUN} differs from {\tt DEFLAM} just because it evaluates its
	arguments,while {\tt DEFLAM} does not.
\end{quote}

\actionthree {DEFSUB}{\sbrna}{\sbral}{\sbrbo}\index{DEFSUB}
\begin{quote}
	{\tt DEFSUB} can work as {\tt DEFLAM} or as {\tt DEFMAC}.
	This subroutine constructor is included for efficiency reasons rather than
	theoretical reasons; hence its description is delayed to appendix A.
\end{quote}


\subsubsection{First level recognizers}

\predicateone {FUNCTION?}{\obj}\index{FUNCTION?}
\begin{quote}
	{\tt FUNCTION?} is a function recognizer, which evaluates to true if \obj\
	is the name of an {\HG} function, false otherwise. 
	{\tt FUNCTION?} can not tell a built-in function from a \flconst\ defined
	function, or a compiled function from an interpreted one.
	{\tt FUNCTION?} returns the correct information only if {\tt\infmode} is
	true; else it prints a warning message to signal that an un-allowable
	information has been required and executes an {\tt HGKM-ABORT} instruction: hence
	every {\tt FUNCTION?} call must be ``surrounded'' by a {\tt TRAP}
	call\footnote{Further information about {\tt TRAP} and {\tt HGKM-ABORT} can be
	found in ``Error handling'' section}.
\end{quote}

\predicateone {MACRO?}{\obj}\index{MACRO?}
\begin{quote}
	{\tt MACRO?} is a macro recognizer, which evaluates to true if \obj\ is the
	name of an {\HG} macro, false otherwise.
	Note that {\tt MACRO?} can not tell a system macro from a user defined
	macro.
	Furthermore, {\tt MACRO?} has the same behavior as {\tt FUNCTION?} with
	respect to {\tt\infmode}.
	Finally we want to mention a small problem: both {\tt MACRO?} and
	{\tt FUNCTION?} evaluate to true in case of built-in subroutines, because
	there is no information telling if a such subroutine has been defined as
	macro or as function.
\end{quote}

\functionone {GET-FUN}{\obj}{\funcdesc}\index{GET-FUN}
\begin{quote}
	{\tt GET-FUN} returns a function description, in one of the following forms:
	%
	\begin{itemize}
		\item
			{\tt (HGKM <function name> COMPILED)} if the \newline function is
			compiled; 
		\item
			{\tt (HOST <function name> <bound variables list> <body>)} if the
			\newline function is an un-compiled system function;
		\item
			{\tt (HGKM <function name> <bound variables list> <body>)} if the
			\newline function is an uncompiled \flconst\ defined function.
	\end{itemize}
\end{quote}


\subsection{Error handling}

\functionone {TRAP}{\form}{\value}\index{TRAP}
\begin{quote}
	{\tt TRAP} evaluates \form\ and if no error occurs during its evaluation
	{\tt TRAP} returns the same value as the form ({\tt LIST} \form).
	If a system error occurs during an evaluation then {\tt TRAP} returns an
	atomic object that can be tested for determining what the error was.
	If the function {\tt HGKM-ABORT} is called during evaluation of \form\ then {\tt
	TRAP} will return the value of the argument of {\tt HGKM-ABORT}.
\end{quote}

\actionone {HGKM-ABORT}{\form}\index{HGKM-ABORT}
\begin{quote}
	{\tt HGKM-ABORT} returns the value of \form\ to the most recent {\tt TRAP}; every
	{\tt HGKM-ABORT} call must be lexically surrounded at least by a {\tt TRAP}
	function call.
\end{quote}

\predicateone{ERROR-OCCURRED?}{\value}\index{ERROR-OCCURRED?}
\begin{quote}
	{\tt ERROR-OCCURRED?} checks whether a {\tt TRAP}ped \form\ {\tt HGKM-ABORT}ed.
\end{quote}

\functionone{TRAP-GET-RESULT}{\value}{\obj}\index{TRAP-GET-RESULT}
\begin{quote}
	{\tt TRAP-GET-RESULT} selects from \value\ the result of the evaluation of
	\form.
\end{quote}


\subsection{Input-output}

In this section {\HG} input-output handling primitives are presented.
{\HG} uses unidirectional channels implemented on Common LISP streams.
A channel is an object representing in {\HG} environment an external object,
i.e.\ a file or the keyboard.
A channel needs to be open to become active.
The system keeps track of open channels, storing information about each active
channel in a global data structure, the channel list.
Every channel list entry contains
%
\begin{itemize}
	\item
		the channel itself
	\item
		the data flow direction ({\tt IN} or {\tt OUT})
	\item
		the type of objects that can be communicated through the channel, namely
		s-expressions ({\tt SEXP}) or characters ({\tt CHAR}).
	\item
		the information on the object bound to the channel (the device on which
		is stored the physical file in case of file I/O, the keyboard in case of
		console I/O, etc).
\end{itemize}


\subsubsection{Opening and closing channels}

\functionone {HGKM-ISOPEN}{\filena}{\chan}\index{HGKM-ISOPEN}
\begin{quote}
	\filena\ is a form which evaluates to a string.
	{\tt HGKM-ISOPEN} creates and returns a channel for s-expressions
	corresponding to the file \filena, input direction.
	When trying to open a channel corresponding to a non existing file, {\tt
	HGKM-OPEN} executes an {\tt HGKM-ABORT} instruction, to signal that an illegal
	operation has been required: hence {\tt HGKM-OPEN} must be {\tt TRAP}-ped.
\end{quote}

\functionone {HGKM-ICOPEN}{\filena}{\chan}\index{HGKM-ICOPEN}
\begin{quote}
	\filena\ is a form which evaluates to a string.
	{\tt HGKM-ICOPEN} creates and returns a channel for characters corresponding
	to the file \filena, input direction.
	When trying to open a channel corresponding to a non existing file, {\tt
	HGKM-OPEN} executes an {\tt HGKM-ABORT} instruction, to signal that an illegal
	operation has been required: hence {\tt HGKM-OPEN} must be {\tt TRAP}-ped.
\end{quote}

\functionzero {TTY-IOPEN}{\chan}\index{TTY-IOPEN}
\begin{quote}
	{\tt TTY-IOPEN} returns a channel for console input.
\end{quote}

\functionone {HGKM-OSOPEN}{\filena}{\chan}\index{HGKM-OSOPEN}
\begin{quote}
	\filena\ is a form which evaluates to a string.
	{\tt HGKM-OSOPEN} creates and returns a channel for s-expressions
	corresponding to the file \filena, output direction.
\end{quote}

\functionone {HGKM-OCOPEN}{\filena}{\chan}\index{HGKM-OCOPEN}
\begin{quote}
	\filena\ is a form which evaluates to a string.
	{\tt HGKM-OCOPEN} creates and returns a channel for characters corresponding
	to the file \filena, output direction.
\end{quote}

\functionzero {TTY-OOPEN}{\chan}\index{TTY-OOPEN}
\begin{quote}
	{\tt TTY-OOPEN} returns a channel for console output.
\end{quote}

\actionone {HGKM-CLOSE}{\chan}\index{HGKM-CLOSE}
\begin{quote}
	{\tt HGKM-CLOSE} closes \chan\ and updates the channel list taking off the
	information about \chan.
	{\tt HGKM-CLOSE} must be trapped because, when \chan\ does not exist, an
	{\tt HGKM-ABORT} occurs to signal that an illegal operation has been required.
\end{quote}


\subsubsection{Input output specifying channels}

\functionone {CHAN-READ}{\chan}{\obj}\index{CHAN-READ}
\begin{quote}
	{\tt CHAN-READ} reads an s-expression from \chan\ and returns it; to handle
	the case in which the object read is an {\tt EOF} a {\tt TRAP} function call
	is required. 
	\chan\ must have been created by {\tt TTY-IOPEN} or by {\tt HGKM-ISOPEN}.
\end{quote}

\functionone {CHAN-CHI}{\chan}{\chr}\index{CHAN-CHI}
\begin{quote}
	{\tt CHAN-CHI} reads a character from \chan\ and returns it; to handle the
	case in which the character read is an {\tt EOF} a {\tt TRAP} function call
	is required. 
	\chan\ must have been created by {\tt TTY-IOPEN} or by {\tt HGKM-ICOPEN}.
\end{quote}

\actiontwo {CHAN-PRINC}{\obj}{\chan}\index{CHAN-PRINC}
\begin{quote}
	{\tt CHAN-PRINC} writes \obj\ on \chan.
	\chan\ must have been created by {\tt TTY-OOPEN} or by {\tt HGKM-OSOPEN}.
\end{quote}

\actionone {CHAN-CRLF}{\chan}\index{CHAN-CRLF}
\begin{quote}
	{\tt CHAN-CRLF} writes a carriage return on \chan.
	\chan\ must have been created by {\tt TTY-OOPEN} or by {\tt HGKM-OSOPEN}.
\end{quote}

\actiontwo {CHAN-CHO}{\chr}{\chan}\index{CHAN-CHO}
\begin{quote}
	{\tt CHAN-CHO} writes \chr\ on \chan.
	\chan\ must have been created by {\tt TTY-OOPEN} or by {\tt HGKM-OCOPEN}.
\end{quote}

\functionone{CHANNEL-GET-FNAME}{\chan}{\filena}\index{CHANNEL-GET-FNAME}
\begin{quote}
	{\tt CHANNEL-GET-FNAME} returns the name of the file corresponding to the
	channel \chan.
\end{quote}


\subsubsection{Input output with implicit channels}

\actionone {SELECT}{\chan}\index{SELECT}
\begin{quote}
	{\tt SELECT} is the primitive to set the values of default channels.
	If into the channel list the direction associated with \chan\ is {\tt IN},
	then \chan\ will be selected as default input channel, else as default
	output channel.
\end{quote}

\functionzero {HGKM-READ}{\obj}\index{HGKM-READ}
\begin{quote}
	{\tt HGKM-CHI} reads an s-expression from the selected input and returns it.
	to handle the case in which the object read is an {\tt EOF} a {\tt TRAP}
	function call is required.
	The selected input must have been created by {\tt TTY-IOPEN} or by {\tt
	HGKM-ISOPEN}. 
\end{quote}

\functionzero {HGKM-CHI}{\chr}\index{HGKM-CHI}
\begin{quote}
	{\tt HGKM-CHI} reads a character from the selected input and returns it; to
	handle the case in which the character read is an {\tt EOF} a {\tt TRAP}
	function call is required.
	The selected must have been created by {\tt TTY-IOPEN} or by
	{\tt HGKM-ICOPEN}. 
\end{quote}

\actionone {HGKM-PRINC}{\obj}\index{HGKM-PRINC}
\begin{quote}
	{\tt HGKM-PRINC} writes \obj\ on the selected output.
	The selected output must have been created by {\tt TTY-OOPEN} or by
	{\tt HGKM-OSOPEN}.
\end{quote}

\actionone {CRLF}{\chan}\index{CRLF}
\begin{quote}
	{\tt CRLF} writes a carriage return on the selected output.
	The selected output must have been created by {\tt TTY-OOPEN} or by
	{\tt HGKM-OSOPEN}.
\end{quote}

\actionone {HGKM-CHO}{\chan}\index{HGKM-CHO}
\begin{quote}
	{\tt HGKM-CHO} writes \chr\ on the selected output.
	The selected output must have been created by {\tt TTY-OOPEN} or by
	{\tt HGKM-OCOPEN}.
\end{quote}

\actionzero {HGKM-INITIO}\index{HGKM-INITIO}
\begin{quote}
	{\tt HGKM-INITIO} initializes I/O by setting the following I/O-related
	variables:  {\tt CHAN-LIST}, {\tt STANDARD-INPUT}, {\tt STANDARD-OUTPUT},
	{\tt SELECTED-INPUT}, {\tt SELECTED-OUTPUT}.
\end{quote}


\pagebreak
\section{Extended {\HG}}

In this section extended {\HG} is presented.
Extended {\HG} is a collection of primitives which operate on basic {\HG} data
structures or on newly defined ones.

Extended {\HG} is not founded on theoretical considerations like basic {\HG}.
Its semantics can't be axiomatized in first order logic.
Extended {\HG} has been developed mainly for sake of performance and programming
facility, and it has grown up ``on demand'': that is, when it has been realized
that a set of primitives was useful it has been implemented. This is why
extended {\HG} does not look homogeneous.

During this revision, an attempt has been made to keep extended {\HG} as small
as possible: hence, several primitives which are actually part of the source
code are not presented in this manual because they seem to be rather useless.
Further revisions of the source codes will face the problem of removing them.


\subsection{Truth values}

\begin{itemize}
	\item {\tt (NOT \ldots)}\index{NOT}: as in CL.
	\item {\tt (AND \ldots)}\index{AND}: as in CL.
	\item {\tt (OR \ldots)}\index{OR}: as in CL.
	\item {\tt (IST {\it [sym]})}\index{IST}:
	  {\tt IST} and its counterpart {\tt ISF} should theoretically allow GETFOL
	  to use real truth values instead that {\tt T} and {\tt NIL}. {\tt IST}
	  returns {\tt TVAL} when {\it sym} is not {\tt MTL}, {\tt FVAL} otherwise.
	\item {\tt (ISF {\it [sym]})}\index{ISF}:
	  {\tt ISF} returns {\tt FVAL} when {\it sym} is {\tt MTL}, {\tt TVAL}
	  otherwise.
\end{itemize}


\subsection{Symbols}

\begin{itemize}
    \item {\tt (CAPITALIZE {\it [sym {\tt | } str]})}:\index{CAPITALIZE}
        returns the argument with all its letters capitalized.
    \item {\tt (CAPITALIZE-SYM {\it [sym]})}:\index{CAPITALIZE-SYM}
        returns {\it sym} with all its letters capitalized.
    \item {\tt (CAPITALIZE-STR {\it [str]})}:\index{CAPITALIZE-STR}
        returns {\it str} with all its letters capitalized.
    \item {\tt (CONCAT-SYMS {\it [sym$_1$][sym$_2$]})}:\index{CONCAT-SYMS}
        returns the concatenation of {\it sym$_1$} and {\it sym$_2$}.
    \item {\tt  (PAIR2CHR {\it [pair]})}:\index{PAIR2CHR}
        makes {\it pair} into the corresponding symbol.
    \item {\tt  (INT2SYM {\it [int]})}:\index{INT2SYM}
        makes {\it int} into the corresponding symbol.
    \item {\tt  (NAT2SYM {\it [int]})}:\index{NAT2SYM}
        makes a natural {\it int} into the corresponding symbol.
    \item {\tt  (NAT2SYM {\it [nat]})}:\index{NAT2SYM}
        makes the natural {\it nat} into the corresponding symbol.
    \item {\tt (NUMERAL {\it [num]})}\index{NUMERAL}
        looks if {\it num} is a numeral.
\end{itemize}


\subsection{Integers}

These primitives are trivial, and do not require any explanation.
Note that {\tt INTEQ} (the integer equality predicate) must be used instead of
{\tt EQ} (the object equality) when compiling because some CL compiler
substitutes the variable n (where n goes from $-1024$ to $1023$), which
evaluates to the integer $n$, with its actual value. This can create unexpected
behavior which is very difficult to detect and fix.

\begin{itemize}
	\item {\tt (MINUS ...)}\index{MINUS}
	\item {\tt (PLUS ...)}\index{PLUS}
	\item {\tt (DIFFERENCE ...)}\index{DIFFERENCE}
	\item {\tt (TIMES ...)}\index{TIMES}
	\item {\tt (QUOTIENT ...)}\index{QUOTIENT}
	\item {\tt (REMAINDER ...)}\index{REMAINDER}
	\item {\tt (NATNUM ...)}\index{NATNUM}
	\item {\tt (INTEQ ...)}\index{INTEQ}
	\item {\tt (LT ...)}\index{LT}
	\item {\tt (LE ...)}\index{LE}
	\item {\tt (GT ...)}\index{GT}
	\item {\tt (GE ...)}\index{GE}
\end{itemize}


\subsection{Characters}

\begin{itemize}
	\item {\tt (ASCII {\it [chr]})}:\index{ASCII}
		  creates and returns the interned symbol corresponding to {\it chr}.
	\item {\tt (SYM-INTERN {\it [obj]})}:\index{SYM-INTERN}
		   creates and returns the interned symbol corresponding to {\it obj}.
	\item {\tt (MAKE-UNINTERNED {\it [obj]})}:\index{MAKE-UNINTERNED}
		   creates and returns an un-interned symbol corresponding to {\it obj}.
	\item {\tt (SYM-GETCHRFIRST {\it [sym]})}:\index{SYM-GETCHRFIRST}
		   returns the character corresponding to the first character of
		   {\it sym}.
\end{itemize}


\subsection{Strings}

\begin{itemize}
	\item {\tt (SUBSTR? {\it [str$_1$][str$_2$]})}:\index{SUBSTR?}
		   evaluates to true if {\it str$_2$} begins with {\it str$_1$}.
	\item {\tt (CONCAT-STRS {\it [str$_1$][str$_2$]})}:\index{CONCAT-STRS}
		  returns the concatenation of {\it str$_1$} and {\it str$_2$}.
	\item {\tt (SUBSTR {\it [str][int]})}:\index{SUBSTR}
		   returns the string of the first {\it int} characters of {\it str}. If
		   {\it int} is too big an error occurs.
	\item {\tt (INSTR? {\it [str$_1$][str$_2$]})}:\index{INSTR?}
		   evaluates to true if {\it str$_1$} appears in {\it str$_2$}.
\end{itemize}


\subsection{Lists}

\begin{itemize}
   	\item {\tt (LIST ...)}:\index{LIST}
        as in CL.
   	\item {\tt (EQUAL ...)}:\index{EQUAL}
        as in CL.
   	\item {\tt (COPY {\it [obj]})}:\index{COPY}
        returns an object {\tt EQUAL} and not  {\tt EQ} to {\it obj}.
    \item {\tt (LENGTH {\it [list]})}:\index{LENGTH}
        returns the length of {\it list}.
    \item {\tt (HGKM-MEMBER {\it [obj][list]})}:\index{HGKM-MEMBER}
        as in CL.
		The equality test is {\tt EQUAL}.
    \item {\tt (HGKM-MEMQ {\it [obj][list]})}:\index{HGKM-MEMQ}
	    like {\tt HGKM-MEMBER}, but with {\tt EQ} test.
		For instance:
		%
		\begin{verbatim}
			> (HGKM-MEMQ '(C D) ((A B)(C D) E))
			FALSE
		\end{verbatim}
    \item {\tt (APPEND ...)}:\index{APPEND}
        as in CL.
    \item {\tt (UNION ...)}:\index{UNION}
        as in CL.
    \item {\tt (REVERSE {\it [list]})}:\index{REVERSE}
        returns the reverse of the top level of {\it list}.
    \item {\tt (SUBST {\it [obj$_1$][obj$_2$][obj$_3$]})}:\index{SUBST}
        substitutes {\it obj$_1$} to every {\tt EQUAL} occurrence of
		{\it obj$_2$} in {\it obj$_3$}.
    \item {\tt (CL-ASSOC ...)}:\index{CL-ASSOC}
        as CL's {\tt ASSOC}: it tests with {\tt EQ}.
    \item {\tt (HGKM-ASSOC ...)}:\index{HGKM-ASSOC}
         the same as {\tt CL-ASSOC}, but it tests with {\tt EQUAL}.
    \item {\tt (SECOND {\it [list]})}:\index{SECOND}
         returns the second element of {\it list}.
    \item {\tt (LAST {\it [list]})}:\index{LAST}
         as in common lisp: it returns the last {\em cons} of {\it list}.
    \item {\tt (HGKM-NTH {\it [int][list]})}:\index{HGKM-NTH}
        returns the {\it int}-th element of {\it list}.
        Note that {\it int} must be positive, otherwise an error occurs.
    \item {\tt (NCONC ...)}:\index{NCONC}
         as in CL.
    \item {\tt  (POS {\it [sym][list]})}:\index{POS}
        returns the first position of {\it sym} in {\it list}.
        If it does not exist {\tt POS} returns {\tt 0}.
    \item {\tt  (TAIL {\it [list][obj]})}:\index{TAIL}
        if {\it obj} occurs in {\it list} the list of elements following
        {\it obj} (not including {\it obj}) is returned; otherwise {\tt MTL}
        is returned. 
    \item {\tt  (FLATTEN {\it [list]})}:\index{FLATTEN}
        {\it list} is flattened and returned.
    \item {\tt  (PRELIST {\it [list][int]})}:\index{PRELIST}
        creates a list of the first {\it int} elements of {\it list}.
    \item {\tt  (SUFLIST {\it [list][int]})}:\index{SUFLIST}
        creates a list by deleting the first {\it int} elements of {\it list}.
    \item {\tt  (INLIST? {\it [list$_1$][list$_2$]})}:\index{INLIST?}
        looks if {\it list$_1$} is in {\it list$_2$} (only at top level).
    \item {\tt  (SUBTRACT {\it [list$_1$][list$_2$]})}:\index{SUBTRACT}
        returns the list of the elements of {\it list$_1$} which are not in
        {\it list$_2$}. 
    \item {\tt  (INTERSECT {\it [list$_1$][list$_2$]})}:\index{INTERSECT}
         returns the intersection
    \item {\tt (SUBSET {\it [list$_1$][list$_2$]})}:\index{SUBSET}
        looks if {\it list$_1$} is a subset of {\it list$_2$}.
        of {\it list$_1$} and {\it list$_2$}.
    \item {\tt  (INSERT! {\it [obj][list]})}:\index{INSERT!}
        inserts {\it obj} in {\it list} making side effect.
        {\tt INSERT!} assumes that {\it obj} can only occur once in {\it
        list}. 
    \item {\tt  (HGKM-DELETE {\it [obj][list]})}:\index{HGKM-DELETE}
        deletes {\it obj} from {\it list} without doing side effect.
        {\tt HGKM-DELETE} assumes that {\it obj} only occurs once in {\it
        list}. 
    \item {\tt  (DELETE! {\it [obj][list]})}:\index{DELETE!}
         the same as {\tt HGKM-DELETE} but with side effect.
    \item {\tt  (OUTOF {\it [obj][list]})}:\index{OUTOF}
        removes all the occurrences of {\it obj} from {\it list} without
        doing side effect.
    \item {\tt  (OUTOF! {\it [obj][list]})}:\index{OUTOF!}
         the same as {\tt OUTOF} but with side effect.
    \item {\tt  (UNLIST {\it [list]})}:\index{UNLIST}
        substitutes the last cons of {\it list} with its {\tt CAR}.
        For instance:

        {\tt (UNLIST (QUOTE (A))) $\Longrightarrow$ A}

        {\tt (UNLIST (QUOTE (A B))) $\Longrightarrow$ (A . B)}

        {\tt (UNLIST (QUOTE (A B C))) $\Longrightarrow$ (A B . C)}
    \item {\tt  (REVMAP {\it [list]})}:\index{REVMAP}
        puts the last element of {\it list} at the top of
        {\it list} itself.
    \item {\tt  (REPETITION-GET {\it [list]})}:\index{REPETITION-GET}
         returns the first element occurring twice in {\it list}.
    \item {\tt  (LIST-N {\it [obj][int]})}:\index{LIST-N}
         returns the list of {\it int} occurrences of {\it obj}.
    \item {\tt  (PRINTLIST {\it [list]})}:\index{PRINTLIST}
         prints all the elements of {\it list}.
\end{itemize}


\subsection{Arrays}

\begin{itemize}
    \item {\tt (TUPLE {\em [sym]} {\em [len]})}:\index{TUPLE}
        sets the value of the variable {\em sym} to be an array of length
        {\em len}. 
    \item {\tt (SETTUP {\em [array]} {\em [index]} {\em [data]})}:
	\index{SETTUP}
        sets the {\em index}-th entry in {\em array} to {\em data} destroying
        the previous information.
    \item {\tt (GETTUP {[\em array]} {\em [index]})}:\index{GETTUP}
        returns the data stored in {\em array} at the position indexed by
		{\em index}.
\end{itemize}


\subsection{Ordered lists of integers}

\begin{itemize}
	\item {\tt (UNIONLE {\it [list$_1$][list$_2$]})}:\index{UNIONLE}
		given two lists of natural numbers, each in ascending order, creates the
		ascending ordered union. 
	\item {\tt (UNIONGE {\it [list$_1$][list$_2$]})}:\index{UNIONGE}
		given two lists of natural numbers, each in descending order, {\tt
		UNIONGE} creates the descending ordered union.
	\item {\tt (ORDER {\it [list]})}:\index{ORDER}
		returns the list of natural numbers occurring in {\it list} in
		increasing order.
	\item {\tt (INSERT {\it [int][list]})}:\index{INSERT}
		inserts {\it int} in {\it list}, an ascending-ordered list of natural
		numbers.
	\item {\tt (hole {\it [list]})}:\index{hole}
		being {\it list} argument a list of integers greater than or equal to
		$-1$, {\tt hole} returns the smallest number not appearing in
		{\it list}.
		
		Example:
		%
		\begin{verbatim}
			> (hole (QUOTE (-1 0 1 3)))
			2
			> (hole (QUOTE (0 2 3)))
			-1
		\end{verbatim}
\end{itemize}


\subsection{Associative lists}

Elements of associative lists are implemented as dotted pairs.
%
\begin{itemize}
    \item {\tt (alistelement-get-1st {\it [alist-elem]})}:
	\index{alistelement-get-1st}
        returns the {\tt CAR} of the dotted pair.
    \item {\tt (alistelement-get-2nd {\it [alist-elem]})}:
    \index{alistelement-get-2nd}
         returns the {\tt CDR} of the dotted pair.
    \item {\tt (alistelement-mak {\it [obj1][obj2]})}:
    \index{alistelement-mak}
        returns the pair {\tt ({\it obj1} . {\it obj2})}
\end{itemize}
%
Associative lists are associative addressing data structures; they are
implemented as lists of dotted pairs.
%
\begin{itemize}
    \item {\tt (alist-get-first {\it [alist]})}:\index{alist-get-first}
         returns the first element of {\it alist}
    \item {\tt (alist-get-rest {\it [alist]})}:\index{alist-get-rest}
         returns the rest of {\it alist} 
    \item {\tt (alist-cons {\it [alist][alist-elem]})}:\index{alist-cons}
         adds {\it alist-elem} to {\it alist}
    \item {\tt (HGKM-REPLACE {\it [obj$_1$][obj$_2$][alist]})}:
    \index{HGKM-REPLACE}
        if {\it obj$_1$} is the first element of an alist element, then
		{\tt HGKM-REPLACE} replaces the second element with  {\it obj$_2$};
        otherwise, {\tt HGKM-REPLACE} adds the {\tt CONS} of {\it obj$_1$} 
        and obj$_2$ to {\it alist}.
    \item {\tt Alist-empty}:\index{Alist-empty}
        is the constant representing an empty associative list
    \item {\tt (ALIST-EMPTY? {\it [alist]})}:\index{ALIST-EMPTY?}
        looks if {\it alist} is empty
    \item {\tt (alist-add {\it [alist][obj$_1$][obj$_2$]})}\index{alist-add}
        adds the pair {\tt ({\it obj$_1$} . {\it obj$_2$})} to {\it alist}
        if not already present
    \item {\tt (alist-del {\it [alist][obj]})}\index{alist-del}
        removes the pair pointed by {\it obj} from {\it alist}
    \item {\tt (alist-union {\it [alist$_1$][alist$_2$]})}\index{alist-union}
        adds to {\it alist$_2$} all the elements of {\it alist$_1$},
        overwriting those elements of {\it alist$_2$} which are also in
		{\it alist$_1$} 
    \item {\tt (alist-cons-shared {\it [alist][alist-elem]})}:
    \index{alist-cons-shared}
        same as {\tt alist-cons} but with side effect on {\it alist}.
\end{itemize}


\subsection{Extending the evaluation syntax}

\begin{itemize}
	\item {\tt (MAPCAR ...)}:\index{MAPCAR}
		  as in CL.
\end{itemize}


\subsection{Extending the input output}

\begin{itemize}
	\item {\tt (HGKM-LOAD {\it [str]})}:\index{HGKM-LOAD}
		  loads the file named {\it str}.
	\item {\tt (HGKM-LOAD-FROM-CHANNEL {\it [chan]})}:
	\index{HGKM-LOAD-FROM-CHANNEL}
		 loads s-expressions from the default channel until an {\tt EOF} is
		 found.
\end{itemize}


\appendix
\section{Some notes about implementation}

While dealing with big software systems, in phase of development it may be very
helpful to know if a subroutine is built-in or user-defined, and if it has been
defined as function or macro.
If {\tt \infmode}\ is true when {\HG} is created, all this information is made
available: for each subroutine declaration, zero-level constructors set a
property recording that the subroutine is built-in; first-level constructors
set a property recording that the subroutine is user-defined and whether it has
been defined as a function or as a macro; reading these properties, first level
recognizers can tell if a subroutine is built-in or system-defined, and if it is
a function or a macro.
If {\tt \infmode} is false no property is set by the zero and first level
constructors, and first level recognizers, when called, will signal that no
answer is possible because no information is available.
For a detailed description of each first-level constructor and recognizer, with
respect to {\tt \infmode}, the reader is referred to basic {\HG} chapter.

Of course, when {\tt \infmode} is false, a large amount of storage is saved, and
a more compact {\HG} executable file is obtained.
Finally we remark that, for sake of consistency, the value of {\tt \infmode}
should not change while {\HG} is being created.

A complex software system may have big performance problems, because of the
overhead due to the great number of function calls, even if it runs compiled.
Therefore, the only way to speed up the system is to define as many subroutines
as possible as macros. These macros are expanded at the moment of compilation,
obtaining in this way a code which is modular and well structured, and which
does not have the overhead of function calls.
There are two drawbacks to this approach: the former, and least important, is
that the program becomes big because of the macro expansion; the latter, and
critical, is that debugging becomes very hard, because all information related
to functional definition is lost.

In {\HG} this problem may be solved using the first-level constructor {\tt
DEFSUB}, that allows to define subroutines either like macros or like functions,
without changing source code.
As explained below, {\tt DEFSUB} definitions depend on the value of a system
constant, {\tt \debmode}.

A {\tt DEFSUB} call has a syntax as defined in basic {\HG} chapter: it requires
three arguments, that is a {\it subroutine name}, an {\it arguments list} and a
{\it body}.
The third argument, {\it body}, is a function body, as needed in {\tt DEFUN}
subroutines definitions.
If {\tt \debmode} is true, {\tt DEFSUB} works like {\tt DEFLAM}.
If {\tt \debmode} is false, {\tt DEFSUB} executes a {\tt DEFMAC} call, modifying
the third argument: that is, starting from {\it body}, {\tt DEFSUB} builds a
macro body that, when macro-expanded, returns {\it body}.

Hence {\tt DEFSUB} is a subroutine constructor that has a simple function
definition syntax and behaves either like a function or macro constructor,
depending on the value of {\tt \debmode}.
If {\tt \debmode}\ is true debugging is possible, because {\tt DEFSUB}
subroutines have been declared as functions; if {\tt \debmode}\ is false
debugging is very hard, but the execution is speed up because no function call
overhead is present.

Note that {\tt DEFSUB} is subject to some restriction, as remarked in basic
{\HG} chapter: this is due to the complexity of macro body creation.

Finally note that, to reach some performance improvements, a {\tt DEFSUB}
definition of a general subroutine {\tt SBR} must be loaded before compiling
subroutines calling {\tt SBR} itself; else, macro expansion can not be made at
compile time, and the {\tt SBR} is executed interpretively and expanded at run
time.


% ................................. bibliography ..............................
\bibliographystyle{alpha}
\bibliography{}

% .................................... index ..................................
%............................... HGKM MANUAL .................................
%.............................................................................


\documentstyle[11pt]{../styfiles/GFmanual}

%................................. macros ....................................
%.............................................................................

\setcounter{totalnumber}{1}

\def\actionzero#1{\noindent ({\tt {#1}}) \hfill action}
\def\actionone#1#2{\noindent ({\tt {#1}} {\it [{#2}]}) \hfill action}
\def\actiontwo#1#2#3{\noindent ({\tt {#1}} {\it [{#2}]}{\it [{#3}]}) \hfill
action} 
\def\actionthree#1#2#3#4{\noindent ({\tt {#1}} {\it [{#2}]}{\it [{#3}]}{\it
[{#4}]}) \hfill action}

\def\functionzero#1#2{\noindent ({\tt {#1}}) $\Longrightarrow$ {#2} \hfill
function} 
\def\functionone#1#2#3{\noindent ({\tt {#1}} {\it [{#2}]}) $\Longrightarrow$
{#3} \hfill function}
\def\functiontwo#1#2#3#4{\noindent ({\tt {#1}} {\it [{#2}]}{\it [{#3}]})
$\Longrightarrow$ {#4} \hfill function} 
\def\functionthree#1#2#3#4#5{\noindent ({\tt {#1}} {\it [{#2}]}{\it [{#3}]}{\it
[{#4}]}) $\Longrightarrow$ {#5} \hfill function} 

\def\predicatezero#1{\noindent ({\tt {#1}}) \hfill predicate}
\def\predicateone#1#2{\noindent ({\tt {#1}} {\it [{#2}]}) \hfill predicate}
\def\predicatetwo#1#2#3{\noindent ({\tt {#1}} {\it [{#2}]}{\it [{#3}]}) \hfill
predicate}
\def\predicatethree#1#2#3#4{\noindent ({\tt {#1}} {\it [{#2}]}{\it [{#3}]}{\it
[{#4}]}) \hfill predicate}

\def\sysconst#1{\noindent{\tt {#1}} \hfill system constant}

\def\chan {{\it channel\/}}
\def\chr {{\it chr\/}}
\def\chrlist {{\it chr-list\/}}
\def\debmode {\%DE\-BUG\--MO\-DE\%}
\def\dev {{\it device\/}}
\def\dir {{\it direction\/}}
\def\filena {{\it filena\/}}
\def\flconst {first level constructor}
\def\flrec {first level recognizer}
\def\form {{\it form\/}}
\def\formlist {{\it form list\/}}
\def\funcdesc {{\it function description\/}}
\def\infmode {\%IN\-FOR\-MA\-TION-MO\-DE\%}
\def\inte {{\it int\/}}
\def\LIST {{\it list\/}}
\def\LISTone {{\it list\/$_1$\/}}
\def\LISTtwo {{\it list\/$_2$\/}}
\def\obj {{\it obj\/}}
\def\objn {{\it obj\/$_n$\/}}
\def\objone {{\it obj\/$_1$\/}}
\def\objtwo {{\it obj\/$_2$\/}}
\def\pair {{\it pair\/}}
\def\sbral {{\it argument list\/}}
\def\sbrbo {{\it body\/}}
\def\sbrna {{\it subroutine name\/}}
\def\str {{\it str\/}}
\def\sym {{\it sym\/}}
\def\symone {{\it sym\/$_1$\/}}
\def\symtwo {{\it sym\/$_2$\/}}
\def\value {{\it value\/}}

% ................................. document ..................................
% .............................................................................

% \makeindex
\begin{document}

\title{HGKM User Manual}
\author{{\bf Fausto Giunchiglia} and {\bf Alessandro Cimatti}}
\date{4 March 1994}
\version{2.1}
\abstract{
	This manual describes the implementation of a simple programming language,
	{\HG}. 
	{\HG} was originally defined and developed by Richard Weyhrauch and Caroline
	Talcott at the Department of Computer Science of Stanford University.
	The version described here sticks as much as possible the conventions and the
	specifications of the original version.
	The implementation is almost totally new and some new functionalities (e.g.
	{\tt DEFSUB}) have been added.
	Theoretical {\HG} was defined so that its semantics could be easily
	represented in first order logic.
	The version of {\HG} described in this manual acts as the implementation
	language for {\GF}.
	This manual has been written only to facilitate the
	implementation/modification of {\GF}.
}
\addresses{
   \begin{tabular}[c]{l@{\hskip 2em}l}
       {\bf Fausto Giunchiglia}:	    & {\bf Alessandro Cimatti}           \\
       Mechanized Reasoning Group       & Mechanized Reasoning Group,        \\
       IRST, Povo, 38050 Trento, Italy  & IRST, Povo, 38050 Trento, Italy    \\
       e-mail: {\tt fausto@irst.it}     & e-mail: {\tt cx@irst.it}           \\
       phone: +39 461 314359            & phone: +39 461 314351
   \end{tabular}
}
\published{
	\begin{tabular}{l}
		DIST Technical Report No. 9107-05 (1991). \\
		DIST -- University of Genoa,\\
		Via Opera Pia 11A, 16145 Genova, Italy.
    \end{tabular}
}

% ................................... cover ...................................

\thispagestyle{empty}
\maketitle

% ............................. table of contents ...............................

\newpage
\pagenumbering{roman}
\tableofcontents
\newpage
\pagenumbering{arabic}


\section{Introduction}

This manual describes an enhanced version of {\HG}, which has been revised by
the authors within the {\GF} project\cite{giunchiglia34,giunchiglia12,giunchiglia29}.
In effect, the most recent version of {\GF} has been implemented on this
version of {\HG}.

{\HG} is a programming language which has been implemented on top of Common
LISP; it is powerful and simple to use. {\HG} has been developed mainly for
three reasons: to increase {\GF}'s portability, to improve {\GF}'s performance
and to introduce a simple mechanism of data typing.

A language we will call {\it basic} {\HG} was developed including only a small
set of data types and a few built in primitives. This language is enough to
provide the full expressive power of {\HG}.

However, in order to facilitate the development of {\GF}, more data types and
primitives have been defined.
These additions form {\it extended} {\HG}.

This manual is not a tutorial.
It is a reference manual for people already familiar with LISP and
s-expressions notation.


\subsection{Basic concepts}

A {\HG} object belongs to a data type. For each data type three classes of
primitives are defined:

\begin{description}
    \item[predicates:]
        test if a condition is verified or not; predicates are forms which can
        be used as test for branching in conditional forms: that is, evaluating
		a predicate returns a truth value.
    \item[functions:]
        specify a correspondence between {\HG} data and returns an {\HG} object;
		the data type of returned object is specified too; a function produces
		no side effects.
    \item[actions:]
        specify a side effect, and do not return an {\HG} object.
\end{description}
%
\noindent
Furthermore for each data type we define some
%
\begin{description}
    \item[system constants:]
        symbols to which corresponds a fixed value
\end{description}


\subsection{Remarks about notation}

The syntax of primitives and system constants will be described using the
following notation:

\begin{description}
	\item \sysconst{<constant name>}
	\item \predicatethree{<predicate name>}{arg type}{arg type}{\ldots}
	\item \functionthree{<function name>}{arg type}{arg type}{\ldots}{res type}
	\item \actionthree{<action name>}{arg type}{arg type}{\ldots}
\end{description}

\begin{quote}\bf
	After the syntactic description we have a section in which the semantics are
	described.
	This section appears narrower than the rest of the text, in this way.
\end{quote}

The name of the primitive is printed in {\tt typewriter} style.
Its arguments are language forms which will be evaluated.
With {\it [arg type]} we mean the data type of the {\HG} object that the
evaluation of the argument form should return.
If the result of the evaluation does not match the specified data type a fatal
error is generated.

For instance, \inte\ denotes a form which evaluates to an integer.
If an operation requires more than one argument of a single type, the bracketed
type name will have a numbered subscript.
For example we write {\it str\/$_1$} and {\it str\/$_2$} to denote two different
forms that have strings as values.
In the case an operation does not care about the type of its arguments we write
\obj\ to indicate a form which can evaluate to any object in the {\HG} world.

In section 2 we present basic {\HG}; in section 3 extended {\HG}; finally, in
appendix we describe some aspects related to the implementation.


\newpage
\section{Basic {\HG}}

In this section we shall describe basic {\HG}: we present the structured data
types, the way in which the language forms are evaluated by the interpreter,
some special forms for subroutine definitions, the error handling and the
input/output primitives.


\subsection{Data structures}

Basic {\HG} data types are:

\begin{itemize}
	\item truth values {\it [trth];}
	\item symbols {\it [\sym];}
	\item integers {\it [\inte];}
	\item characters {\it [\chr];}
	\item strings {\it [\str];}
	\item pairs {\it [\pair];}
	\item lists {\it [list].}
\end{itemize}

In this chapter we shall refer to each data type by the corresponding name
enclosed in square brackets.

The syntax of data types is defined by this grammar:

\begin{bnf}
	\T{obj}			\sep  \T{atomic obj} | \T{compound obj} \\
	\T{atomic obj}	\sep  \T{trth} | \T{sym} | \T{int} | \T{chr} | \T{str} \\
	\T{compound obj}\sep  \T{pair} | \T{list} | \T{chr-list}\\
	\T{pair}	    \sep  ( \T{obj} . \T{obj}) \\
	\T{list}		\sep  empty-list | ( \T{obj} . \T{list} ) \\
	\T{chr-list}	\sep  empty-list | ( \T{chr} . \T{chr-list} )
\end{bnf}

In this grammar we refer explicitly to a data structure, the list of characters
{\it [chr-list],} which is not a primitive data type.
However it is very used in {\HG}, and it is interesting to define its syntax too.
Note that a list of characters is different from a string: in basic {\HG} a
string is an atomic object, while a list of characters has a compound structure.


\subsubsection{Truth values and equality predicate}

\sysconst {TRUE}\index{TRUE}
\begin{quote}
	{\tt TRUE} is a system constant which always evaluates to true.
\end{quote}

\sysconst {FALSE}\index{FALSE}
\begin{quote}
	{\tt FALSE} is a system constant which always evaluates to false.
\end{quote}

\predicatetwo{EQ}{\objone}{\objtwo}\index{EQ}
\begin{quote}
	{\tt EQ} is the equality predicate, and is defined for all data types.
	{\tt EQ} evaluates to true if both \objone\ and \objtwo\ are the same
	object, false otherwise.
\end{quote}


\subsubsection{Symbols}

{\HG} symbols are like Common LISP identifiers, with the only difference that
they are case sensitive.
{\tt abc}, {\tt CAR}, {\tt NIL}, {\tt THISISALONGATOM} are examples of {\HG}
symbols.

\bigskip

\predicateone{SYMBOL?}{\obj}\index{SYMBOL?}
\begin{quote}
	{\tt SYMBOL?} evaluates to true if \obj\ is a symbol, false otherwise.
\end{quote}

\actiontwo{SETVAL}{\sym}{\obj}\index{SETVAL}
\begin{quote}
	{\tt SETVAL} changes the value associated with the specified symbol to \obj.
\end{quote}

\functionone{GETVAL}{\sym}{\obj}\index{GETVAL}
\begin{quote}
	{\tt GETVAL} returns the value associated with \sym.
\end{quote}

\actionthree{SETPROP}{\symone}{\obj}{\symtwo}\index{SETPROP}
\begin{quote}
	{\tt SETPROP} sets the property of \symone\ with key \symtwo\ to have the
	value \obj.
\end{quote}

\functiontwo{GETPROP}{\symone}{\symtwo}{\obj}\index{GETPROP}
\begin{quote}
	{\tt GETPROP} returns the value of the property \symtwo\ of \symone.
	If there is no such property {\tt GETPROP} returns the false truth value.
\end{quote}
 
\functiontwo{REMPROP}{\symone}{\symtwo}{\obj}\index{REMPROP}
\begin{quote}
	{\tt REMPROP} removes from \symone\ the property with an indicator {\tt EQ}
	to \symtwo. 
	It returns the false truth value if there is no such indicator.
\end{quote}


\subsubsection{Integers}

Integers are the only type of numbers allowed in {\HG}.
\bigskip

\predicateone {INTEGER?}{\obj}\index{INTEGER?}
\begin{quote}
	{\tt INTEGER?} evaluates to true if \obj\ is an integer, false otherwise.
\end{quote}

\predicateone {ZERO?}{\obj}\index{ZERO?}
\begin{quote}
	{\tt ZERO?} evaluates  to true if \obj\ is the integer $0$, false otherwise.
\end{quote}

\predicateone {POSITIVE?}{\obj}\index{POSITIVE?}
\begin{quote}
	{\tt POSITIVE?} evaluates to true if \obj\ is a positive integer, false
	otherwise.
\end{quote}

\predicateone {NEGATIVE?}{\obj}\index{NEGATIVE?}
\begin{quote}
	{\tt NEGATIVE?} evaluates to true if \obj\ is a negative integer, false
	otherwise.
\end{quote}

\functionone {ADD1}{\inte}{\inte}\index{ADD1}
\begin{quote}
	{\tt ADD1} adds one to an integer.
\end{quote}

\functionone {SUB1}{\inte}{\inte}\index{SUB1}
\begin{quote}
	{\tt SUB1} subtracts one from an integer.
\end{quote}


\subsubsection{Characters}

{\HG} characters are all the characters to which is associated an ASCII code.
An {\HG} character is represented by the integer between $0$ and $255$
corresponding to its {\tt ASCII} code.

\bigskip

\predicateone {CHARACTER?}{\obj}\index{CHARACTER}
\begin{quote}
	{\tt CHARACTER?} evaluates  to true if \obj\ is a character, false otherwise.
\end{quote}

\functionone {CHR2SYM}{\chrlist}{\sym}\index{CHR2SYM}
\begin{quote}
	{\tt CHR2SYM} returns the symbol whose print name consists of the characters
	in \chrlist.

	For instance:
	%
	\begin{verbatim}
		> (CHR2SYM '(65 80 80 76 69))
		APPLE
	\end{verbatim}
\end{quote}

\functionone {SYM2CHR}{\sym}{\chrlist}\index{SYM2CHR}
\begin{quote}
	{\tt SYM2CHR} is the logical inverse of {\tt CHR2SYM}: it returns the list
	of the characters which compound the symbol.

	For instance:
	%
	\begin{verbatim}
		> (SYM2CHR 'APPLE)
		(65 80 80 76 69)
	\end{verbatim}
\end{quote}

\functionone {CHR2INT}{\chrlist}{\inte}\index{CHR2INT}
\begin{quote}
	{\tt CHR2INT} takes a list of characters which represent digits and returns
	the corresponding integer.
	If the list begins with a minus sign ``-'' then {\tt CHR2INT} returns the
	appropriate negative number.

	For instance:
	%
	\begin{verbatim}
		> (CHR2INT '(49 51 55)
		137
		> (CHR2INT '(45 49 51 55))
		-137
	\end{verbatim}
\end{quote}

\functionone {\tt INT2CHR}{\inte}{\chrlist}\index{INT2CHR}
\begin{quote}
	{\tt INT2CHR} is the logical inverse of {\tt CHR2INT}.
	{\tt INT2CHR} returns the list of characters corresponding to the digits of
	the integer.
	If the integer is negative then the first element of the list will contain
	the number $45$, which is the {\tt ASCII} code for the minus sign ``-''.

	Example:
	%
	\begin{verbatim}
		> (INT2CHR 137)
		(49 51 55)}
		> (INT2CHR -137)
		(45 49 51 55)
	\end{verbatim}
\end{quote}


\subsubsection{Strings}

A string is an arbitrary sequence of {\it n\/} characters, indexed from $1$ to
{\it n\/}.
A string is delimited by a pair of double quotes ({\tt "}) which do not belong
to it. {\HG} strings are case sensitive.
{\tt "Symbol"}, {\tt "number"}, {\tt "ThisIsAString"} are examples of {\HG}
strings.

\bigskip

\predicateone {STRING?}{\obj}\index{STRING?}
\begin{quote}
	{\tt STRING?} evaluates to true if \obj\ is a string, false otherwise.
\end{quote}

\functionone {LENSTR}{\str}{\inte}\index{LENSTR}
\begin{quote}
	{\tt LENSTR} returns an integer \inte\ equal to the number of characters in
	\str.
\end{quote}

\actionthree {SETSTR}{\str}{\inte}{\chr}\index{SETSTR}
\begin{quote}
	{\tt SETSTR} sets to \chr\ the \inte-th position of \str.
\end{quote}

\functiontwo {GETSTR}{\str}{\inte}{\chr}\index{GETSTR}
\begin{quote}
	{\tt GETSTR} returns the \inte-th character of \str.
\end{quote}

\functionone {\tt STR2CHR}{\str}{\chrlist}\index{STR2CHR}
\begin{quote}
	{\tt STR2CHR} returns a list of characters which appear in the string.
\end{quote}

\functionone {CHR2STR}{\chrlist}{\str}\index{CHR2STR}
\begin{quote}
	{\tt CHR2STR} returns a string of the length of \chrlist\ that contains the
	characters that appear in \chrlist.

	Example:
	%
	\begin{verbatim}
		> (STR2CHR "APPLE")
		(65 80 80 76 69)
		> (CHR2STR '(104 111 109 101)
		"home"
	\end{verbatim}
\end{quote}


\subsubsection{Pairs}

An {\HG} pair is the ordered collection of two {\HG} objects.

\bigskip


\predicateone{PAIR?}{\obj}\index{PAIR?}
\begin{quote}
	{\tt PAIR?} evaluates to true if \obj\ is a pair, false otherwise.
\end{quote}

\predicateone{ATOMIC?}{\obj}\index{ATOMIC?}
\begin{quote}
	{\tt ATOMIC?} evaluates to true if \obj\ is an atomic object, namely
	a symbol, an integer or a string, false otherwise.
\end{quote}

\functiontwo {CONS}{\objone}{\objtwo}{\pair}\index{CONS}
\begin{quote}
	{\tt CONS} returns a pair consisting of \objone\ and \objtwo.
\end{quote}

\functionone {CAR}{\pair}{\obj}\index{CAR}
\begin{quote}
	{\tt CAR} returns the first element of \pair.
\end{quote}

\functionone {CDR}{\pair}{\obj}\index{CDR}
\begin{quote}
	{\tt CDR} returns the second element of \pair.
\end{quote}

\actiontwo {SETCAR}{\pair}{\obj}\index{SETCAR}
\begin{quote}
	{\tt SETCAR} changes the first element of \pair\ to \obj.
\end{quote}

\actiontwo {SETCDR}{\pair}{\obj}\index{SETCDR}
\begin{quote}
	{\tt SETCDR} changes the second element of \pair\ to \obj.
\end{quote}


\subsubsection{Lists}

A list is either the empty list or a pair whose second element is a list.

\bigskip

\sysconst {MTL}\index{MTL}
\begin{quote}
	{\tt MTL} evaluates to the empty list.
\end{quote}

\predicateone {MTL?}{\obj}\index{MTL?}
\begin{quote}
	{\tt MTL?} evaluates to true if \obj\ is the empty list, false otherwise.
\end{quote}

\predicateone {LIST?}{\obj}\index{LIST?}
\begin{quote}
	{\tt LIST?} evaluates to true if \obj\ is a list, false otherwise.
\end{quote}

\functiontwo {LCONS}{\obj}{\LIST}{\LIST}\index{LCONS}
\begin{quote}
	{\tt LCONS} returns a list with \obj\ in the front of \LIST.
\end{quote}

\functionone {FIRST}{\LIST}{\obj}\index{FIRST}
\begin{quote}
	{\tt FIRST} returns the first element of \LIST.
\end{quote}

\functionone {REST}{\LIST}{\LIST}\index{REST}
\begin{quote}
	{\tt REST} returns the rest of \LIST.
\end{quote}

\actiontwo {SETFIRST}{\LIST}{\obj}\index{SETFIRST}
\begin{quote}
	{\tt SETFIRST} changes the first element of \LIST\ to \obj.
\end{quote}

\actiontwo {SETREST}{\LISTone}{\LISTtwo}\index{SETREST}
\begin{quote}
	{\tt SETREST} changes the rest of \LISTone\ to \LISTtwo.
\end{quote}


\subsection{Evaluation of forms}

This section describes how a form is evaluated by {\HG} interpreter.
The different types of forms are:

\begin{itemize}
	\item null operation {\tt <nop>}
	\item constant {\tt <const>}
	\item variable {\tt <var>}
	\item quotation {\tt <quote>}
	\item subroutine application {\tt <appl>}
	\item conditional evaluation {\tt <if>}
	\item operation sequence {\tt <seq>}
	\item local environment definition {\tt <let>}
	\item environment modification {\tt <setq>}
	\item program {\tt <pgm>}
\end{itemize}
%
In the rest of the chapter we shall refer to the different types of forms to be
evaluated by the corresponding names which are enclosed between angle brackets.
The syntax of the language forms is defined by the following grammar.

\begin{bnf}
	\T{form}           \sep \T{atomic form} | \T{compound form} \\
	\T{atomic form}	   \sep \T{nop} | \T{const} | \T{var} \\
	\T{compound form}  \sep \T{quote} | \T{appl} | \T{if} | \T{let} | \T{setq} |
							\T{seq} | \T{pgm} \\
	\T{nop}			   \sep ( NOP) \\
	\T{quote}		   \sep ( QUOTE \T{form}) \\
	\T{appl}		   \sep (\T{subroutine name} \{ \T{form} \})\\
	\T{if}			   \sep ( IF | IFN \T{predform} \T{form1} \T{form2}) \\
	\T{let}			   \sep ( LET ( \{ ( \T{var} \T{form} ) \} ) \{ \T{form} \})
							\\ 
	\T{seq}			   \sep ( SEQ \{ \T{form} \} )\\
	\T{pgm}			   \sep ( PGM ( \{ \T{var} \} ) \{ \T{pgm-stm} \} )\\
	\T{pgm-stm}		   \sep \T{form} | \T{rtn} | \T{jump} | \T{label}\\
	\T{rtn}			   \sep ( RETURN \T{form} ) \\
	\T{jump}		   \sep ( BRT | BRF \T{label} \T{predform} ) | ( GOTO
							\T{label} )\\
	\T{label}		   \sep \T{symbol}
\end{bnf}
%
Some explanation must be supplied: by constant ({\tt <const>}) we mean either
a truth value, an integer, a character or a string; by predicative form
({\tt <predform>}) we mean a particular form that evaluates to a truth value;
by program statements ({\tt <pgm-stm>}) we mean the special constructs that
compose a {\tt <pgm>} form statement list.
In the rest of this section we shall describe the semantics of each form,
referring to the grammar which has been presented.

\bigskip

{\bf Null operation}\index{NOP}
\begin{quote}
	The evaluation of the {\tt <nop>} form does nothing and does not return a
	value.
\end{quote}

{\bf Constant}
\begin{quote}
	The evaluation of a constant form (truth value, integer, character or
	string) returns the constant itself. Hence {\HG} constants are invariant
	with respect to the evaluation.
\end{quote}

{\bf Variable}
\begin{quote}
	A {\tt <var>} form is a symbol. This symbol must be the name of an
	initialized variable.
	In this case the result of the evaluation of the form is the value of the
	variable in the current environment; otherwise a fatal error occurs.
\end{quote}

{\bf Quotation}\index{QUOTE}
\begin{quote}
	The evaluation of a {\tt <quote>} form returns the argument form without
	evaluating it.
\end{quote}

{\bf Subroutine application}
\begin{quote}
	{\tt <subroutine name>} must be a symbol which is the name of an {\HG}
	function or predicate.
	It can be the name of a built-in primitive or a user-defined function.
	The argument {\tt <form>}s are evaluated and bound to the corresponding
	formal parameters of the function or predicate (call-by-value).
	The result of the application is the value obtained applying the specified
	subroutine to the arguments.
	The type of the values deriving from arguments must match properly the data
	type of the formal parameters, otherwise a fatal error occurs.

	Examples of subroutine applications are:
	%
	\begin{verbatim}
		> (CONS 1 13)
		(1 . 13)
		> (MTL? MTL)
		T
	\end{verbatim}
\end{quote}

{\bf Conditional expressions}\index{IF}\index{IFN}
\begin{quote}
	If the {\tt <predform>} evaluates to true, then the {\tt IF} conditional
	expression returns the evaluation of the first {\tt <form>}, otherwise of
	the second one.
	{\tt IFN} stands for {\tt IF NOT}; its semantics are trivial.

	For instance:
	%
	\begin{verbatim}
		> (IF (MTL? X) FALSE TRUE)
	\end{verbatim}
	%
	evaluates to false if {\tt X} evaluates to the empty list, true otherwise.
\end{quote}

{\bf Sequences of forms: \verb+<seq>+}\index{SEQ}
\begin{quote}
	The value of a {\tt SEQ} form is the value of its last form in the
	environment defined by the evaluation of previous forms.
\end{quote}

{\bf Local environment definition}\index{LET}
\begin{quote}
	{\tt LET} is used to create and initialize local variables.
	The result of evaluating a {\tt LET} is given by the evaluation of the last
	{\tt <form>} in a new environment in which each {\tt <var>} is associated
	with the value of the respective {\tt <form>}.
	If one of the initializing {\tt <form>}s contains any reference to any of
	the {\tt <var>}s, its value is taken from the lexically exterior
	environment.
	Hence the semantics of a {\tt <let>} form is the same as:
	%
	\begin{verbatim}
		(SEQ
		  (SETQ TEMP-1 <form-1>)
		  ...
		  (SETQ TEMP-n <form-n>)
		  (SETQ var-1 TEMP-1)
		  ...
		  (SETQ var-n TEMP-n)
		  <form>)
	\end{verbatim}
	%
	where {\tt TEMP-1} through {\tt TEMP-n} are symbols which do not appear in
	any {\tt <form-i>} or in the last {\tt <form>}.
\end{quote}

{\bf Environment modification}\index{SETQ}
\begin{quote}
	The evaluation of a {\tt <setq>} form produces a side effect on
	environments, and does not return a value.
	The evaluation of a {\tt <setq>} form sets the value associated with the
	variable {\tt <var>} to the evaluation of {\tt <form>}.
	If {\tt <var>} is a global variable the form {\tt (SETQ <var> <form>)} is
	equivalent to the form {\tt (SETVAL (QUOTE <var>) <form>)}.

	For instance:
	%
	\begin{verbatim}
		> (SETVAL (QUOTE X) 1)
		1
		> (LET ((X 5)) (SEQ (SETQ X 3) X)))
		3
	\end{verbatim}
	%
	evaluates to $3$ and leaves the value of {\tt X} to be $1$.
	%
	\begin{verbatim}
		(SEQ
		  (SETVAL (QUOTE X) 1)
		  (LET ((X 5)) (SEQ (SETVAL (QUOTE X) 3) X)))
	\end{verbatim}
	%
	evaluates to $5$ and leaves the value of {\tt X} to be $3$.	
\end{quote}

{\bf Program}\index{PGM}
\begin{quote}
	The {\tt <pgm>} form provides a facility for executing sequential programs,
	something like {\tt PROG}s in Common Lisp.
	The default order of executing {\tt <pgm-stm>}s is the sequential order.
	This can be modified by the use of {\tt <jump>} or of {\tt <rtn>}
	instructions.
	The {\tt <var>}s are local to the {\tt <pgm>} form and they have no initial
	value. 
\end{quote}

{\bf Jumps}\index{BRT}\index{BRF}\index{GOTO}
\begin{quote}
	The {\tt <jump>} statements can be either conditional ({\tt BRT} and
	{\tt BRF}) or unconditional ({\tt GOTO}).
	{\tt BRT} branches to the {\tt <label>} if the {\tt <predform>} evaluates to
	true.
	{\tt BRF} branches to the {\tt <label>} if the {\tt <predform>} evaluates to
	false.
	{\tt GOTO} branches to {\tt <label>}.
\end{quote}

{\bf Labels}
\begin{quote}
	{\tt <label>}s are statements which are simply symbols.
	They are not evaluated but provide markers for the {\tt <jump>} statements.
\end{quote}

{\bf Return}\index{RETURN}
\begin{quote}
	At the evaluation of a {\tt <rtn>} the {\tt <pgm>} statements evaluation is
	halted and the value of {\tt <form>} is returned.
	As shown by the grammar, a {\tt <rtn>} statement can only be used as
	statement into a {\tt <pgm>} form.
\end{quote}


\subsubsection{Eval and apply}

In this section we describe two special form for evaluating forms and applying
functions to forms.

\bigskip

\functionone {HGKM-EVAL}{\form}{\obj}\index{HGKM-EVAL}
\begin{quote}
	{\tt HGKM-EVAL} evaluates \form\ in the current environment and returns the
	resulting \obj.
\end{quote}

\functiontwo {HGKM-APPLY}{\funcdesc}{\formlist}{\value}\index{HGKM-APPLY}
\begin{quote}
	{\tt HGKM-APPLY} binds each element of the form list (without evaluating it)
	to the corresponding element of the bound variables list of the function
	description, and returns the value of evaluating the body of the function
	description in this new environment.

	Example:
	%
	\begin{verbatim}
		> (HGKM-APPLY (GET-FUN 'APPEND) '((A B) (C D)))
		(A B C D)
		> (HGKM-APPLY (GET-FUN 'APPEND) '(LIST '(B) '(C D)))
		(B C D)
	\end{verbatim}
\end{quote}


\subsection{Special forms}

In {\HG} special forms to define functions and macros are provided.
We call this special forms \flconst s, to distinguish them from Common LISP
constructors (level zero constructors) which have been used to define built-in
forms. 
In this section we also present {\HG} special forms to inspect and describe
built-in and user-defined subroutines: we shall refer to these special forms
using the term "\flrec s".


\subsubsection{First level constructors}

The \flconst s are

\begin{itemize}
	\item {\tt DEFLAM}
	\item {\tt DEFMAC}
	\item {\tt DEFSYS}
	\item {\tt DEFMACSYS}
	\item {\tt DEFSUB}
	\item {\tt SET-FUN}
\end{itemize}

A \flconst\ call has the usual form:

\begin{quote}\tt
	(<constructor name> {\it [\sbrna] [\sbral] [\sbrbo]})
\end{quote}

The result of a such call is the definition of a subroutine named {\it \sbrna},
in which {\it \sbral\/} is the list of formal parameters and {\it \sbrbo\/} is
the statement list.
The syntax of user defined {\HG} subroutines is the same as Common LISP
subroutines: hence are allowed all CL options, namely {\tt \&REST, \&OPTIONS,}
etc.

The way in which a subroutine is implemented depends on which is the \flconst
used in defining it, and on the value of two installation variables at the
moment of definition.
Anyway the implementation is almost transparent to the user, and so here we give
only a brief description of implementation features.

We shall explain now the meaning of an installation variable, which is named
{\tt \infmode}: the main idea is that when this variable evaluates to true some
system information is set up at the moment of installation, while when it
evaluates to false no information is available.
The following are special forms which {\it do not} evaluate their arguments.

\bigskip

\actionthree {DEFLAM}{\sbrna}{\sbral}{\sbrbo}\index{DEFLAM}
\begin{quote}
	{\tt DEFLAM} is a function constructor; which always allows the redefinition
	of previously defined functions.
	If {\tt\infmode} evaluates to false, then {\tt DEFLAM} just defines the
	function.
	If {\tt\infmode} evaluates to true, {\tt DEFLAM} also sets a property
	telling that {\it \sbrna\/} is a first level defined function and, in case
	of redefinition, prints a warning message.
\end{quote}

\actionthree {DEFMAC}{\sbrna}{\sbral}{\sbrbo}\index{DEFMAC}
\begin{quote}
	{\tt DEFMAC} is a macro definer: hence {\it \sbrbo\/} must be a macro
	statement list, that is must be ``expansible'' like Common LISP macros.
	{\tt DEFMAC} always allows to redefine already defined macros.
	If {\tt\infmode} evaluates to false then {\tt DEFMAC} just defines the
	macro; else {\tt DEFMAC} also sets a property telling that {\it \sbrna} is a
	first level defined macro and, in case of redefinition, prints a warning
	message. 
\end{quote}

\actionthree {\tt DEFSYS}{\sbrna}{\sbral}{\sbrbo}\index{DEFSYS}
\begin{quote}
	{\tt DEFSYS} is a function definer which allows to redefine already defined
	functions only when {\tt\infmode} evaluates to false; when {\tt\infmode} is
	true and {\it \sbrna} has already been defined, {\tt DEFSYS} will signal
	that it can not execute the redefinition.
	With respect to functions which have not been defined yet {\tt DEFSYS} is
	just like {\tt DEFLAM}.
\end{quote}

\actionthree {DEFMACSYS}{\sbrna}{\sbral}{\sbrbo}\index{DEFMACSYS}
\begin{quote}
	{\tt DEFMACSYS} is a macro definer which allows to redefine already defined
	macros only when {\tt\infmode} evaluates to false; when {\tt\infmode} is
	true and {\it \sbrna} has already been defined, {\tt DEFMACSYS} will signal
	that it can not execute the redefinition.
	With respect to macros which have not been defined yet {\tt DEFMACSYS} is
	just like {\tt DEFMAC}. 
\end{quote}

\actionthree {\tt SET-FUN}{\sbrna}{\sbral}{\sbrbo}\index{SET-FUN}
\begin{quote}
	{\tt SET-FUN} differs from {\tt DEFLAM} just because it evaluates its
	arguments,while {\tt DEFLAM} does not.
\end{quote}

\actionthree {DEFSUB}{\sbrna}{\sbral}{\sbrbo}\index{DEFSUB}
\begin{quote}
	{\tt DEFSUB} can work as {\tt DEFLAM} or as {\tt DEFMAC}.
	This subroutine constructor is included for efficiency reasons rather than
	theoretical reasons; hence its description is delayed to appendix A.
\end{quote}


\subsubsection{First level recognizers}

\predicateone {FUNCTION?}{\obj}\index{FUNCTION?}
\begin{quote}
	{\tt FUNCTION?} is a function recognizer, which evaluates to true if \obj\
	is the name of an {\HG} function, false otherwise. 
	{\tt FUNCTION?} can not tell a built-in function from a \flconst\ defined
	function, or a compiled function from an interpreted one.
	{\tt FUNCTION?} returns the correct information only if {\tt\infmode} is
	true; else it prints a warning message to signal that an un-allowable
	information has been required and executes an {\tt HGKM-ABORT} instruction: hence
	every {\tt FUNCTION?} call must be ``surrounded'' by a {\tt TRAP}
	call\footnote{Further information about {\tt TRAP} and {\tt HGKM-ABORT} can be
	found in ``Error handling'' section}.
\end{quote}

\predicateone {MACRO?}{\obj}\index{MACRO?}
\begin{quote}
	{\tt MACRO?} is a macro recognizer, which evaluates to true if \obj\ is the
	name of an {\HG} macro, false otherwise.
	Note that {\tt MACRO?} can not tell a system macro from a user defined
	macro.
	Furthermore, {\tt MACRO?} has the same behavior as {\tt FUNCTION?} with
	respect to {\tt\infmode}.
	Finally we want to mention a small problem: both {\tt MACRO?} and
	{\tt FUNCTION?} evaluate to true in case of built-in subroutines, because
	there is no information telling if a such subroutine has been defined as
	macro or as function.
\end{quote}

\functionone {GET-FUN}{\obj}{\funcdesc}\index{GET-FUN}
\begin{quote}
	{\tt GET-FUN} returns a function description, in one of the following forms:
	%
	\begin{itemize}
		\item
			{\tt (HGKM <function name> COMPILED)} if the \newline function is
			compiled; 
		\item
			{\tt (HOST <function name> <bound variables list> <body>)} if the
			\newline function is an un-compiled system function;
		\item
			{\tt (HGKM <function name> <bound variables list> <body>)} if the
			\newline function is an uncompiled \flconst\ defined function.
	\end{itemize}
\end{quote}


\subsection{Error handling}

\functionone {TRAP}{\form}{\value}\index{TRAP}
\begin{quote}
	{\tt TRAP} evaluates \form\ and if no error occurs during its evaluation
	{\tt TRAP} returns the same value as the form ({\tt LIST} \form).
	If a system error occurs during an evaluation then {\tt TRAP} returns an
	atomic object that can be tested for determining what the error was.
	If the function {\tt HGKM-ABORT} is called during evaluation of \form\ then {\tt
	TRAP} will return the value of the argument of {\tt HGKM-ABORT}.
\end{quote}

\actionone {HGKM-ABORT}{\form}\index{HGKM-ABORT}
\begin{quote}
	{\tt HGKM-ABORT} returns the value of \form\ to the most recent {\tt TRAP}; every
	{\tt HGKM-ABORT} call must be lexically surrounded at least by a {\tt TRAP}
	function call.
\end{quote}

\predicateone{ERROR-OCCURRED?}{\value}\index{ERROR-OCCURRED?}
\begin{quote}
	{\tt ERROR-OCCURRED?} checks whether a {\tt TRAP}ped \form\ {\tt HGKM-ABORT}ed.
\end{quote}

\functionone{TRAP-GET-RESULT}{\value}{\obj}\index{TRAP-GET-RESULT}
\begin{quote}
	{\tt TRAP-GET-RESULT} selects from \value\ the result of the evaluation of
	\form.
\end{quote}


\subsection{Input-output}

In this section {\HG} input-output handling primitives are presented.
{\HG} uses unidirectional channels implemented on Common LISP streams.
A channel is an object representing in {\HG} environment an external object,
i.e.\ a file or the keyboard.
A channel needs to be open to become active.
The system keeps track of open channels, storing information about each active
channel in a global data structure, the channel list.
Every channel list entry contains
%
\begin{itemize}
	\item
		the channel itself
	\item
		the data flow direction ({\tt IN} or {\tt OUT})
	\item
		the type of objects that can be communicated through the channel, namely
		s-expressions ({\tt SEXP}) or characters ({\tt CHAR}).
	\item
		the information on the object bound to the channel (the device on which
		is stored the physical file in case of file I/O, the keyboard in case of
		console I/O, etc).
\end{itemize}


\subsubsection{Opening and closing channels}

\functionone {HGKM-ISOPEN}{\filena}{\chan}\index{HGKM-ISOPEN}
\begin{quote}
	\filena\ is a form which evaluates to a string.
	{\tt HGKM-ISOPEN} creates and returns a channel for s-expressions
	corresponding to the file \filena, input direction.
	When trying to open a channel corresponding to a non existing file, {\tt
	HGKM-OPEN} executes an {\tt HGKM-ABORT} instruction, to signal that an illegal
	operation has been required: hence {\tt HGKM-OPEN} must be {\tt TRAP}-ped.
\end{quote}

\functionone {HGKM-ICOPEN}{\filena}{\chan}\index{HGKM-ICOPEN}
\begin{quote}
	\filena\ is a form which evaluates to a string.
	{\tt HGKM-ICOPEN} creates and returns a channel for characters corresponding
	to the file \filena, input direction.
	When trying to open a channel corresponding to a non existing file, {\tt
	HGKM-OPEN} executes an {\tt HGKM-ABORT} instruction, to signal that an illegal
	operation has been required: hence {\tt HGKM-OPEN} must be {\tt TRAP}-ped.
\end{quote}

\functionzero {TTY-IOPEN}{\chan}\index{TTY-IOPEN}
\begin{quote}
	{\tt TTY-IOPEN} returns a channel for console input.
\end{quote}

\functionone {HGKM-OSOPEN}{\filena}{\chan}\index{HGKM-OSOPEN}
\begin{quote}
	\filena\ is a form which evaluates to a string.
	{\tt HGKM-OSOPEN} creates and returns a channel for s-expressions
	corresponding to the file \filena, output direction.
\end{quote}

\functionone {HGKM-OCOPEN}{\filena}{\chan}\index{HGKM-OCOPEN}
\begin{quote}
	\filena\ is a form which evaluates to a string.
	{\tt HGKM-OCOPEN} creates and returns a channel for characters corresponding
	to the file \filena, output direction.
\end{quote}

\functionzero {TTY-OOPEN}{\chan}\index{TTY-OOPEN}
\begin{quote}
	{\tt TTY-OOPEN} returns a channel for console output.
\end{quote}

\actionone {HGKM-CLOSE}{\chan}\index{HGKM-CLOSE}
\begin{quote}
	{\tt HGKM-CLOSE} closes \chan\ and updates the channel list taking off the
	information about \chan.
	{\tt HGKM-CLOSE} must be trapped because, when \chan\ does not exist, an
	{\tt HGKM-ABORT} occurs to signal that an illegal operation has been required.
\end{quote}


\subsubsection{Input output specifying channels}

\functionone {CHAN-READ}{\chan}{\obj}\index{CHAN-READ}
\begin{quote}
	{\tt CHAN-READ} reads an s-expression from \chan\ and returns it; to handle
	the case in which the object read is an {\tt EOF} a {\tt TRAP} function call
	is required. 
	\chan\ must have been created by {\tt TTY-IOPEN} or by {\tt HGKM-ISOPEN}.
\end{quote}

\functionone {CHAN-CHI}{\chan}{\chr}\index{CHAN-CHI}
\begin{quote}
	{\tt CHAN-CHI} reads a character from \chan\ and returns it; to handle the
	case in which the character read is an {\tt EOF} a {\tt TRAP} function call
	is required. 
	\chan\ must have been created by {\tt TTY-IOPEN} or by {\tt HGKM-ICOPEN}.
\end{quote}

\actiontwo {CHAN-PRINC}{\obj}{\chan}\index{CHAN-PRINC}
\begin{quote}
	{\tt CHAN-PRINC} writes \obj\ on \chan.
	\chan\ must have been created by {\tt TTY-OOPEN} or by {\tt HGKM-OSOPEN}.
\end{quote}

\actionone {CHAN-CRLF}{\chan}\index{CHAN-CRLF}
\begin{quote}
	{\tt CHAN-CRLF} writes a carriage return on \chan.
	\chan\ must have been created by {\tt TTY-OOPEN} or by {\tt HGKM-OSOPEN}.
\end{quote}

\actiontwo {CHAN-CHO}{\chr}{\chan}\index{CHAN-CHO}
\begin{quote}
	{\tt CHAN-CHO} writes \chr\ on \chan.
	\chan\ must have been created by {\tt TTY-OOPEN} or by {\tt HGKM-OCOPEN}.
\end{quote}

\functionone{CHANNEL-GET-FNAME}{\chan}{\filena}\index{CHANNEL-GET-FNAME}
\begin{quote}
	{\tt CHANNEL-GET-FNAME} returns the name of the file corresponding to the
	channel \chan.
\end{quote}


\subsubsection{Input output with implicit channels}

\actionone {SELECT}{\chan}\index{SELECT}
\begin{quote}
	{\tt SELECT} is the primitive to set the values of default channels.
	If into the channel list the direction associated with \chan\ is {\tt IN},
	then \chan\ will be selected as default input channel, else as default
	output channel.
\end{quote}

\functionzero {HGKM-READ}{\obj}\index{HGKM-READ}
\begin{quote}
	{\tt HGKM-CHI} reads an s-expression from the selected input and returns it.
	to handle the case in which the object read is an {\tt EOF} a {\tt TRAP}
	function call is required.
	The selected input must have been created by {\tt TTY-IOPEN} or by {\tt
	HGKM-ISOPEN}. 
\end{quote}

\functionzero {HGKM-CHI}{\chr}\index{HGKM-CHI}
\begin{quote}
	{\tt HGKM-CHI} reads a character from the selected input and returns it; to
	handle the case in which the character read is an {\tt EOF} a {\tt TRAP}
	function call is required.
	The selected must have been created by {\tt TTY-IOPEN} or by
	{\tt HGKM-ICOPEN}. 
\end{quote}

\actionone {HGKM-PRINC}{\obj}\index{HGKM-PRINC}
\begin{quote}
	{\tt HGKM-PRINC} writes \obj\ on the selected output.
	The selected output must have been created by {\tt TTY-OOPEN} or by
	{\tt HGKM-OSOPEN}.
\end{quote}

\actionone {CRLF}{\chan}\index{CRLF}
\begin{quote}
	{\tt CRLF} writes a carriage return on the selected output.
	The selected output must have been created by {\tt TTY-OOPEN} or by
	{\tt HGKM-OSOPEN}.
\end{quote}

\actionone {HGKM-CHO}{\chan}\index{HGKM-CHO}
\begin{quote}
	{\tt HGKM-CHO} writes \chr\ on the selected output.
	The selected output must have been created by {\tt TTY-OOPEN} or by
	{\tt HGKM-OCOPEN}.
\end{quote}

\actionzero {HGKM-INITIO}\index{HGKM-INITIO}
\begin{quote}
	{\tt HGKM-INITIO} initializes I/O by setting the following I/O-related
	variables:  {\tt CHAN-LIST}, {\tt STANDARD-INPUT}, {\tt STANDARD-OUTPUT},
	{\tt SELECTED-INPUT}, {\tt SELECTED-OUTPUT}.
\end{quote}


\pagebreak
\section{Extended {\HG}}

In this section extended {\HG} is presented.
Extended {\HG} is a collection of primitives which operate on basic {\HG} data
structures or on newly defined ones.

Extended {\HG} is not founded on theoretical considerations like basic {\HG}.
Its semantics can't be axiomatized in first order logic.
Extended {\HG} has been developed mainly for sake of performance and programming
facility, and it has grown up ``on demand'': that is, when it has been realized
that a set of primitives was useful it has been implemented. This is why
extended {\HG} does not look homogeneous.

During this revision, an attempt has been made to keep extended {\HG} as small
as possible: hence, several primitives which are actually part of the source
code are not presented in this manual because they seem to be rather useless.
Further revisions of the source codes will face the problem of removing them.


\subsection{Truth values}

\begin{itemize}
	\item {\tt (NOT \ldots)}\index{NOT}: as in CL.
	\item {\tt (AND \ldots)}\index{AND}: as in CL.
	\item {\tt (OR \ldots)}\index{OR}: as in CL.
	\item {\tt (IST {\it [sym]})}\index{IST}:
	  {\tt IST} and its counterpart {\tt ISF} should theoretically allow GETFOL
	  to use real truth values instead that {\tt T} and {\tt NIL}. {\tt IST}
	  returns {\tt TVAL} when {\it sym} is not {\tt MTL}, {\tt FVAL} otherwise.
	\item {\tt (ISF {\it [sym]})}\index{ISF}:
	  {\tt ISF} returns {\tt FVAL} when {\it sym} is {\tt MTL}, {\tt TVAL}
	  otherwise.
\end{itemize}


\subsection{Symbols}

\begin{itemize}
    \item {\tt (CAPITALIZE {\it [sym {\tt | } str]})}:\index{CAPITALIZE}
        returns the argument with all its letters capitalized.
    \item {\tt (CAPITALIZE-SYM {\it [sym]})}:\index{CAPITALIZE-SYM}
        returns {\it sym} with all its letters capitalized.
    \item {\tt (CAPITALIZE-STR {\it [str]})}:\index{CAPITALIZE-STR}
        returns {\it str} with all its letters capitalized.
    \item {\tt (CONCAT-SYMS {\it [sym$_1$][sym$_2$]})}:\index{CONCAT-SYMS}
        returns the concatenation of {\it sym$_1$} and {\it sym$_2$}.
    \item {\tt  (PAIR2CHR {\it [pair]})}:\index{PAIR2CHR}
        makes {\it pair} into the corresponding symbol.
    \item {\tt  (INT2SYM {\it [int]})}:\index{INT2SYM}
        makes {\it int} into the corresponding symbol.
    \item {\tt  (NAT2SYM {\it [int]})}:\index{NAT2SYM}
        makes a natural {\it int} into the corresponding symbol.
    \item {\tt  (NAT2SYM {\it [nat]})}:\index{NAT2SYM}
        makes the natural {\it nat} into the corresponding symbol.
    \item {\tt (NUMERAL {\it [num]})}\index{NUMERAL}
        looks if {\it num} is a numeral.
\end{itemize}


\subsection{Integers}

These primitives are trivial, and do not require any explanation.
Note that {\tt INTEQ} (the integer equality predicate) must be used instead of
{\tt EQ} (the object equality) when compiling because some CL compiler
substitutes the variable n (where n goes from $-1024$ to $1023$), which
evaluates to the integer $n$, with its actual value. This can create unexpected
behavior which is very difficult to detect and fix.

\begin{itemize}
	\item {\tt (MINUS ...)}\index{MINUS}
	\item {\tt (PLUS ...)}\index{PLUS}
	\item {\tt (DIFFERENCE ...)}\index{DIFFERENCE}
	\item {\tt (TIMES ...)}\index{TIMES}
	\item {\tt (QUOTIENT ...)}\index{QUOTIENT}
	\item {\tt (REMAINDER ...)}\index{REMAINDER}
	\item {\tt (NATNUM ...)}\index{NATNUM}
	\item {\tt (INTEQ ...)}\index{INTEQ}
	\item {\tt (LT ...)}\index{LT}
	\item {\tt (LE ...)}\index{LE}
	\item {\tt (GT ...)}\index{GT}
	\item {\tt (GE ...)}\index{GE}
\end{itemize}


\subsection{Characters}

\begin{itemize}
	\item {\tt (ASCII {\it [chr]})}:\index{ASCII}
		  creates and returns the interned symbol corresponding to {\it chr}.
	\item {\tt (SYM-INTERN {\it [obj]})}:\index{SYM-INTERN}
		   creates and returns the interned symbol corresponding to {\it obj}.
	\item {\tt (MAKE-UNINTERNED {\it [obj]})}:\index{MAKE-UNINTERNED}
		   creates and returns an un-interned symbol corresponding to {\it obj}.
	\item {\tt (SYM-GETCHRFIRST {\it [sym]})}:\index{SYM-GETCHRFIRST}
		   returns the character corresponding to the first character of
		   {\it sym}.
\end{itemize}


\subsection{Strings}

\begin{itemize}
	\item {\tt (SUBSTR? {\it [str$_1$][str$_2$]})}:\index{SUBSTR?}
		   evaluates to true if {\it str$_2$} begins with {\it str$_1$}.
	\item {\tt (CONCAT-STRS {\it [str$_1$][str$_2$]})}:\index{CONCAT-STRS}
		  returns the concatenation of {\it str$_1$} and {\it str$_2$}.
	\item {\tt (SUBSTR {\it [str][int]})}:\index{SUBSTR}
		   returns the string of the first {\it int} characters of {\it str}. If
		   {\it int} is too big an error occurs.
	\item {\tt (INSTR? {\it [str$_1$][str$_2$]})}:\index{INSTR?}
		   evaluates to true if {\it str$_1$} appears in {\it str$_2$}.
\end{itemize}


\subsection{Lists}

\begin{itemize}
   	\item {\tt (LIST ...)}:\index{LIST}
        as in CL.
   	\item {\tt (EQUAL ...)}:\index{EQUAL}
        as in CL.
   	\item {\tt (COPY {\it [obj]})}:\index{COPY}
        returns an object {\tt EQUAL} and not  {\tt EQ} to {\it obj}.
    \item {\tt (LENGTH {\it [list]})}:\index{LENGTH}
        returns the length of {\it list}.
    \item {\tt (HGKM-MEMBER {\it [obj][list]})}:\index{HGKM-MEMBER}
        as in CL.
		The equality test is {\tt EQUAL}.
    \item {\tt (HGKM-MEMQ {\it [obj][list]})}:\index{HGKM-MEMQ}
	    like {\tt HGKM-MEMBER}, but with {\tt EQ} test.
		For instance:
		%
		\begin{verbatim}
			> (HGKM-MEMQ '(C D) ((A B)(C D) E))
			FALSE
		\end{verbatim}
    \item {\tt (APPEND ...)}:\index{APPEND}
        as in CL.
    \item {\tt (UNION ...)}:\index{UNION}
        as in CL.
    \item {\tt (REVERSE {\it [list]})}:\index{REVERSE}
        returns the reverse of the top level of {\it list}.
    \item {\tt (SUBST {\it [obj$_1$][obj$_2$][obj$_3$]})}:\index{SUBST}
        substitutes {\it obj$_1$} to every {\tt EQUAL} occurrence of
		{\it obj$_2$} in {\it obj$_3$}.
    \item {\tt (CL-ASSOC ...)}:\index{CL-ASSOC}
        as CL's {\tt ASSOC}: it tests with {\tt EQ}.
    \item {\tt (HGKM-ASSOC ...)}:\index{HGKM-ASSOC}
         the same as {\tt CL-ASSOC}, but it tests with {\tt EQUAL}.
    \item {\tt (SECOND {\it [list]})}:\index{SECOND}
         returns the second element of {\it list}.
    \item {\tt (LAST {\it [list]})}:\index{LAST}
         as in common lisp: it returns the last {\em cons} of {\it list}.
    \item {\tt (HGKM-NTH {\it [int][list]})}:\index{HGKM-NTH}
        returns the {\it int}-th element of {\it list}.
        Note that {\it int} must be positive, otherwise an error occurs.
    \item {\tt (NCONC ...)}:\index{NCONC}
         as in CL.
    \item {\tt  (POS {\it [sym][list]})}:\index{POS}
        returns the first position of {\it sym} in {\it list}.
        If it does not exist {\tt POS} returns {\tt 0}.
    \item {\tt  (TAIL {\it [list][obj]})}:\index{TAIL}
        if {\it obj} occurs in {\it list} the list of elements following
        {\it obj} (not including {\it obj}) is returned; otherwise {\tt MTL}
        is returned. 
    \item {\tt  (FLATTEN {\it [list]})}:\index{FLATTEN}
        {\it list} is flattened and returned.
    \item {\tt  (PRELIST {\it [list][int]})}:\index{PRELIST}
        creates a list of the first {\it int} elements of {\it list}.
    \item {\tt  (SUFLIST {\it [list][int]})}:\index{SUFLIST}
        creates a list by deleting the first {\it int} elements of {\it list}.
    \item {\tt  (INLIST? {\it [list$_1$][list$_2$]})}:\index{INLIST?}
        looks if {\it list$_1$} is in {\it list$_2$} (only at top level).
    \item {\tt  (SUBTRACT {\it [list$_1$][list$_2$]})}:\index{SUBTRACT}
        returns the list of the elements of {\it list$_1$} which are not in
        {\it list$_2$}. 
    \item {\tt  (INTERSECT {\it [list$_1$][list$_2$]})}:\index{INTERSECT}
         returns the intersection
    \item {\tt (SUBSET {\it [list$_1$][list$_2$]})}:\index{SUBSET}
        looks if {\it list$_1$} is a subset of {\it list$_2$}.
        of {\it list$_1$} and {\it list$_2$}.
    \item {\tt  (INSERT! {\it [obj][list]})}:\index{INSERT!}
        inserts {\it obj} in {\it list} making side effect.
        {\tt INSERT!} assumes that {\it obj} can only occur once in {\it
        list}. 
    \item {\tt  (HGKM-DELETE {\it [obj][list]})}:\index{HGKM-DELETE}
        deletes {\it obj} from {\it list} without doing side effect.
        {\tt HGKM-DELETE} assumes that {\it obj} only occurs once in {\it
        list}. 
    \item {\tt  (DELETE! {\it [obj][list]})}:\index{DELETE!}
         the same as {\tt HGKM-DELETE} but with side effect.
    \item {\tt  (OUTOF {\it [obj][list]})}:\index{OUTOF}
        removes all the occurrences of {\it obj} from {\it list} without
        doing side effect.
    \item {\tt  (OUTOF! {\it [obj][list]})}:\index{OUTOF!}
         the same as {\tt OUTOF} but with side effect.
    \item {\tt  (UNLIST {\it [list]})}:\index{UNLIST}
        substitutes the last cons of {\it list} with its {\tt CAR}.
        For instance:

        {\tt (UNLIST (QUOTE (A))) $\Longrightarrow$ A}

        {\tt (UNLIST (QUOTE (A B))) $\Longrightarrow$ (A . B)}

        {\tt (UNLIST (QUOTE (A B C))) $\Longrightarrow$ (A B . C)}
    \item {\tt  (REVMAP {\it [list]})}:\index{REVMAP}
        puts the last element of {\it list} at the top of
        {\it list} itself.
    \item {\tt  (REPETITION-GET {\it [list]})}:\index{REPETITION-GET}
         returns the first element occurring twice in {\it list}.
    \item {\tt  (LIST-N {\it [obj][int]})}:\index{LIST-N}
         returns the list of {\it int} occurrences of {\it obj}.
    \item {\tt  (PRINTLIST {\it [list]})}:\index{PRINTLIST}
         prints all the elements of {\it list}.
\end{itemize}


\subsection{Arrays}

\begin{itemize}
    \item {\tt (TUPLE {\em [sym]} {\em [len]})}:\index{TUPLE}
        sets the value of the variable {\em sym} to be an array of length
        {\em len}. 
    \item {\tt (SETTUP {\em [array]} {\em [index]} {\em [data]})}:
	\index{SETTUP}
        sets the {\em index}-th entry in {\em array} to {\em data} destroying
        the previous information.
    \item {\tt (GETTUP {[\em array]} {\em [index]})}:\index{GETTUP}
        returns the data stored in {\em array} at the position indexed by
		{\em index}.
\end{itemize}


\subsection{Ordered lists of integers}

\begin{itemize}
	\item {\tt (UNIONLE {\it [list$_1$][list$_2$]})}:\index{UNIONLE}
		given two lists of natural numbers, each in ascending order, creates the
		ascending ordered union. 
	\item {\tt (UNIONGE {\it [list$_1$][list$_2$]})}:\index{UNIONGE}
		given two lists of natural numbers, each in descending order, {\tt
		UNIONGE} creates the descending ordered union.
	\item {\tt (ORDER {\it [list]})}:\index{ORDER}
		returns the list of natural numbers occurring in {\it list} in
		increasing order.
	\item {\tt (INSERT {\it [int][list]})}:\index{INSERT}
		inserts {\it int} in {\it list}, an ascending-ordered list of natural
		numbers.
	\item {\tt (hole {\it [list]})}:\index{hole}
		being {\it list} argument a list of integers greater than or equal to
		$-1$, {\tt hole} returns the smallest number not appearing in
		{\it list}.
		
		Example:
		%
		\begin{verbatim}
			> (hole (QUOTE (-1 0 1 3)))
			2
			> (hole (QUOTE (0 2 3)))
			-1
		\end{verbatim}
\end{itemize}


\subsection{Associative lists}

Elements of associative lists are implemented as dotted pairs.
%
\begin{itemize}
    \item {\tt (alistelement-get-1st {\it [alist-elem]})}:
	\index{alistelement-get-1st}
        returns the {\tt CAR} of the dotted pair.
    \item {\tt (alistelement-get-2nd {\it [alist-elem]})}:
    \index{alistelement-get-2nd}
         returns the {\tt CDR} of the dotted pair.
    \item {\tt (alistelement-mak {\it [obj1][obj2]})}:
    \index{alistelement-mak}
        returns the pair {\tt ({\it obj1} . {\it obj2})}
\end{itemize}
%
Associative lists are associative addressing data structures; they are
implemented as lists of dotted pairs.
%
\begin{itemize}
    \item {\tt (alist-get-first {\it [alist]})}:\index{alist-get-first}
         returns the first element of {\it alist}
    \item {\tt (alist-get-rest {\it [alist]})}:\index{alist-get-rest}
         returns the rest of {\it alist} 
    \item {\tt (alist-cons {\it [alist][alist-elem]})}:\index{alist-cons}
         adds {\it alist-elem} to {\it alist}
    \item {\tt (HGKM-REPLACE {\it [obj$_1$][obj$_2$][alist]})}:
    \index{HGKM-REPLACE}
        if {\it obj$_1$} is the first element of an alist element, then
		{\tt HGKM-REPLACE} replaces the second element with  {\it obj$_2$};
        otherwise, {\tt HGKM-REPLACE} adds the {\tt CONS} of {\it obj$_1$} 
        and obj$_2$ to {\it alist}.
    \item {\tt Alist-empty}:\index{Alist-empty}
        is the constant representing an empty associative list
    \item {\tt (ALIST-EMPTY? {\it [alist]})}:\index{ALIST-EMPTY?}
        looks if {\it alist} is empty
    \item {\tt (alist-add {\it [alist][obj$_1$][obj$_2$]})}\index{alist-add}
        adds the pair {\tt ({\it obj$_1$} . {\it obj$_2$})} to {\it alist}
        if not already present
    \item {\tt (alist-del {\it [alist][obj]})}\index{alist-del}
        removes the pair pointed by {\it obj} from {\it alist}
    \item {\tt (alist-union {\it [alist$_1$][alist$_2$]})}\index{alist-union}
        adds to {\it alist$_2$} all the elements of {\it alist$_1$},
        overwriting those elements of {\it alist$_2$} which are also in
		{\it alist$_1$} 
    \item {\tt (alist-cons-shared {\it [alist][alist-elem]})}:
    \index{alist-cons-shared}
        same as {\tt alist-cons} but with side effect on {\it alist}.
\end{itemize}


\subsection{Extending the evaluation syntax}

\begin{itemize}
	\item {\tt (MAPCAR ...)}:\index{MAPCAR}
		  as in CL.
\end{itemize}


\subsection{Extending the input output}

\begin{itemize}
	\item {\tt (HGKM-LOAD {\it [str]})}:\index{HGKM-LOAD}
		  loads the file named {\it str}.
	\item {\tt (HGKM-LOAD-FROM-CHANNEL {\it [chan]})}:
	\index{HGKM-LOAD-FROM-CHANNEL}
		 loads s-expressions from the default channel until an {\tt EOF} is
		 found.
\end{itemize}


\appendix
\section{Some notes about implementation}

While dealing with big software systems, in phase of development it may be very
helpful to know if a subroutine is built-in or user-defined, and if it has been
defined as function or macro.
If {\tt \infmode}\ is true when {\HG} is created, all this information is made
available: for each subroutine declaration, zero-level constructors set a
property recording that the subroutine is built-in; first-level constructors
set a property recording that the subroutine is user-defined and whether it has
been defined as a function or as a macro; reading these properties, first level
recognizers can tell if a subroutine is built-in or system-defined, and if it is
a function or a macro.
If {\tt \infmode} is false no property is set by the zero and first level
constructors, and first level recognizers, when called, will signal that no
answer is possible because no information is available.
For a detailed description of each first-level constructor and recognizer, with
respect to {\tt \infmode}, the reader is referred to basic {\HG} chapter.

Of course, when {\tt \infmode} is false, a large amount of storage is saved, and
a more compact {\HG} executable file is obtained.
Finally we remark that, for sake of consistency, the value of {\tt \infmode}
should not change while {\HG} is being created.

A complex software system may have big performance problems, because of the
overhead due to the great number of function calls, even if it runs compiled.
Therefore, the only way to speed up the system is to define as many subroutines
as possible as macros. These macros are expanded at the moment of compilation,
obtaining in this way a code which is modular and well structured, and which
does not have the overhead of function calls.
There are two drawbacks to this approach: the former, and least important, is
that the program becomes big because of the macro expansion; the latter, and
critical, is that debugging becomes very hard, because all information related
to functional definition is lost.

In {\HG} this problem may be solved using the first-level constructor {\tt
DEFSUB}, that allows to define subroutines either like macros or like functions,
without changing source code.
As explained below, {\tt DEFSUB} definitions depend on the value of a system
constant, {\tt \debmode}.

A {\tt DEFSUB} call has a syntax as defined in basic {\HG} chapter: it requires
three arguments, that is a {\it subroutine name}, an {\it arguments list} and a
{\it body}.
The third argument, {\it body}, is a function body, as needed in {\tt DEFUN}
subroutines definitions.
If {\tt \debmode} is true, {\tt DEFSUB} works like {\tt DEFLAM}.
If {\tt \debmode} is false, {\tt DEFSUB} executes a {\tt DEFMAC} call, modifying
the third argument: that is, starting from {\it body}, {\tt DEFSUB} builds a
macro body that, when macro-expanded, returns {\it body}.

Hence {\tt DEFSUB} is a subroutine constructor that has a simple function
definition syntax and behaves either like a function or macro constructor,
depending on the value of {\tt \debmode}.
If {\tt \debmode}\ is true debugging is possible, because {\tt DEFSUB}
subroutines have been declared as functions; if {\tt \debmode}\ is false
debugging is very hard, but the execution is speed up because no function call
overhead is present.

Note that {\tt DEFSUB} is subject to some restriction, as remarked in basic
{\HG} chapter: this is due to the complexity of macro body creation.

Finally note that, to reach some performance improvements, a {\tt DEFSUB}
definition of a general subroutine {\tt SBR} must be loaded before compiling
subroutines calling {\tt SBR} itself; else, macro expansion can not be made at
compile time, and the {\tt SBR} is executed interpretively and expanded at run
time.


% ................................. bibliography ..............................
\bibliographystyle{alpha}
\bibliography{}

% .................................... index ..................................
\input{hgkmman.ind}

\end{document}


\end{document}


\end{document}


\end{document}
